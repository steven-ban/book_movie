\subsection{《一小时的故事》}

作者:凯特·肖邦

一些短篇,讲述了19世纪初美国南部法语聚集区的日常故事,主题是女性的所感所思。这大概是为什么编入《被误读的女权·女性主义源流》中了。故事里有男女求爱、婚姻生活、离家出走、家庭变故……行文使用较多法语,语言流畅优美,如同散文。

作者生活在美国19世纪后半叶,作品有关于赞同出轨的情节,可见她是反对婚姻制度而追求爱情的,但因此成为女权主义者有点搞笑。本书的短篇里好几个设计女性婚内出轨和对丈夫的厌弃或反抗。本书在她生前并未发表。

\subsubsection{阿泰纳伊斯}
女主新婚不久,婚姻还算美满,但不适应,多次跑回娘家,丈夫去找,她回家后借助哥哥的力量消失,出走到一个城镇上投奔朋友。那里她遇到一个年轻人,仰慕她,但她在枯燥离别的生活里待腻了,开始想家,最后回家和丈夫团聚。

\subsubsection{她的信}
妻子瞒着丈夫和情人通信,随后她得了绝症,死前把信密封,写上“死后让丈夫把信毁掉”的话。

妻子死后,丈夫怀着信任和正直,在犹豫中把信投进河里。之后他越来越怀疑,反复梦见那条河流。最终,他在抑郁中投河自尽。

心理描写步步推进,冷静细腻,情节简单却惊心动魄,优美的小说。

\subsubsection{紫丁香}
寡居的女人到修道院寻找安慰,为人们弹琴,但最终被赶了出来。

\subsubsection{塞莱斯坦夫人的离婚}
丈夫酗酒半年不归,夫人辛苦劳作,想离婚,主教劝阻不听,律师帮她,也爱上了她。律师找到夫人,她说丈夫回来了,她不离婚了。

\subsubsection{一小时的故事}
短小精悍的情节,也就两三千字:妻子听说丈夫死亡,嚎啕大哭,独自在房间里,想到自己不爱丈夫,马上要自由,不禁转为平静,继而狂喜,在医生到来前因开心引发心脏病而死。