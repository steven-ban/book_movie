\subsection{《太空旅客》}
从地球到”家园二号“,飞船需要航行120年,因此旅客们需要完全在休眠舱中度过,因为人的寿命似乎并没有随宇航技术(或许还有其他技术)的进步而延长。
 
但是,飞船出了散热故障,以致于一个旅客Jim30年后就已经醒来,这意味着他需要独自一个度过90年才有可能到达目的地,但很不幸,似乎他并没有足够的寿命。他孤独地生活了一年,忍无可忍,于是手动唤醒了一个美女作家Aurora。两人很快坠入爱河,但得知真相的Aurora如何看待这件事呢? 

《太空旅客》(Passengers)的这个设定其实非常迷人。当一个人获得到掌握另一个人生死的时候,他便是神,神无对错,但人终究有对错。在这一点上,Jim的做法当然是谋杀。 

作为典型的好莱坞制作,后面的剧情其实乏善可陈,只不过是给Jim一个将功补过的机会而已,他冒着生命危险修正了飞船错误,救下了5000人的生命,Aurora也原谅了他;同时,Jim把重新休眠的机会给了Aurora。但是,仅仅在唤醒Aurora这件事上,Jim的过失其实很大,是做恶。换作是我,孤独终老也不会谋杀另外一个人的青春。 

这个电影值得吐槽的不仅是谋杀问题,飞船系统故障概率也太大了点。另外,Jim和Aurora似乎不是一个阶级,吃的饭都不一样。
 
其实这个设定更值得人沉思的一点是,这不仅仅是科幻,父母给了子女生命,这一点多像Jim的所做所为啊!
 
值得推荐的理由还在于:大表姐好美啊!大表姐的脸是倒瓜子,有点肥,但肥得恰到好处,披着长发有成熟女人味,结合身材让人蠢蠢欲动。