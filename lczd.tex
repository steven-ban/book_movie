\subsection{《浪潮之巅》}
\subsubsection{阅读中的记录}
本书记录了硅谷各大公司几十年(甚至上百年)的沉浮兴衰,重点在于分析其盛衰原因,总结规律;作者也同时给出了硅谷动作的一些因素,如资本、投资、技术等等。

AT\& T公司

AT\& T公司下有贝尔实验室,诞生了很多世界第一的发明:电话,射电天文望远镜,晶体管,数字交换机,UNIX操作系统,C语言,发现电子的波动性,提出信息论,发射第一颗商用通信卫星,铺设第一条商用光纤……

为什么美国小公司能很快成为跨国垄断公司?反垄断法,逼着它搞技术进步。

蓝筹股,道琼斯工业指数,blue chips,蓝色筹码是赌局中面值最大的

失败原因:在华尔街的操纵下不断拆分,没抓住互联网的机会

IBM

企业特点:保守。

成立之初就服务于政府和事业单位,二战时民用转军用,二战后小沃森执掌IBM,年增长率30\% 以上。但错过微机革命,原因:企业基因就是服务于大企业、反垄断造成对盗版和破解不管不用、微软的崛起。

IBM实验室是创新的源泉,很多发明都源自这里:

计算机硬盘
FORTRAN
计算机内存(DRAM)
关系型数据库
通信工程中的BCJR算法
微型硬盘
SEM
专利申请大户,鼓励员工申请专利

2008年金融危机对IBM影响较小,原因:

主要客户均是大型企业用户
不断淘汰低盈利部门
全球化程度高
苹果

英特尔

2008年金融危机损失巨大,因为受“反摩尔定律”影响较大

微软

Bill Gates的成功秘诀:

保守与冒险的平衡
心比天高,脚踏实地
多赚小钱,积累成大钱
甲骨文

思科

雅虎

惠普

摩托罗拉

谷歌

诺基亚

3M

GE

\subsubsection{科技公司后的运转机制}
计算机工业的技术规律

摩尔定律,对IT行业的影响:

为了使摩尔定律生效,IT公司必须在短时间内完成下一代产品的研发
由于有了硬件的支撑,以前不敢想的应用可以被实现
公司的研发必须针对多年后的市场
安迪-比尔定律:软件和硬件相互促进,相互消耗

反摩尔定律:IT公司如果维持销售产品和销售量不变,每18个月营业额减半。

风投

投行

斯坦福大学

\subsubsection{书评}
吴军语音识别领域出身,算是计算机学科的行家,本身也“有幸见证历史”,经历过互联网行业从九十年代诞生到2010年的十多年的变迁,认识很多硅谷的高管和技术专家,能获得一些“小道消息”,使得这本书故事性很好,料也算比较独家比较足,虽然大部分仍旧只是二道贩子,不是第一手资料。

本书