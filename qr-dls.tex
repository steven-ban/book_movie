\subsection{《情人》}
\subsubsection{标注}

我知道,女人美不美,不在衣装服饰,不在美容修饰,不因为施用的香脂价钱贵不贵,穿戴珍奇宝物、高价的首饰之类。我知道问题不在这里。问题究竟何在,我也不知道。
人家常说,我这头发最美,这话由我听来,我觉得那意思是说我不美。

\subsubsection{读后感}

1.杜拉斯的行文絮絮叨叨絮絮叨叨絮絮叨叨(重要的事情说三遍),从遇见情人的穿着到情人给她讲自己的故事,从脾气怪异的母亲(我的总结)到不争气的两个哥哥,再到遇见的几个白人朋友,东拉西扯,颇为意识流。老实说,我不喜欢这样的文风,她的文采我也丝毫不欣赏。

2.作者以讲述情人的理由,不断温习回顾自己年轻时的心态经历,回顾自己的亲人和朋友。这乃是一部回忆录。 
3.杜拉斯的讲述是啰嗦的,口语的,仿佛你在听一位年老妇人讲述自己的青春。

4.这本小说是王小波极力推崇的,他称赞杜拉斯写小说的手法独树一帜。另外,他推荐王道乾的这个译本。翻译地确实不错。

5.我不懂女人,所以这本充满女性气息的小说,我看不懂。满分五分,我给三分。