\subsection{《王氏之死:大历史背后的小人物命运》}

标签: 汉学 \  中国历史 \  清史 \  民谷俗史 \  明清法律 \  地方史

作者:史景迁\footnote{参考\url{https://baike.baidu.com/item/\%E5\%8F\%B2\%E6\%99\%AF\%E8\%BF\%81}}

\subsubsection{笔记}
\emph{七出}:(双方同意的婚姻也为法律所允许)
\begin{itemize*}
	\item 无子
	\item 淫佚
	\item 不事舅姑
	\item 口舌
	\item 盗窃
	\item 妒嫉
	\item 恶疾
\end{itemize*}

妻子不想离婚,可有\emph{三不去}:
\begin{itemize*}
	\item 妻子曾为公婆守丧三年
	\item 娶时丈夫贫贱,后来富贵
	\item 妻子无家可归
\end{itemize*}

李孝悌认为,\emph{过去三十年多,美国学者在中国史研究的领域中,表现最突出的要算是中国近代社会史了}。

\subsubsection{书评}

这是著名汉学家史景迁的著作,出版后引起了西方和中国的极度重视。史景迁所用的史料,无非是惯常的县志和官员文人笔记,甚至是作为“稗官野史”的《聊斋志异》。但透过这些“简单”的史料,史景迁构建出一幅清晰的、丰富的、绚烂的生活画卷。史景迁继而使用优美的文笔,勾勒出这个画卷里一个个生动的故事,借以明清的社会风俗、法律、政治经济常识,将这个故事讲得妙笔生花。

故事发生在1670年左右的山东省郯城县。这个县是个普通的北方县城,特殊之处在于《郯城县志》的记载(这和其他的县志的记载并没有什么不同)、1670年左右黄六鸿的笔记(十分翔实准确,这是特殊之处)以及邻近的淄川县的当时蒲松龄的巨著《聊斋志异》。

这是一个天下初定的时代。一百年来,中国的社会发生了翻天覆地的变化:17世纪初经济在明中叶繁荣之后的余波、明末灾荒带来的减产和土匪横行、满人入关后的屠杀、社会的逐渐恢复。上两代的人可能还保留着以往的回忆,而年轻的人已经在开始新的稳定的生活。

书名《王氏之死》,其实是一个再简单不过的故事:王氏的丈夫是任某,他们是郯城下属的一个村里的普通农民,生活贫苦。王氏与他人私通,离开了村子,在外面转了一圈后,由于当时的中国有着严格的流民控制政策,无法在陌生的环境下长住,于是又回到了村子,住在一个道观里,被高某和丈夫任某相继发现。高某与任某发生肢体冲突,之后王氏被任某带回家继续过日子。一个大雪纷飞的深夜,任某在床上掐死了王氏,然后把尸体转移到高某家附近,随后诬告是高某所为。县官王六鸿经过审问和实地侦察(包括验尸),获得了真相,释放了高某及其家人,但考虑到任家老太爷只有任某一个人,死刑后任家无后,于是只是杖责了事。很简单的一件事情,史景迁结合了明清时的法律、民俗、道德和经济,放在当时的一个真实的社会环境下考察和书写,让这个故事立体起来,同时对于现代读者和西方读者而言,新颖有趣,足见功力。

史景迁还考察了当时郯城附近的匪患和军民、民族冲突。一旦灾荒持续,农业减产,便会饿死人,很多人便离乡乞讨或者落草为寇,社会治安便会变坏,这在明末越发严重。一些乡野恶霸掌握和武力,或与土匪勾结,便横行乡里,甚至公然和官府对抗。而清军入关后,对待民众也是奸淫掳掠,甚至在社会稳定后的削藩战斗中,官军也是纪律差劲,鱼肉乡里。中国的老百姓,天下兴亡都是很苦的。

《聊斋志异》对于中国来说,是一部小说,而作者选取的故事,鬼怪色彩很少,基本都是普通的人际间故事,即使带了一点志怪色彩,也是从反映人的心态来解析的。对于这些故事是否能拿来做史料与普通人的生活等同,我本人是持有疑问的。

评分:6/10。






