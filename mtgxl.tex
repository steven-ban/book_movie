\subsection{《每天读一点狗的心理》}

\subsubsection{书评}
林乐毅写过一本《图解猫的心理》,两年前对猫感兴趣,想和猫一起晒太阳看书,看了几章,但今年也没养猫,反而在老婆的极力提议下养了狗。狗需要人的时刻陪伴,但上班族没有那个时间和精力,因此似乎养猫更合适一些。但半年时间养下来,狗似乎也没有那么脆弱。养猫主要还是不操心,只要给好吃的好玩的,你一天不理都可以,但养狗就要每天陪着散步玩耍,否则狗自己会疯掉。

说回这本书。这本书按章组织了几个主题,但从逻辑上讲几个主题之间似乎没有明确的界限。每一章里作者都以一句话作为主题来组织内容,再加上一些示例性质的插画。内容上是很乱的,看完之后一点头绪都没有,忘得差不多了。插画也画得不怎么样,并且我也不是喜欢看画来获取信息的人。

就养狗的知识上来说,不够专业和系统,太零碎,我看完以后也不知道怎么养狗。作者只是反复强调狗的生活习性,但没有给出严格的学术性的建议,因此没啥价值。

总之如果满分五星的话,这本书我只愿意给两星,不推荐给养狗的人看。

\subsubsection{标注}
1. 肉类中,狗依次喜欢牛肉、猪肉、羊肉、鸡肉和马肉

2. 狗虽然能尝出辣味和甜味,却对咸味比较迟钝。对狗来说,人类的食物味道过浓,并不好。清淡些的食物比较好。

3. 在狗鼻子的深处有感觉气味的嗅觉细胞,其数量相当于人类的500万倍,据推测有1.5亿到30亿之多。
