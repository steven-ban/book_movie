\subsection{《西西弗神话》}

标签: 加缪 \ 存在主义 \ 荒谬

作者:加缪

本书是加缪的短篇文集,主要是基于存在主义哲学来论述“荒谬”这一概念,涉及一些文学方面的评价,包括萨德、尼采、陀斯妥耶夫斯基笔下的卡拉马佐夫兄弟和其他一些哲学家的观点。说实话,我在哲学上没有什么涉猎和视野,看不太明白。

可以摘抄一些译者的译序来帮助理解(事实上我对正文的理解还不如这一篇短短的译序来得透彻)加缪的观念。

加缪用品达的两句诗为《西西弗神话》题词,译者认为是加缪毕生的座右铭和行为准则,也是高度概括他的生存哲理——\emph{不求永生,竭尽人事}:
\begin{quotation}
吾魂兮无求乎永生,
竭尽兮人事之所能。
\end{quotation}

荒诞作为哲学术语来源于古代一个基督徒的拉丁语怪论:\emph{上帝的儿子死了,绝对可信,因为这是荒诞的;他被埋葬之后又复活了,绝对确实,因为这是不可能的}。这当然这是一种悖论,一种无稽之谈,但是像尼采这样的哲学家就很欣赏(悖逆天道),加缪则称之为“哲学自杀”——\emph{理性阐述往往不得要领,于是利用理性阐述的失败来为信仰荒诞作辩护}。加缪认为,\emph{荒诞正是清醒的理性对其局限的确认……所谓荒诞,是指非理性与非弄清楚不可的愿望之间的冲突,弄个水落石出的呼唤响彻人心的最深处}。加谬说:
\begin{quotation}
理性有自己的范畴,在自己的范畴里是有效的。这正是人类经验的范畴。所以我们想要把一切都搞个水落石出。反之,我们之所以不能把什么都搞清楚,荒诞之所以应运而生,恰恰因为碰上了有效而有限的理性,碰上了不断再生的非理性。……荒诞,则是确认自身界限的清醒理性。
\end{quotation}

可见,加缪的荒诞说建立在矛盾论之上,是人对单一性和透明性的欲望与世界不可克服的多样性和隐晦性之间的矛盾。加缪的荒诞说不是一种概念,而是“荒诞感”(人与其生活的离异,演员与背景的离异)“激情”(人是无用的激情,明知无用仍充满激情;明明知道自由已到尽头,前任无望,为反抗绝望而不断冒险,这叫荒唐)“感知”(人面对自身不合情理所产生的反感,对自身价值形象感到堕落,有了这份自知之明,就叫“荒诞感知”)“精神疾病”(人一旦被剥夺了幻想和光明,便感到自己是现世的局外人,随时想逃脱自我,又无可奈何置身其间,因焦虑而消沉,陷入绝望所患的一种抑郁症,可能会导致自杀),试图弄清楚它是否会导致自杀。

借此得出“荒诞人”的定义:与世界、与时间形影不离的人,不为永恒做任何事情,又不否定永恒的人。加缪称赞西西弗这样的荒诞人,既因为他的激情,也因为他的困苦。荒诞人直面人生,不逃避现实,摒弃绝对虚无主义,怀着反抗荒诞人世的激情,坚持不懈,或许能创造一种人生价值。在西方的哲学体系里,反抗者反抗的是上帝和造物主,要先否定上帝才有个体。加缪说:
\begin{quotation}
与忍耐派、清醒派等各流派相比,创造派最为有效,也是人类唯一尊严的见证,令人震撼:执著地反抗人类自向的状况,坚持不懈地进行毫无结果的努力。创作要求天天努力,自我控制,准确估量真实的界限,有分有寸,有气有力。这样的创作构成一种苦行。这一切都为“无为”,都为翻来覆去和原地踏步。也许伟大的作品本身并不那么重要,更重要的在于要求人经得起考验,在于给人提供机会去战胜自己的幽灵和更接近一点赤裸裸的现实。
\end{quotation}

加缪列举和分析了一些哲学家“反抗”的方式:
\begin{longtable}{p{0.2\textwidth} | p{0.6\textwidth}}

    \caption{加缪笔下哲学家反抗的方式} \\
    \hline
哲学家 & 主要观点 \\
\hline
\endfirsthead

(接上表) \\
哲学家 & 主要观点 \\
\hline
\endhead

\hline
\endfoot
萨德 &  “绝对否定”:上帝牺牲了创造物,上帝和永生都不存在,就应许可人成为上帝 \\
施蒂纳 &  把现实个体的“我”凌驾于上帝与制度,引发无数个体冲突,极端虚无主义,崇拜毁灭,要么死亡,要么重生 \\
尼采 &  反抗者只有放弃一切反抗才能成为神明,甚至放弃为修正这个世界而创造神祇的反抗。世界丧失神明的意志,因此不能再被判决。否定上帝责任,才能解救世界。“对马克思而言,自然受人控制以便服从历史,而对尼采来说,自然让人服从以便控制历史,这正是区别基督徒与希腊人之所在。” \\
马克思主义者 &  通过研究马克思主义最终走向革命的共产党人屈指可数,通常先归依而后读圣贤。可以毫不离谱地说超现实主义者最终走向马克思主义正是因为他们最恨当今的马克思主义。 \\
伊凡·卡拉马佐夫 &  陀斯妥耶夫斯基笔下自己的化身,认同“要么人人得救,要么谁也别得救”,反抗的尽头是自己变成上帝 \\
\end{longtable}

总之这一系列的哲学信条和推理,以及加缪对他们不同的批判与继承,都是相当抽象和形而上的,普通人未必会认同或者否定。这也可以看出欧洲哲学从人文和理性主义走出后的颓丧和失落。这里对于虚无的论述,有佛教和老子的影子,可见东西方哲学,有点殊途同归之感(虽然有极大不同),反观中国哲学则显得早熟和先进(虽然逻辑推理远在西方之下)。

评分:3/5。