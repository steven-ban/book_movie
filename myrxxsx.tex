\subsection{《没有人给他写信的上校》}

标签: 马尔克斯

作者:马尔克斯

这么多年了,他(死者)是我们这里第一个自然死亡的人。

二十年了,我们一直等着他们兑现每次大选后对我们许下的那一大堆诺言,可到头来我们连儿子都没保住……连儿子都没保住!

\subsubsection{人物}

\begin{longtable}{p{0.1\textwidth} | p{0.15\textwidth} | p{0.4\textwidth}}

    \caption{《没有人给他写信的上校》人物表} \\
    \hline
姓名 & 特点 & 事件 \\
\hline
\endfirsthead

(接上表) \\
姓名 & 特点 & 事件 \\
\hline
\endhead

\hline
\endfoot

上校 &75岁 & 曾经是革命军,投降后一直在等退伍金 \\
 ? & 上校的妻子 & 有哮喘 \\
 阿古斯丁 & 上校死去的儿子 & 因斗鸡被警察杀死 \\
 ? & 死者 & 吹铜号的,比阿古斯丁小一个月,1922年出生 \\
堂萨瓦斯 & 阿古斯丁的教父 & 准备400比索买下上校的鸡,再以900比索卖出去 \\
? & 医生 & 总是拿油印的政治新闻给人看 \\
\end{longtable}

\subsubsection{书评}
这个短篇只有5万字,但写得十分精彩,1957年的马尔克斯已经展露出自己无以伦比的文采了。故事的情节也很简单:退伍上校一直在等政府发的退伍金,每周都去邮局看邮件,但是那些忘恩负义的政客彻底忘记了这个(以及和他一样很多的)曾经给国家流过血卖过命的老人,一些老人死去了,而这个国家的政权极不稳定,政府轮番垮台;上校的儿子因为斗鸡而被警察打死,他们留着这只鸡,晚上把它绑在床腿上,白天喂它;他们生活贫困,到了最后不得不卖掉家具、钟表、画、鞋子,甚至要卖掉这只鸡;奸诈的堂萨瓦斯低买高卖,上校在他面前唯唯诺诺,可到最后也没有卖掉。

这本书的视角就是上校,整体气氛十分压抑。上校、上校妻子的生活中的唯一希望就是等上校的退伍金,但是15年了还是等不来;他们的唯一记忆就是儿子和儿子的鸡,随着儿子记忆的远去,儿子留下来的鸡成了他们生活中唯一的关注点。妻子遭受着哮喘病痛的折磨,却把家里打扫得井井有条。他们试图卖掉钟去,希望以此换来40比索,但好面子(其实是自尊心极强)的上校怀揣钟表却没有把它卖出去。能把鸡卖给狡诈的堂萨瓦斯,换来900比索,这是他们生活里的一抹亮色,他在堂萨里斯面前唯唯诺诺,却换来他只愿出400比索的承诺,剩下的差价则被堂萨瓦斯赚去。最后上校决定不卖鸡,而是卖掉自己的鞋子。妻子一遍遍地问他不卖鸡他们吃什么,他回道“吃屎”,他准备拿这只斗鸡去赌博(儿子就因此死于警察枪下)赚钱,妻子也曾经买过彩票。虽然只有5万字,但一个衰败的、贫困的、毫无希望的南美社会栩栩如生展现在读者面前。“吃屎”是他们唯一能做的,唯一在生活里还存在着的渺茫的希望。故事在这里戛然而止,剩下的内容已无须多言,这个人物已经立起来了,他或生,或死,如何过完以后的日子,是需要读者脑补的,作者的使命已经完成。读完此书,不得不为拉美那种破败的社会而叹息。

评分:5/5。