\subsection{《阳光下的罪恶》}

“在太阳底下,到处都有邪恶的事。”——波洛

\subsubsection{人物}

\begin{longtable}{p{0.1\textwidth} | p{0.2\textwidth} | p{0.3\textwidth} | p{0.35\textwidth}}

    \caption{《阳光下的罪恶》人物表} \\
    \hline
人物 &	特点 & 家庭住址 & 事件 \\
\hline
\endfirsthead

(接上表)\\
\hline
人物 &	特点 & 家庭住址 & 事件 \\
\hline
\endhead

\hline
\endfoot
波洛	 & &伦敦怀特里文大厦,伦敦西一区 & \\	
加德纳太太 & 美国人	& & \\
奥德尔-加德纳先生 &美国人 &	纽约	& \\
艾琳-加德纳 & 加德纳夫妇的女儿	 & & \\
艾米丽-布鲁斯特小姐	 & & 南门街,泰晤士河森伯里区	& \\
凯尔索先生	& & &向加德纳夫妇推荐海盗岛 \\
巴里少校	& &卡顿街18号,对詹姆斯,伦敦西南一区 & 喜欢聊起在印度的经历 \\
斯蒂芬-兰恩牧师	 & 五十多岁 & 伦敦	 & \\
克莉丝汀-雷德芬太太 & & & 有恐高症,曾当过老师		\\
帕特里克-雷德芬先生	& & 克劳斯门,赛尔顿,雷斯堡王子市	& \\
贺拉斯-布拉特先生	& &皮克斯街5号,伦敦东部中二区 & 有红色的船 \\
艾莲娜-斯图尔特 & 女明星,男人为她倾倒,女人恨她	& & 曾主演《送往迎来》,十七八年前被控谋杀亲夫,后来证明丈夫有服食砒霜的习惯,被判无罪,然后马歇尔上尉娶了她,被指控当作柯丁顿勋爵的第三者,导致后者离婚,然后被柯丁顿勋爵甩掉,随后嫁给马歇尔上尉,三年前,老爵士罗杰-厄斯金死后把所有财产赠给给她,	目前似乎看上了雷德芬先生(达恩利观点),认识雷德芬先生 \\
肯尼斯-马歇尔上尉 & 单身汉,在海滩上唯一不看艾莲娜-斯图尔特的男人 & 厄普科特大厦73号,伦敦西南区 & 第一个妻子在生女儿琳达的时候死去 \\
琳达-马歇尔  & 16岁,恨继母艾莲娜,喜欢罗莎蒙德	& & \\
罗莎蒙德-达恩利 & 著名女装公司的老板,设计服装,单身 & 卡丁甘大厦,西一区 & 两年前来过海盗岛,和马歇尔上尉是青梅竹马的老朋友,惋惜马歇尔娶了艾莲娜,看到自己儿童时期的幻象 \\
卡斯尔太太 & 旅馆的老板,业主	& & \\	
考恩一家	& & 雷德山,莱瑟赫德镇	& \\
马斯特曼一家 & & 马尔伯乐大道5号,伦敦,西北区	& \\
\end{longtable}

\subsubsection{事件}
马歇尔上尉认为艾莲娜是为了雷德芬先生来的海盗岛,雷德芬太太也认为雷德芬是为了艾莲娜来的海盗岛

罗莎蒙德劝马歇尔和妻子艾莲娜离婚

艾莲娜受人勒索,她说从丈夫那儿弄不到钱了

艾莲娜和雷德芬先生约会

早上,琳达买了绳子和蜡烛,被雷德芬夫人撞见,两人一起去欧湾写生

艾莲娜独自划船出海(10:15)

雷德芬先生和布鲁斯特小姐一同划船出海,在海滩上发现艾莲娜死亡(11:40),面朝下躺在沙滩上,疑似被掐死

\subsubsection{感想}
“她(指琳达)年轻,有年轻人的那种残忍”。不知道作者为什么这么说,年轻人有什么残忍呢?年老之后这种残忍就会消失吗?

这次破案也没有证据,靠的是波洛把一些事实像拼图一样拼接起来,以及依靠直觉去想象的,如果说真有什么证据,那就是死者被利用后向凶手账户上汇的钱,而这个证据之前根本没有出现。读者也只能靠猜想来推断凶手。波洛推理的依据是凶手是惯犯,可以从类似的案件里寻找启发,这一点流于片面和肤浅,降低了推理的趣味。

死者看上去像是水性杨花的女人,能够玩弄男人于股掌之上,其实是心思单纯甚至愚蠢的人,被玩弄和欺骗,成为凶手贪欲下的牺牲品。

作者有意误导读者,让读者认为死者的丈夫和女儿最可疑,手法还是比较高明的。

评分7/10。