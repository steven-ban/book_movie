\subsection{《红太阳是怎样升起的:延安整风运动的来龙去脉》}
\subsubsection{一些标注}
毛具有党内无人企及的极其丰富的中国传统文化的知识,他不仅极其熟悉并爱好唐诗宋词、《昭明文选》、红楼、水浒、三国、野史稗记一类古典文学,同时也嗜读鲁迅杂文,然而毛对鲁迅之外的五四以来的新文学作品却很少涉猎,一是兴趣不大,二是长年深居军中无机会阅读。毛对外国文学作品就知之更少。

「红色的三十年代」是一个世界性的现象,法西斯主义的崛起和由对西方制度的怀疑、动摇而产生的幻减感和深刻的精神危机,促使西方一部分知识分子将人类的前途寄托于斯大林进行的苏联共产主义试验上,因此,从三十年代初至1939年苏联参与瓜分波兰前,许多著名的知识分子纷纷向左转,而在向左转的知识分子中,情感丰富,且对政治和社会生活变化最具敏感性的文学家又占有最大的比重。

王实味与一般左翼人士不同之处在于,他不仅执着于五四自由、民主的理念,他更受到青年马克思人道主义思想,和继承发扬了这种思想的第二国际社会民主主义思想,以及反斯大林主义的托洛茨基部分观点的强烈影响。

整风运动确实是一场对马列原典的革命,它以教化和强制为双翼,以对俄式马列主义作简化性解释为基本方法,将斯大林主义的核心内容与毛的理论创新,以及中国儒家传统中的道德修养部分互相融合,从而形成了毛的思想革命的基本原则。

从审查人员的交待材料发现「敌人」固然是一种行之有效的方法,但是这种方法也有缺点,既费时又费事,且不能大面积地发现「敌人」。针对这种情况,康生又采取另一谋略,这是暗中布置特工在各单位可疑人群中故意散布「反动言论」,以钓出「反革命」。然而这种方法的效果也不太明显,因为在审干、反奸的紧张气氛中,绝大多数干部都谨言慎行,提高了警惕性,一般不会主动上钩。 

1943年4月3日,中共中央发出第二个「四三决定」,正式号召参加整风的一切同志大胆说话,互相批评,以大民主的方式,来批评领导,揭露错误。此项决定的真正意图在于「引蛇出洞」「暴露敌人」。

在审干、反奸、抢救运动中,假枪毙是一种常见的斗争和惩罚方式。经过种种酷刑拷打,如果被审查者仍拒不交待,这时审查机关负责审讯的干部就会想到利用假枪毙的方式再作一次榨取囗供的努力。选择假枪毙的时间一般在月黑风高之夜,将嫌疑分子五花大绑押往野地,嗖嗖几声枪响,于弹从耳边飞过,给受刑者造成极大的心理与肉体伤害,许多人甚至会长时间精神失常。原中共地下河南省委书记张维桢在中央党校受审期间,就曾被拖出去假枪毙。

在康生和各单位审干小组施行的精神、肉体双重折磨下,大批「特务」被制造出来,人们互相「揭发」,甚至许多夫妻也互相「咬」对方是「特务」。各单位、学校的「日特」”国特“「叛徒」鱼贯上台自首,有的还被树为「坦白」典型,胸佩大红花,骑在马上,风尘仆仆地巡回各地现身说法。1943年夏秋之后,各机关、学校大门紧闭,门口由警卫把守,延安的人们已中断互相往来,「谁也不敢理谁」(王德芬语),在偌大的延安城,也需持介绍信才能办事。入夜,延安万籁俱寂,听不到一点声音,陷入一片恐怖、沉寂之中。

由于华中根据地紧邻沪、宁、杭等大城市,许多知识分子先后投奔新四军,知识分子干部在新四军中的比例要大大高于华北的八路军。

韦君宜当时在绥德,亲耳听到杭大一位副校长介绍抗大的反特斗争原则。这位副校长说:别人说反对逼供信,我们就来个信供逼。我们先「信」,「供」给你听,你不承认,我们就「逼」!韦君宜说,这个副校长后来在文革中「闹得全家惨死」,「我不知道他曾否回想过1943年他自己说的这些话!」

陈彦的一些评论:(个人觉得很中肯,建议看看)
\begin{quotation}
延安整风之后,任何知识,尤其是同人文、社会相关的知识,只要未经过毛泽东意识形态的过滤,就是罪孽,就需要被批判,而掌握这种知识的知识份子就应该接受改造,就需要赎罪。

如果说毛泽东思想的胜利实际上是共产主义舶来意识形态与中国专制传统的双重胜利的话,那么这个胜利就不仅仅意味着「留苏教条派」的出局,同时更意味着五四精神的失败。延安精神的确立,正是五四精神被淘汰的产物。

没有长期的战乱,毛就不可能利用其军事才能压倒众多的知识份子出身的中共前领袖;没有落后的农民国的现实,毛就不可能轻而易举地将其融合中国传统和斯大林极权主义的专制主义强加于40年代初仍然充满理想色彩的中国共产党。
\end{quotation}

萧功秦的评论:
\begin{quotation}
二十世纪是理想主义的世纪,是乌托邦主义焕发出无穷魅力与光环的世纪,也是革命以谁也不知道的逻辑来试图改造人性的世纪,是"建构理性主义"给予人们以新生活的意义,同时又摧毁着人们的诗情梦幻与追求的世纪。

我深为钦佩的是作者在字里行间所显示出来的极高史学悟性与对史料的独到的穿透力。
\end{quotation}

作者高华自已的话:
\begin{quotation}
如果说本书的叙述中有什么价值倾向的话,那就是我至今还深以为然的五四的新价值:民主、自由、独立、社会正义和人道主义。

吾细读历史,站在二十世纪全局观二十年代后中国共产革命之风起云涌,心中自对中共革命抱持一种深切的同情和理解。吾将其看成是二十世纪中国民族解放和社会改造运动的产物,认为在历史上自有其重大正面价值和意义。

有关延安整风期间的中共中央政治局、书记处、中社部、中组部的档案文献,除少量披露外,绝大部分迄今仍未公开。

刘知几云,治史要具史才、史学、史识,其最重要之处就是秉笔直书,「在齐太史简,在晋董狐笔」。
\end{quotation}

\subsubsection{感想}
本书史料翔实,全面细致,不仅把整风运动的来龙去脉说了个底朝天,还透过三十年代以来中共在思想和组织上的蜕变,把这个党的这段历史完整展现出来。

看这本书,觉得中共是一个极权的、红色恐怖的、极端集体主义的政党,在长期残酷的斗争中,矛头不仅对准敌人,更对准同志,甚至对待同志比对待敌人更冷酷。

中共的这种特质,当然和源自列宁主义的组织精神一脉相承,同时又效法斯大林的恐怖式清洗和铁血手腕。另外,中共的个人崇拜、要集体不要个人的气质,和农民泥腿子众多也有关系,使得穷人翻身做领导,同时严厉钳制知识分子的批判意识,拿枪杆子牢牢控制笔头和嘴巴。虽然中共知识分子众多,但缺乏毛那种目空一切的手段,最终被毛牢牢控制。

中共的气质同时更多来自毛泽东个人的诗人气质和帝王手段。毛熟读旧书,胸有点墨,能借势能藏形,能当山大王,也能当精神领袖。毛善用权谋,对死敌孤立围攻,对属下恩威并施胡萝卜加大棒,对心腹留心眼加以牵制,拉一派打一派,掺沙子玩阳谋。毛对科学对现代思想所知不多,拿天下做自己的玩物,对知识分子和“自由主义”打压起来毫不同情毫不手软。

延安整风将中共塑造成毛泽东的党,同时将“对同志的残酷斗争”注入中共的基因,埋下了49年以后历次运动的伏笔。建政后中共将这种高度集权而个人又高度原子化高度孤立的组织方式推向大陆每一寸土地,成为之后文革的样板。文革在整风运动就已经成为必然。

当然,中共的这种蜕变,和二十世纪初共产主义的流行、和中国农业社会缺乏自由主义土壤、和中国传统社会法家手段的流弊不无关系,甚至和中国1895年后逼仄的生存空间和残酷的内战环境紧密相关。整风运动本身就是战争的产物,战争关涉生与死,它不仅仅会把对立的双方变成不共戴天的仇雠,甚至会把仅仅是立场不同的同志变成敌人。在战争机器的动员下,根本没有中间地带,个人必须站队必须宣誓效忠,必须在这个或那个组织的庇护下苟延残喘。经过整风运动,中共的战争动员能力进一步加强,为将来的内战胜利赢得了宝贵的机会。但战争终究会消失,当步入和平后还抱着战争心态不放,会最终在这种紧张的斗争氛围中迎来混乱,文革就是例证。幸好这个党还没有丢失最基本的反省和调整能力,能够在八十年代迎来一次新的蜕变。

整风运动造成中共思想的肃杀,这对热爱自由热爱思想的年轻人是一个莫大的打击。即便不是真正的右派和自由主义者,即便仅仅讲一点人文平等,都要视为思想甚至政治错误。当时投奔延安的知识分子,真是进了地狱。

毛的封神,另外和当时高层的妥协默许纵容以及暗中支持有关。如果这可用理想主义和共产主义事业心来解释的话,那么建政后一波一波卷土重来的新的整风只会让从高层到基层、从政府到知识分子都厌烦反感甚至反对,这也为文革的结束及改革开放埋下伏笔。毛的可悲在于,始终对权力威信紧握不放,置党中央政治局于不顾,拉拢无权无势之人反对身边的既得利益者和潜在对手,甚至一直对“二把手”展开更酷烈的打压摧残,导致追随自己的刘高林始终无法坐大,儿子死在朝鲜战场,四人帮烂泥扶不上墙,华资历太浅,身死之后没有人能继续他的“革命理想”和革命手段,自然也被随后的元老们请下神坛,毛的作为也被打入冷宫。