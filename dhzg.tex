\subsection{《大汗之国:西方眼中的中国》}

作者:史景迁(Spence J.)

\subsubsection{标注}
1. (1910年)傅满洲的诞生,使得邪恶的中国人在国际上有了明确的定位。

2. (美国)在十九世纪末有关中国的小说里,中国人普遍被描绘得富于心机、危险、不可靠、邪恶。

3. 莱布尼茨认为,在追求宇宙万物的知识时,中国也许该扮演主要角色,因为他自忖在协调极端事物上他的理念与中国思想不谋而合,因此,若欲寻找天主教与基督教之间兼容并蓄的中间立场,必得大力仰赖中国人的信仰。也只有这种统合,才有可能使世界臻于和平且协调的新世纪。

4. (18世纪初)当时的主流是,借着亚洲的优点彰显西方社会内在的弊病。

5. 为了补充前述解释,利玛窦创造了一个新的名词“天主”,以免传教士和中国教友使用上帝时造成文化上的混淆。

6. 为了适切翻译基督教一神概念的神(God),利玛窦采取了极富机智而又允执厥中的立场。他认为中国字的“上帝”,意思接近“全能主宰”或“最高统治者”,因此相当适用。这就“上帝”的用法而言,是项新的用法。

7. 过于概括性的言论通常都是偏离事实的,个人经验也很难反映出所谓世界潮流。

8. 西方一般民众对中国的认识,仍然带有殖民心态与说不清道不明的迷思,三分猎奇、三分轻蔑、三分怜悯,还有一分“非我族类”的敌意。

\subsubsection{书评}

如书名所言,本书所要探讨的并非中国,也非中西方交流史,而是西文人如何看待及回应中国这一文明体的简要历史。当然,这是一个大题目,万余字可做出概括,但若要清晰而完整地展现这一文明间的认识,十本书都未必够。本书作者史景迁,善于借助故事的手段来讲述历史,本书即是通过几十位西方人物对中国的认识及展现来试图勾勒出这一认识的历史。

本书由郑培凯作的序已经给出了“西方人眼中的中国”的大体内容:
\begin{quotation}
西方一般民众对中国的认识,仍然带有殖民心态与说不清道不明的迷思,三分猎奇、三分轻蔑、三分怜悯,还有一分“非我族类”的敌意。
\end{quotation}

其实,何止是一般民众,不专门研究中国的知识分子,对中国的了解也十分贫乏。这本书里展现的西方人的刻板印象,即是一种主观的偏激的认识,\emph{过于概括性的言论通常都是偏离事实的,个人经验也很难反映出所谓世界潮流}。

\begin{longtable}{p{0.1\textwidth}|p{0.25\textwidth}|p{0.6\textwidth}}
\caption{不同历史时期西方看待中国的角度}\\
\hline
历史时期 & 人物 & 主要观点 \\
\hline
\endhead

13世纪 & 圣方济会的修士威廉·鲁不鲁乞 & 第一位用欧洲语言来讨论中国 \\
13-14世纪 & 马可·波罗 & 极有可能没有去过中国,其游记的内容缺少实证,但影响深远,比如启发了哥伦布 \\
16世纪 & 葡萄牙船员加莱奥特·佩雷拉 & 被中国流放后写下报告,细节丰富 \\
16世纪 & 多明我会的修士克路士 & 观察精准,有丰富的细节 \\
16世纪 & 葡萄牙冒险家、小说家平托 & 内容全部抄袭自克路士和佩雷拉,恶意丑化中国人 \\
16世纪末 & 利玛窦 & 在中国曾为官,描述准确客观,“天主”一词就是利玛窦发明的 \\
17世纪 & 闵明我 & 借清朝统治者“文明”的行为来抨击教廷,对中国充满溢美之词 \\
17世纪 & 荷兰人奥尔佛特·达帕 & 外交使节,第一位描述北京之行,但主要是主观感受 \\
17世纪 & 耶酥会士皮方济神父 & 第一个以客观方式描述北京 \\
17世纪晚期 & 英国日记体作家约翰·艾弗林 & 介绍来自中国的珍品 \\
17世纪 & 英国小说家笛福 & 以小说《鲁滨逊漂流记》中对中国表现得敌对和歧视 \\
17世纪末 & 莱布尼茨 & 研究易经,认为中国应该在探索世界上有所作为,中国文化使欧洲的宗教问题和谐,将中国与欧洲并列,认为中国应该开放 \\
18世纪 & 孟德斯鸠 & 与巴黎的中国人黄嘉略通信,深入探讨中国文化,认为中国专制 \\
18世纪下半叶 & 伏尔泰 & 研究《赵氏孤儿》,反思中国文化 \\
18世纪下半叶 & 德国学者、历史学家约翰·戈特弗莱·赫尔德 & 认为中国一成不变,僵化落后 \\
18世纪 & 约翰·曼德维尔,莫尔,康帕内拉,孟德斯鸠等 & 当时的主流是借亚洲的优点彰显西方社会内部的弊病 \\
18世纪上半叶 & 俄国彼得大帝的使节伊斯梅洛夫的下属,医师约翰·贝尔 & 观察细致,评论正面 \\
18世纪中叶 & 英国贵族乔治·安生 & 自信好战,喜欢欺负弱小,贬损中国 \\
18世纪 & 戈德史密斯 & 评论《赵氏孤儿》,讨论中国哲学,讽刺当时的中国热 \\
18世纪 & 贵族审美家赫勒斯·活波尔 & 将当时奢华的中国风发挥到极致 \\
18世纪末 & 马戛尔尼 & 对中国有好感,但访问中国后知道两国贸易已不可能 \\ 
19世纪初 & 简·奥斯汀 & 根据马戛尔尼和亲戚访问中国的内容,对中国有浮光掠影的评论 \\
19世纪中叶 & 美国人埃尔萨·简·吉莉 & 长住中国,直接接触中国社会,险些被暴民伤害 \\
19世纪中叶 & 简·艾德金斯 & 在上海生活 \\
19世纪70年代 & 玛丽·克劳馥·弗雷泽 & 在中国生活,对中国印象不好 \\
10世纪00年代 & 美国传教士之妻莎拉·康格 & 在中国生活 \\
20世纪10年代 & 卡夫卡 & 通过小说《中国长城》来探讨皇帝与权威 \\
20世纪10年代 & 博尔赫斯 & 通过短篇小说《曲径分岔的花园》探讨迷宫、可能性和选择 \\
20世纪70年代 & 卡尔维诺 & 通过小说《如果在冬夜,一个旅人》中忽必烈和马可·波罗的互动来解释“说故事的艺术出现无限多的变化” \\
\hline
\end{longtable}

第七-九章也简单介绍了中国人在美国的情况、法国和美国知识分子如何看待中国风情等。相比于刻薄的英国人,法国人和美国人对中国的看法更加全面,但由于对真正的中国社会缺少认真观察和研究,很多(特别是法国人)虽然有实地考察,但念念不忘的仍是欧洲和法国的问题。因此,他们最多只能攻其一点,难以对中国的全貌有全面和深刻的认识。当然,当中国在20世纪的革命中展现出惊人的力量时,世界都惊呆了,虽然在很多西方人眼中,中国的共产主义运动只不过表明这个文明的专制属性根本没有变化。

世界仍在变化,而中国的变化尤甚。现在21世纪已经过去了将近20年,中国的改革开放使得中国的经济和文化实力有了快速的提高。到了今天,世界已经不可能再像近代以来蔑视和贬低中国文明,中国很可能重回马可波罗或利玛窦时代的巅峰地位。如果西方的知识分子仍像以前那样不认真研究和了解中国,那么他们也就太自大了。不过,我不相信这样的情况。随着中国的崛起,中国文化必将成为世界的潮流,汉学必将成为世界的显学。

本书的内容并没有那么丰富,但广度和跨度较大,更多地是会带给读者一种思考和反省:为什么这些西方人眼里的中国会如此多样?是中国真的如他们所言,还是欧洲在现代化的过程里自身视角的变动呢?

评分:3/5。