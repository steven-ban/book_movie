\subsection{《从公司到国家:美国制度困局的历史解释》}

标签:美国历史 \ 美国宪法 \ 美国政治制度

作者:范勇鹏

本书是复旦大学中国研究院的副院长范勇鹏老师的新作。在观视频上看过范老师的一些讲解,其人风度翩翩,湿润如玉,讲话节奏很好,娓娓道来,而且简洁没有废话,学术性较强,于是买了他的著作来看。

本书是对美国政治制度和美国历史的祛魅之作。长久以来,美国人在中国人的印象中是偏正面的,自由主义者就更不要说了。虽然美国历史上有过残酷的对印第安人的迫害、欺压压迫他国的历史,但往往都被一笔带过,大家对美国新教徒远涉重洋寻找自由、反抗英国压迫、建立人类历史上最为“先进”的民主制度、华盛顿高风亮节拒绝再次连任归乡、林肯“解放黑奴”、美国黑人的独立运动等津津乐道,认为美国虽然不是完美的,但也是各国中政治制度最为先进和民主的。近代以来中国知识分子和政治精英也多推崇美国,孙中山即是大量吸收了美国的民主共和制来设计民国的政治制度。本书的目的,即是从美国如何建立、如何确定政治制度的内容入手,来对这些观点进行反驳。

本书主要观点如下:
\begin{itemize*}
    \item 美国具有优越的地理环境,建国更多是基于利益而非恐惧(国家产生的两个理由)。早期的移民更多是业主、股东、劳工、黑奴的商业行为而非为了“自由”,《独立宣言》的签署者有2/3是英国国教徒。清教徒更多是追求自由而非民主,是追求利益的实用主义者。南美西班牙殖民者往往是男性,倾向于当地的印第安人和黑人混血,而北美是举家前来,混血少,种族差异大。
    \item 代议制产生的三个因素:日耳曼原始自由制度(部落性)、中世纪封建制度(封建性)与近代资本主义及其中央集权(集权性)努力之间的复杂互动。代议制本身并不等于民主制。对于欧洲而言,自由是古老的,(君主)专制才是近代之事。欧洲中世纪存在着原始自由制度、封建制与王权之间的斗争,王权专制是整体趋势。英国中世纪存在盎格鲁-撒克逊人的野蛮自由制度和诺曼人的封建制度。英国大宪章是贵族限制王权,随后存在着王权、大贵族、小贵族、地产所有者之间的相互博弈、联合与分裂。法国大贵族过于强大,反而导致王权与下层联盟压制大贵族,建立君主专制。西班牙的蛮族征服者接受罗马遗产,建立起较强的王权,摩尔人的征服激活了日耳曼蛮族的原始自由制度,产生了与英国不同的代议制萌芽。代议制的产生不是为了限制王权,相反是中世纪末国家集权过程的产物。美国早期存在原始自由、封建领地、商业公司和王室集权的因素共同决定了美国代议制的特征。早期殖民地分为公司殖民地(存在议会,弗吉尼亚、马萨诸塞)、业主殖民地(类似于封建领地,马里兰、卡罗来纳、纽约、新泽西、宾夕法尼亚)和契约殖民地(又称自治殖民地,基于契约,普利茅斯、罗得岛、纽黑文、康涅狄格,规模较小,后一部分纳入王室)。
    \item 罗马征服城市后军队退回,市政治理权归市民,政治权交给罗马中央。威尼斯(与大陆隔绝,易守难攻,易于贸易,海军强大,商人众多,选举大公由大贵族世袭,大部分居民不能参与大议会,贵族集体讨论代替大公权威,商法发达且成熟,商人掌握权力后把自己变成贵族)、荷兰(低地国家,商人和金融阶级用财富把持官职,共和联邦,缺少中央集权)、英国(商人与尼德兰的金融行业多有关联,“光荣革命”是为了让荷兰人移民过来,1832年之后成为虚君制和议会至上,资本进入议会)历史中存在着商业公司因素,商人统治的性质带来了现代共和制、权力分立和议会制度,英国“光荣革命”后的宪制是集大成者。
    \item 美国独立不是为了反抗暴政(税不高),而是为了有产阶级的利益。美国的成立是基于商业契约的公司国家,其制度必然带来有限责任政府。一旦独立,则具有主权,主权不可分割,因此各州失去自己的主权,建立邦联,没有三权分立,没有独立行政和司法部门,各州独立防务,无法制订统一的对外贸易政策,各自发生纸币,产生建立国家的需要。制宪会议是各种利益的妥协,无法达到完全一致,最后的联邦政府仅与州际利益有关(征税权,外事权,战争权,州际贸易管理权)
    \item 政治学里有人将国家视为人格(霍布斯),把英国王室当成独立法人。美国宪法完全就是公司契约或者合同的写法。第一条强调“本宪法授予(herein granted)的全部立法权”归国会(不是全部立法权),未列举的立法权归各州。第二条规定“行政权被授予总统(The executive power shall be vested in a President)”,没有任何限定,成为之后总统扩权的法理依据,但整体上对总统的行政权不是很重视,其权力受到国会的制约,也没有明确规定内阁部长与总统的隶属关系(国会和总统争夺行政体系),还没有规定全国性的行政官僚机构。美国宪法制度存在权威不足、三权掣肘和行政官僚制度混乱软弱的先天性问题。对司法权更不重视。美国政治制度下权威的来源不是什么君主的尊荣、鲜血(权威是一种习惯,是权力的最后归结),没有得到普遍承认(不是所有人都要独立),因此只能不断神化立宪过程和宪法,建立一种法人、集体主权,多个法人合并导致双层主权。美国是一个“有限责任公司”,最终归责不明确,导致推脱。
    \item 一般而言,总统制通常是由选举分别产生立法和行政机关,总统既是国家元首又是政府首脑(但行政权并不完全在总统手中),行政机关与立法机关之间是分立和制衡的关系,接近于经理管理模式。议会制又称内阁制,由选举产生立法机关,议会选举中获胜的党组建内阁,立法权与行政权不分立,行政权紧密融合于立法权,议会不会过多干预行政权,司法权也很少骚扰行政机关,减少了权力冲突。但这种分法并不绝对。美国早期是上级任命官员,19世纪末期才在理性主义、专业主义和西进运动(西部需要治水)下建立基于考试和中立的官僚体系。国会和总统任命两套功能相近的班子,国会过度关注程序,喜欢诉讼形成了律师利益群体。
\end{itemize*}

总之,美国的政治制度的产生是一个历史事件,要放在特殊的历史地理情境中考察,绝非美国人和很多自由主义者口中的“人类灯塔”“山巅之城”。美国的“公司国家”色彩是资本主义制度的反映,虽然有一些我们值得借鉴的地方,但也有不少缺点,这些缺点拖了两百多年很多还没有解决,我们没有必要去神化美国的政治制度。

当然,本书限于阅读对象和篇幅,很多地方是观点和事实的罗列,没有严密逻辑的支撑,不是教科书那种严谨。不过,读这种书最重要的仍然是打开视野,通过作者观点、作者引用的其他学者的观点和事实,作为进一步了解美国政治制度的跳板。

评分:5/5。