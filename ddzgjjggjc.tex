\subsection{《当代中国经济改革教程》}
作者:吴敬琏

这是一本\emph{一本运用现代经济学的分析工具考察中国改革的著作}。

标注:
\begin{itemize*}
	\item 匈牙利经济学家科尔奈早就论证过,计划经济是一种短缺经济。也就是说,它的常态是总需求远大于总供给。不过,由于在计划经济体制下绝大部分商品实行固定价格制度,需求过旺和供给不足通常并不表现为价格上涨,而是在行政压制下隐性地存在着,并以配给制度和额外的寻求成本等形式表现出来。
	\item 从短期的观点看,在1978年末改革开放以来30年的时间里,中国宏观经济引人注目的特征是反复出现经济过热和通货膨胀。
	\item 凯恩斯主义宏观经济政策的主张是,当出现总需求不足和经济萧条的情况时,用扩张性的宏观经济政策增加总需求来加以救助;反之,当出现了总需求过旺和经济过热的情况时,则主张用紧缩性的宏观经济政策减少总需求来加以抑制。
	\item 一位卫生部前副部长指出,中国政府投入的医疗费用中,80%用于党政干部。
	\item 在计划经济条件下,传统社会主义国家社会保障的典型特征是政府直接向社会成员提供实物福利,而不是像在市场经济条件下那样,主要是建立针对某些风险的福利维持计划。
	\item WTO 规定(1)缔约一方给予任何一方的优惠,也给予所有缔约方;(2)缔约方之间相互保证给予对方的自然人、法人和商船与本国自然人、法人、商船相同的待遇(国民待遇);(3)缔约一方不得对任何缔约方实施歧视性待遇,要使所有缔约方能在同样的条件下进行贸易;(4)关税是WTO所认可的唯一合法保护方式,缔约各国应不断降低关税的总体水平;等等。
	\item 为什么进口替代的工业化不能像预期的那样发挥作用呢?根据克鲁格曼的分析,最重要的原因是:发展中国家制造业发展程度低下,通常是由于缺乏熟练的劳动力,企业家、管理人才和社会组织方面存在问题等多方面的原因造成的;贸易保护政策非但不能为这些国家的制造业创造出竞争力,相反会使这些部门和企业效率下降。而且,进口替代战略还会因为给予少数受到保护的精英企业获得垄断利润的特权,而使二元经济及收入分配不均和失业等问题加剧。
	\item 增值税的转型改革主要围绕两个方面进行:一是由生产型增值税转为消费型增值税,二是扩大增值税覆盖范围。
	\item 20世纪90年代初期,开发预算外收入已成为从中央到地方的普遍现象。到1996年末,中央政府明文规定的收费项目有130多项。经过地方政府和主管部门的层层加码,到县和县级市一级的不完全统计,各种收费项目已达1000项以上,各种基金420项以上。1992年预算外资金达到3854.92亿元,为当年全国财政收入的110.67%。
	\item 增值税不同于所得税,它是一种间接税,它的纳税人并不是税负的最终承担者。增值税的特点是对产品在其生产过程每个阶段的价值增值征税。
	\item 在各种税种中,将维护国家权益、实施宏观调控所必需的税种划为中央税;将同经济发展直接相关的主要税种划为中央与地方共享税;将适合地方征管的税种划为地方税。
	\item 美国财政学家奥茨(Wallace E. Oates)指出,财政联邦制在具备如下三个前提条件时,明显优于其他分权体制:(1)当地方政府提供该产品的成本小于中央政府,且不存在负的外部性的时候;(2)采取财政分权制度提供某种公共物品带来的福利增进与该物品的需求弹性成反比;(3)这种福利增进也与居民的迁移性成正比。
	\item 亚当·斯密在《国富论》中提出的税收四原则,即平等原则、确定原则、便利原则和经济原则,是对税收原则最系统和完整的表述。
	\item 在市场经济条件下,公共财政的基本职能,就是为政府向社会提供公共物品筹措和分配资金。
	\item 投机对于市场的有效运作有它不可或缺的作用。【216】这是因为,如果金融市场上只有长线投资者,市场就没有流动性,价格也不能被发现。投机者寻求风险收益使交易得以连续进行。但是问题在于,投机活动只有在与投资等活动结合在一起,实现良性互动时,才能够透过证券市场实现资本资源的优化配置,从而对经济起积极作用。单纯的投机炒作并不能提高效率和增加财富,只不过是“钞票在不同人的口袋之间搬家”的零和博弈【217】,
	\item 股市合规性监管最重要的职能,就在于通过强制性的信息披露制度,缓解信息不对称的问题,保护投资者的利益不受侵犯。
	\item 现代经济学认为,股票市场的基本功能是通过股市交易和股价变动,使资本资源流向效率较高的地方,实现资本资源的优化配置;与此同时,利用股价对公司绩效的度量作用,对公司经营作出评价并对经理人员进行监督。
	\item 作为资本市场分析基准(Benchmark)的MM定理【208】指出,在满足信息完全对称等一系列假设的条件下,企业的市场价值与它的融资结构无关。但在现实经济生活中的信息不对称条件下,不同的融资方式对企业市场价值的影响是不同的。
	\item 中国经济改革的第二个阶段以增量改革为基本特征。所谓增量,在很大程度上是指在原有国民经济中逐步增添新的非国有的经济成分。
	\item 作为公共企业,国有企业不管是否上市,其透明度都应该达到上市公司的水平,国有企业巨额利润不经财政预算程序而自动转为投资资金,是近年来中国经济增长过度依赖投资而消费增长乏力的一个重要原因。
	\item 1998年以后,中国政府采取了分拆改组的办法来打破垄断,形成竞争局面。
	\item 市场经济国家更普遍采用的个体农户与市场的对接形式,其实是在农业产前、产中、产后服务的各个环节上建立农民合作组织。
	\item 农业生产资料价格的上升幅度却远远超过农产品价格的上升水平。只有农业部门的剩余劳动力都被工业部门所吸收,农业部门劳动者的工资才会提高,整个经济才会转入现代增长。
	\item 制度变革本质上就应该是整体推进的,虽然在实施上可以分步进行,否则就会存在巨大的制度运行成本。
	\item 由于计划经济的资源配置方式在本质上要求集权,分权的计划经济是较之集权的计划经济还要糟的计划经济。
	\item 直到1976年“文化大革命”宣告结束,由于存在社会主义只能采取行政命令配置资源这样的意识形态障碍,市场取向改革很难在政治上被接受,向地方政府下放计划权力几乎成了唯一可能的改革选择。(1)1958~1978年:行政性分权,改革的重点是中央政府向下属各级政府放权让利。(2)1979~1993年:增量改革,改革主要在国有部门以外的经济领域中推进,并以民营经济的成长壮大来支持和带动整个国民经济的发展。(3)1994年至今:整体推进,以建立市场经济体系为目标进行全面改革。
	\item 在马克思和恩格斯看来,社会主义之所以会取代资本主义,源于资本主义社会中生产力和生产关系之间,即社会化的生产力和资本主义私人占有制度之间的冲突。
\end{itemize*}