\subsection{《剪刀男》}

标签: 日本 \ 推理

作者:【日】殊能将之

\subsubsection{人物}

\begin{longtable}{p{0.12\textwidth} | p{0.3\textwidth} | p{0.5\textwidth}}

    \caption{《剪刀男》人物表} \\
    \hline
姓名 & 特点 & 事件 \\
\hline
\endfirsthead

(接上表) \\
姓名 & 特点 & 事件 \\
\hline
\endhead

\hline
\endfoot

小西美菜 & 埼玉县 & 一年前遇难,当时高一,成绩很好 \\
松原雅世 & 江户川区 & 遇难者,死时嘴巴周围的脸被割掉 \\
樽宫由纪子 & 目黑区鹰番,高中女生 & 成绩很好,但同校传言她私生活淫乱,与多个年长男性交往,对家人冷淡 \\
樽宫一弘 & 樽宫由纪子的父亲 & 非亲生,前妻过世 \\
樽宫敏惠 & 由纪子的亲生母亲 & 之前离婚 \\
樽宫健三郎 & 由纪子的哥哥 & 非亲哥,是樽宫一弘亲生的 \\
“我”(安永知夏) & 冰室川出版社实习 & 胖,喜欢吃,一直尝试在周六晚上自杀,但都不成功,跟踪并杀掉那些成绩好的女生 \\
日高光一 & 目击证人 & 胖子,喜欢吃,对剪刀男事件感兴趣,曾经跟踪死者 \\
医生 & 约60岁 & “我”总是和他对话 \\
冈岛部长 & “我”的上司,50岁女性 & 对“我”比较关心 \\
矶部龙彦 & 目黑西署刑事科 & 调查中喜欢上安永知夏 \\
下川宗夫 & 矶部龙彦的同事 & \\
堀之内 & 高级警官,擅长心理分析 & FBI学习过 \\
椿田亚矢子 & 由纪子的同学兼密友 & 经常和由纪子一起 \\
松元 & 警察 & 善于观察人的心理 \\
岩左邦马(Kunima) & 由纪子的体育老师,与由纪子交往 & \\ 
近藤诚斗 & 警察 & 擅长拍照 \\
村木晴彦 & 警察 & \\
堀之内靖治 & 上级警察,犯罪心理分析官(marusai) & FBI学习过,指使手下警察去调查,自己每天都在警察局,调查过之前的剪刀男案件 \\
黑梅夏绘 & 周刊arukana的自由记者 & \\
\end{longtable}

\subsubsection{事件}
\begin{itemize*}
    \item “我”之前已经完成两起命案,小西美菜和松原雅世都是成绩不错的高中女生,被“我”杀掉,而松原雅世更是因为“我”想探究她英语发言为什么好而把脸割掉。医生其实是“我”构想出来的,长相和父亲一样。“我”第一次杀人时剪刀不锋利,之后两起便都把剪刀磨得十分锋利。
    \item “我”伪装成快递员蹲点监视樽宫由纪子家,之后车站尾随她,
    \item 由纪子经常和一个男人(不是她的父亲,是堀之内)出入车站附近的商店,十分亲密。
    \item 准备充分,十一月十一日这天,将要去杀她,她却一直不来。九点以后,“我”放弃,往回走时发现她在公园里被用和“我”杀害前两位的相同的手法(脖子上通过塑胶绳而勒死,然后用剪刀刺入喉咙)杀掉了,我捡了尸体旁边一个金属质打火机,上面写着“K”。
    \item “我”开始调查与死者有关的较为亲近的人
    \item 当地刑警也开始调查。当天就盘问了目击者:一个胖子,还有一个“女性目击者”。堀之内对那把额外的剪刀很感兴趣,他在开内部会时不小心说出死者“是因为弓箭社的练习才那么晚放学的”,这引起其他警察的怀疑。
    \item 葬礼上,樽宫敏惠表现得比较镇定,而樽宫一弘则比较悲伤,健三郎突然激动地跑掉。“我”去葬礼上观察,而矶部也去了葬礼上观察。(这里有暗示,“我”是在亚矢子之后上去拈香的,穿的是黑色西装。矶部:不知道为什么身穿丧服的女性总是特别漂亮?他注意到了胖胖的目击者,他穿关羽毛外套)
    \item “我”开始调查死者周围的人:她的闺蜜、体育老师、弟弟和母亲。随着调查的深入,“我”和警方都获知了死者的情况:看上去成绩好,乖乖女,但是其实内心缺爱,以自己的年轻和美貌去戏弄那些看上自己的男人。
    \item 堀之内冒着大雨30分钟内浑身淋透赶到警察局讨论案情,受到下川等人的怀疑,于是他们开始拍他照片并去寻找目击者。调查的重点被堀之内导向剪刀男的目击者而非被害人的交往对象。有目击证人证实堀之内曾经与被害人一起,并且后来还去了命案现场的公园。
    \item 日高光一找到“我”,并把“我”叫到他的出租屋内,趁他看打火机的时候偷袭他,把他绑起来,讯问得知他不是凶手(之前“我”推论他是真凶),依然把他杀了。堀之内跟踪过来,他之前也不知道剪刀男的真实身份,也认为剪刀男是日高光一。“医生”指出他从堀之内犯案时就推断出来是伪装犯罪,只有堀之内才知道剪刀男的真实做案手法(勒死后扎上剪刀,但不性侵),并且为了掩盖反面把尸体摆放得整齐,尽力模仿剪刀男不性侵的手法。之前“我”把另一把剪刀的事件在媒体上公开,引诱堀之内前来,实现了“医生”找到真凶的目的。堀之内杀掉由纪子的原因,是他受到了后者的耍弄,以为她真的要和自己好,已经怀孕。
    \item 堀之内要杀掉“我”,矶部赶来,“我”撞向堀之内的枪口,被打中腹部,堀之内开枪自杀。这样堀之内就有了三个命案的嫌疑(之前警方已经怀疑他杀死了由纪子)。
    \item 矶部去看望医院里的安永知夏,后者对他隐瞒了日高光一被杀的真相,“我”顺利脱罪。
\end{itemize*}

\subsubsection{书评}

本书实际上有两个故事:一个是“我”作为之前的剪刀杀人事件的凶手的事故,一个是以矶部为视角的警察破案的故事。这两个视角以章节的形式交替进行,典型的POV式写法。本书的杀人诡计很简单,叙述诡计则很厉害,主要是围绕着“谁是凶手”展开,不仔细看伏笔很难发现,并且在整个破案的过程里,放了大量的烟雾弹,给了读者很多干扰。“我”经常见到“医生”,而这是“我”根据父亲的形象而虚构出来的,说明“我”内心活动极为丰富,甚至有双重人格。不过准确地说,“医生”才是“我”的真实人格,是压抑的自我,而“我”才是“医生”的妄想型人格。不过,这个诡计的成立,需要很大的巧合(越巧妙的诡计所需要的巧合越多,客观上的可能性就越低):堀之内、日高光一和“我”同时出现在堀之内与死者在速食店的约会里;堀之内杀害死者后“我”和日高光一恰好不久就发现尸体。

这本书写得十分细密,不仅有惯常的推理,也会向警察等人物中加入一些职业心理,比如对上级的恭敬、对同行的妒忌等等。对于死者生前的心理状态,本书也是着力描写,死者在进入继父的新家庭后,与其他人无法打开心扉,对母亲不是爱,也不是恨,而是“漠不关心”。这本书中有一个叙述诡计:“我”是剪刀男不假,可“我”的真实身份呢?作者没有刻意说出“我”的身份,读者受“剪刀男”和警方对日高光一的调查(这段写得十分精彩,坑挖得太好了,烟雾弹放得太足了)的暗示,会认为“我”就是男胖子日高光一,但其实之前已经有伏笔这完全是两个人。“我”吃了老鼠药发现马桶里有红色,以为自己“出血”了,这也算一种暗示吧。这里有一个细节,“我”上吊都没死,一直觉得自己“胖”,而那个胖胖的目击证人明显形成了一种误导,让读者认为两者是同一个人。

各路人士对剪刀男的心理分析可谓可笑。他们从精神分析的角度上来揣测剪刀男是一个残暴的、性无能的御宅族,与“我”的真实想法——仅仅是想一探究竟——完全不同。这可能是作者借“我”之口发出的议论,还暗示本作并非十分关注罪犯的社会动因,杀人仅仅就是想杀人而已:
\begin{quotation}
(“我”说)杀人当然没有关系……想杀人就杀,想和很多男人上床就上床,不想和家人说话就不要说,想和没有血缘关系的姐姐上床就上床,就是这么简单。如果只是因为想做而不能做,或是想做却被阻止,而感到痛苦或是背着众人偷偷摸摸地做,就是愚蠢的人才有的行为。想做就去做,只要自己愿意负责。

(医生说)人之所以不杀人,只是因为一些微不足道的原因,像是亲眼目睹死亡时的不快感、血液的味道令人恶心、触摸到尸体时的阴森感等等的琐碎理由,而不是什么伦理道德。这样的观念,只要简单的一个翻转,就会引起错乱的喜悦。因为违反禁忌,所以愉悦;因为脱离常轨,所以喜悦;因为错乱,而以为自己是特殊的存在。
\end{quotation}

这当然反映了某种脱离社会规范的心理,这种心理和很多犯罪动机必是有原因不同,是“反社会派”的。不过,本书的现代小说技法很全面,整体上节奏不错(后期节奏过快,中间烟雾弹放的有点多,比如刑警拿照片找人,“我”调查死者的亲友等),涉及精神分析(医生和“我”之间的妄想型人格的关系),还有诗歌、音乐、经典推理小说的点缀等等,使得本书不仅作为一部普通的推理小说而言不错,即使是放入普通小说的范畴也极为优秀。

评分:4/5。