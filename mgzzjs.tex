\subsection{《美国种族简史》}
使文化区别成为争论不休的问题,究竟对文化存留或种族进步是否有推动作用,这绝没有清晰的答案。

他把对个人的歧视与对一个种族的歧视区分开来了。

光顾华人餐馆的显然不都是中国人。

迁居美国是一个带有选择性的过程,前来的是自己祖国人口中那些有雄心或有本事的分子。

怎么可能用同一个尺度去衡量追求不同目标的人们的进步呢?进一步说,怎么可以要别人(即“社会”)对这些以一把尺子衡量出来的差别担负起因果关系或道义方面的责任呢?

我们并不沉湎于过去之中,但过去却纠缠着我们不放。”

每个人都生在世代相传的特定文化模式之中,这种文化模式有着深远的影响,

美籍墨西哥人参加投票时,主要投民主党的票,

新来的墨西哥移民还会遭到其他美籍墨西哥人的冷落,后者出生于很早以前就来到美国的墨西哥裔家族。这些人常常自称为“西班牙人”或“拉丁人”,以避免混同于新来的墨西哥移民。一如其他种族的中产阶级看待新来的同胞一样,他们把自己国家的移民视为肮脏、无知和缺乏教养之辈。这一方面反映了他们当中普遍存在的一种恐惧心理,即新来的移民会不利于美国社会接受整个墨西哥种族,另一方面也反映了墨西哥本国刻板的等级制度。

无论节制生育对那些为摆脱贫困而辛勤奋斗的个人来说是何等有利,但对于整个波多黎各种族来说,节制生育意味着他们当中最成功的人士给下一代留下的知识、财富和社交关系是较少的,因为下一代绝大多数都出生在父母并无这些东西可遗留的家庭。

地位显赫或引人注目的黑人,迄今一直大多出身于黑人精英世家,或出身于西印度群岛人家族。

奴隶制在南方实行了两个世纪之久,黑人惯于被认为是毫无权利的。所以南北战争之后,南方白人不仅对黑人获得解放感到愤懑,而且对黑人在言行态度方面有任何表现,显示他们与白人一样也是人,或和白人有共同权利的迹象时,都不能接受。

由于其主动精神世世代代一直受到压抑,又由于他们缺乏激励,只想把工作做到仅足以不受惩罚的程度,奴隶们养成了磨蹭和逃避工作的习惯,这种习惯在奴隶制本身消失之后很久仍然在他们身上存在着。

几个世纪以来,有近1000万的非洲奴隶被运到了西半球。

日本对西方的感情很复杂,既憎恶其傲慢,又欣赏其赖以称雄的成就。

华人今天的成功,基本上取决于这样一个简单的事实:他们比别人多干活,并受过更多(也更良好)的教育。

在中国文化中,妇女的作用历来都是从属的,至少对外是如此。但是,即便是在中国本土,中国男子也博得了世界上最怕老婆的名声。在美籍华人当中,单是男女比例失调这个现状就足以使女人易于占上风。尽管华人丈夫对外是一家之主,但是妻子决定家中大事的现象很普遍,而且不限于家务事。

当时曾出现过一个成语,叫做“未必属中国佬的机会”,意思就是说,事情无望了。

美国工会长期不遗余力地站在排斥华人移民的前列,想把华裔居民赶出美国。

爱尔兰人在政界、金融、工会领导、体育和新闻方面一马当先,而犹太人则在经商、技术行业、学术和科学方面独占鳌头。