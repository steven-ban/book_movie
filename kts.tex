\subsection{《开膛史》}

标签: 历史 \ 手术 \  医学

作者:苏上豪,台湾高雄人,现为台北市博仁综合医院心脏血管外科主任,在网上发表医学科普文章多篇,也写小说。著有《国姓爷的宝藏》

《开膛史》封面:\url{http://cover.read.duokan.com/mfsv2/download/s010/p01MuWnWmvYU/00deWcYGYH5ETD.jpg}

看名称《开膛史》以为这是一本兴味盎然的外科手术简史,看了1/3才发现这只是一个外科医生的随笔,历史上的考据并不严谨,内容也不够丰富,东一下西一下闲侃外科手术的变迁历史,因此更像一部故事集。名称改为《外科手术杂谈》更合适。

人类的外科手术受人类对科学、解剖学、生物学认识深度的支配。古希腊时代即有麻醉和外科手术,但十分简陋;中世纪长期受教廷思想的桎棝,外科手术停滞不前;随着启蒙运动和科学革命的兴起,外科手术才和其他科学技术手段一样发生了巨大的变革和进步。关于外科手术的历史,还有一个冷知识:中世纪时很多外科手术是理发师做的,因此现在的理发店门口的红白相间的柱子,红色代表血管,白色代表止血带,柱子代表止血时病人握着的柱子。这个说法没有考证是否正确,但确实挺有意思的。

对于中医,作者认为,中医外科的没落是由于其\emph{内科化}的原因。

原来作者所说的\emph{大体},是尸体的意思。

普通人认为外科手术神秘甚至恐怖,因为它关乎性命。但作者身为外科医生,对外科手术却有着调侃的味道,手术在他看来只不过是外科医生的小戏法而已,真正神秘的人体本身。外科的历史充满了冒险、幸运和偶然,并非人们通常认为的那种“高大上”和严谨。

医患矛盾并非大陆独有,台湾也有,甚至美国加拿大也经常出现医院暴力行为。另外,作者在本书中讲到,台湾的医保的保费低,导致现在很多人养成了心理习惯,稍微出点手术费就不情愿,甚至大闹医院。医保的钱是大家的钱,你不出他就要出,保费过低长期演进恐怕会导致崩盘。另外,台湾的产科在萎缩,医疗人员减少,这和出生率低有很大原因。

启示:人们应当相信外科医生,尊重科学和专业人士;同时,人们也要把自己接受医疗手段的真实情况(如不良反应等)与医生充分沟通,因此很多药物反应因人而异。

评分:6/10。