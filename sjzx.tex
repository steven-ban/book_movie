\subsection{《世界秩序》}

作者:【美】基辛格

在这本书中,基辛格以美国人的视角来解读世界局势,其核心价值观是以美国的国家利益为最高目的,维持欧洲、中东和亚洲的“均势”,保护当地的美国利益,通过自身直接的经济、文化和军事干预,以及盟友的作用,相辅相成打压反美力量。这和十九世纪英国在欧洲通过多方不断结盟的方式来维持“离岸均势”一脉相承。

一个地区一旦出现了“权力真空”,局势就可能失控,美国稳定的利益就会受损,此时美国就会出手,作为务实的国家,美国毫不避讳武力打击和经济制裁。

美国是个清教徒国家,宗教色彩浓厚,喜欢推销自己的文化。以前是上帝,现在是美式民主。这种没来由的“热心肠”在中东和亚洲屡屡碰壁。同时,美国的外交是务实的甚至是赤裸裸的,作为世界老大和自封的“世界警察”吃相难看,而且和“自由主义国际政治观”常常抵触。
基辛格对中国的看法在本书中由于篇幅限制而内容单薄,感兴趣的读者可以看看他的《论中国》。

整体来看,基辛格对中国的了解在西方一群政客里还算全面,但对于中国的民族心理缺乏深入了解。

由于基辛格早就不从政了,对美国现在的外交工作影响没那么大,特朗普的“美国优先”和重回孤立主义明显和基辛格不合。

总之这本书值得一看,但私货太多。作为一本了解美国人对世界局势的看法的书值得一读,但私货比较多,读者应该保留自己对欧洲、中东和亚洲的看法,不必受作者的观点所左右。

美国人一直以来认为天降正义,自己的政治制度是最文明最先进的,怀有天真浪漫的心思去传播自由民主的价值观,这是美国外交的底色和“理论基础”。

同时,基于现实主义的考虑,美国必须维持自己的国家安全。由于美国地处北美,没有强敌环伺,整个十九世纪都处于孤立境地,信奉孤立主义,除了本土战争,基本不插手欧洲和亚洲局势。

随着十九世纪末美西战争结束,美国的领土扩张走向极限,不得不面对欧洲长期以来相互制衡和结盟的均势政策。威尔逊这样的学者型总统在一战派兵就不积极,认为美国应当怀揣理想和正义。

但现实主义者如西奥多·罗斯福就认为美国不能老是事不关己,应当有所作为甚至主导局势来维护本国利益。这在二战及之后的冷战越来越重要,压过了理想主义外交或价值观外交政策。随着美国国力日盛,现实主义者认为美国应当越来越多地“合理”利用自己的力量,维持区域的均势,这是欧洲实用主义外交的延伸,没什么理想主义的成分。

朝鲜战争和越南战争暴露出美国外交的这个矛盾:(1)价值观外交很难实施,特别是在亚洲和中东,民主的政治结构在缺少中产阶级的社会极不稳定,更不要说很多国家的局势不稳、宗教干涉甚至领导世俗政治、民族多样等问题了,因此价值观外交的后果往往是越来越乱,粗放型民主政治导致的腐败和动乱甚至还不如以前,缺少正义感;(2)为了维护国家利益而和非民主国家结盟、开战导致的人道主义危机愈演愈烈,国内一片反对,本身就缺乏正义性。

20世纪美国的外交就在这两极之间反复摇摆,到了21世纪也没见统一的迹象。这和当初英国的殖民地政策类似,短视缺少体系的政策导致美国独立时英国束手无策(见《英帝国的崩溃与美国的诞生》 )。我觉得这是盎格鲁撒克逊人的通病,缺少大陆国家的体系化和统一化,灵活有余而系统化不足,经验有余而理论不足。

显然,理性的外交应当基于现实并合理估计结果,不能一厢情愿。美国应当更加务实,不要老拿政治制度来说事。至于把现实主义的操作包装成为了他国民主和人民自由,估计也就哄哄傻子了。

西方理想主义者有一个很荒谬的论断:得到被统治者认可的政府不会向邻国开战(罗纳德·里根)。

那么我想问问里根:黑人和女人是被统治者吗?你们在赏给黑人和女人投票权之前没有打过仗吗?打仗需要全体国民的同意吗?你们打朝鲜战争、越南战争、两次伊拉克战争、阿富汗战争前没有被国会授权吗?

另外,两次世界大战是非民主国家打的吗?希特勒不是民选的吗?

再一次证实,盎格鲁撒克逊人的理论严密性简直是个竹篮,全是窟窿。

现在随着美国带着走反全球化道路,世界各国会越来越封闭,全球化趋势受到很大冲击,国家间的利益冲突会越来越多地摆到桌面上。在这样的趋势下,个人的民族主义立场会越来越明显,而白左和世界主义那一套会受到更多人唾弃。
