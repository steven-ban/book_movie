\subsection{海洋深处}

标签: 白鲸记 \ 航海  \  捕鲸 \ 海难 \  漂流 \ 人性

英文名:In the Heart of the Sea

电影海报:\url{https://gss3.bdstatic.com/-Po3dSag_xI4khGkpoWK1HF6hhy/baike/c0\%3Dbaike180\%2C5\%2C5\%2C180\\2C60/sign=509333fef41fbe090853cb460a096756/cb8065380cd79123bc8598fcab345982b3b780d4.jpg}

\subsubsection{情节}
本片改编自美国作家赫尔曼·梅尔维尔的《白鲸记》,讲述了在19世纪初年美国远洋捕鲸的“壮举”。当时未发现石油,西方的照明用油严重依赖于鲸油,同时鲸骨也可以作为裙撑,因此捕鲸业成为美国重要的支柱产业。本片中Chase是一名经验丰富的水手,他的祖先被驱逐出他所在的捕鲸岛,而他自己则渴望成为船长,但在航海前资本家钦定了本地航海大亨的毫无经验的儿子George Pollard来做船长,只让Chase做了大副。船上一行人共约20人。在航行三天以后,船长指挥大船进入风暴中以“提升士气”,结果船支损坏(之后被修复),而Chase强烈反对这个做法,两人之间的矛盾公开化。

在捕猎了一只较小的鲸鱼后,他们再难找到大鲸鱼,而Chase想获取2000桶鲸油以获得下次的船长职位。到了南美厄瓜多尔后与其他船员交谈,发现在位于赤道的太平洋深处有大量鲸群并有一只大白鲸,于是他们启航去捕猎。在目的地上,他们果然发现了大量座头鲸和抹香鲸,但遇见了传说中的大白鲸,船被破坏,他们只能抢救出相应的食物和工具,用三只小船回头向南美的方向漂流。他们在海上经历了饥渴和漫长的长达一个多月的无风漂流,最终到达一个小岛,靠岛上的淡水和鸟类为食。休整一周后,除了三个决定留下以后,其他人决定再次出发向南美。在漂流中,他们的食物吃完了,只能靠捞一点鱼和船下的贝类为食。一个黑人死了,Chase决定吃了他的尸体。随着更长时间的漂流,他们开始决定抽签决定谁死并作为其他人的食物。这样经过了两个月的漂流后,他们分别被一个大船救起和发现了南美陆地,而此时只剩下了两条船共五个人,加上留在荒岛上的三个人(他们全部存活了下来,虽然岛里还有之前躲避海难的人的尸体),最后只剩下这8个人存活。具体事件剖析可以参考南方周末的文章:\url{https://movie.douban.com/review/7795671/}。

\subsubsection{影评}
本片作为华纳于2015年冲击奥斯卡的电影,视觉效果很震撼,特别是鲸鱼和风暴中的海面,一帧一帧毫不吝啬经费。故事本身而言是名著改编,本身的剧情走向是没有问题的,但本片时长2小时,却塞进了太多内容:Chase想做船长的勃勃雄心、Chase与毫无经验的船长之间的认知冲突(他们后来又合解了,一开始看船长不可一世的样子以为是一个反派人物,但在后来求生时他正直、公正的态度又彻底洗白了)、捕鲸和炼鲸油的细节、人与自然的反思、求生时的人性考验……甚至还采用倒叙的方式描述了赫尔曼·梅尔维尔在50年后采访精神紧张的生还者以写作《白鲸记》的内容。总之,本片给人的感觉是内容和主题太多,而每一个主题都没有向里深挖,有点浅尝辄止。如果本片能在一两个地方深挖而放弃这种样样都要的心态,可能本片的深度会上一个台阶。

评分:6/10。