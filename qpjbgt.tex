\subsection{《枪炮、病菌与钢铁》}
\subsubsection{一些标注}
各大陆民族长期历史之间的显著差异,不是由于这些民族本身的天生差异,而是由于他们环境的差异。

习惯上都是把产业革命武断地定为从18世纪的英国利用蒸汽动力开始,但事实上一种以水力和风力为基础的产业革命在中世纪时就已在欧洲的许多地方开始了。

中国的华北人和华南人在遗传上和体质上都存在相当大的差异:华北人最像西藏人和尼泊尔人,而华南人则像越南人和菲律宾人。

技术的发展是长期积累的,而不是靠孤立的英雄行为;技术在发明出来后大部分都得到了使用,而不是发明出来去满足某种预见到的需要。

小部落人口少,这一点不但说明了为什么他们承受不住从外面带来的流行病,而且也说明了为什么他们没有能演化出自己的流行病去回敬外来人。

轴线走向影响了作物和牲口的传播速度,可能还影响文字、车轮和其他发明的传播速度。

它们生活在群体里;它们在群体成员中维持着一种完善的优势等级;这些群体占据重叠的生活范围,而不是相互排斥的领域。

判断野生动物是否能够被驯化的因素:
\begin{itemize*}
	\item 群居结构。
	\item 容易受惊的倾向。
	\item 凶险的性情。
	\item 圈养中的繁殖问题。
	\item 生长速度。
	\item 日常食物。
\end{itemize*}

大型哺乳动物驯化的年代从绵羊、山羊和猪开始,到骆驼结束。公元前2500年后,就再也没有出现过任何有重大意义的动物驯化了。

粮食生产的出现涉及粮食生产与狩猎采集之间的竞争问题。

当地谷物和豆类组合的驯化,标志着许多地区粮食生产的开始。

是人口密度增加迫使人们求助于粮食生产,还是粮食生产促使人口密度增加?

粮食生产是逐步形成的,是在不知道会有什么结果的情况下所作出的决定的副产品。
皮萨罗成功的直接原因包括:以枪炮、钢铁武器和马匹为基础的军事技术;欧亚大陆的传染性流行病;欧洲的航海技术;欧洲国家集中统一的行政组织和文字。

澳大利亚/新几内亚所有大型动物的消失对其后的人类历史带来了严重的后果。这些动物绝种了,本来可以用来驯化的所有大型野生动物也就被消灭了,这就使澳大利亚土著和新几内亚人再也没有一种属于本地的家畜了。

先是由于某种原因出现了政治集权,然后才有可能建设复杂的灌溉系统。

人类在智力上存在着差异,但并没有可靠的证据足以证明这种差异是与技术上的差异平行发生的。

不同民族之间相互作用的历史,就是通过征服、流行病和灭绝种族的大屠杀来形成现代世界的。

\subsubsection{读后感}
1.翻译不怎么样,很多地方按照英语习惯进行直译,没有按照汉族的表达习惯来润色调整,导致读起来不够流畅。

2.人类从七百万年前从非洲起源,历经漫长的进化,才达到今天世界各地形形色色的人种和文明。为何有些文明能够征服另外一些文明?即使不看这本书,也能大概明白欧亚大陆的文明是如何发展起来的,但却难以说出为何如此?气候?

3.作者声称反对地理决定论,但给出的答案却接近地理决定论:人类历史进入文明时代的标志,是农业生产方式的出现和繁荣,以此为基础出现了大规模的国家、文字、金属冶炼和大规模的人口,以及以密集人口为基础的传染病。具有这些特征的文明,能够以类似三体进攻地球那样的态势去灭亡那些“落后”文明。

4.为何农业生产会最先出现在新月沃地并最终在欧亚大陆上散布,而非在美洲、非洲和澳洲那样停滞不前?作者从地理和生物学角度给出了令人信服(最起码是让我信服)的答案:农业生产的兴起依赖于容易驯化的谷物(能量来源)、豆类(蛋白质来源)和能够提供大规模力量的畜力,而这些都依赖于其野生种群在当地的分布,并且即使少了某几样也难以使原始人类产生代替捕猎习惯的动力。而恰巧,新月活地出现了现在世界上的主要上述种群。

5.即使能够产生农业,让农业能够传播、散布并且各种作物能够不断交流也是重要原因。在这一点上,欧亚大陆的“同纬度传播”优于美洲大陆的“同经度传播”(我杜撰的名词),并且后者在中美洲附近的狭窄通道和赤道附近的热带雨林里产生了天然的屏障。