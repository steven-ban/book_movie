\subsection{《鼠疫》}

标签: 法国文学 \  加缪

作者:(法)加缪

\subsubsection{主要人物}

\begin{table}[htpb]
    \centering
    \caption{《鼠疫》中的主要人物}
    \begin{tabular}{l|p{0.3\textwidth}|p{0.3\textwidth}}
人物 & 特征 & 事件 \\
\hline
贝尔纳·里厄大夫 & 35岁 & 主人公 \\ 
让·塔鲁 & 奥兰城的暂居住客,社会活动家,为人宽厚 & 主要的志愿者,里厄大夫的得力助手 \\
约瑟夫·格朗 & 市政府职员,里厄大夫的病人,生活似苦行僧,任劳任怨 & 妻子离开他 \\
雷蒙·朗贝尔 & 巴黎大报社的记者 & 因出差滞留奥兰城,疫情开始时把自己当成局外人,想回巴黎与恋人团聚,后加入志愿队 \\
帕纳卢 & 神父,教会权威,擅长讲道 & 后加入志愿队 \\
卡斯泰尔 & 经验丰富的老大夫 &  \\
科塔尔 & 矮小的吃年金者 ,走私者 & 曾上吊自杀,被格朗救下 \\
\hline
    \end{tabular}
\end{table}

\subsubsection{标注}

1. 重读的过程,我越来越被一种东西所打动:廉耻。我相信这是伟大艺术的一种根本品质。(路易 纪尤)

2. 这是人所能想象出来的最为惊心动魄的一个神话,写人对抗恶的搏斗,写这种不可抗拒的逻辑,终将培育起正义的人;他首先起来反对创世和造物主,再反对他的同胞和他自身。(加缪自己的评价)



\subsubsection{书评}
这本书的情节极为简单:鼠疫出现在了一个小城,城市出现了混乱,然后一个医生建立了救治志愿小队,坚持治疗,最终疫情退去,小城回归正常的平静;这其中,包括大夫和形形色色的人展现出了面对死亡和封闭的不同的态度:有的人不幸地以不同的状态死去了,有的人一改涣散坚定地抗击疫情,有的任劳任怨最终死去,有的思想开了小差但仍然为了美好生活而回归战线。在叙述手法上,这本书也是很简单的线性叙事,没有那种复杂的情节,作者自己仅化身”叙述者“做很少的评论,大部分事件仍借用当事人的日记等手段来托出。

鼠疫(推而广之的,所有的疫情)的本质究竟是什么?它让人染病,并且相互传染,为了抗击它,必须得封闭一个城市,人人自危,不能让往常一样出门,同时城市的封闭也导致和远方的亲人、爱人的隔绝。这本书写于二战,鼠疫也是对战争的一种比喻:战争也让人产生同样的想法。鼠疫会结束,也会卷土重来,战争也是这样。

作者对待鼠疫的态度,乃至于对待救治这一行为的态度,似乎并不是“苦大仇深”的,而是有某种距离感和冷静感。作者秉持一种“人道主义”的态度,把鼠疫当成一种人类不得不面对的东西:
\begin{quotation}
天灾人祸是常见之事,不过,当灾难临头之际,世人还很难相信。人世间流行过多少次瘟疫,不下于频仍的战争。然而,无论闹瘟疫还是爆发战争,总是出乎人的意料,猝不及防。……他们不相信灾祸,灾祸无法同人较量,于是就认为,灾祸不是真实的,而是一场噩梦,总会过去的。然而,并不是总能过去,噩梦接连不断,倒是人过世了,首先就是那些人本主义者,只因他们没有 采取防范措施。我们的同龄,论罪过也并不比别人大,只不过他们忘记了应当谦虚,还以为自己无所不能,这就意味着灾难不可能发生。他们继续经营,准备旅行,发表议论。他们怎么能想到鼠疫要毁掉他们的前程,打消他们的出行和辩论赛?他们自以为自主自由,殊不知,只要还有灾难,就永远不能自主自由。

鼠疫给我们的同胞带来的头一种印象,就是流放感。……时刻压在我们心头的这种空虚、真真切切的这种冲动,即非理性地渴望回到过去,或者相反,加快时间的步伐,还有记忆的这些火辣辣的利箭,这些正是流放感。

从词义的内涵讲,鼠疫就曾意味着流放和分离。

说到底,鼠疫究竟是什么呢?\emph{鼠疫就是生活,不过如此。}
\end{quotation}

这种对待灾难的逃避、自以为是的态度,真是与多灾多难、与灾难同行的中国人有着天壤之别。可能是欧洲人太过优渥了,太过顺利了(至少近代以来是这样),只要做着“正确”的事情(比如祷告、民主制度、个人自由)就可以避免灾厄。中国人则不然:中国一直与灾难做斗争,知道否极泰来、阴阳互生的道理,因此对待灾难更从容,更认真,更淡然。2020年的新冠疫情,中国人积极应对,不惜代价,朝着明确的目标前进,而欧洲人和受欧洲影响很大的社会,则不停以“自由”为借口,混乱应对,完全没有科学精神。显然,欧洲人并没有从历史上的瘟疫和加缪的思考中学到深刻的东西。

除此之外,加缪还把鼠疫比喻成一种有关意志力的“思想上的滑坡”:
\begin{quotation}
鼠疫,每人身上都携带,因为,任何人,是的,世上任何人都不能免遭其害。我也知道,必须时时刻刻小心谨慎,以免稍不留神,就面对别人的脸呼吸,将疫病传给别人。天然生成的,是细菌。其余的东西,诸如健康、正直和纯洁,都是意志的一种表现,而人的意志永远也不应该停歇。一个正派人,就是几乎不把疫病传染给任何人的人,就是尽量少疏忽走神的人。真得有意志,还要绷紧神经,才始终不会疏忽大意。
\end{quotation}

事实上即使是加缪本人,他对疫情、对医生的认识也并不是无懈可击的。加缪说“献身于卫生防疫组织的人,他们那样做,其实也算不上丰功伟绩,只因他们知道那是唯一可做的事情,不下决心去做反倒是不可思议的”,“结论始终限于他们所知道的这一点:必须以这种或那种方式进行斗争,绝不能跪下求饶。问题全在于控制局面,尽量少死人,少造成亲人永别。为此也只有一种办法,就是同鼠疫搏斗。这个真理并不值得赞扬,这只是顺理成章的事”。这对于“崇高”的解构我实在无法认同。医生和防疫人员在面对疫情时,本身的工作就是崇高的,无论医生本人是否认识到这一点。中国人有这样的认识,因此在疫情期间医务人员本身即有一种强烈的使命感,民众也有这种感激心理,也愿意听从医生和防疫人员的安排,因此社会上能够形成一种合力,一种必胜的信念也油然而生。反观《鼠疫》中的人们,以及2020年新冠肺炎期间欧美的民众,完全没有这种面对灾难的自觉,依然与平常生活时一样我行我素。因此,与加缪相反,我觉得上面那种“真理”是值得赞扬的,这种“顺理成章”对于医务人员来说也是使他们变得崇高和伟大的东西。在加缪看来,似乎世上并没有崇高之事,特别是那些为民众牺牲之事,眼下的很多“自由主义者”似乎也热衷于把集体主义、为国为民牺牲奉献的精神进行解构,实质上还是不承认这种伟大精神。这自然不是真的,因为这个世界上,永远存在这些伟大的人和伟大的事。

加缪在最后写道:
\begin{quotation}
今后再遇到类似情况,还应该做些什么:所有当不成圣贤,以不甘心横遭灾祸的人,当然要将个人的伤痛置之度外,努力当好医生,抗击瘟神及其武器乐此不疲制造的恐怖。
\end{quotation}

可见从这样的灾祸中,加缪只是认为人只要做“自己该做的事情”就可以抵挡灾难,然而殊不知在更大的超出他所能想象的灾难面前,是需要某些人挺身而出的,是需要社会从既往的灾难斗争史中优化自己的社会动员能力以抗击灾难的,而这种“社会动员能力”,既包括那种让某些人甘愿做圣人的奉献自身的意识形态,也包括像2020年中国抗击新冠肺炎的那种社会总动员的组织技术。

小城是依靠什么与鼠疫做斗争并取得胜利呢?政府似乎做了该做的事情:封城、治疗、血清、上报、执行,然而结果并不如人意,或者说,措施不够有力,以致于城中人心惶惶,失去了秩序,甚至失去了活下去的信心。防疫的胜利,一方面靠的还是里厄医生这样的人的坚守,还有就是,鼠疫比较弱小,让这种简单的社会动员打败了。2020年的新冠肺炎,完全是另外一种级别的对手,我们看到欧洲社会无法像中国那样坚决地封城、有效地协调配合,封城不坚决,民众轻敌认为轻易就能战胜疫情,这导致了传染人数和死亡人数在欧美地区的急剧攀升,丢尽了“发达国家”和“民主国家”的脸面。

因此,和很多人对《鼠疫》的溢美之词相比,我对它的评价就没有这么高。它对灾难的想象力过于贫乏,以至于最多相当于大禹治水的水准(甚至还不如)。2020年的新冠肺炎,显然已经超出了《鼠疫》的这种“惊心动魄”。我对《鼠疫》的评价,远低于加缪自己的《局外人》,后者对生活真实性的剖析实在是一针见血,让人有一种惊艳的感觉,相比之下前者暗淡了不少。

评分:3/5。