\subsection{《心花路放》}
1.这年头,从文青圈外到文青圈内,从男文青到女文青,黑女文青成为时髦。《心花路放》里的康小雨是典型的女文青和大龄剩女,闺密婚礼上遇见旧爱,打肿脸说自己不单身,回到家却还要披着雨衣修水管。水管修不好,一团乱麻时听到男文青耿浩的《去大理》有了共鸣,于是千里迢迢到大理寻找爱情。女文青到大理是为了寻找爱情,男文青以及所有男人去大理是为了寻找艳遇,其实都是一回事。但大理已经是俗人遍地的地方,不再是文青们的天堂。文青大概都有些洁癖,特别是精神洁癖,而女文青在感情上的洁癖比其他地方只会更重,所以只看得上会弹吉他的耿浩,看不上遍地只会喝酒吹牛的俗人。康小雨在大理的情节是典型的文艺片桥段,不值一提,观众恐怕会想到她是在等待离婚的耿浩到大理来一场邂逅,然后就是一happy ending,宁浩的聪明也在于此,所以结尾的反转才会如此巧妙。说回康小雨,再有洁癖也难耐寂寞,而女文青又天生容易寂寞,于是遇见找寻很久的耿浩就一见钟情。

2.很多电影里,一见钟情是美好的结局,但《心花路放》里只是个开始。不做男文青的耿浩开店挣不了钱,后来女文青的康小雨就傍了大款离婚。在耿浩的心里,她那是爱钱,自己钱少所以痛失爱妻。整个电影,就是耿浩拼命忘记过去的过程。他去大理,一方面是陪着基友送戏服和学习约炮,一方面,是去寻找作为男文青的自己。影片最后,康小雨似乎已经不再是女文青,但耿浩终于做回了男文青,并且以此身份又收获了一场邂逅。你看,女文青早早死去,男文青却越挫越勇,并且似乎只有坚持做男文青才能保持异性缘,做生意谈钱太俗,所以丢了老婆。这电影是给男文青洗白,给女文青抹黑。

3.所以坚持做女文青的女文青们肯定不喜欢这部电影。整个电影院观影期间,爆笑的以男生居多,我座位旁边一的女生坚持全场没出声,让我敬佩这定力,不过想想也正常。电影里的少儿不宜镜头太多,本来就不是给女性观众设计的,情节也不讨好女性,特别是拉拉那段,这得得罪多少女同性恋观众啊。男人看了哈哈大笑,女人看了只会心里骂一声真粗俗。

4.中国内地男导演对女人的审美趣味相当之低。远的,如张艺谋的电影里,女人要么被压迫被迫害,要么单纯地不正常,永远被动,永远是迎接姿态。近的,如韩寒的《后会无期》里,苏米怀了别人孩子假装妓女骗钱,成为恶和脏的化身,却最终被江河这个男文青和纯屌丝洗白和拯救,导演自己成全了一场意淫的好姻缘。其实《后会无期》里,导演对苏米这种女性是一种俯视的姿态,作为导演化身的江河,本身没能力找富家千金,只能找地位比自己还不如的应召女郎,几乎是以命令的语气想和妓女在一起,而后者,也只有经过江河的廉价拯救和关爱才重生,并最终靠在成功后的江河肩头。总之,给妓女立好牌坊,就可以明媒正娶了。国师和岳父都是读书人,情怀兼济天下,镜头里的女人是孤独的画像,看似深刻实则卑微无力。和张艺谋的隔靴搔痒和韩寒的意淫不同,宁浩不是男文青,《心花路放》里,宁浩一贯的置身事外的姿态进行调笑和记录,并以一个男导演的视角把女人搬上荧幕啪啪啪啪,影片里的女人是抽象的,是背对观众的,是情节里的,就像大理的邂逅,过去了出了旅馆就不再记得了。

5.宁浩的导演技术在内地导演里是翘楚,这部电影的剧本是好剧本,镜头语言也纯熟老练,是部有观赏性的成功商业片,起码值回33块钱的票价。