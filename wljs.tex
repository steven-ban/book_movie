\subsection{《未来简史》}

作者:尤拉尔·赫拉利

虽然名为“简史”,但未来并未发生,根本无史可言,并且本书主要是作者基于人类文化史、基因学、社会学、人工智能、大数据等对未来的预测。因此,书名除了和作者的成名作《人类简史》有所对应外,其实并没有反映出本书的主要内容和写作宗旨。相对地,副标题“从智人到智神”才更贴合本书的主题。

本书写作大概定稿于2017年初,晚于《人类简史·从动物到上帝》,早于《今日简史·人类命运大议题》(突然发现这本书“人类命运”有点像中国提的“人类命运共同体”)。本书的内容,与《人类简史》对人类历史的概括(认知革命、农业革命、工业革命、宗教等)紧密相联,同时关于人类目前面临的问题和《今日简史》有很大程度的重叠(例如大数据、互联网、人类在数据洪流下的精英与大众的分化、医疗革命对精英的提升等等、世俗主义等等)。

本书的宗旨,在于根据人类历史,畅想未来的发展趋势,契合当前数据科学、基因科学的发展。本书关于宗教的祛魁、对世俗主义的思考、对人文主义和自由主义的思考,和《今日简史》遥相呼应。从作者的关注点和主要观点来看,本书内容和《今日简史》《人类简史》基本相同。

人类为了解释世界,也为了团结群体,发明了宗教。文艺复兴以来,人文主义兴起,让人类开始关注自己的体验,并把自己的体验放在第一位。对自我的肯定,对自己的重视,是自由主义、共产主义、民族主义和纳粹主义的根基。但是,大数据的兴起,使得人受到外界的影响越来越深刻,向内心探寻也就发现“自己”并不是铁板一块,而是可能受外界的洗脑;人类并不真的能够认清自我,相反基于人体健康、社交的数据才更能清晰和理性地了解人,也更能为人做决定,而这些数据是由谷歌、Facebook这样的私人公司掌握的。人文主义的危机,带来了民主选举的合法性问题,也带来了贫富分化的问题。

有一个让我感兴趣和震惊的事实。人类对自己的记忆里,有两个“自我”,一个是体验自我,一个是叙事自我。前者是一个人在生命中每一刻的短期记忆,但在长期情况下则不存在;后者在人类对自己进行评估、做出决策时才是考虑的重点。但人类的叙事自我的“算法”很粗暴,就是把一件事或一段时期的最高体验和最低体验做一个平均,并用这个平均值来代替此事或此段时间的体验,因此抹除了体验自我的种种波动。女人生孩子虽然痛,但生下孩子后的内呔啡和激素带来的快乐会抹平痛苦,从而在一段时间后不再想起撕心裂肺的痛苦,这是这个道理。

另外一个事实,是谷歌可以根据一时一地的人们对“头痛”“喷嚏”的搜索量波动来估计流感的爆发,这比医院的上报更快更准确,也能够让政府更快地发布预警和做出应对。

另外一个事实,是被人类驯化的动物,如猪、羊、牛、鸡等均是社群化的动物,反过来它们都有较大的社交需求,但养殖场里猪都是独自进食,被囚禁在栅栏里无法有效活动,可以想见它们有多么“痛苦”。

本书的关于人类认知的内容,很让我惊讶,这也可以看出作者作为一个历史学者,知识涉猎之宽。

评分:8/10。
