\subsection{松本清张}
\subsubsection{《苍白的轨迹》}
一个恋爱探险故事。基本的杀人手法比较简单,就是用货车的大灯晃花人眼,近前用木棍打晕,再扔下山崖。杀人动机也比较单纯,典型的日式复仇故事:哥哥的作品被人以欺骗手段去发表,没想到却被那人作为交易让另外一个作家去发表,于是作为妹妹便联合旧友的弟弟杀人。
 
推理基本上就是一本道,读者没有参与的乐趣,只能靠猜,算不上本格。唯一的亮点(我并不喜欢)就是由一对恋人来作为侦探,但互动仅仅在于恋爱的开始阶段,羞涩与撩拨,但没有亲密。
 
满分五星,只能给三星。

\subsubsection{《死亡螺旋》}
味冈也太怕警察了,神经质的心理是案件的主要推进动力。其实第一桩的遇见尸体案件,只要向警察说明清楚,同时告诉警察不要张扬,事情就解决了,也就没有后面的事情了。 

当然,味网被BOSS害,被同盟害,被下属害,多方夹击心理崩溃,也不算太失常,毕竟手法太“高明”了。 

但是还是有破绽的。比如红叶庄杀人事件,尸体被长途运输4-5个小时,虽然有了水中搅拌作用防止尸斑过快生长,但很可能会留下痕迹,警察很容易发现;尸体并未接触房间内的任何东西,因此肯定没有留下指纹,反向思维的话,警察也很可能发现这个疑点认为第一杀人现场不是在红叶庄房间内。红叶庄事件是这里面最悬疑重要的一个案件,直接造成了味冈的心理崩溃,因此这个事件还需要更好的打磨。

\subsubsection{《隐花平原》}
读书过程里一直没有注意书名“隐花平原”的存在,当看到教团以集体建房来吸引教众时,还以为反映的是日本的宗教呢,“隐花平原”是不是指的隐藏的净土呢?直到读到最后才翻看书本介绍时才恍然大悟:
\begin{quotation}
在植物链的底层,有一种名为“隐花植物”的低等植物,它们一生都没有机会在阳光下开出鲜艳的花,只能在黑暗中缠绕爬行。在广袤平原下阴暗潮湿的角落中,隐花植物以各自奇形怪状的藤须盘根,交错形成一朵巨大而盛烈的恶之花……
\end{quotation}
如果看的是纸质书而非电子书,每次翻书都看到这段话来提醒自己,那么在修二拜访完芳子后恐怕就能猜出来“隐花”比喻的就是像玉野文雄这样的私生子,杀人事件都是他指使的。但是,最后几页和西东的遗书使案情走向彻底发生了反转,这是几个同父异母的兄弟间互相憎恨互相伤害的悲惨故事,而作为侦探的修二一不小心就可以死于非命!至于整个事件的原因,仅仅是花房忠雄年轻时做的孽,四处留情广播露水情缘,生下的几个儿子生活经历不同,当他们发现自己的处境相关巨大时,便生出不平愤懑之意,时间一长成为最大的仇恨,不惜对自己的亲兄弟下手。可见,在一个家庭里,爱没给够,也要用钱来凑,否则留下的会是多大的悲剧!

松本清张前半断的叙述不紧不慢,我一直以为最终的凶手是教团联合银行借刀杀人,没想到教团只是打了个酱油,前期打酱油的西东刑警才是最终BOSS。作为推理小说,松本清张并没有给出足够多的线索让读者自己解开谜团,这一点很我有点失望。当然,社会派嘛,轻推理重动机也是常态。

\subsubsection{《时间的习俗》}