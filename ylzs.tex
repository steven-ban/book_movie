\subsection{《娱乐至死》}

作者:(美)尼尔·玻兹曼

本书是1985年出版的对于美国电视文化进行反思的著作,影响很深。

美国的交流史分为印刷机时代、电视时代和互联网时代。最后一个互联网时代作者没有讲,因为那个时候还没有互联网,但现在我们可以用类似的比较方式去进行考察了。

美国是一个知识分子建立的国家,最初的移民们虽然著述少,但极喜欢读书,出版业发达。此时是美国的”印刷机时代“(或铅字时代)。这个时代里,人们交流思想的方式是文字,因此表达方式是正式的、修辞的(华丽的)、逻辑的。使用这种方式也使得人们对思想的认识具有相应的理性和严谨的特点。

电报出现以后,信息的流动突破了当时当下的局限,不相关的地理间隔(远)和时间间隔(短,迅速及时)发生了联系,而在作者看来,这样的联系并非必要。电报的及时性为报纸上的新闻注入了活力,但人们更关注那些远方的无聊信息,无法理解远方的深刻的思想。摄影的产生使得文字居于第二位,人们喜欢照片甚于文字,但与文字相比,照片只表现了此时此刻的一种发生的孤立的状况,而非逻辑性的理性的存在,这为人们的理解埋下了隐患。

电视时代来临以后,所有的传播都是娱乐化的,电视为娱乐而生,电视上的语言也与文字时代的严谨背道而驰。为了满足这处娱乐,播音员和主持人的形象必须是正面的而非怪异的,因此他们变得千篇一律。在作者看来,电视对信息的表现和传播并非仅仅是一种形式,它影响了“实质”,影响了人们产生和接收信息的内容。具体来说,电视上信息分为一段一段的,每一段之间缺少联系,当人们观看电视节目时,接受到的信息是孤立的凌乱的。由于时间有限,对于每段新闻只能做出简短报道,因此人们获取的信息也是肤浅的,并且很快就会忘掉。电视时代,人们获取的不再是连贯的完整的信息而是“情绪”。

作者还论述了电视上的宗教节目。作者认为,宗教类电视节目取消了宗教活动的神圣性,人们可以在电视前做出任何动作,这和现场的仪式感是完全不同的东西。因此,对于电视观众来说,宗教成了一种娱乐和表演,上帝(宗教的目的)不再重要,传教士本人更为重要。另外,电视节目本质上都是使用娱乐的手段吸引观众的眼球,借此讨好观众,这和宗教的本质相背离。宗教是给教徒们他们真正需要的东西,而不是拿一些他们喜欢的东西来讨好他们。

美国是一个宗教氛围很浓厚的国家,作者这样的论述属于对美国文化的批判,这在中国就不存在了,毕竟中国人整体上对各种宗教都没有那么虔诚。

对于政治,电视也深刻地改变了它的运行方式。政治家依靠电视的传播手段,像广告一样推销自己,选民对于政客不熟悉,只清楚他们的长相和外在,而对他们具体的政治观点不太关心。政客们为了讨好选民,采用广告的方式包装自己,而非像印刷时代那样通过严肃的政策演说。电视广告是讨好人的,它创造出一种美好生活的影像让观众来模仿。电视同时创造出繁杂的信息洪流,真正的思考被淹没,因此不通过审查,政府(主要是美国政府)就达到了愚民的目的。在印刷时代,书籍负载着思想来传播,消灭书籍才能消灭相应的思想,而电视时代则没有这样的担忧。因此,作者认为,\emph{电视时代验证了赫黎胥的预言,而非奥威尔的预言}(至少在美国是这样)。人们在快乐的氛围中麻痹自己,放弃思考,随波逐流,享受在这样的“美丽新世界”中。

美国一些人把电视用于教育,声称人们在娱乐时有更好的学习效果。但是观看电视时人们注意力不集中,无法进行系统性和逻辑性的思考,这和教育的目的背道而驰。作者显然反对这样的方式,认为它完全可以被传统的教室和书本替代。

作者对印刷机时代的喜爱和对电视娱乐化的痛恨表明了作者的\emph{精英和古典意识,甚至是保守主义的立场},但作者没有看到的是,随着电报、广播和电视时代的来临,信息的交流成本更加低廉,人们互相之间交流更加方便,随后的互联网时代更加突显了这种演进。诚然,越来越多的人参与到信息的产生和交流中,信息的形式也更加破碎和随意,甚至如同作者所痛陈的那样,更加娱乐化和猎奇化。但是,严谨的思想还是能传达给更多的人,虽然改变了形式和逻辑。人们沐浴在信息的海洋里。我记得许知远采访马东时,许知远认为,过去只有5\% 的人能发声,而现在剩下的95\% 也能发声,因此这个世界变得嘈杂和喧嚣。这95\% 的相比于过去5\% 的精英化声音没有必要吗?其实想想,5\% 的人永远是存在的,只是现在他们的声音越来越不彰显。在过去,5\% 的人发声了,但真正想听这些声音的人只有一小部分,另外的部分虽然即使听到这仅有的声音,也不能理解和吸收;现在不同了,95\% 的人在自娱自乐,但5\% 的人的声音可以渗透给更多的人,产生更多的随机性,95\% 里那些渴望思考和自由的人有了机会去听。因此,电视时代和互联网时代虽然更加娱乐化,但那些逻辑严谨的思想还是能够被人听到的。在这个方面,我对作者的忧虑表示反对。情况可能如他说的那样,但没有那么严重,现在与过去的对比也没有那么鲜明。

作者批判的是美国的文化现象,但电视带来的娱乐精神在中国也大量存在。不过,对于中国人而言,电视意味着更多的东西。中国人大量拥有电视,是80年代以后的事情,借助于电视,这个国家的基层人民得以了解国家的方方面面,对于很多农村人来说,这是他们了解外界社会的重要或仅有的几个渠道之一。电视促进了人们的交流,促进了不同地域的人的相互理解,是有正面意义的。与美国不同的是,中国的电视节目是管制的,只有经过政府允许的节目才能播出,因此中国的电视的政治性较强,同时由于创造能力和制作能力水平较低,中国电视节目的娱乐性没有那么强,它与现实之间的关系似乎没有那么密切,人们看完电视之后,会加到真实的生活里,不会混淆电视与生活。

中国在经历了20年的电视时代后,迅速进入互联网时代,这和美国的代差就很少了。在互联网时代,人与人之间的交流成本更加低廉,同时信息也更加破碎,流动更加迅速。互联网对于中国人,才相当于电视对于美国人。在互联网上,中国人才迎来了真正的普遍的娱乐,他们看到更真实的世界,现实和网络开始混淆,同时他们也在互联网上互相抱团,相互孤立,甚至相互仇视。美国的电视所带来的问题,互联网也为中国人带来了。所以,即使是30年后再看这本书,对于中国人而言也能起到反思的作用。

互联网,特别是移动互联网的发展,大大促进了信息的流动,也促进了人与人之间的联结。然而,这种联结虽然有一定的进步性,但也带来了一些问题。如果说电视改变了美国人的思考方式,那么互联网也深刻改变了中国人的思考方式。互联网催生了自媒体,它们首先是需要赢利的,因此会拼命扩大受众并夯实受众,因此会聚集起一批有类似观点和思想的人。由于自媒体资金有限,知识水平也有限,因此他们不可能高质量地输出内容,只能拼命通过观点来聚拢人气。人群聚集起来,往往是懒于思考的,何况互联网本身是个娱乐大平台,因此这些自媒体的信息是高度娱乐化的,是浅薄的,是情绪的。他们贩卖的是易于传播的情绪,例如快乐、例如仇恨,例如焦虑。现实往往是平淡的,是痛苦的,是连续的,是立体的,但\emph{到了自媒体这里,现实里的事情被拆分,被孤立,被标签化,被对立化,被戏剧化,以应对人们的猎奇、放松和聚集心理}。所以,一个人如果只通过上网来了解人们的思想,肯定是偏颇的。

作者对电视采取了反对的基本论调,我觉得他的分析虽然有理,但有所夸大。我对电视或互联网是持欢迎和肯定态度的。它们为人类带来了新的理解世界的方式,旧的方式仍然是存在的(人们在互联网时代还可以看电视,还可以看书),对于一个理性的自由的人而言,一个新工具的使用并不一定代表旧工具的消亡,他仍然可以生活地严谨和理性。对于大众来说,新手段的流行会使旧有的优点消失,但也会带来更多的优点。电视和互联网都带来了便利、平等和未知的可能。

30年过去了,不知道作者对互联网又怎么看呢?

评分:4/5。