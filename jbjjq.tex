\subsection{《中国历史上的基本经济区》}

标签: 中国历史,经济,区域


作者:冀朝鼎,1903-1963,中共党员,《北平无战事》中的原型人物之一。

\emph{基本经济区}的概念,是指历朝历代统治者只要占领控制了这一区域,均可进而控制全国,因此更集在地在此地进行水利工作。这个概念是作者自行提出的。根据这一概念,历代的“基本经济区”为:

\begin{table}[htpb]
\centering
\caption{中国历代基本经济区的变动}
\begin{tabular}{l|p{0.7\textwidth}}
朝代 & 基本经济区的变动 \\
\hline
秦汉 & 泾水、渭水、汾水 + 黄河下游。西汉末年河内(河南南部)水利兴起,刘秀借以建立东汉。\\
魏晋南北朝 & 四川与长江下游开始崛起,三国时期借以成为割剧势力。\\
隋唐 & 长江流域成为基本经济区,大运河发展\\
五代、宋、辽、金 & 长江流域的作用进一步突显,但围田作为优良土地资本被豪强占据,引发了阶级矛盾\\
元明清 & 统治者试图发展海河流域以摆脱对长江流域的依赖,因而大量修建直隶的水利工程,但收效并不大\\
\end{tabular}
\end{table}

黄河河水饱含泥沙,虽然会带来水患,但泥沙肥力高,会抵消对农业的不利影响,因此成为中华文明的发源地。长江流域水量大,淤泥肥力强,加之以完善而复杂的水利管理,可种植水稻。

作者引用了顾颉刚在30年代的论断,认为大禹的故事属于杜撰,实际上是浙江的事,并且年代很晚(西周)。这些疑古言论都没有依据,现在看来没有什么价值。不过,作者并不严格认为顾的言论是正确的。

本书提出“基本经济区”这一概念,分析了秦至鸦片战争这一漫长的历史阶段中政治发展的经济学基础,为有创见的学说。但从全篇来看,篇幅较小(仅8万字),分析较为浅近,格局较为粗疏,治水数据缺少核实,因此可作为一家之言,但难成不易之论。