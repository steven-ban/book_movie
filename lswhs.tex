\subsection{《两宋文化史》}
\subsubsection{标注}
禁兵大致以一百人为都,五都为营,五营为军,十军为厢,分隶三衙。

北宋时,京开封中下户人家,不重生男,重生女,教以卖艺;

园林的兴废,象征着洛阳的盛衰,而洛阳之盛衰者,天下治乱之候也。”

南宋张淏《艮岳记》说:“越十年,金人犯阙,大雪盈尺,诏令民任便利斫伐为薪。是日,百姓奔往,无虑千万人,台榭宫室,悉皆拆毁,官不能禁也。”

嫁娶固不可无媒,而媒者之言不可尽信。

六礼,即经过纳采、问名、纳吉、纳征、请期、亲迎六个步骤的程序,才算完成了婚礼,成为合法的正式的夫妻。

先秦之时,迟婚为多,《周礼》规定男子三十而娶,女子二十为嫁。

冬至阳气起,君道长,故贺;夏至阴气起,君道衰,故不贺。”(《广记》)

据说有一年重阳节时,因此物古代六经中没有“糕”字,刘梦得作诗也不敢用,另一诗人宋子京作《九日食糕》诗讥笑此事说:“刘郎不肯题糕事,虚负人生一世豪。”

宋太宗时把中秋与新年、端午列为三大节日。

宋徽宗生日为五月五日,因俗忌改为十月十日,并称为“天宁节”。

寒食,冬至与元旦为宋代三大节。

唐宋以来的皇宫中仍用“外朝”与“内朝”之别。每逢国家大典,如改元、大赦、元旦、冬至等大朝会以及阅兵、受俘、接见外国使者等,均在“外朝”大庆殿举行各种隆重仪式,即人们所称的“金銮殿”;而皇帝日常接见群臣商讨国家大事却在“内朝”之殿,即垂拱殿,仪礼可以不拘,举止较为方便。

两宋都城妇女的服饰是相当华丽奢侈的,往往不顾朝廷的禁令。

赵宋最高统治者的基本国策最重要、也是最有效的一条,就是宽容精神。

宋朝是“为与士大夫治天下,非与百姓治天下也”

赵宋王朝的基本国策——比较开明的文化政策,尊重知识、尊重人才的宽容态度

宋代两税实际征收额的增长,主要是通过附加税的途径实现的。

宋代的法律法规大致有律、敕、例等三大类。

宋代立法大权操于皇帝一人之手。

宋代监察机关,在中央有专职的御史台与谏院,有兼职的门下省给事中与中书省中书舍人;在地方有路一级的监司,州一级有通判,北宋时路还有走马承受公事。