\subsection{《古墓之谜》}

作者:【英】阿加莎·克里斯蒂

\subsubsection{人物}

\begin{longtable}{p{0.2\textwidth} | p{0.3\textwidth} | p{0.35\textwidth}}
        \caption{《古墓之迹》人物表} \\
        \hline
    姓名 & 特点 & 事件 \\
    \hline
    \endfirsthead
    
    (接上表) \\
    姓名 & 特点 & 事件 \\
    \hline
    \endhead
    
    \hline
    \endfoot
艾米-莱瑟兰 & 护士,32岁,本书记述者 & \\	
凯尔希太太 & 莱瑟兰的看护人	& \\
莱利医生 & & \\	
赖特一家 & 凯尔希太太的朋友	& \\
路易丝-莱德纳太太 & 漂亮,很瘦 & 前任丈夫在她20岁的时候(15年前,1918年)死在战场上,善变,会无缘无故地心烦意乱 \\
埃里克-莱德纳先生 & 美籍瑞典人,带领考研挖掘队,谦逊,不张扬,讨人喜欢	& \\
希拉-莱利 & 不喜欢莱德纳太太,喜欢埃莫特 & \\
弗雷德里克 & 莱纳德太太的前夫,德国间谍 & 被莱纳德太太揭发,处决时逃跑,尸体被发现在火车上,不能百分这百确定是本人 \\
威廉 & 弗雷德里克的弟弟,崇拜哥哥 & \\

\end{longtable}

考古队其他成员:
\begin{longtable}{p{0.15\textwidth} | p{0.3\textwidth} | p{0.2\textwidth} | p{0.3\textwidth}}
    \caption{《古墓之迹》考古队其他成员} \\
    \hline
姓名 & 特点 & 职责 & 事发时在做的事(供词) \\
\hline
\endfirsthead

(接上表) \\
姓名 & 特点 & 职责 & 事发时在做的事(供词) \\
\hline
\endhead

\hline
\endfoot
比尔-科尔曼 & 英国建筑师,话特别多,喜欢莱利 & & 在哈沙尼 \\
拉维尼神父 & 迦太基来的法国神父,高个子,眼神敏锐,善于观察 & 辨认碑文,代替因病不能来的伯德博士 & 在卧室里工作 \\
约翰逊小姐 & 英国约克郡人,50岁左右,很欣赏莱纳德先生 & 总管杂务 & 在客厅里拓印圆筒印章上的刻痕 \\
 &	矮胖的美国人,很安静 & 拍照	& \\
莫卡多先生 & 又高又瘦,面色萎黄	& &	在实验室工作 \\
莫卡多夫人 & 非常年轻,看起来阴险,讨厌莱德纳太太 & & 在自己房间里洗头 \\
莱特尔先生 & 白白胖胖 & & 在暗房里洗相片 \\
大卫-埃莫特 & 喜欢莱利	& & \\	
理查德-凯里 & 英俊,很瘦,建筑师,很沉默,但以前不是这样的 & & 去了挖掘场 \\
厨师 & 印度小伙子 & & 坐在拱门的外面,一边和卫兵聊着天,一边在给鸡拔毛 \\
易卜拉欣和曼苏尔 & 男仆  & & 在一点一刻的时候过去和他们一起聊天。他们在那儿有说有笑地一直待到两点半 \\

\end{longtable}

\subsubsection{事件}

房间示意图

事件:
\begin{itemize*}
    \item 5年前(1918年),莱纳德太太发现前夫是个德国间谍,告诉了自己陆军部的父亲,前夫弗雷德里克被处决时逃跑,死在火车上,尸体被发现
    \item 莱纳德太太每次想和别人结婚时,都会收到前夫的信,威胁如果结婚就杀死她
    \item 和莱纳德结婚时连续收到两封威胁信,有人试图开煤气杀死他们
    \item 三周前收到来盖有伊拉克邮票的威胁信,之后不断收到威胁

    \item 考古队在伊拉克的雅瑞米亚遗址进行挖掘
    \item 莱纳德太太变得很焦虑,有幻想症
    \item 团队里出现了紧张的气氛
    \item 作者莱瑟兰来到营地
    \item 次日,莱纳德太太在没有任何人陪同的情况下去散步
    \item 莱纳德与莱瑟兰一起散步,见到神秘男人,长得像莱纳德梦中窥视她的人
    \item 拉维尼神父和神秘男人交谈,被莱瑟兰撞见
    \item 夜晚有人闯进营地
    \item 莱纳德夫人向莱瑟兰讲述故事缘由
    \item 莱纳德夫人的笔迹很像匿名信上的笔迹
    \item 两点四十五分,发现莱纳德夫人死去,前额靠近右太阳穴的地方受到了致命一击,死亡时间推断:一点十五到一点四十五之间

    \item 开始调查
    \item 波洛找各个人谈话,寻问他们对死者的看法
    \item 约翰逊在屋顶上发现了嫌犯进来的手法,当晚喝浓盐酸死去
    \item 当晚拉维尼神父失踪

\end{itemize*}

真相:凶手是死者的老公,也是死者的前夫。弗里德里克在火车失事后,冒充了莱德纳的身份,成为一个考古学家,二十年后再次娶了莱德纳太太。为了防止身份泄露,他冒充自己的身份多次写下恐吓信。

杀人手法:使用面具在晚上恐吓妻子,之后在屋顶上趁别人不注意把面具吊下,放到妻子窗户旁边,妻子开窗观看,然后将石磨吊下去砸死她,再拉上来。之后选择一个时间进入房间,关上窗户,移动尸体到床边,然后报案。

这个杀人手法实施难度挺大的,比较不靠谱,很容易失败,而一旦失败就完全暴露了,因此本案的诡计设计并不合理。

神父是个文物惯偷,这一点上有点轻视其他人的专业水准了,我不相信多次考古的人会对蜡制的赝品弄错。

波洛的推理方式过于强调人格分析(类似于精神分析,这种方式本来就不太严谨,没有证据支持)和动机分析,而拿不出一个证据,最后只能强行以凶手自首来完成破案。

因此,本篇的推理水平是比较低的,满分10分的话,恐怕只能给6分。

评分:6/10。
