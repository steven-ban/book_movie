\subsection{《恶魔的圣诞节》}

作者:【日】横沟正史

这是金田一探案系列的三个案件,故事本身并不复杂,典型的本格派推理,核心在于诡计的设计:
\begin{description}
    \item[恶魔的圣诞节] 一个十六岁的少女因为母亲父亲之间的离婚纠纷,变成了凶残的杀人凶手。继母是个爵士乐歌手,她的经纪人发现了端倪,于是找金田一协助,金田一让她去自己的公寓里等自己,然而凶手却打电话伪装成经纪人,尾随经纪人进入了公寓,杀死了经纪人,手段是氰化钾,随后她洋洋得意,把墙上的日历撕了几页,显示出圣诞节那一天。在圣诞节晚上,父亲和继母招待几个客人,继母与情人在自己的房间幽会,被情人的情人发现并尾随,凶手父亲也到了隔壁的洗衣间,少女在背后杀死了父亲,但父亲对她有愧疚,极为替她掩护,从洗衣间爬到了继母房间门口,这一幕被继母情人的情人发现,但没有声张。继母从自己房间打开了门,发现了死亡的丈夫,东窗事发。诡计本身并没有十分高明,就是一个简单的“暴风雪山庄”模式,推理的过程也较为简单,最大的线索就是凶手写的小说,被继母鼓励,小说里她把自己比喻成住宿舍的少女,而继母是那个舍管。鑫田一猜出了凶手,事后去他们的宴会上,揭穿了凶手,但凶手想要杀害继母,在酒里下了毒,金田一偷偷调换给了凶手,凶手喝掉死去。
    \item[女怪] 漂亮女人嫁给有财有势的男人,但男人因为破落而经常打她,于是她忍无可忍,用长针从耳后插入他的脑里,解剖时没有发现针,于是定义为脑死亡,但是针其实留在了头骨里,成为一项重大的语气。多年以后,尸体被一个装神弄鬼的人挖出,以此威胁女人,女人因害怕而与他交往,这个人爱上了她,为了得到她而扮演了另外一个柔情的人,但在恐惧驱使下的女人用同样手法杀死了这个人,转而等待那个柔情的人。故事很扯,且不说针留在头骨里很容易被发现(有针孔),而且一个人扮演另外一个人且和同一个女人亲密接触很难伪装。
    \item[迷雾山庄] 著名影星爱捉弄人,却被自己的侄子侄女算计,主要手段是借刀杀人,这个“刀”是影星一边的人,他通过在迷雾中偷偷挪走路牌,引导金田一走到和自己家外观相似的一幢别墅门口,影星自己表演被人杀死,然而这种玩笑完成后,影星就被自己的侄子侄女杀死,然后被拖到自己的别墅后面抛尸,“刀”也被杀死藏在那幢楼的阁楼里。在时间上,侄子侄女利用当地去东京的火车,在中间站下车,传递信息,然后从东京来的人就知道了案件进展,实施下一步行动。然而,“刀”在伪装时踢到一块石头,但出血是假,凶手却忽略了这一点,造成破绽,因此不得不继续作伪下去,要将阁楼里的尸体的脚趾头割下来,却自此被警察逮了个正着。他们做恶,仅仅是为了好玩。
\end{description}

这几个短篇讲的都是金田一的探案故事,故事情节比较离奇,比较像阿加莎的风格。之前读了几篇松本清张的小说,后者是社会派,凶手犯罪都有着复杂深刻的社会原因,往往是社会的黑暗面。而本书主种本格式的推理,则将重点放在做案本身,《女怪》这种还比较离奇,实施成功的可能性很低,可以说十分刻意了。金田一这个形象,是不修边幅的,他的头发乱如鸟窝,但对案件则有着痴迷,把犯罪和破案当成艺术。整体而言,这几篇里最好的就是《恶魔的圣诞节》,情节比较紧凑,节奏感很好,比较刺激,而且凶手的残忍与她的年龄形成了反差,迷惑性也很强,另外罪行本身和父亲为其掩饰也形成了足够的反差,有着日本特有的赎罪文化。《迷雾山庄》则较为一般,读者很容易猜出来迷雾中用房屋外观来迷惑人的诡计,利用对向火车中途下站来制造联络也比较新奇,而犯罪动机则比较薄弱。《女怪》最差,没有什么推理性,而且易容也比较扯淡。

评分:3/5。