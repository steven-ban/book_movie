\subsection{《乌合之众》}

1.这本书并不是一部真正的心理学著作,通篇没有严谨的论述,没有科学的推理,只有片段的观察和武断的判断。因此它只是一本通俗读物,不应该把它当成严肃的社会心理学的学术著作。有趣的是,作者反对的群体性弱点里,那些“乌合之众”恰恰喜欢领袖们这样的论断。

2.由于缺乏严谨的论述,因此看这本书决不能被作者的论断吓住,要思考究竟对不对,特别是反例是否存在,从当前例子能否必然得到当前结论。

3.作者语言极富于煽动性,如果你对这本书的观点全盘接受,那么你也会成为不会思考的乌合之众。

4.作者对群体缺乏思考的论断我严重不同意,它举的例子,如中世纪的欧洲和法国大革命,往往是由于当时某个群体的信息闭塞和科技落后导致的,或是缺乏科学态度。当然,群体行动力和思考力的严重不一致我是同意的,群体即是法律我也极为认同,但群体的盲目并不是这些因素的必然结果。

\emph{(待续,目前只看了全书的前百分之二十)}

5.作者似乎对“民族性”或“国民性”相当看重,认为这是一个国家民众的生活习惯,影响这个国家的政治制度和行为。比如,作者反复提到的法国人的冲动易于煽动,影响着法国的近代历程,特别是法国大革命和1848年波旁王朝的复辟。

6.作者对法庭和议会这样的“乌合之众”指责颇多,即使它们都是受到教育的人的集合,但在愚蠢和易于煽动上和没受到教育的人一样。作者警惕议会制度会因为议员为自己前程着想会过多增加选民福利,从而增加政府规模和权利,反过来会限制公民自由,这在战后成为了现实。

7.虽然如此,但作者坚定维护民主制度,认为这是最不坏的制度。

8.作者对社会主义表示警惕,认为这种把理想放在现世的宗教,会在它建立的那一刻便开始消亡。不得不说,这个论断还是很精准的,特别是对列宁主义这一派。

9.总之,作者是个有远见和极度聪明的人,但书中观点是不是正确,恐怕是仁者见仁智者见智的事情了。反正,我对耍聪明讨论严肃问题又缺乏严谨论证的书,评价不高。亚马逊上给了三星评价。