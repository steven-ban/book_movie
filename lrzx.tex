\subsection{《烈日灼心》}
首先,片名“烈日灼心”和情节究竟有啥关系?能不能换上更贴合情节的片名,比如“劫后余死”?

情节略乱,小高潮不断,但都不够嗨,让人悬着一颗心,却始终落不了地。伊警官目睹邓超是gay后,情节开始混乱,gay的情节让人出戏,案情扑朔迷离让人摸不着头脑。高楼大战略狗血。镜头上,喜欢用黑幕来分割情节,结果加重了这种凌乱。

杀人三人组的逻辑说不通,特别是为了掩盖罪行偏偏用一些弱智手段:邓超非要整天在警察眼皮子底下活动,难道不担心警察查他户口籍贯?智商162的高虎存在感极弱,最后的自杀也完全莫名其妙;道哥倒还好比较正常,可是和王珞丹的感情戏完全没必要,很弱智,王珞丹再大叔控,爱上郭涛也太没有理由;伊警官一直在纠结,对邓超又是怀疑又是欣赏倚重,目睹邓超和吕颂贤的激情戏后神色慌乱,难道他对邓超也有某种暗含的情愫?

这部戏里gay情节沦为噱头,对推动情节作用不大,对同性恋也存在不少误解,更多是以一种猎奇的看客心态来看待,而非一种严肃的观点,恐怕会让真正的gay不舒服。作为一部电影,还希望能照顾到那些社会少数人群。比如王珞丹您能让郭涛性唤起就能证明他不是gay吗?too naive啊小姑娘,难道你不知道有双性恋的存在吗?邓超为了将伊警官的注意力引开,不惜牺牲自己的色相和肉体(甚至第一次),情节太弱智,你把同性恋都看成不管什么肉都吃的色狼了吗?同性恋也要找自己爱的人才能开心地上床的好吗?

演技方面,郭涛和伊警官层次都很丰富,邓超在小鲜肉和疑似gay的道路上越走越远,而王珞丹,由于人物设计上的败笔,根本不需要任何演技来展现。

电影里的一些细节:做个儿童心脏手术才五万块钱?几个人东拼西凑还凑不齐