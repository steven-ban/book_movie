\subsection{《白色病毒》}

标签: 悬疑

作者:肯·福莱特

\subsubsection{人物}

\begin{longtable}{p{0.16\textwidth} | p{0.3\textwidth} | p{0.5\textwidth}}

    \caption{《白色病毒》人物表} \\
    \hline
人物 & 身份 & 事件 \\
\hline
\endfirsthead

(接上表) \\
人物 & 身份 & 事件 \\
\hline
\endhead

\hline
\endfoot

托妮·加洛 & 病毒实验室的安全主管 &之前在警局干过;受斯坦利信任而对他心生爱意 \\
贝拉 & 托妮的妹妹,家里有孩子 & 对照顾母亲不是很尽心 \\
奥黛特 & 托妮的朋友,在反恐部门工作 & 给了托妮很多支援 \\
迈克尔·罗斯 & 病毒实验室研究人员,单身 & 深爱的母亲一年前去世,支持动物福利,不忍心而把兔子从实验室带出来,染病毒死亡 \\
斯坦利·奥克森福德 & 病毒实验室所属公司(奥克森福德公司)的董事长和大股东,60岁左右 & 对家庭成员包括已经死去的玛塔均充满爱意,对托妮也萌生爱意 \\
玛塔 & 斯坦利的亡妻,意大利人 & 一年半前去世 \\
弗兰克 & 托妮·加洛在警局时的男友 & 自负且对与托妮的分手耿耿于怀,一直给她找麻烦 \\
基特·奥克森福德 & 斯坦利的儿子,有大量债务 &  之前策划偷公司的钱,被托妮·加洛识破后被父亲开除,对父亲和家人心生不满;谋划抢劫奥克森福德公司里的致命病毒卖给恐怖分子以还清债务 \\
米兰达·奥克森福德 & 斯坦利的小女儿,基特的姐姐  & 前夫是个恶霸流氓,有个儿子汤姆;想让姐姐奥尔加对奈德好点 \\
黛西 & 女恶棍 & 力量很大,光头 \\
克雷格 & 米兰达自己的儿子 & 喜欢索菲,与她一起行动 \\
奈吉尔 & 做一些非法生意 & 抢劫病毒的主要领导 \\
埃尔顿 & & 抢劫病毒的一员 \\
奈德 & 米兰达的现男友,杂志编辑 & 不喜欢冲突,懦弱 \\
索菲 & 奈德上段婚姻的女儿 & 酷,叛逆 \\
奥尔加·奥克森福德 &斯坦利的大女儿,律师 & 讲话严谨,对弟弟不满,但对丈夫感情较深 \\
雨果 & 奥尔加的丈夫 & 与米兰达有旧情 \\
\end{longtable}


\subsubsection{书评}
之前,基特从公司做假账偷钱,被托妮·加洛侦察出来,然后被扫地出门。12月8日,迈克尔·罗斯通过从男性更衣室\emph{中转替换}的方式从实验室偷出兔子。12月23日午夜,迈克尔·罗斯染病毒死亡,托妮·加洛去调查,故事拉开帷幕。

12月24日,托妮·加洛被斯坦利原谅并没有免职,基特准备绕过实验室的安保系统偷药物,因迈克尔·罗斯死亡而做罢。基特伙同奈吉尔一行人,拟定了一个算是周密的计划,可以绕开实验室的认证系统:他去偷走父亲的卡,写入自己的指纹信息,并黑掉实验室的电话系统,装作电信公司的人去修理,然后黑掉监控系统,把昨天的画面替换进去,进入实验室偷走病毒。当天是平安夜,下着大雪,即使是警察也很难赶过来,并且实验室的人都放假休息,可以说是一个好时机。

故事就这样紧锣密鼓地开始了。小说采用了pov式和类似美剧《24小时》的写法,一般十几分钟或者半个小时一个单元,每个单元里采用一个人的视角来推进剧情。整个故事,就发生在24日凌晨到25日午夜这48个小时内,主要的发生地点就是病毒实验室和斯坦利的家,而又以基特一行人在风雪中抢走病毒回到斯坦利的家里为高潮——这伙劫匪因为掩饰不了而开枪,逐一把斯坦利家人找到并关起来,期间发生了肢体冲突;斯坦利的家人在面对共同的威胁下齐心协力,想办法找到电话去报警,特别是克雷格和索菲两个人去开了斯坦利的法拉利,开车逃跑途中撞倒了黛西,故事的紧张程度到达了顶点,而这一切都发生在不断的风雪和寒冷中;此时托妮已经赶到,正在想办法制伏歹徒并报警。当然,故事的结尾是个美好的大结局——众人齐心协力报了警,斯坦利劝说基特不要拿着病毒跳下去,并且病毒的买家也被跟踪制伏,更重要的是,托妮获得了斯坦利的青睐和他家人的接纳,两人确立了关系(虽然差了三十岁),米兰达认清了奈德软弱的外表下可以为了保护家人而与歹徒战斗,奥尔加在雨果被曝出与米兰达偷情后依然原谅了他,克雷格在潜行中照顾了索菲,两人确立了关系。

总之,本书是通俗或者类型小说里比较好的,层层推进的紧张气氛技法十分成熟老道,非常适合改编成电影。不过,说它能够通宵读完明显是夸大了。本书有三十多万字,一晚上读完不太可能,不过故事到了30\% 的时候就已经非常吸引人不停读下去了。故事的起因是恐怖主义,一些人想买病毒去造成大规模的伤亡,本书并没有去探讨这背后的更为深刻的社会学意义。基特是因为赌博欠钱而铤而走险去偷病毒的,他因为被父亲管教而心生不满,在家人和抢动之间摇摆并决定和家人决裂(最后听了父亲的劝),这是故事推进的主要动力。托妮是为了解决问题的,她对于安保工作的推理可以说是缜密,同时她作为一个单身女性,对斯坦利心生爱慕,不断在地纠结、狂喜、失落中的感情中循环,不过她的优秀带来了斯坦利对她的肯定和接受,相比之下斯坦利的形象就扁平的多了,就是一个成功老男人的形象,对家人比重看重,思维理性内敛,对亡妻念念不忘,同时也接纳了托妮的爱。克雷格作为一个十六岁的青少年,被酷酷的索菲吸引,想吸引她的注意,在困难中照顾了索菲,赢得了她的芳心,同时那种少男的忐忑不安、害羞心态也十分真实。本书的氛围感营造也不错,圣诞节的暴雪把那种凄厉、无助心态衬托得十分惊险。

评分:4/5。