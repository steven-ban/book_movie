\subsection{《你一定爱读的极简欧洲史》}
\subsubsection{一些感想:}
1.非常非常非常推荐的一本历史书,我觉得,好的历史书,都要这么写。很多厚厚的历史书只是一些材料和前人观点的堆砌,读起来生硬如同教科书。看钱穆先生的《中国历史政治得失》,几句话把一种政治制度解释得清清楚楚,又能联系前代做一些比较和分析,读起来轻松愉快,而《饥饿的盛世》虽然材料全而语言好,但功力实在差了好多。所以,买历史书,一定要买大家的,即使薄薄的一本,都比平常的历史学者的书耐看。

2.在普通的教科书或读物里,欧洲历史是一次次乱而又乱的混战、莫明其妙的君主和落后的庄园经济。而在本书里,欧洲历史清晰地被划分三个核心元素间的此消彼长和互相依存:希腊罗马的哲学和政治制度、基督教和日耳曼战士文化。陌生而复杂的中世纪历史在基督教保存希腊罗马文明、日耳曼承认基督教的分析下令人豁然开朗。

\subsubsection{标注:}
“不是所有的东西都属于国王”,这是欧洲自由和繁荣的基石(和保障)。

谈天气不是为了没话找话说,而是一群人在忧心自己的命运。

如今的英文,几乎所有东西都有两个以上的词汇,举“国王”和“国王的”为例,英文本是king、kingly,后来加入了royal、regal、sovereign。数量上,英文词汇要比法文和德文多出数倍——它毕竟是法文和德文的混合加总。

在19世纪下半叶之前,德意志和意大利一直是处于分裂的局面,这两个国家一直到很晚才统一,而且比那些较早统一的国家,更倾向于浪漫主义时期所萌生的强烈民族主义。这两国于20世纪采行了最具侵略性也最排他的民族主义,世称“法西斯主义”。

教皇和皇帝之争要说有什么意义,那就是教皇从没说过自己是皇帝,皇帝也从不以教皇自居。双方都承认对方的存在有其必要,争的只是彼此的相对权力。这是西罗马帝国一个非常重要的特色,也是它和东罗马帝国的分野所在。

柏拉图相信,我们在世间的所见所感,只是存在于另一个崇高灵魂界中的完美形体的影子。世界上有普通的桌子,但有一张完美形体的桌子一直存在于某个别处。即使是个抽象的观念,例如正义和良善,也是以完美的形体存在于某个他处。人类便是来自那个灵魂界,必须透过心智和精神的锻炼,才能重新发现这个完美。