\subsection{《底牌》}

作者:【英】阿加莎-克里斯汀

\subsubsection{人物}
\begin{longtable}{p{0.2\textwidth} | p{0.3\textwidth} | p{0.35\textwidth}}
    \caption{《底牌》人物表} \\
    \hline
姓名 & 特点 & 事件 \\
\hline
\endfirsthead

(接上表) \\
姓名 & 特点 & 事件 \\
\hline
\endhead

\hline
\endfoot
夏塔纳先生 & 死者,知道很多人的秘密,透露出其他人以前的丑事	& \\
阿里亚德尼-奥利弗太太 & 侦探小说家,惊悚小说家 & \\
巴特尔警司 & 供职苏格兰场 & \\	
瑞司上校 & 50岁左右,特工 & \\
罗伯茨医生 & & 曾害死病人及其丈夫 \\
洛里默太太 & 60岁左右,女权主义者,冷静,擅长打牌 & 害死丈夫 \\
德思帕少校 & & \\
安妮-梅瑞迪斯小姐 & 20出头,温柔 & 曾杀死自己的雇主,家境贫困但爱慕虚荣,试图害死闺蜜 \\
\end{longtable}


\subsubsection{事件}

用餐时的座次,是个环形:\\
\begin{centering}
奥利弗太太 \ 夏塔纳先生(主位) \ 洛里默太太 \	波洛 \	罗伯茨医生 \\
巴特尔警司 \ 梅瑞迪斯小姐 \ 德斯帕少校 \\
\end{centering}

打桥牌:
\begin{itemize*}
    \item 九点三十分开始
    \item 第一轮:洛里默太太和梅瑞迪斯一组,德斯帕少校和罗伯茨一组
    \item 第二轮:梅瑞迪斯和罗伯茨一组,对抗洛里默太太和德斯帕少校
    \item 第三轮:洛里默太太和罗伯茨一组,对抗梅瑞迪斯和德斯帕少校
    \item 第四轮:梅瑞迪斯和罗伯茨一组,对抗洛里默太太和德斯帕少校
    \item 输赢情况:“洛里默太太每轮都是赢家。梅瑞迪斯小姐第一轮赢了,后两轮输。罗伯茨小赚一点,梅瑞迪斯小姐和德斯帕输了一些。”
    \item 波洛和奥利弗太太一组,对抗巴特尔和瑞斯
    \item 12点10分打完,德帕斯少校死亡,衣服上有类似饰钉的东西,被捅一刀
\end{itemize*}

真相:杀死夏塔纳先生的是罗伯茨医生,他找准打牌的时机,故意为奥利弗太太叫了个大满贯,这样其他人都集中精神打牌,他趁牌友不注意杀死了夏塔纳先生。之后他为掩饰罪行,写假信给其他嫌疑人,同时一大早拜访奥利弗太太注射药物害死她,并脱罪给她。

这部小说没有黑斯廷斯,只有包括波洛在内的四个人分别办案的过程。案件本身没有任何诡计可言,重点在于假设和推论的过程,在这个过程中逐一查清四个嫌疑人的历史和个性,然后一个一个排除。除了凶手外,其他几个嫌犯都有前科和疑点,小说的重点就是一一调查和解释这些疑点。推理本身的比重并不大,主要还是依靠波洛对各人性格的分析并推论出可能的作案手段。我觉得在真实的办案中这种方式未必奏效,同时也使得本篇小说有点脱离本格推理的套路。

评分:7/10。