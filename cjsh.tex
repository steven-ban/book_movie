\subsection{《成吉思汗与今日世界之形成》}

标签: 历史 \  成吉思汗 \  蒙古史 \  人类学


\subsubsection{成吉思汗其人}
成吉思汗是一个伟人,一个天才。剥离掉他蒙古文明的背景,放到任何一个文化中,他都是一个不世出的领导者。他从小就格局宏大,杀死自己同母异父的兄长就是最好的证明,而这仅仅是一个十岁左右的孩童所为。蒙古虽然野蛮,但家族中依然有一定的秩序,成吉思汗这绝对是对现有权力秩序的挑战和重建。他成功了,并且从一个成功走向另一个成功。同时,他并非一个莽夫,而是有着严密的计划和对形势精准的判断。他与札木合结盟,又通过他人的力量来消灭他的势力,并且最后依然以好友(俺答)的礼节来对待他。他对自然有着崇敬,有权力和欲望也有节制,对家人礼敬,对朋友友好,这简直是一个明君的模范。因此,如毛泽东那样说的“一代天骄,只识弯弓射大雕”是不准确的,他只是于农业文明不熟悉,但在领导和人格魅力方面绝对可以和秦皇汉武、唐宗宋祖相提并论。

成吉思汗从12世纪末期开始崛起,仅仅30年的时间内,就完成了蒙古族的统一,并运用卓绝的军事天分攻克了金、西夏、花拉子模,将蒙古族的版图扩充到史上最大,给了子孙继续扩充的空间。他虽然是蒙古族,但与放牧的族人不同,他保留着亲生父亲那样的渔猎习性,将这样的习性放到军事中,而非蒙古族那样的掠夺。他可以集中优势兵力,将敌人有生力量全部歼灭(而非游而不击或击破敌人后抢劫一通就逃走)。他还具备优秀的组织能力,创立了十人-百人-千人-万人这样的层级性的军事集团,进行兵团之间的组合作战。嫡系部分的所有人,地位均高于万人单位中的所有人,这种严格的等级制赋予了成吉思汗对军队的绝对控制。这样的能力,绝非普通的军事统帅所为。要知道,他的这些创举,对于农业文明而言可能不算什么,但对于游牧民族而言(特别是蒙古族而言)却是很不容易的。他根据自己的战争经验,调整军事管理和进攻的策略,不断在战争在学习,赋予了蒙古军队卓越的战术和战略能力,为后代们开疆拓土提供了条件。

\subsubsection{蒙古帝国的兴衰}
在成吉思汗事业的基础上,他的子孙对东亚、南亚、中亚、俄罗斯、西亚、东欧及中欧开展了大规模的军事征服。从战术的角度上来讲,蒙古骑兵具有当时农业文明不具备的高度机动性。蒙古人从小就学习骑马和射箭,因此他们是天生的骑兵。他们属于轻骑兵,不负担大量的装备和辎重,只携带必要的弓箭、刀、奶制口、水袋,可以长徒迅速奔袭。他们的弓箭射距和力量优于同期的欧洲弓箭手。由于这种机动性,在大规模的野战中,他们可以快速移动,诱使数倍于自己的敌兵追击并进入他们的伏击圈,然后迅速冲垮敌人的阵型,借机屠杀敌军。相反,在围城时,他们采取围城打援的方式,并将上次屠城后的平民放入城内,散步他们的恐怖。敌人最终会因为恐惧和粮食短缺而丧失战斗力,随后的胜利也就顺理成章了。他们从东亚学会了火药,可以采用炮击的方式摧毁城墙,这大大增加了围城的效率。因此,在军事上,蒙古骑兵是当时最强的,而且在多次的战争中不断增强自己的战斗能力,扫除天下也就是理所当然的事情了。

蒙古骑兵的节节胜利,将蒙古的版图向西扩充到匈牙利、土耳其、埃及,向南扩充到印度和中国南海,向东直至太平洋,向北扩充到北冰洋,成为欧亚大陆的霸主。然而,一系列因素限制了他们继续扩张。骑兵的载体在于马,而到了低纬度地区,马就难以找到合适的草料,同时高温使马出汗和生病,因此他们止于东南亚和埃及。同时,蒙古人没有真正的水军,从朝鲜和南宁俘获的水军也没有给予他们战略上的支持,由于台风影响,他们在日本战场失败了。可见,蒙古骑兵仅仅是温带和寒带的陆地霸主,在13世纪后半叶终于到达了自己的极限。另外,中欧战场的胜利并没有给他们带了更多的收获,因此当时欧洲的人口并不比中东和东亚多,物产也少,这使得蒙古人没有合理动机继续向西征服大西洋沿岸。

军事的胜利使地区上很多文明和地方势力灭亡,从而造就了一个庞大的帝国。蒙古人不会种地,他们需要的是掠夺农业文明的物产,美其名曰“贸易”,其实就是烧杀抢,因此他们自己并没有什么东西是农业文明需要的,他们只是把东西从一个文明抢出来,再去找其他的农业文明来换。不过,这样的“交易”过程依然催生了不同地区间的贸易活动。蒙古人在文化上落后,但在技术、工匠、商人的使用上则显得更加“全球化”,也显得更加“先进”。他们设立了大量的驿站,消除了不同文明之间的政治隔离,使得从欧洲到东亚的遥远距离可以安全地相连,大大增加了不同文明间的交流和相互影响。中国的“四大发明”沿着蒙古帝国的驿站(以及之前的征服战争)传入到中东和欧洲,促进了欧洲的文艺复兴。蒙古本身信仰萨满教,对于基督教、伊斯兰教则采取了开放包容的姿态,这比当时的欧洲和伊斯兰地区则“现代”得多(这一点也得到了作者的过于拔高的夸奖)。

但到了成吉思汗孙子辈时,蒙古帝国则开始了分裂。他的子孙们没有像他那样谦逊和包容,他们展开了大汗之位的争斗,并最终分裂了成为几个汗国。在东亚,忽必烈成为最多人承认的大汗,他征服了南宋,建立了元朝,并成功地“汉化”,成为一个更像汉人的君主。他建都于北京,成为明清都城的基础。他废除了科举,将南宋的人民设为最低等的人,其实就是拒绝了更大程度的汉化。他用蒙古的法律代替了南宋更加先进的法律(这里作者显然对南宋的法律、政治缺少必要了解,认为南宋充满了腐败和低效,而蒙古人则更加先进,这显然是错误的),使读书人流落到低等的位置。元朝只稳固统治了50年,之后就陷入了农民起义的反叛声浪里,最终被南方的汉人推翻,蒙古人被赶回蒙古老家,结束了短暂的统治。而其他几个汗国坚持的时间长一些,但也纷纷被当地的农业文明同化,最终没有完成启蒙运动和工业化,从而被历史淘汰。

蒙古的历史表明,作为一个“落后”民族,虽然科技树上军事领先,但在其他方面短板太多,依然难以持续地在世界上称王称霸。蒙古人没有治理一个大帝国的经验,相配套的一系列治国政策、文化、政治也均不合格,很快就淹没在历史的洪流中。成吉思汗的儿子和孙子们为了权力而互相争斗,可见他们缺少一个有效的权力制衡机制,使得政局不稳定,于是很快就走向了分裂和衰落。

\subsubsection{对世界格局的影响}
蒙古作为一个统一的大帝国,虽然统一的时间很短,但仍然以其巨大的国土面积、重视商业的政治倾向、尊重工匠的习俗,大大加快了世界各地区之间的相互交流,使东方的科技、文化传到西方,同时也将中东的宗教传入东方。因此,蒙古帝国加快了各个地区的历史进程,特别是对欧洲的启蒙运动做出了巨大的推动作用。蒙古的君主们没有像伊斯兰世界和基督教世界那样搞宗教控制或政教合一,而是采取了包容的态度,因此各个宗教得以流传和传播。一个有趣的现象是,蒙古帝国很少去进行农业社会那样的“建设”,他们不修城堡,不修农田,不修寺院,而是修了最多的桥梁。桥梁象征一种连接,象征蒙古帝国对中古时代各个文明的连接。蒙古帝国带来的这种文化交流和文明的碰撞,随着蒙古帝国的衰落而渐渐减弱,最终在欧洲对全世界的殖民中彻底消亡。

蒙古帝国的人,对文化有一种开放的姿态,这一点受到了作者很高的赞美。蒙古帝国本身信仰原始的萨满教,他们治下的宗教有多种,他们没有设立国教,而是使各种宗教百花齐放,一同辩论。这一点,比当时的政教合一的欧洲强多了。作者花了大量篇幅,来对比蒙古人和当时的欧洲人。欧洲人无法战胜蒙古人,于是在恐惧下对自己治下的犹太人下了毒手,作者在此表达了巨大的愤慨。

\subsubsection{殖民主义下的历史建构}
成吉思汗究竟是什么样的人?不同的人会给出不同的答案。当时的时代里,欧洲的贵族和君主们认为他是上帝派来惩罚他们的,对他抱以恐惧。然而,一切历史都是当代史,进入近代以后,欧洲的知识分子对成吉思汗进行了抹黑和丑化,认为他邪恶而残忍。而亚洲人则在民族主义观念举起时对成吉思汗进行了美化,对他的丰功伟绩进行了歌颂。一切历史都是当代史,这样的重构,都只是对蒙古历史的一种再认识,然而每个人都可以基于相同的史料而得出不同的结论。

\subsubsection{书评}
本书并非单纯的历史,而是从自己的人类学专业出发,对这段历史做出的一种解释:蒙古帝国的征服和统一,为人类历史的进程带来了何种影响?也即属于一种对蒙古帝国史的\emph{文化人类学解读}。本书在历史学的意义上因此并不很大,而是提供了一种文化上的观点,当然,由于缺少足够的史料来佐证,因此这样的文化观点也不是很有说服力。作者的史学功底很差,特别是面对不熟悉的东亚史时更是暴露无遗。他认为蒙古人的法律要优于“野蛮”的南宋法律,认为元朝的文化政策优于宋朝的文化政策,这一点就是对中国历史缺少认识的体现。正如我一贯认为的那样,若非专门的中国史学学,西方普通知识分子对中国的认识完全是扭曲的、错误的、意识形态化的、模板化的,他们无法理解中国的政治文化传统,以一种刻板印象的西方视角来观察中国,由于史料的缺乏,他们很容易做出一些啼笑皆非的结论,本书作者即是其一。当然,蒙古历史如此漫长,试图在一本32万字的书里讲得面面俱到也是不可能的。

本书作者为了写这本书,翻阅了大量的典籍,特别是《蒙古秘史》这样的史料,同时开展了大量的实地走访。在认为本书即将完成时,他在实地考察中发现了新问题,于是又展开了为期五年的研究,搜集了更多史料,才最终写完了本书。可以说,作者的治学精神是很珍贵的,值得每一个学人去学习。

评分:7/10。