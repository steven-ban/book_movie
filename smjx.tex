\subsection{神秘巨星}

标签: 印度电影 \  阿米尔汗 \  家庭暴力 \  穆斯林 

电影海报:\url{https://gss0.bdstatic.com/94o3dSag_xI4khGkpoWK1HF6hhy/baike/c0\%3Dbaike92\%2C5\%2C5\%2C92\%2C30/sign=f4fb75d538fa828bc52e95b19c762a51/a08b87d6277f9e2f4fcdecd81430e924b999f362.jpg)}

尹希娅是一名15岁的印度女孩,她热爱音乐,有着成为音乐歌星的梦想。她有天赋,经常弹唱。然而,她有一个严厉而专制的父亲,他是一名穆斯林,认为丈夫对家庭有着绝对的统治。他动不动就以各种理由殴打老婆,以学习为重的理由阻止尹希娅练吉它。他重男轻女,疼爱自己唯一的小儿子,而对尹希娅十分冷漠。他虽然是一个高级工程师,受过现代科学教育,家庭条件不错(尹希娅上的似乎是类似私人中学那样的好学校),但缺乏丝毫的现代人文思想。他想举家移民到沙特阿拉伯并把尹希娅嫁给像他那样的穆斯林。就是在这样的家庭环境里,尹希娅偷偷练习弹唱,并穿着黑色罩袍将自己的视频上传到YouTube上,并因此大火。他父亲知道后,因为她母亲将项链卖掉给她买笔记本电脑并且她成绩只有30分,大怒,把吉它弦砸断,并禁止她再唱歌。

即使是在这样的环境中,尹希娅依然没有放弃希望。她在自己的小男友的帮助下飞到了孟买,在过气音乐制作人(阿米尔汗饰)的帮助下录制新歌,同时联系律师帮母亲离婚。母亲的忍气吞声让她不理解和愤怒,但从姑婆那样她了解到真想:母亲第一胎怀上她时,周围的人都让她打掉,但她从医院逃掉了并在外一年生下了尹希娅。在印度,每年都会有很多人打掉女胎,这倒和中国几十年前很想像。母亲的忍气吞声,都是为了在这样的家庭环境下多给尹希娅留下一点空间让她有梦想可以追。最终,在飞往沙特的机场上,父亲因行李过多让尹希娅丢下吉它,并接着扬言要打老婆,尹西娅和母亲忍无可忍,在离婚协议上签了字并离开。在颁奖典礼上,尹希娅虽然没有获得最佳女歌手,但获奖者谦虚地请她上台,尹希娅在台上感谢了母亲。

本片虽然是商业片,主线剧情是典型的套路和程式,但保留了阿米尔汗一贯的关注印度社会问题的特点。印度的重男轻女、因女婴而堕胎、穆斯林家庭的封建落后、家庭暴力……这些“调料”都较好地融入了整体的故事中。线索虽然多,但最终都圆了回来,收得还算合理。

人物上,尹希娅有梦想(虽然这种梦想十分世俗,想出名的比重更大一些),有性格(性格火暴,动不动就发怒,遗传了她爸),有想法(单刀赴会找音乐制片人并促成母亲离婚,这种性格似乎遗传了她母亲)。她母亲看似软弱迁就,实际内心十分坚定。这都是有血有肉的人物,使得本片的人物动机都有迹可寻。尹希娅的父亲也很出彩,虽然稍显各式化,但演绎出一个很恐怖的家暴者,让人想起中国电视剧《不要和陌生人说话》中的冯远征的表演。阿米尔汗的音乐经纪人虽然是配角,但维持了他影帝的水准,风流倜傥而不失正义。

主演塞伊拉·沃西\footnote{\url{https://baike.baidu.com/item/塞伊拉·沃西/22350724}}也是《摔跤吧,爸爸!》里少女吉塔的扮演者,不得不说这次的扮相很漂亮,和吉塔时的表演大相径庭,15岁少女的纯真、大胆、细腻都看起来滴水不漏,没有丝毫违合感,看来演技很了得。同时,和她演对手戏的小男友表演也很好。\footnote{尹希亚和她的小男友\url{https://gss0.bdstatic.com/-4o3dSag_xI4khGkpoWK1HF6hhy/baike/c0\%3Dbaike92\%2C5\%2C5\%2C92\%2C30/sign=978cc7d4fb36afc31a013737d27080a1/3bf33a87e950352a529faf1c5f43fbf2b2118b2c.jpg}}

母亲的扮演者是梅·维贾\footnote{https://baike.baidu.com/item/梅·维贾/22343934},表演很有层次感。\footnote{
梅·维贾\url{https://gss3.bdstatic.com/-Po3dSag_xI4khGkpoWK1HF6hhy/baike/c0\%3Dbaike80\%2C5\%2C5\%2C80\%2C26/sign=a092e5e9aa8b87d6444fa34d6661435d/203fb80e7bec54e7255f4dc6b2389b504ec26ab2.jpg}}

评分:8/10。