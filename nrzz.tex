\subsection{《女人之罪》}

这本小说的完成度不到三分之一,情节前后连不上,处于手稿阶段。

作者玛丽不擅长写小说,除了主角玛利亚大段的独白和回忆外,其他人物形象单薄,性格脸谱化严重,整体叙事节奏也缺乏韵律感。和《为女权辩护》比较来看,玛丽这种喜欢华丽修辞和引经据典的行文习惯更适合写议论和辩驳的雄文而非以故事见长的小说。

情节上,这是一部真正的玛丽苏小说,主角玛利亚恨不得化身仁爱睿智的圣母,不幸掉进渣男遍地的世界,被欺骗,被背叛,一次次忍让包容却换来得寸进尺和世人的不理解,最终自杀(手稿大纲里有,没写进正文)。如果相较普通玛丽苏小说有什么拔高的话,玛利亚那不屈不挠的理性抗争如同野火般在她生命里燃烧不止,以及对十八世纪末英国婚姻制度和社会风貌的鞭挞领先了时代。可是,与经过工人运动和科学革命洗礼的现代女权主义相比,玛丽的女权主义停留在资产阶级的文学幻想中,对底层人民缺乏理解和包容(字里行间流露出对贫穷的鄙弃和恐惧),对贵族生活和旧式婚姻心存幻想,缺少真正的平民主义立场。

夸奖一下翻译,文笔超棒。

满分五分的话,给三分。