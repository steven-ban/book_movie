\subsection{《三体》}
三体是一流的想象,二流的文笔,和三流的人物塑造。人物塑造之所以差,是因为这本书几乎把力量都放在了情节上,留给人物本身的东西并不多,也可能是大刘不善于描画人物,并以人物性格和际遇来推动情节发展。说实话,从人物那时说出来的话,除了史强这样的糙汉子外,其他那些科学家、知识分子、政府官员几乎一模一样,丝毫没有体现出人物的职业和性格特点。甚至于知识分子的思考,也都是大刘硬加给他们的,看不出来必然是哪个人说的。大刘还喜欢给人物添加“评论”,让人物自已把大刘的思想一股脑儿说出来。一句话,人物是为大刘的想像力和情节服务的,完全是大刘思想的话筒。文笔这所以说是二流,是因为大刘奇瑰的想像力加上对三体世界和人类未来世界的详细描画太吸引人。大刘对景物的描写还是可以的,特别有电影感和未来感,带有一种荒凉和瑰丽。

对于这本书硬科幻的一面,我想不用多说了,看过书的读者会对大刘对文明本质的思考深深迷恋,并不厌其烦地将科学元素揉入想像之中。

对于三体人而言,由于其科学水平已远高于人类,我认为他们既然都把质子展开并在其上制造了集成电路,并可以随意干涉人类的基础研究,那么,其实用智子阻止人类的技术进步也完全不在话下,根本不用再担心人类会出现什么技术爆炸了。既然智子可以阻止粒子对撞机里产生正确的结果,那么也完全可以阻止人类制造航天器或基因改造的进度。

智子本质上是个质子,因为用它来进行长距离旅行应该是很困难的,随便一个原子就可以阻碍其传播。中子不带电,所以不易捕获,这是三体人使用质子来展开的原因,但质子带正电,很容易被太空中的原子捕获,也行根本就来不到地球。

事实上,三体行星上的生物也是需要水的,但是在三体太阳系不规则气候的条件下,水根本不可能稳定存在,甚至在产生生命前,三体行星也很有可能被三个太阳之一捕获。当然,如果真被捕获,那么也没有这本小说了,所以对待这些情景设置和前提不能太较真。

但三体人在掌握如此高的科技水平的情况下,特别是已经有了外太空航行技术,完全可以在远离三体恒星的质心的轨道上建立地外居住点,如此一来就不用担心太阳活动的不规则,可以将三体运动对行星的影响降到最低限度。当然,距离恒星太远会太冷,但这只是能源问题,相信这对三体的逆天科技来讲根本不是个阻碍。

小说里的人物可以用脑残来形容,主角一个一个都有自杀倾向,但您想自杀就自杀吧,别都带上其他人啊!不能想你地球的政治家们会让罗辑这样的人来做什么“面壁者”和“执剑者”,完全可以出现一些更加强势更加有责任心的人来做这些事情。三体人更脑残,轻易就被罗辑吓唬住了。说实话,有了地球文明作为对照,三体人对待黑暗森林的态度应该更乐观才是,更害怕的应该是地球人。再者,即使人类毁灭,顺便把太阳系也毁灭掉,三体人也完全不用在太阳系这棵大树上吊死。我觉得以地球人的能耐,即使两百年后科技大发展,想把太阳能灭掉还是不可能,这点大刘太乐观了。从情节上说,智子完全可以掌握戴在罗辑腕上的开关,并截获所有相关信号,让地球三体组织再做出一个来,再代替罗辑控制毁灭装置。反过来讲,罗辑这种挟持地球文明和全人类生命的愚蠢行为早就应该被枪毙几百次了,完全不是什么英雄。万一他手上的装置出了点什么故障,三体人没来,地球人就已经灭绝了。

看完《三体》我又看阿西莫夫的《银河帝国》,以前说大刘文笔差,这才发现阿西莫夫也不遑多让。希望理科生们好好锤炼文笔,好好理解社会。社会和人性不是理工科里对象的非黑即白和理性,而是丰富的多。