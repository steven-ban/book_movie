\subsection{《黑洞与时间弯曲》}

\subsubsection{评价}

这是一本黑洞研究的活历史,对黑洞研究的方方面面的概念提出、争论到确定都有介绍。

行文严谨,可喜的是每个概念都有出处,解释很严格,和普通物理书的区别仅仅是没有那么多方程和公式,适合物理学的学生阅读,普通理工科的人去阅读可能有点艰涩。

译者文笔很好。

\subsubsection{一些标注}
\begin{itemize*}
	\item 阿尔及利亚哲学家诺贝尔奖得主阿尔伯特·加缪所作的《西绪福斯神话》。
	\item 光的波长只有半微米,主要是留在恒星和行星大气中的热原子发出的,所以它为我们带来了关于这些星体的大气的信息。无线电波的波长长1 000万倍,主要是在磁场中近光速螺旋运动的电子发出的,于是,它向我们坦白了星系核射出的磁化喷流,吞没喷流的巨大的星系间的磁化射电叶,以及脉冲星的磁化辐射束。X射线的波长比光短1 000倍,大多数是被吸积到黑洞和中子星的超高热气体中的高速电子发出的,因此它直接反映了黑洞和中子星吸积气体的情况。
	\item 当生成黑洞的坍缩恒星消失在黑洞视界里时,它也失去了以任何方式影响黑洞的能力;最重要的是,恒星的引力不再是黑洞的维持者。这时候,黑洞还能继续存在完全是因为引力的非线性:没有了恒星,黑洞时空曲率仍将继续产生其非线性;这样自我生成的曲率像非线性“肢”一样将黑洞粘在一起。
	\item 相反,时空曲率大(如在大爆炸或黑洞附近)时,爱因斯坦广义相对论引力定律预言,曲率是高度非线性的——是宇宙中极端非线性现象之一。
	\item 为了寻找黑洞,天文学家应该找那些光谱表现出周期性的红一蓝一红一蓝频移的恒星。这类移动是恒星有一颗伴星的确凿证据。
	\item 谁也不能根据黑洞的性质来判别形成它的恒星的构成是物质的还是反物质的,是质子的还是电子的,或者是中子的还是反中子的。借惠勒的话,更准确地说,黑洞几乎无毛,它仅有的毛是质量、自旋和电荷。
	\item 如果一个物体(一颗星,一个星团或者别的什么)经历高度非球形压缩,那么只有当它在各个方向的周长都小于临界周长时,它才会形成包围自己的黑洞。
	\item 改变我们对黑洞的认识的那些年轻人是谁?是三位杰出先生的学生、博士后和学生的学生;那三位先生是:美国新泽西州普林斯顿的惠勒(John Archibald Wheeler),苏联俄罗斯莫斯科的泽尔多维奇(Yatov Borisovich Zel'dovich)和英格兰剑桥的席艾玛(Dennis Sciama)。他们通过他们的徒子徒孙,在黑洞的现代认识上留下了自己的烙印。
	\item 时间卷曲的一个结果是,从恒星发出的光会经历引力红移。因为光的振荡频率由光发射处的时间流决定,从星体表面的原子发出的光在到达地球时,将比从星际空间的同类原子发出的光具有更低的频率。频率降低的量完全与时间流变慢的数量相同。较低的频率意味着较大的波长,所以,来自星体的光必然会以星体表面时间膨胀的数量向光谱的红端移动。
\end{itemize*}