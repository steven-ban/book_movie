\subsection{《国民党的“联共”与“反共”》}

标签: 历史 \  党史

作者:杨奎松

本书是著名党史专家杨奎松的的力作,系统梳理了国民党从孙中山提出联俄联共到1949年国民党败退台湾前将近30年的国共关系史。本书以国民党为视角,重点讲述了国民党内各重要人物、派系对共产党的态度和政策,其中又重点以孙中山、蒋介石、汪精卫等关键人物为中心,探讨了他们对共产党态度的转变过程,可以说非常细致入微地展现了那段时间的国共斗争史。

作者在前序部分就明确指出,现有国共关系史如中国大陆官方那样重点以共产党为重心,缺少对国民党方面史料的整理与分析,特别是对新出现的如蒋介石日记、台湾方面出版的史料的忽视,造成了叙事上的单一。当然,作为外行,我无法对杨奎松本身的治学手法有任何否定的资本,不过这本书不仅有详尽的史料,还有对于史料的地道分析,同时语言极为流畅易读,是不错的历史著作。

就“国民党的联共与反共”这一主题来看,国民党的对共政策经历了以下几个阶段:
\begin{enumerate*}
    \item 孙中山晚年。孙中山由于护法运动的失败,转而思考国民革命的失败原因,同时十月革命和随后苏维埃革命,使他萌生出联合工农、苏俄和共产党的想法,同时在党建上加强思想指导,防止军阀坐大影响政治进程。这一时期,孙中山对新生的共产党,是极为赞赏的,虽然他对于共产主义没有兴趣,也反对共产党的激进手段,但仍然认为让做事积极、紧密联系工农和苏俄的共产党加入国民党,从而带领国民党走向新生是必要手段。他遏制住党内的反共声音,与共产党保持了良好的关系。
    \item 孙中山去世后,国民党内的反共声音开始走向前台,国民党元老中反共的人士组成“西山会议派”,对当权的汪精卫等人施加压力。这其中,汪精卫还是极为赞同联共的,而蒋介石为获得党内权力,则渐渐与反共人士组成联盟,以取得后者的支持。北伐中工农革命让国民党内的大地主恐慌,蒋介石走向杀害工农(手段极为残忍卑劣,暗杀、流氓公然打杀完全还是帮会做派)和共产党的反革命道路,汪精卫等人也在这个过程中妥协,两党走向分裂。此后,国民党对对共政策,基本就是蒋介石的个人独裁决定。
    \item 之后的土地革命时期,共产党开始掌握武装与独立政权,与国民政府展开军事斗争。然而在抗日的背景下,特别是西安事变后,蒋介石迫于国内外压力也不得不与共产党展开合作,无法再做到“赶尽杀绝”,共产党由于地盘小和抗日因素,也需要与国民党进行妥协与合作,于是两党开始谈判,整编军队(保留军队总是一直是之后两党谈判的重心,共产党想要保留更多军队,而国民党想让共产党放弃军队的指导权),一致抗日。
    \item 抗日战争中,共产党的实力大为发展,这引起蒋介石和党内反共人士的猜疑,皖南事变是这一矛盾的总爆发。之后两党一直有谈判,问题的焦点在于是否可以让共产党保留军队以及保留多少。这一阶段,共产党是一边斗争一边谈判,某种程度上掌握着谈判的主动权。
    \item 抗日战争胜利前夕,苏联出兵,形势突然对中国有利,于是两党的对立情绪开始升温,并着手战后的斗争。共产党这时的手段极为成熟,一方面抢占根据地,一方面政治上施压,姿态极为主动。战后全国均向往和平,两党开始接触谈判,但在共产党保留军队这一老问题上僵持不下,最终走向军事斗争,然而国民党很快失势,政权崩盘,之后的“划江而治”成为空想,此时共产党也放弃谈判,态度变得强硬,准备武力夺权全国政权,国民党失败,“转进”台湾。
\end{enumerate*}

孙中山作为革命先行者,在国民党内具有极高的地位,但他个人独裁气息很浓,对待工农和其他党内同志也是高一截的自负态度,而蒋介石继承了孙中山的这一独断性格,并且更加极端、独裁、好权。蒋介石待人并不真诚,喜好猜忌,这事实上造成了他的孤立。而且,他在立场上反共,在认识到也极为自负(孙中山也自负),乃至于影响了自己的判断力。在解放战争这样的节点上,他已经与共产党斗争了二十多年,还动不动以匪相称(立场问题,情有可原),认为共产党军队是“乌合之众”,这就是极为错误的了。在军事上国民党一直被胖揍,在政治上一直被共产党借势(和平主张),这样的对手,怎么可能是“乌合之众”呢?更何况这时的共产党,已经掌握割据政权达到二十年,根据地建设已经有了不少成效(能在如此恶劣的条件下坚持二十年本来就是很了不起的成就了),这时候蒋介石应当做的,显然是要对共产党有一个清晰的认识:他们是极为厉害可怕的对手,与这样的对手相斗争,需要谨慎,需要以更大的精神和魄力,需要联合更多的力量,否则就会像后来发展的那样失败。蒋介石如果想的话,完全可以搜集解放区的情报,研究共产党治理社会的手段,从而得出它真实的治理水平,但是他显然没有这么做。

国民党败退大陆后,蒋介石痛定思痛,认为共产党组织能力强,国民党组织能力弱,总结了“经验教训”,但仍然未能认清共产党之所以成功,根子在于顺应了国内革命的大势,在于团结了最广大的贫苦大众,在于从思想到行动上都保持了对改革社会的热情,而国民党则相反,不仅放弃了孙中山的“联合工农”的主张,还走向了反面,与张绅和买办成为盟友,站在了工农的对立面。经历如此大的挫败,仍然认识不到根本原因,这就说明蒋介石的自负已经到了自欺的地步。国民党的失败,一方面这是领导人的这一性格使然,但根本的原因,还是在于党建的涣散,派系林立,虽然上层独裁,但做不到有效的集权,因而往往互相扯皮;更为深层次的原因,则在于站在了革命(特别是土地革命)的对立面,与土豪劣坤、帝国主义结盟,被有效组织起来的共产党和红色政权蚕食。

站在现在回望这段历史,不仅为国民党难堪大任感到惋惜,也同时感到共产党的理想主义、组织能力、军事能力极为高明,在如此困难复杂的局势下,能够力挽狂澜,夺取胜利,实在是实至名归。这其中,共产党上至党中央,下至基层组织,都能调动起积极性,敢于斗争、善于斗争,成为历史的创造者。

评分:5/5。