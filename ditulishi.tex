\subsection{《地图中的历史》}

英文名:Maps that Changed the World

作者:【英】John O.E. Clark

这本书的书名有一定欺骗性。虽然讲的是“地图中的历史”,但其实是地图的一些简要介绍,特别是各种类型的地图(如天体图、各大洲图、地铁运行图、鸟瞰图、军事地图等)是如何出现的,跟实际的“历史”并没有太大关系。本书以一节一节的形式展开,每节介绍一种或一个地图,有一定的知识性,但都比较枯燥,真正喜欢历史的,未必会看得上这种粗浅的介绍。

地图的绘制涉及测量学、天文学等知识,需要实地测绘,精确性是第一位的。同时地图的绘制又代表着人们对某一地或世界的认知或政治态度,因此往往受到当时政治实体的主观性的影响。比如中国地图上要把台湾划进去,显然很多反华国家就不会这么做。以此而言,作者的论述是很精辟的:几乎所有地图都是对世界片面、带有偏见的展示:它们不得不在各种区域、形状、距离和方位的准确性之间寻求平衡(这些需求是冲突的)——哪怕是制图者唯一目标是符合地理或自然规律。

书中有大量的军事地图,采用当时军事事件的方式来介绍,比如布尔战争、一战、二战等,对不了解战史的人而言,比较枯燥。另外,书中有大量美洲殖民时期的由某一方制作的地图,都深入自己政治立场和信息收集甚至实地勘测的限制,比较片面,借此也可了解历史上各势力对美洲的争夺。

作者字里行间都是英国人对这个世界的看法,遇到法国德国就黑着来,遇到自己则站在英国立场,虽然不明显,但都能看出来,这也属于英国人的传统手艺了。

评分:2/5。