\subsection{《费正清中国回忆录》}

英文名:Chinabound: A Fifty Year Memoir

作者:【美】费正清(John King Fairbank)

标签: 费正清 \ 美国汉学 

\subsubsection{标注}
有些人不相信汉语拼音书写体系是最恰当的学习方法,甚至怀疑是一个阴谋。(事实上,它过去确实是公开地限制人们进入上流社会的阴谋)。(这句话没明白,汉语拼音是近代才有的东西,中国学字都是靠开蒙读经,不是靠拼音。)

让中国通过现代化强盛起来以反抗日本侵略,这是蒋廷黻和其他人士支持南京国民政府的主要原因。

美国的外交政策分为三种情况:向东对欧洲采取避免卷入战争的“我们不介入”的策略;向南对拉丁美洲则采取门罗主义的“你们别介入”的策略;而向西越过太平洋采取门户开放的“我们都介入”的策略。

共产党信奉“重视自由主义者所重视的改革和自由,并谴责国民党的罪恶。这使得自由主义者将共产党视为自由主义者,也使得国民党右翼将自由主义者视为共产党”。我仍然试图使人们理解,共产主义在中国是正义的,但在美国是邪恶的。这是客观的实际情况。

\subsubsection{书评}

这本书内容极其丰富,厚厚的一大本,可以说把费正清从出生到80年代之间几十年的与中国研究有关的过程记录地面面俱到,细致了他办公室中的某个秘书的生活习惯。

费正清是一个真正的“中国通”。他三十年代、四十年代都长期居住在中国,七十年代还再次回到大陆;她交游广泛,与林徽因、金岳霖、乔冠华、陈立夫等不同派别、职业的人都有接触。他三十年代乘车穿越华北平原,见到了中国最为真实的一面,这是美国的政客和普通学者很难做到的事情。他站在一个美国人特别是美国官方(这一点特别重要)的角度观察和理解中国,对当时的国民党和共产党都保持着距离,他见识到了国民政府的腐败和无能,在1943年就预见到了国民党的失败;他反对共产主义,但认为美国政府不应当完全关上与新中国合作的大门,“对于中国革命我们应持理解的态度,而非冲突和歇斯底里”。他在麦卡锡主义疯狂的五十年代初也受到了美国政府的怀疑和讯问,而“虎踞台湾”的国民党又把大陆的失败部分归责于他,可谓两面不是人。他与基辛格谈话,对中国的理解可以说某种程度上促使了后者与中国在外交上的接触。更不要说,他基于自已对中国的了解,二十年代就开始研究中国,建立哈佛大学的汉学中心——哈佛费正清东亚研究中心,培养出了大批的汉学家,对西方的中国研究有很深刻的影响。

他的学术著作,我没有读过,也非史学专业,因此无法评价。不过,他对于中国乃至远东学术的整体性是有洞见的,“对中国问题和日本问题作不同的分卷论述令我深感遗憾,因为它违背了将东亚文明作为一个整体的华夏文化区的本义……通过华夏文化区(包括越南)的概念来体现中国文化的主导地位自然不得不放弃。不过,这种主导地位总有一天会东山再起的”。

当然也要看到,他毕竟是美国人,对当时中国的看法依旧不是平视而是俯视的。“我的研究方向是通过对中国的研究来救赎美国,尽管这个方向可能更加狭窄与专业化。”他妻子威尔玛(费慰梅)曾经买走巩义和洛阳石窟中的佛像和其他文物,客观上导致了中国的文物流失,损害了中国的国家利益。他反对共产党,虽然与某些共产党员交流沟通,但仍旧是站在一个美国自由主义知识分子的角度上来看待这件事实,而没有看到共产党在中国革命中取得政权的必然性。他对中国的了解,虽然可以说是全面和深入(相对于其他外国人而言),但是还是没有中国人那么深刻。他虽然在中国游历四方,但主要还是和一些人文学科的人交流,缺少对中国第一手的经济、政治、军事资料的研究。他对下层人士的想法可能不够重视,不知道中国占绝大多数人口的农民、工人是怎么想的,怎么理解这个国家的,怎么构画国家的未来的,这导致他对中国近代的理解,停留在中国是铁板一块的认知上,没有从内部结构上来理解中国,这一点就不如后来的孔飞力。他在四十年代离开中国后,中国有三十年时间没有再来中国,没有经历过中国这三十年翻天覆地的变化,对中国共产党的评价可能也受到了冷战思维的影响而变得更加负面。不过,有一说一,他一直提倡美国与中国共产党展开对话,关注中国,这一点是其他不搞中国研究的政客所缺乏的。

中国人近代以来对美国的态度是复杂而微妙的,而费正清体会不出一部分中国人对帝国主义的仇恨:
\begin{quotation}
对我而言,“帝国主义”只是一种虚设的意识形态术语,它被如此广泛地使用,以至于现在它包括了所有的国际交往,因而没有实际意义。福特基金会对于学术研究所提供的资助可以有力地推动研究工作,只靠着道德上的故作姿态是无法达到的,无论这种姿态及其含义如何正确,都是如此。从这个角度来说,我显然乐意做一名帝国主义者。
\end{quotation}

帝国主义当然不会虚设的意识形态(真要这么说,自由主义难道不是虚设的意识形态吗?),而是中国近代以来受西方列强欺负的自然而然的感受。

他对于苏联官方管制学术研究和言论表示反对和鄙视,对中国文革中的无知和政治挂帅这都表现出不屑:
\begin{quotation}
我对接受一个彻底的意识形态的立场,或者假装赞成这种立场的人,有一种天生的厌恶感。我有个人的立场,决不会跟着某一个政党的路线走。
\end{quotation}

他对于台独势力也做出正确的判断,当然,让学者帮助决策有必要性,但是属于夹带私货了:
\begin{quotation}
    我对于任何“台湾独立运动”的力量表示疑虑,并对以公民民主投票的办法裁定民族自决在中国政治思想上的合法性表示怀疑。让无知的大众举手表决这种公议办法,永远比不上由学者为统治者出谋献策那样有效。
\end{quotation}

二战以后,费正清被派到中国来,是作为大使和情报相关人员的,美国政府想利用他们这些学者的能力服务于自己的国家利益。费正清自己也清楚,不忘对美国政府进行告诫(1943年):
\begin{quotation}
为了我们在此的长期利益考虑,我们必须鼓励培养那些具有领导能力且按着我们想要的方法发展的中国人。所有人都认为没必要与留美归国学生保持联系,然而事实上,他们是学术领域的领袖人物,是我们对华主要的人才投资,同时也是连接中美的桥梁。如果没有他们,我们就会陷入不幸的处境,就如同与苏联的关系一样。
\end{quotation}

他进面提议应当从管理上提升中美文化交流。可看长期以来,美国就希望通过这种“文化感召”和福特基金会的方式来影响中美关系,并导入向自己的利益。这是“颜色革命”和思想渗透的雏形。

费正清的时代里,中国在世界上的地位可能与印度、土耳其这样的国家差不多,虽然人口众多,但经济落后,国力衰弱,长年战争,西方人对待中国人,并非以一种平等的姿态,而是居高临下,这在这本书的字里行间里也能看出来。但是,费正清基于一个治史者的学术自觉,依然去认真了解中国的方方面面,并融入他对美国政府做出的政策建议中。到了2021年的今天,中国已经成了世界上第二大经济体,在经济、科学、军事等方面做出了极其耀眼的成就,但西方研究中国的所谓汉学家们,却长久以来陷入到一种唯意识形态论的陷阱中。他们认为,中国是一个极权和专制的国家,民众没有民主权利,各项自由受到限制,要么会在发展中崩溃,要么就成为强大而邪恶的要粉碎西方一切的恶龙。媒体或者普通大众这样认为确实无可厚非,毕竟这有很多国际政治上的考虑,但是那些汉学家们,是不是有在中国各处走一走、看一看,并潜心研究中国的政治制度和现实呢?现在看来似乎并没有。这真是令人遗憾和感慨的事情,一个互联网大行其道的社会,反而让很多学者的视野更加封闭了。如果费正清看到今天美国精英对中国的认识,看到今天中美关系走到将要对抗的地步,不知道会怎么想。

评分:4/5。