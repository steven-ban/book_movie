\subsection{《世界经典侦探推理悬疑小说大全集》}

虽然名称是“侦探推理悬疑小说”,但由于题材相近,叙事手法相近,还是统称为推理小说吧,虽然很多篇并没有多少推理成分。

之前看过的推理小说,除了很早很前柯南道尔的《福尔摩斯》,鬼谷女的《碎脸》《失魂雪》,东野圭吾的《白夜行》《嫌疑人X的献身》《秘密》《解忧杂货店》《梦幻花》,松本清张和阿加莎的一些小说之外,对推理小说没有什么涉猎。
推理小说有非常精确的流派,大体分为本格派和社会派,前者重推理,后者重动机。但有上面的很多小说,推理都有漏洞,主要还是动机更吸引人。即使是推理有优势比较烧脑的,也是重犯罪设计而非严谨的实施。事实上,很多推理小说中的犯罪手法过于复杂,反而更容易露出破绽。

本篇小说均是中短篇,也有很多上述问题,很多小说也比较平庸,因此挑一些印象深刻的来说说。

“悬疑篇”里的故事基本不涉及推理,只是叙事手法上比较间接,造成了悬疑的效果。平均来看,还是“推理篇”里的故事更好看一些。

\subsubsection{《火山喷火口杀人案》}

案件本身并不复杂:高中同窗兼同寝的柿达沼也和香取馨都是体育和学业兼优的人,但香取馨在追求柿达沼也的大妹美代子后始乱终弃,美代子自杀,于是柿达沼也立志报仇。一年后,沼也写信给香取以及其他三个室友,召集他们到火山口决斗,设计在火山口的平台上,谁能过去再回来就胜。沼也先上去(练习了很多次),在平台上放了烟,但烟里有麻醉药,他预计香取馨也会上去并且依他的个性也会点烟,因此在回来的路上会跌入岩浆,借此神不知鬼不觉除掉宿敌,而且香取确实中计接受了挑战。

这个小说如果仅仅是这种流水式的复仇式设定,那跟其他篇小说也没啥区别。但本小说胜在以第一人称“我”在十年后回忆的口吻写成,“我”也是室友之一,爱上了沼也的二妹登志子,而决斗前香取已经向登志子求婚,一直被香取压制和污辱的“我”悲愤交加,在香取还的麻醉药刚发作时就举起石头砸向香取,让他跌落走道而坠入岩浆身亡。

因此这其实是一个双线故事,故事的大部分时间都是以“我”的视角来讲一个压抑和复仇的心理经历,沼也的计谋只在最后和盘托出。整个小说语言简练,气氛和心理渲染得很到位,不读到最后一行根本不知道事情的真实。同时故事跨越了“初遇登志子-大学再见香取-决斗-五人分开对犯罪事实秘而不宣-达也写遗书披露事实”的时间段,又以回忆口吻写成,很有时间的交错感。

同时,在叙述时极尽细节之能事,心理渲染层层递进,文采相当了得。因此,几乎是一个满分的短篇了。

\subsubsection{《万无一失的谋杀》}

典型的日式社会派推理,谋杀者的视角。

周吉是做股票交易的,出身普通,他的老婆真弓和“麻友”画家和佐十郎偷情,被周吉发现,于是他构思了一个“万无一失”的连续杀人计划。

杀害和佐十郎很简单:在他轿车里放上冰块,冰芯是安眠药,因此随着冰块融化,安眠药或麻醉药释放,和佐十郎就会意识不清并发生车祸。

随后是杀害妻子真弓,核心是伪造不在场证明。周吉利用家里匾画上玻璃在两层均可留下指纹不容易发现的原理,编造故事做伪证来欺骗警察。

得意、愤怒的周吉浪起来了,在杀害妻子真弓前把计划和盘托出。尽管布局严密,妻子真弓还是一眼看出破绽,但这没有组织周吉的谋杀。当面对警察时,周吉傻眼了:妻子真弓曾经让画店老板修理过匾画,上面留下了老板特殊的指纹。

本篇堪称精品,罪犯角度的叙事让人体会到复仇的怒火和残酷的夫妻关系,充满日式小说的阴暗和绝望。
五分。

\subsubsection{《奇特的珠宝窃贼》}

阿婆的短篇,没看懂。

\subsubsection{《雨中的半耳男人》}

作者横沟正史,没有什么推理,不耗费脑细胞。

故事也很老套:父亲为了挖金出海,发生海难死亡,两名海员逃生,分赃不均。多年后一名海员将财宝留给船长父亲的子女,另一名海员来抢,最终被男主擒获。

人物没啥性格,不值一看。

\subsubsection{《黑手帮》}

作者:江户川乱步

标题很惊悚,情节很狗血。

案件实施难度很高,现实中不可能成功。

\subsubsection{《敦厚的诈骗犯》}

作者:西村京太郎

一个落魄的演员,为了骗保让自己的儿子上大学,故意敲诈一个交通肇事者的理发店主,用自己的演技激怒他,最后理发店主杀死了演员,遮掩了自己的交通肇事罪,同时演员在最后一刻让理发店主说死亡是过失而非故意。

这是一个用谎言和善意来掩盖罪恶的故事,很猎奇,但道德观很扭曲。演员本身性格温厚,热爱电影,但演技拙劣,一心想骗保,但在惊弓之鸟的店主这儿却被深信不疑,虽然真的求死得死,也算满足我自己的电影梦。店主的憋屈愤怒让人感同身受,作者的描写功底还是很深的。

\subsubsection{《塔楼奇案》}

作者:莫里斯-勒布朗

侦探瑞宁运用推理揭露了奥坦丝叔叔20年前对兄弟谋财害命的丑事,解救了奥坦丝。

推理的基础没有给读者展现,基本靠脑补,不是本格推理。

\subsubsection{残酷的确证}

作者:松本清张

丈夫不善言谈郁郁寡欢,妻子喜欢聊天颇受欢迎,这样的夫妻组合能发生什么样的故事?松本清张这部短篇里,展示出典型的日本式夫妻生活。

丈夫带客人(如同事)来家里吃饭,妻子都会侃侃而谈主管交际局面,长此以往颇受同事称赞。同事和自己都经常出差,但却是错开的。自卑而缺乏安全感的丈夫据此怀疑妻子可能和同事出轨,但没有证据。

于是他想出了一个奇葩的点子:主动去找一个脏的站街女,让自己染上淋病,然后借性事传染给妻子,再由妻子传染给同事,这样就能发现他们的丑事。

这样奇葩而中二的想法可能也只有日本人能想出来了。且不说染上淋病很难治好,坑了自己和妻子的身体健康,难道这种事情就不能开诚布公地谈一谈吗?或者花钱找个侦探来调查一下也好啊?(当然,丈夫不是没有想过,但就是不想花这个钱。)

总之自己和妻子顺序地染上了淋病,同时丈夫发现同事似乎也有类似的病,因此高兴的断言他们之间确实存在丑事,并兴奋地讽刺同事,完成了自己的报复。

这种猥琐阴暗的心理太可怕了,也太可怜了。

随后他发现妻子被人杀死在家里,并读到了真正通奸者的信。原来妻子的那个情人是肉店老板,两人情投意合决定放弃自己婚姻里的性事。但肉店老板发现了妻子身体不干净,气愤之下杀死了她。

对于丈夫来说,真是赔了夫人又折兵。

这个故事具有反转和巧合,算得上是一个精巧的故事,但构思痕迹明显,并非一流的故事。

同时主人公崩坏的三观和日式夫妻间的隔阂,恐怕在其他的社会中也不是很普遍的现象。

\subsubsection{《牙齿》}

作者:(日)水上勉

纵贯30年的复仇与谋杀:30年前,铸铁厂工人将工友的尸体推入1000多度的熔炉中,30年后又被死者的弟弟以相同的手法报复。

手法残忍猎奇,如同那1000多度的火红的炉火一样。典型的日本式的绝望与冷酷。

人物众多,但手法高超,司机与警察的调查工作如同两条线交汇结合,叙述上很有美感。

评分:5/5。

\subsubsection{《箱子》}

作者:日下金介

情人勾引妻子害死丈夫,诱骗借丈夫钱的店主到家里蹲桩,妻子将店主锁在家里的箱子里,情人用安眠药迷晕丈夫勒死他,随后把店主放出来。

情人让寿司店主为自己做不在场证明。但事与愿违寿司店主发生车祸,不在场证明成为谎言,从而露陷。

案件简单,但夹杂回忆的插叙手法如同电影闪回,镜头感十足。

\subsubsection{《死者的暗示》}

父子两个刑警以对话的方式揭示案情并展开推理,角度新颖。

结尾并没有给出结论,但推理的过程犹如抽丝剥茧,令人欲罢不能。
