\subsection{《浮生六记》}

作者:【清】沈复

沈复,字三白,号梅逸。本书著于1808年,书名来自于李白“夫天地者,万物之逆旅也;光阴者,百代之过客也。而浮生若梦,为欢几何?”。本书刊行于1877年,之后风靡书坊。本书按目录,共分六部分,后两部分为托名伪作:
\begin{itemize*}
    \item 闺房记乐。
    \item 闲情记趣。
    \item 坎坷记愁。
    \item 浪游记快。
    \item 中山记历。
    \item 养生记道。
\end{itemize*}

本书属于笔记体,行文自由,语言率真活泼。本书最为亮眼的地方,就是作者与芸娘从小时相识、相恋、结婚及婚后生活的点点滴滴。青梅竹马是何种情形?妻子和知己合二为一是何种状态?古代文学很少描绘,最为详细的当为《红楼梦》,宝玉与黛玉之间的感情状态是少有的细致而感人,但黛玉早死,他们若结婚又会如何?我们不得而知。而《浮生六记》把这种灵魂伴侣的感觉都描绘出来了。芸娘(陈芸)少时读书,有生活情趣,是作者的表姐,与作者算是青梅竹马,眼界情趣相近,这在古代是少有(至少是少见)的伉俪情深并且双峰齐立比翼齐飞。林语堂称赞芸娘是“中国文学中最可爱的女人”。然而,就是这样一个可爱的女人,却因贫因病四十一岁而亡,让人唏嘘。《闺房记乐》《闲情记趣》有多少让人温馨、莞尔,后面的《坎坷记愁》就多么让人揪心和难过。当然,《浪游记快》里的游记都是蜻蜓点水,我不太喜欢。

越美好的东西,它的消亡越是具有悲剧性。如果沈复能对家庭对妻子多一些责任,去多挣一些钱,不要把太多时间花在游山玩水、听戏赏花中,甚至在与兄弟的宅斗中能多一些心眼,取得父母的支持,岂不是可以避免悲剧的发生?当然,作者本身就是父亲过继给了亲戚,在出游在外的时间(比如去广州)家中虽有芸娘,但仍然与倡伎为伍,自己潦倒借朋友的房子住了一年多,这让现代人不齿,不过在那个年代,也不算十分出格的行为,只是作者醉心于游乐,显然没有尽到传统社会中一家之主的责任。芸娘虽好,但情商不足,管家能力不高,内有性情但过于外露,行事不过脑子,竟然鼓动给公公纳妾,失欢于婆婆,被妯娌利用打压、挑唆,乃至于一败涂地。总的来说,夫妻两人都没有理家之财,做好人有余,而自保手段不足,足可为今天的人借鉴。

作者出生于幕僚家庭,终身未举,虽然出身不算低微,但不治生产,中年之后渐至潦倒,最后也走上幕僚的道路。虽然他们的生活,虽然不是吃糠咽菜,甚至诗酒游玩、赏花作画、伴妓听优,但也不像《红楼梦》那样钟鸣鼎食,以至于受兄弟排挤、父母厌恶,发生宅斗,没有继承什么财产,被兄弟赶出家门,芸娘后期生病香消玉殒,作者的儿子早死,女儿给人做童养媳,作者也潦倒到四海为家给人做幕僚。总之,这部笔记,包含着作者大半生的酸甜苦辣。就算是后两部是伪作,但《养生记道》中关于生活应当少欲、规律的习惯,也值得现代人借鉴。

评分:4/5。