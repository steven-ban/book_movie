\subsection{《论中国》}

作者:【美】亨利·基辛格(Henry Kissinger)

基辛格这个名字对于中国人并不算陌生。1971年他作为尼克松的特使到达中国,与周恩来留下了惊天的握手照片,成为中美关系正常化的一次破冰行动。随后,在基辛格的推动下,中美关系在冷战中持续升温,成为改变历史进程的重要力量。他历经毛泽东时代、邓小平时代和江泽民时代,成为中国外交甚至是中国历史的重要见证者。在这本书中,基辛格作为一位中国专家,不仅带来了中美外交中的独家一手资料,还对中国本身进行了深刻而独到的思考。

基辛格认为,中国的外交观念与西方不同。西方人经历了近代以来的外交变革,考虑问题更多是以实力为支点,而中国人在传统的朝贡体系在作为地区的支配性力量有几千年,对待周边的外国人是以“蛮夷”的眼光来确定姿态的,中国不需要主动与它们交流,只需要维持自己的正统即可。如果局势对中原王朝不利,中国人也会以文化上的高姿态来挑拨各方关系。在外交思维上,基辛格认为中国人的外交行动受“围棋”影响甚大,讲求势的变化和包围的重要性,特别是对敌方的即使是没有直接军事意图的行为也会感到压迫。然而,近代以来,由于中国的落后,被强制性地打开了国门,遍尝屈辱,在外交中缺少发言权,对西方列强有求必应。

到了1949年以后,中国的外交由于意识形态和地缘政治的影响,不得不一边倒向苏联。然而,苏联和中国有不同的国家利益,斯大林和毛泽东不可能因为意识形态相同而损害各自的国家利益,因此双方的合作时刻隐藏着巨大的分裂危险。这种倾向在朝鲜战争中体现出来:苏联意图在朝鲜施加巨大影响力,甚至想在中国东北维持军事力量,但毛泽东知道中国近代以来的领土上的屈辱,无法坐视苏联控制中国东北。两国虽然签证了军事合作条约,但在金日成的统一朝鲜的过程中均不想失掉本国利益,苏联更不想与美国兵戎相见。最终,中国在朝鲜打仗,苏联进行了后方援助,中国也失去了收复台湾的好机会。

中苏关系在整个50年代都伴随着争吵,最后走向了决裂。中国提出“和平共处五项原则”,这和社会主义国家的结盟相违背;毛泽东以巨大的牺牲决心,扬言不惧怕核战争,同时暗中发展核武器,开展“金门炮战”,这让赫鲁晓夫很不满意。同时,中国也不可能脱离社会主义阵营,因此与美国的交流一直中断到70年代。到了60年代,中苏之间开始交恶,而中国力量远远弱于苏联,在漫长的国境线上保持着巨大军事压力,因此有动机打开外交局面,并且国内的局势也变得动荡。中国开始考虑独立于苏联行动,改善与西方国家的关系。此时对于美国而言,越南战争和苏联的进取态势都促使它考虑和中国的关系改善。

基辛格就是在这样的背景下开始与中国接触的。两国必须抛弃意识形态和台湾问题的巨大分歧,建立一个对话渠道,放下心理上的戒备,共同对付苏联。外交谈判是琐碎的曲折的,由于基辛格是这一系列事件的直接参与者,他的这部分的回忆是本书的重中之重。在他的眼里,毛泽东具有中国传统上的帝王的姿态,而周恩来是儒家标准里的翩翩君子。中美此时搁置了大量的争议,紧紧抓住两者的共同的眼前利益,取得了共识上的最大公约数,以极大的想象力和创造力,带来了冷战国际关系史上的巨大成果。

邓小平时代迎来了中美关系的蜜月期。邓小平务实、直接的外交风格和大胆的改革魄力和不仅为中国带来了经济上的进步,还带来了外交上的巨大优势。中国利用这种优势更坚决地进行改革开放。同时期,美国的各届领导人也维持着务实风格,与中国共同应对苏联。

美国的外交政策受民间的影响很大。美国政府更多将外交看成一种讨价还价的行为,需要使国家利益最大化,因而需要考虑和放弃很多东西,这样才能促成合作和协议。然而,美国民众(至少基辛格是这样认为)保持着天真而简单的思维方式,他们会把意识形态与对外交政策的期待混为一谈。比如在89事件后,美国民众就促使政府对中国进行经济制裁和外交施压,完全不顾中国的实际情况,而这在中国人看来,是赤裸裸的“干涉他国内政”。中国关系在90年代以来,一起受此影响。基辛格没有明确说明美国政府是否有意利用这一点来作为外交谈判的乱码,但从历次事件来看,中国的“不民主”“不自由”完全成为美国对中国施压的工具,这和美国普通民众的看法并没有直接的关系。

90年代以来,中美关系虽然面临着巨大阻力,并且随着中国实力的增强,这些阻力越来越大,但两国的经贸联系更加紧密,实际上不可能再迎来50年代之后的那种隔绝。本书成书于2012年左右,当时的中美关系相对还比较平和。然而到了特朗普时代,中国和美国却越来越多地走向了对抗,在2018年甚至爆发了贸易战:美国不满于中国的贸易顺差,不满于中国的汇率政策,也不满于中国对于外资的限制。长远来看,美国更多地是担心中国持续的经济军事实力增长会挑战它的全球霸权,特别是在高端产业上和它开展直接的竞争。显然,中国不会也不可能持续作为美国的工厂,而是需要建立自己的全面的工业,在世界经济中占据更为有利的位置。这场贸易战,到了2018年底已经来回谈判了多次,中国做出了妥协,让更多市场向美国开放,降低贸易壁垒,甚至改革自己的知识产权执法环境。两者经贸依存度的提升,所带来的上述摩擦是必然的。然而,中国人在90年代的外交环境,比现在要恶劣得多,相信中国人能够应对这样的挑战。

评分:9/10。