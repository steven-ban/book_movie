\subsection{《物理学的困惑》}

标签: 物理学 \  科普  \ 弦理论

作者:[美]李·斯莫林

《物理学的困惑》封面:\url{https://gss1.bdstatic.com/-vo3dSag_xI4khGkpoWK1HF6hhy/baike/w\%3D268/sign=108297bf29a446237ecaa264a0207246/b7003af33a87e950488a1ea919385343faf2b452.jpg}

\subsubsection{五大问题}

书名《物理学的困惑》,作者在第一章中,就开宗明义指出\emph{当代物理学的五大问题}:
\begin{itemize*}
	\item 将广义相对论与量子理论结合成一个真正完备的自然理论。
	\item 解决量子力学的基础问题:要么弄清理论所代表的意义,要么创立一个新的有意义的理论。
	\item 确立不同的粒子和力能否统一在一个理论中,并将其解释为一个单独的基本作用,即\emph{粒子和力的统一}。
	\item 自然是如何选择量子物理标准模型中的自由常数值的?
	\item 解释暗物质和暗能量。或者,假如它们不存在,那么如何在大尺度上修正引力理论,为什么修正?更一般地说,为什么宇宙学标准模型的常数(包括暗能量)具有那样的数值?
\end{itemize*}

\subsubsection{统一之路}
20世纪最伟大的两个理论——广义相对论和量子力学——长久以来不兼容,两者方程的结合会产生无穷大的概率,而现有的实现都表明这两个理论至少在各自的观察尺度下都是正确的。物理学在过去300年的巨大胜利,使得人们时常有着这样的想法:存在一个最基本的理论来描绘宇宙的秘密。因此,20世纪30年代以来,所有物理学家都面临着如何将这两个理论解释在一起的问题。

超弦理论是四十年来最受欢迎的理论。其具体内容我无力做介绍,因为即使是专业如本书,也只是提供了该理论的一部分形象化的图景。但是,超弦理论虽然经历了第一次(发现弦理论能够生出标准模型)和第二次革命(发现了隐藏的M理论可能存在以统一过多的理论形式),但仍然存在着巨大的缺陷。首先,它只是一种纯粹的理论,本身\emph{没有提供新的预言},因此在现有的物理实现条件下无法得到证明或证伪;其次,它具有多个不同的形大(至少五个)及不同的解,采用了无数个复杂的看不出原因的卡丘流形结构,虽然很多人声称这多个理论的背后有一个更为深刻和基本的“M理论”,但谁也没有发现它,而我们的世界为何需要这么多理论或解释呢?这不和统一之路背道而驰吗?

超弦理论可能是美的,但一个美的看似合理的理论未必是正确的,作者在第2章就给出了历史上的这些例子。作者介绍的物理学的历史,也是把弦理论放在这样一个宏观的背景下比较,与之前的辉煌相比,弦理论虽然运用了大量数学试图做一个统一,但本身是黯然失色的。

\subsubsection{开撕超弦理论物理学家}
在对超弦的历史及诸多疑点进行介绍和分析后,斯莫林开始对这个团体开战,内容十分火爆。在第16章里,作者提到现在的物理学研究人员过多,而大学职位有限,于是管理者只能对那些有可能拿到经费支持的人开放,而管理者的行政权力过大,导致年轻物理学家的竞争越来越激烈,他们扎堆去做那些容易拿项目补助的课题,如超弦理论。

在超弦圈子里,形成了\emph{小团体思维}。虽然现在理论物理学大部分人都在做超弦理论,但这个圈子却显得十分小家子气。他们对自己的理论盲目自信,轻信其他研究者的似是而非的不严谨的结论(如作者费了很多笔墨介绍的“有限性问题”,其实原始作者并没有给出严格的证明,但被大多数超弦研究者误解并引用,又如他们不经严格证明就认为M理论已经存在);他们对权威亦步亦趋,很少提出自己的反对意见,甚至二十多年来没有提出自己的独特见解,而是一窝蜂地和权威们保持一致;他们对圈外人士保持着敌意,至少是轻蔑。

作者对这些现象做出的批判,看得人很过瘾。本书前面讨论的问题很艰深,难于理解,但这一部分却给出了这么多“猛料”,也算是值了。其实,理论物理学的科普本来就存在这样的两难处境:如果写得通俗易读,则会采用大量的类比,这会降低其严谨性;如果写得严谨,则难免出现大量专业术语,让科普受众望而却步。本书则是理论+段子的混搭,让人看得很过瘾。

科学是什么?在作者看来,科学是科学家们在不完备信息下认识世界的一种方式,他们首先需要维护科学研究的道德规范,承认他人在证据充分下的观点,同时需要保持自己的怀疑精神,相信科学会进步。整体来看,作者对科学是抱有乐观信念的,这和我的观点也一致。科学是人类目前为止认识世界最成功也是最好的方法,虽然它经常出错,也会和复杂的人性及历史纠缠不清,但它促进了人类社会的进步,也是世界进步的主要推动力。

在作者看来,超弦圈子的物理学物理学家,绝大多数只是“工艺师”或“工匠”,他们的基础好,只是在前人画好的“安全”的圈子里做研究,然后出成果谋取职位,缺少对物理学最前沿的深刻洞察,也正是他们在整合广义相对论和量子力学上无所不用其极,诞生了标准模型、量子色动力学等。而“预言家”则不同,他们的数理基础可能差于工匠们,但能够长时间地追寻最根本的物理学问题,他们要做一场彻底的革命,也怀疑量子力学是否真的“正确”,以此来确实新物理学的模样。这些人与其说是物理学家,不如说更像是艺术家或作家,但是他们与物理学“主流”相背,注定得不到终身教职,但正是他们的思考可能会带来物理学的真正革命。有趣的是,作者把物理学的“工匠化”或专业化与20世纪30年代以来欧洲大陆物理学的衰退与美国物理学的兴起联系在一起,后者关注的是“实用”的方式,而前者则更关注根本性的研究方式。值得吐槽的是,作者的这些论调,有点像郭德纲对“主流相声演员”与“非主流相声演员”或“体制内”“体制外”的划分\^V\^。

美国科学界的“终身教授”制度也加强化对主流和老科学家的支持,而忽视了不同的声音。另外,同外评审这样悠久的科学传统也有利于有更多做相同研究方向的“同行”们的推荐。作者没有过多攻击美国科学界的门派,但想想中国大家和研究所里的师承关系形成的派系,这种现象只会更加普遍。

\subsubsection{书评}

本书的内容是十分精彩的,特别是最后一部分对于物理学界的批评猛料很多,是对前面枯燥和深奥理论解释的绝好补充。如果读者喜欢理论物理学,本书不可不读。相比更火的《时间简史》和《宇宙的琴弦》,本书的成书时间更晚,也会带来更多新的发现和思考。

评分:10/10。