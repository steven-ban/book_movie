\subsection{《八次危机》}

标签: 政治经济学 \ 温铁军 \ 改革开放 \ 经济危机

作者:温铁军团队

作者在“国仁文丛”丛书总序里讲明,他们学术团队的主要理论贡献在于:
\begin{itemize*}
    \item 人类文明差异派生论。各文明是独立演化的,不一定遵循产业资本时代西方经典理论家的生态化历史观。
    \item 制度派生及其路径依赖理论。不同文明在各自地理条件下的资源禀赋和要素条件各异,以其之后的历史进程、制度安排有路径依赖上的限制。
    \item 成本递次转嫁论。全球化时代核心国家向外转嫁制度成本而获取收益,发展中国家因此落入低水平陷阱中难以脱身。中国的特殊性在于,在外资和外援中断后,国内的工业化进程没有中断。
    \item 发展中国家外部性理论。二战后发展中国家独立,与宗主国展开主权谈判(交易),具备认知、信息和实力上的劣势,难以摆脱经济政治上的依附地位。
    \item 乡土社会应对外部性的内部化理论。中国有着几千年的“内部化处理负外部性”的村社基础,三农是中国危机软着陆的载体。如果失去了农村的“软着陆”条件,则危机会在城市中产生“硬着陆”,引发矛盾的爆发。
\end{itemize*}


\begin{longtable}{r | l | p{0.35\textwidth} | p{0.35\textwidth}}
\caption{中国建国前后“八次(经济)危机”的情况} \\
\hline
序号 & 时间 & 原因 & 化解 \\
\hline
\endhead

\hline
\endfoot

0 & 1935-1952 & 民国时期帝国主义输入的高通胀累积,法币改革失败(参见作者《去依附》) &土地革命均分土地,朝鲜战争争取苏联大规模投资和技术经验,但全面苏化带来“条块分割,尾大不掉”的痼疾 \\
1 & 1958-1960 &中苏关系恶化,苏联投资中断,中国举债进行工业化造成财政赤字 & 2000万“知青插队”,2000万农村中学毕业生回乡,以劳动力投入代替资本投入,以地方投入为主,带来“大跃进”和饥荒 \\
2 & 1968-1970 &苏联模式的缺陷,三线建设,彻底还清苏联债务 &减少投资,上山下乡 \\
3 & 1974-1976 & 四三计划导致债务危机,财政赤字 & 第三次“上山下乡” \\
4 & 1979-1980 &投资过量导致财政赤字,包括前期积累、福利补贴过量  &财政甩包袱,农村分地,社队企业 \\
5 & 1989-1990 &价格闯关失败,官倒公司大量囤积居奇,为遏制民众挤兑而提高存款利率,银行因负利率而亏损 ,货币超发&硬着陆,放松劳动力进城,引发农民工入城潮 \\
6 & 1993-1994 & 大规模招商引资引发中央对外债务陡增,内部投资过热 &汇率贬值,增发国债和货币,分税制改革,城市国企大下岗,社会公共服务部门私有化,农村税负加重,金融资本异化实体产业,土地商品属性确立 \\
7 & 1997 & 东南亚金融紧缩危机输入 & 中央政府启动国债投资拉动增长(政府进入),增加三农投入 \\
8 & 2008 & 美国次贷危机输入,国内投资市场“两头在外”,内需不足 & 四万亿投资,三农投入 \\
\end{longtable}

本书的主要内容,是作者多年(特别是2000年后)在乡村建设实践中对三农问题和中国经济问题的梳理整理而成的研究成果,本书名为“八次危机”,事实上是对新中国成立以来面临的八次经济危机的原由、过程和解决方式的整理,角度独立于官方的意识形态叙事和新自由主义经济学的解释。本书的论点基于事实而成,而非现有的“经济模型”,作者也反复解释,他是基于做学问的实事求是精神而写此书的,本书也确实对各种“主流”“非主流”经济学进行了一定程度的批评。比如,对于邓时代的改革,作者不乏批评之语,对主流的“改革开放”的政治正确掩盖下的改革失误也多有揭露,而对前三十年,作者破除了很多批评,指出当时“失误”的不得已甚至可取之处。在自由主义者看来,作者对毛时代的肯定,对改革开放后一些政策的批评,对国际上美国霸权主义的揭露,表明他是一个“左派”,事实上作者在国内也是和左派、红派归类在一起的。作者曾经为当前的重庆模式站台,这成为他的一个“黑点”,然而以现在十年后的眼光来看,当时的做法未必就是错误(现在某种程度上不也是“唱红打黑”“遏制房地产”吗?),而是基于当时经验和学术研究基础上的判断。

作者的核心论点,在我看来有两个:
\begin{itemize*}
    \item 一个是任何后发国家想要实现工业化,必须解决好与霸权国家(核心国家)间的“依附”关系,维护好经济上的主权,防止成为霸权国家转移国内矛盾(经济危机)的外部手段。这一点在新中国与苏联决裂、在美苏两个大国之间求生存、三线建设中可以显现出来,但过度的“去依附”只会造成独立,因此需要借助核心国家构建的国际贸易和地缘政治秩序,融入全球化,接受技术、资本上的转移,这也是新中国建国后“一边倒”和苏联结盟和70年代后与美国接触从而融入西方国家的贸易秩序(加入WTO)的显现,这种。很多国家过度依附(如拉美、东南亚)或者完全不依附(如伊朗、朝鲜、古巴等),都使自己陷入发展的陷阱中不可自拔。中国借助自己的巨大体量、领导集体的高瞻远瞩而避免了这种陷阱。
    \item 另一个是国内工业化进程中如何转移经济危机。中国有着独特的小农经济,中国的近代化、现代化是在保留着小农经济而推进的。建国前后的土改事实上保留了这种小农经济,而后的人民公社运动取消了小农经济,但改革开放以后又予以恢复,这事实上是对农民的一种“补偿”和“保障”,耕者有其田保证了农民阶层和农业社会的稳定,防止出现拉美、印度等国家的贫民窟问题。同时,农业社会成为城市工业化转移危机的“蓄水池”。作者指出,历次危机中只要能够向农村转移,就可以实现“软着陆”,否则就会在城市硬着陆,造成矛盾的爆发,比如改革开放以后人民公社取消,上山下乡的政治动员不再实施,因此八九十年代的危机都在城市中爆发,典型如八九学潮;2000年后中央增加三农投入,使得农民负担减轻,收入增加,这为矛盾的解决提供了方便。建国前三十年的三次“上山下乡”事实上均是城市工业化出现经济危机,债务问题凸现,外资和中央投资中断,城市失业,不得不向农村转移劳动力,以劳动力投入取代资本投入,以此方式渡过危机。作者作为“三农问题”专家,曾提出增加三农投入并被中央采纳,为成十六大后农村问题和十七大后生态文明的政策基础之一。
\end{itemize*}

本书行文并不流畅,有些句子过长,不易读,但不影响作者的真知灼见,从作者的研究中可以窥见中国建国后历史发展的独特位面,中国工业化进程的艰难与伟大。而且,中国的这种经验,中国实力的增强,均对其他第三世界国家如何实现工业化提供了宝贵经验。

本书大致成书于2011年,其中关于2010-2020年国际关系(如中美关系)和国内问题的讨论有些过时,有些结论和猜想与后来的发展不符合。例如作者显然没有考虑到美国国内问题积累导致的孤立主义、特朗普上台、中美贸易战与关系恶化的问题,也没有考虑到中国借助一带一路、与俄罗斯加强双边关系导致的外交新局面,没有预见到十八大以后强有力的反腐、党建、环境治理、进一步地开放放权带来的社会治理的巨大进步,也没有预见到中国国内消费市场的扩大和民众消费能力的提升。不过,社会科学的预测作用本来就差,基于当时缺乏更全面深入的第一手资料的分析,能够逻辑自洽就已经可以给后来者以启发了。

另外,作者对三农“蓄水池”的论述是客观和中立的,但我作为农村出身的人,却为农民连连叫苦。作者认为,城镇化应该保留农村的特色,不能把农村变成城市那样千篇一律的钢筋森林,不能在这个过程中被金融资本、地产既得利益者所绑架,然而我实现难以想象,如何在现在的农村面貌下既不走旧的城市化的道路,又保证农民生活水平的有效提升。不把农民变成上班的市民,能够解决他们的生活条件低下问题吗?这本书没有细说,我也想不明白,农村将来的路应当如何走下去?毕竟现在农村的人口和人才流失现象特别严重,一两代人之后大部分乡村可能要消失或合并,符合作者观点的路究竟在何方呢?作者的书我会读下去,这个问题他可能会给出比较全面的答案。

评分:5/5。
