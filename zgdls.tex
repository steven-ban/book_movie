\subsection{《中国大历史》}
\subsubsection{书评}
1.本书注重制度分析,以经济学中的税制和土地为线贯穿中国历史,作为一本通俗读物讲得已经非常透彻了。

2.个人读史不太喜欢细节,喜欢并重视历史之逻辑,轻历史之偶然性。这本书深得我心,如果早读十年,读史上会少走很多弯路。

\subsubsection{标注}
1. 军阀一般为带悲剧性格的英雄人物,他们也并非个个存心做坏事。一位英国观察者指出不少中国军阀可能在英国陆军里成为出人头地的将领。他们将个人之野心和他们所想象的救国救民宗旨合为一谈,因之极难向他们的部下及中国民众解释明 白。张作霖初受日人培植,以后成为热烈爱国者。冯玉样起先被称为“基督将军”, 以后向苏联靠拢。阎锡山组织了一个“洗心团”,给以种种宗教式点缀。唐生智几 乎完全皈依佛教,他以超度的观念补偿他的杀戮。军阀也非个个粗蛮,吴佩孚即系诗人。可是性格淘气的张宗昌据说生平不知所带兵员人数、手中钱数和各房姨太太 数。

2. 中国是无数农村组成的一大集团,当中的弊病尚且大过贪污。她的整个组织即是不能在数目字上管理。如果让中国继续闭关自守,那她也会继续以精神和信仰上的运作掩饰组织上的缺陷。

3. 西方民间的自由源自于封建体制里额外颁发的城市特权。可是市民阶级的资产力量,仍无从构成社会的大改造,只有国际贸易增多,在全国经济里的比重升高,商业财富的力量伸展到农业 财富里去,牵动了全局,才构成实力,如此才可以改造社会。当日政府也仍不肯立 时服输,只是抵御不得,才无可奈何地承认改组,此后便以商业原则作为施政的准 据。

4. 因着汉 满两方缺乏永久的仇恨,使我们想到现在所谓的民族主义其实是近代社会的产物。

5. 如果环境的开展与事实上的情形稍有差异,朱棣在历 史上的声名很可能和隋朝的第二个皇帝杨广等量齐观。

6. 货币贬值也增加政府本身之困难。这也是促成宋朝衰亡的一大原 因。


8. 这种局势的展开也指出中国即使在国防上也要中央集权。全国的国防线 大致与15英寸的等雨线符合,这是世界上最长的国防线

9. 六岁必有灾荒,12年必有大饥谨。
