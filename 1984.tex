\subsection{《1984》}

故事设定在1984年,作者奥威尔想像当时的英国已经被赤化,由苏联这样的国家占领(欧亚国,其实就是苏联赤化欧洲),建立了极权统治,整部小说即是在这种极权的社会环境下写成。

奥威尔参加过共产主义运动,对苏联的极权主义运作方式较为熟悉(如果真的去做对比分析,也可能会得出更加精确的结论,我相信奥威尔对苏联生活很大程度上应该来自想象,应该有失实的地方),在本书中不仅剖析了整个社会的氛围,也重点描写了这种社会下人性的丧失。

极权社会是极度集体化而剥夺人性的社会,是基于战争的社会(如果没有战争,便会制造战争的恐慌)。社会的资源都是先集中,再分配,取消市场经济,也随之取消个人的经济自由。有经济学家说过,计划经济是一种短缺经济,这在本书中很常见。一些稍微好点的物品都需要从黑市购买,市场上只供应政府允许的基本物品,实行严格的配给制,因此无法满足每个人个性化的需要,并且千篇一律,毫无新意。同时,政府实行严格的思想控制,将过去的历史完全抹煞,只突出过去的阶级社会属性,认为过去是残酷的资本主义社会,是经过以前的革命也建立的新社会。新社会里只有一种思想,就是老大哥的思想(隐喻斯大林),老大哥对党内进行大清洗,甚至逮捕了自己以前的战友,以完成自己绝对的控制地位。在社会的任何地方,都有窃听器来记录每个人的话语,并进行分析,一旦发现有了越轨言论就直接逮捕并进行肉体消灭。家庭和单位内部的人会互相检举揭发,政府对这种行为进行鼓励,因此人人自危,更不敢瞎说什么。

在欧亚国,“对立统一”有着荒谬的解释,并贯彻在人的所有思维里:
\begin{quotation}
知道又不知道;明白全部事实,却说着精心编造的谎言;同时拥有两种针锋相对的意见,一方面知道两者之间的矛盾,一方面又两者都相信;利用逻辑来反逻辑;一方面批判道德,一方面又自认为有道德;相信不可能有民主,另一方面又相信党是民主的保卫者;忘掉一切需要忘记的,然后随时在需要记起时再回想起来,接着马上再忘掉——最重要的是,对这个过程本身,也要照此处理。最奥妙之处在于:要清醒地诱导自己进入不清醒状态,然后再次意识不到刚刚对自己实行的催眠行为。甚至理解“双重思想”这个词,也要用到双重思想。
\end{quotation}


政府的文化产品单一,只有不断洗脑式的宣传。如前所述,政府对以前的社会进行了意识形态化的描述,对现在的“美好生活”进行夸大式宣传。这些宣传包括目前的经济形势(指标又提升了百分之几)的报道,其实都是无中生有,大家也都不相信。政府对现在的社会造出了很多“新词”,割断了语言与以前的联系,因此现有的政府对语言进行了垄断式的解释。政府会在社会投放大量的淫秽书刊来满足压抑社会中的性需求,但不准进行开放式的恋爱。男人与女人之间只是进行机械式的结婚,并在政府的号召下实行“义务”为国家生孩子(主角温斯顿的老婆就是被洗脑的人),只有性没有爱。在这种人人自危的枯燥的生活里,思想文化可谓贫瘠、贫乏。

政府销毁了所有以前的书籍,并改写了历史,作者在书中说:
\begin{quotation}
谁掌握历史,”党的标语这样说,“谁就掌握未来;谁掌握现在,谁就掌握历史。”
\end{quotation}

大洋国、欧亚国和东亚国,是这个世界在核战争下的三个国家,它们之间是战争关系,然而战争的目的并非胜利,而是为了消耗掉过剩的产品,让人民不断地工作,从而维持着党统治人民的私欲。战争在拉锯,成败并不重要,只有宣传和经济上的意义。当然,我对这个依据资本论建立起来的社会充满了质疑,这样的社会是不可能产生的,更是不可能长时间 运行的。首先资本论关于剩余价值的那一套未必正确,其实这样的统治形式会很快有反叛者出现,推翻这样的统治,同时建立一个和过去充满了路径依赖的国家。不断进行战争的社会会不断地衰退,最终自我毁灭。当然,奥威尔写这部小说,可能并不需要那么严谨,他的目的是为了黑共产主义和苏联,所依据的上面一套理论也是照搬自共产主义的理论,照着它画靶子打。

故事的主角是温斯顿,在类似宣传机构的单位中工作,内容无非是捏造宣传数据。温斯顿是个党员,而党员受到的监控比普通群众更严格,以至于温斯顿认为只有群众的觉醒才能推翻这个邪恶的政权。温斯顿没有孩子,生活单调,但是他心中没有完全失去对过去生活的好奇,根据他的工作内容,他认为党对过去生活的描述是错误的,过去并非一个毫无温情和充满压迫的社会。他会去旧杂货店购买以前的商品,借此躲避现在宣传的腐蚀。他一直在内心里抗争,没有像妻子一样成为党宣传和压迫下的木偶。

温斯顿像其他很多人一样,看似是党压迫下的一个螺丝钉,但其实内心对这套制度和文化充满了反感和反叛的欲望,一旦有机会,他就会突破牢笼去杀死党和老大哥。终于他有了一个机会:他越来越对过去感兴趣,因为过去隐藏着现有制度的所有谎言,回到过去是所有反叛者的目标。他看到了过去的器物,和一个具有抬头精神的女人谈了恋爱,同时遇到了一个看似普通、实际是哥斯坦因追随着的同事,于是他借此加入了反叛组织。然而,后来发现他加入的“兄弟会”其实不存在,一切都是让他加入的那个人的承诺,他是思想警察的头目,同时温斯顿藏身的老板也是监控人员,于是温斯顿入狱。之后,温斯顿先是受到了关押和酷刑,随后在巨大的电流刺痛下被迫去除头脑中的正常思维,拥抱“老大哥”的思想。在这里,全书出现了高潮:如果说之前的描述和所谓苏联式的极权主义还相同的话,奥威尔在这里对这个极权社会的心灵控制手段大大往前推进了一步。苏式集权和中世纪的宗教裁判所都是靠酷刑来对人进行精神压制和肉体消灭,但受刑者因此而变得伟大;在大洋国,思想警察不仅摧毁你的意志,还让你全面拥抱“新”思想,承认自己的下流与猥琐,直到你真的接受了那个思想,混淆了现实与过去,完全抛弃自己的自由意志,才让你死去,不仅在肉体上杀死你,还要在精神上彻底打倒你,羞辱你,让你尊严丧尽,从而不得不重新生长成为体制的一颗螺丝钉。在最后,温斯顿的思想改造进入了第三阶段,他被带进101房间受审,那里有他最恐惧的东西:老鼠啃脸,在巨大的恐惧下,他放弃了最后的底线,哭喊着让老鼠咬自己的恋人茱莉亚,从此他完全被改造,会承认2+2=5而不认为这后面有什么错,具备“双重思想”可以随着场景改换对事物的认知。故事的最后,他与同样改造后的茱莉亚会面,两人再也不“认识”对方,也没有任何激情和爱情可言了。就这种利用人对疼痛和恐惧而进行的思想清除与灌输而言,这本小说的思想性达到了真正的不朽,成为名副其实的神作。

评分:10/10。