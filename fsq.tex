\subsection{《房思琪的初恋乐园》}

标签: 台湾文学 \  性侵 \  恋童

作者:【台湾】林奕含\footnote{参考\url{https://baike.baidu.com/item/林奕含/20723377?fr=aladdin}}

\subsubsection{人物 }

\begin{table}[htpb]
\centering
\caption{《房思琪的初恋乐园》主要人物。}
\begin{tabular}{p{0.1\textwidth}|p{0.3\textwidth}|p{0.55\textwidth}}
人物 & 特点及关系 & 事件 \\
\hline
房思琪 & 主人公 & 被自己的老师李国文性侵,后来发疯,被送到台中 \\
刘怡婷 & 房思琪的玩伴,在高雄时两家在一层楼里的对面 & 曾经和房思琪很要好,房思琪事发后试图经历她所经历过的一切 \\
刘妈妈 & 刘怡婷的妈妈 &  \\
房妈妈 &  & 房思琪说家里没有性教育,她说性教育是给需要性的人的 \\
钱一维 & 房思琪在高雄时同楼的邻居,父亲是台湾大富豪钱升,四十多岁 & 喝了酒就打许伊纹,事后会基本忘掉 \\
李国文 & 房思琪在高雄时同楼的邻居,老师 & 搬来时房思琪仅12岁,喜欢在女人面前显摆才学,有恋童癖,诱奸女学生。年轻时不会讨女人欢心,于是习惯对女学生下手。 \\
许伊纹 & 二十多岁,钱一维的老婆,长得有点像房思琪 & 比较文学博士,因家暴而流产。房思琪和刘怡婷经常去找她玩,她给两个小女孩讲文学 \\
钱老太太 & 钱一维的母亲 & 思想封建,“肚子是用来生孩子的,不是用来装书的”,想让许伊纹生男孩,喜欢家里干净以此折磨许伊纹 \\
蔡良 & 女老师,李国文的炮友 & 她自己和班上男生上床,和李国文之间相互介绍学生拉皮条 \\
郭晓奇 & 条件中上,补习时李国文的炮友 & 被班主任蔡良介绍给李国文,后来被李国文抛弃,自己自暴自弃滥交。把这事告诉父母,父母心理卑微与李国文夫妻谈,父母认为自己的女儿“淫贱”,向李国文道歉。成绩落后被退学,想自杀未能成功。她把自己的经历上传到网上,得到了人们的羞辱和嘲讽,同时李国文把房思琪的裸照寄给她,她家门口被喷了红漆。 \\
毛毛先生 & 珠宝店老板,暗恋许伊纹 & 经常与许伊纹谈论文学,但对于她有丈夫很心痛。他妈妈知道这事,劝他不要异想天开。许伊纹流产后搬出钱家,他经常去陪许伊纹。他爱许伊纹,无法容忍她与钱一维再在一起。 \\
李太太 & 李国文的结发妻子 & 发现了郭晓奇的事,被李国文欺骗说是她勾引的。
\end{tabular}
\end{table}

\subsubsection{书评}

一个沉痛的故事:女学生被自己的老师诱奸,被欺骗了六年,之后精神失常。这是一个关于性侵的故事,关于强奸幼女的故事。

类似的故事发生在很多地方,包括中国大陆和韩国。我无法想象这些恋童癖的心态,他们无法在成熟女人那里找来崇拜和操控感,于是只会对毫无反抗之力的女童下手。这些可怜的、自卑的、猥琐的灵魂!这并非一个“文学与语言如何成为诱奸与哄骗之物”(青年作家汤舒雯语),文学只不过成了替罪羊,成为一种借口。李国华之流凭借是,是相对于幼女的权力。女学生被哄骗应该“爱”老师,李国文偷换概念,把“爱”解释成“做爱”和“献身”,这是赤裸裸的诱骗啊!“对一个男人最高的恭维就是为他自杀”,这是一种病态的“爱情”,是一种对生命的漠视,是罪恶。

这些女童们在面对中年男人的手段时,毫无还手之力,最年轻的岁月被玷污,之后再无重新来过的机会,一生随之被毁坏。

可惜的是在本书里,人们竟然无法去借助法律手段去制裁李国文这样的畜牲,因此没有保留关键的证据。

另外值得一提的是家庭暴力。钱一维的家庭是传统的封建式的,许伊纹在锦衣玉食的表面下,过的是地狱一般的生活。钱一维在醉酒后会打许伊纹,之后会立刻忘掉。在本书里,邪恶的人是强大的,无论是李国文在女学生面前卖弄才学耍嘴皮子,还是钱家在许伊纹面前炫耀财力。在这场比拼一开始,形势就对真善美不利。

无论是许伊纹,还是房思琪,还是郭晓奇,都是被命运摆弄的人,无法给予命运以反击。她们是美丽的,也是脆弱的。我在读这个故事的过程里,真的希望她们能给予命运以反抗,哪怕是一丁点的反抗(像郭晓奇那样也好),哪怕是在反抗的战斗里死亡,也比这样顾影自怜、苦苦哀求(就像许伊纹哀求钱一维不要打她一样)要好。她们有点像《金瓶梅》里善良软弱、对命运逆来顺受的李瓶儿;作为她们的对立面,潘金莲则是对命运给予了最大程度的反击:她泼辣而狠毒地去报复命运,报复任何她能感受到的命运给予的戕害。

不,她们不。她们的借口是文学,文字的美(其实是修辞的美),但其实,她们阅读的文学都是美丽而脆弱的,一如她们的性格(这种阅读或许一定程度上铸造了她们的性格),要知道这个世界上,有尊严地去反抗命运的文学多得是,甚至革命文学这样酷烈的华丽的反抗乐章,肯定能给她们一些希望和力量。她们过于沉溺于自己的小天地了;如果说房思琪、刘怡婷和郭晓奇还是孩子不会去反抗大人,那么作为房思琪和刘怡婷人生导师的许伊纹就理应受到苛责了(我并不是因为她们是受害者而责备她们,而是因为她们不反抗而责备她们)。她们是有精致的、资产阶级的(许伊纹语)、有教养的、淑女的、文雅的女人,她们被教育成这样,社会对这些“有地位”“受教育”的女人的期待是这样,她们(至少作为成人的许伊纹)接受了这样的角色,并安心扮演这样的角色。她们的错误在于放弃了斗争的义务,在社会构建的性别角色里安之若素,没有认识到\emph{女人应该和男人一样和这个世界战斗,和自己战斗}。如果她们看过波伏娃,看过《第二性》,她们应该会有不一样的人生。波伏娃在《第二性》里对资产阶级女性地位的解释,对她们完全适用。波伏娃说,资产阶级女性安于自己优渥的生活条件,与劳动相脱离(许伊纹是全职太太),不能以个体的身份超越自己,因此注定被男人禁锢。

我甚至害怕,房思琪如果真的像许伊纹说的那样过上“正常”的生活,她长大后会不会像张太太那样,麻木而猥琐地成为暴力的共谋?

更可怕的是,世界在面对她们的悲惨命运时,往往是袖手旁观甚至是助纣为虐。郭晓奇的论坛发贴被泼冷水,人们会轻易地指责受害者,而非施暴者。甚至是郭晓奇的父母(“在这个故事中父母将永远缺席”),也需要因为郭晓奇破坏了李国文的家庭而道歉。他们在李国文请的饭店里感受到自己的自卑,而女儿的惨剧熟视无睹,甚至反过来责备自己的女儿。更让人寒心的是,房思琪的父母在女儿受到性侵的六年里,一直没有防范,也没有沟通和支持,轻松葬送了女儿一生的幸福。该道歉的不应该是那些受害者,而是那些施暴者和帮凶,甚至是麻木不仁的看客。“人对他者的痛苦是毫无想象力的”,这是作者对冷漠大众的挞伐,这和鲁迅的*人类的悲欢并不相通,我只觉得他们吵闹*有异曲同工之妙。社会对性太讳莫如深了,把它看得太“隐私”了,因此认为强奸犯自己有责任。诚然,从社会治安和自我保护的角度上来看,受害者有不被强暴的可能,或许她们在事前的一念之变就会为自己带来安全,但我们仅能责备她们为什么不更小心一点,而不能认为强暴本身是她们的错。希望通过本书,大人们能更深刻地去检讨以前的过失,希望儿童性侵和家庭暴力得到人们的重视并去改变它们。

本书使用了大量的比喻,语言特别女性化。作者如果不自杀,可能在文学上有更好的发展。

据悉,这本书的中文版删除了一小点政治敏感内容,做了一些两岸不同表述的转换(如外国译名),基本不影响阅读。

评分:7/10。

