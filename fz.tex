\subsection{《复杂》}
\subsubsection{读后感}
作者是“复杂”这个领域的研究者,把这一主题涉及的大部分话题都叙述了出来,读一下还是有些收获的。

但是,复杂并没有形成完整的通用的体系,很多人对这一“学科”的指责不无道理。“复杂”是一种系统性的现象的总称,而这一系统往往由简单的“元胞”组成,通过设定元胞的基本结构和规律来模拟系统的整体规律。这一“学科”包括了生物学、物理学、化学、计算机科学、生态学等多个学科,但似乎各个主题间缺少必然的联系,仅仅是通过一些表面上的表现来牵涉到一起。从这本书来看,这章节独立性非常强,并且关系不深刻,一章一个规律的现象比较明显,这也从侧面说明了这一学科的不成熟。

我对“复杂”的前景表示不看好。像“元胞自动机”这样的通过程序来模拟形态的,更多是依赖于具体情形,很难提炼出规律性的结论。

满分如果五星,我给四星。这本书作为通识读物,还是值得看的,即使内容本身有商榷余地,也会给读者带来一些思考和启发。

\subsubsection{标注}
他将进化论者分为三类:适应主义者,认为自然选择才是主要的;历史主义者,相信历史偶然导致了许多进化变化;以及考夫曼这样的结构主义者,关注的是组织结构如何能没有自然选择也能产生。

进化论者由于受到宗教极端主义的攻击,尤其是在美国,并且经常处于守势,因而不愿意承认自然选择可能不是故事的全部。

考夫曼的“第四定律”则提出生命具有复杂化的内在趋势,而这独立于自然选择的任何趋势。

根据进化发育生物学,生物的多样性主要来自开关而不是基因的进化。这也是为什么形态的巨大变化——可能还包括物种形成——可以在很短的进化时间内发生:主导基因不变,但是开关变了。根据进化发育生物学的观点,进化的主要力量正是这种—长期以来一直被视为“垃圾”的DNA的——变化,而不是新基因的出现。

基因调控网络包括功能基因和调控基因,功能基因编码用于细胞结构和运转的蛋白质(和非编码RNA),而调控基因编码的蛋白质则可与目标基因旁边的DNA“开关”相结合,从而开启或关闭相应的基因。

编码RNA对基因和细胞的功能具有调控作用,这些以前都认为是由蛋白质单独完成的。
即使基因的DNA序列不发生变化,基因的功能也会发生可遗传的变化。

生物系统的复杂性主要来自基因网络,而不是单个基因独立作用的简单加总。

单个基因可以编码多个蛋白质。

基因可以在染色体上移动,甚至移动到其他染色体。

基因并不是相互分开的。有些基因相互重叠——也就是说,它们各自编码不同的蛋白质,但是共用DNA核甘酸。有些基因甚至完全包含在其他基因内部。

进化将我们的循环系统塑造成了接近于“四维的”分形网络,从而使我们的新陈代谢更加高效。

决定代谢率的养分输送速率与体重呈指数为3/4的比例关系。

如果画出许多物种的平均生命期和体重的关系,会发现是指数为1/4的幂律。如果画出平均心率与体重的关系,你会得到指数为-1/4的幂律(越大的动物心率越慢)。生物学家们发现了大量的幂律关系,都是分母为4的分数指数。因此,这些关系也被称为四分幂比例律(quarter-power scaling laws)。

3/4次幂比例不仅对哺乳动物和鸟类成立,对鱼类、植物,甚至单细胞生物也成立。

代谢率与体重的3/4次幂呈比例。

代谢率同体重的2/3次幂呈比例。这就是所谓的“表皮猜想(surface hypothesis)”,

相对于体重大小来说,较小动物的代谢率比较大的动物更快。

就是“在不同尺度下具有不变性”。这就是无尺度一词的由来。

小世界性233:一个网络如果只有少量的长程连接,相对于节点数量来说平均路径却很短,则为小世界网络。小世界网络也经常表现出高度的集群性:任选3个节点A、B、C,如果节点A与节点B和C相连,则B与C也很有可能相连。

空间相邻关系的存在会促进合作。

哥德尔提出了可以编码数学命题的方法,从而让它们可以谈论自身。图灵则提出了编码图灵机的方法,让它们可以运行自身。

还原论者喜欢线性,而非线性则是还原论者的梦魇。

混沌指的是一些系统——混沌系统——对于其初始位置和动量的测量如果有极其微小的不精确,也会导致对其的长期预测产生巨大的误差。也就是常说的“对初始条件的敏感依赖性”。

复杂系统试图解释,在不存在中央控制的情况下,大量简单个体如何自行组织成能够产生模式、处理信息甚至能够进化和学习的整体。