\subsection{《编写高质量代码:改善Python程序的91个建议》}

作者:张颖 \ 赖勇浩

\subsubsection{笔记}
1. 需要安全地关闭文件,可以使用with语句:
\lstset{
language=Python,
keywordstyle=\bfseries,
stringstyle=\ttfamily,
numbers=left,
numberstyle=\small,
stepnumber=1
}
\begin{lstlisting}
with open(path, 'r') as f:
	do_sth_with(f)
\end{lstlisting}

2. 通常禁止断言的方法是在运行脚本的时候加上-O标志。

3. 如果对于or条件表达式应该将值为真可能性较主的变量写在or的前面,而and则应该推后。

4. 函数参数在传递过程中将整个对象传入,对可变对象的修改在函数外部以及内部都可见,调用者和被调用者之间共享这个对象,而对于不可变对象,由于并不能真正地修改,因此,修改往往是通过生成一个新对象然后赋值来实现的。

\subsubsection{书评}
这本书是Python的中级应用,主要聚集在内部语言机制和一些使用上的技七和“坑”,对于平时用的比较多的用户而言是值得一看的。某些机制似乎只出现在Python2中,在Python3的新的底层实现上能否可用还存疑。本书的内容大概在2013年之前,随着Python2被取代以及Python3的流行,某些内容可能已经过时了。

对于那些语言上的“坑”,平时在编程过程中就应当少涉及,一是可读性不好,二是可移植性差。

评分:3/5。