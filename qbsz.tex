\subsection{《史蒂夫·乔布斯传》}

\subsubsection{读后感}
2011年10月乔布斯去世,中国的互联网圈里立即引起一片波澜,很多人纷纷购买本书试图了解这一产品之神的伟大事迹,并学习甚至简单地模仿其行事风格。事后发现,其实这只是一片波澜而已。作为科技时代的一个”伟人“及天才,虽然现在还常常被人提起和缅怀(特别是与现在苹果公司缺乏创新和提姆·库克的反差极大的行事风格相比较),但那仅仅是一个潮流甚至炒作式的标榜而已,七年后的人们恐怕很难再去细细品味乔布斯的种种的鲜明人格了。

苹果公司的特立独行和颠覆式创新不需要多言了,只要看看2018年的电子产品就不难发现它究竟多大程度地改变了科技行业\footnote{”科技行业“这种叫法很奇怪,因为这只是”数码消费电子“的一个下位概念,与真正的”科技“并没有多大关系。}和人们的生活。苹果公司的专制式、宗教式的管理文化,完全是其创始人乔布斯的个性的投射。虽然乔布斯手下的人还在,虽然这些在还在维持同样的一个公司,但大海航行靠舵手,革命事业靠伟人,没有了乔布斯天才式的idea,后乔布斯时代的苹果和之前的区别还是相当大的。

乔布斯的专断、敏感来自于他被领养的经历,而他艺术家式的思维则来自于嬉皮士、佛教、书法艺术(他在大家里曾学习书法,当然,是西方的字母文字的书法)的学习和长期感悟。他的精艺求精的匠人习惯,来自于他养父的美国蓝领的职业道德。不用看他的家庭生活实际为何,这种性格肯定不会为他的家庭带来多少正面的影响。他的家庭是破败的,他不断地伤害爱他的人和他爱的人,把自己的家庭搞得支离破碎。这种独特的个性和价值观,甚至伤害到了他自己。胰腺癌并非不能治好,何况对于这样的富豪而言,早发现早治疗基本上可以让他延长更多的寿命,但他长期只吃水果,迷信邪教式的常说,耽误了自己的治疗。然而,就是这样的个性,让他在商业环境里取得了很大的成果。左手人文,右手科技,他缔造了一个庞大的电子消费帝国。他创建和培养了皮克斯这样一个独特的鲜明的动画工作室。

当然,乔布斯也有看走眼的时候。比如贝佐斯的kindle被他嘲笑,认为没有前途。然而十多年后,kindle取得了巨大的成功,特别是在中国市场。值得一提的是,本书就是在kindle上被阅读的。

作为一部分传记,本书的内容可以算是很详细了。如果你是乔布斯的粉丝或者是他产品的粉丝,本书一定值得你看。如果你仅仅喜欢消费电子,本书没有聚焦太多苹果公司本身的事件,可能不会让你过瘾。

评分:8/10。

\subsubsection{标注与评注}
\begin{quotation}
做销售的人经营公司,做产品的人就不再那么重要,其中很多人就失去了创造的激情。
\end{quotation}
这是苹果和乔布斯对同行的无情嘲讽。

2. 伟大的艺术家和伟大的工程师是相似的,他们都有自我表达的欲望。

\begin{quotation}
3. 他的激情、完美主义、阴暗面、欲望、艺术气质、残酷以及控制欲,这一切都跟他的商业理念和最终的创新产品交织在一起。
6. 他的激情让他很难实现真正的涅;内在的平静、内心的平和、为人的圆润这些禅修者的特质,并未在他身上显现出来。
\end{quotation}
乔布斯的”激情“。

7. 当我意识到自己比父母更聪明时,我为自己有这样的念头而感到异常羞愧。我永远忘不了那一瞬间。”他后来告诉朋友,这个发现,再加上自己是被领养的这个事实,让他觉得自己有些孤立,就如与世隔绝一般,脱离了父母,也脱离了世界。

8. 父亲宁静又温和,这些特质后来得到了乔布斯的赞扬而不是仿效。他还是一个坚决果断的人。

\begin{quotation}
9. 史蒂夫身上的关键问题是,为什么他有时候会失控般变得残酷并伤害别人,”他说,“那还要追溯到他一出生便被遗弃这件事上。真正的潜在问题是,史蒂夫的生活中,永远有‘被遗弃’这样一个主题。”
\end{quotation}

乔布斯身上的偏执、热情、苛刻、控制是苹果的灵魂。我在未读本书里,还期待能从乔布斯的成长故事里寻找出他牛逼的原因。但首先本书并没有重点去探寻“为什么”,而是重点讲述了“如何”。读完此书,我只有一个感觉:\emph{不要问天才是怎么成为天才的,天才就是天才,天才是这个世界的意外,无关乎任何人为和刻意,我们只需要知道某个人是天才并包容他就可以了。}

4. 他有种不可思议的能力,能够创造出一些小工具,我们原先不知道自己需要它们,等推出以后我们却发现自己离不开它们,”莱昂斯写道,“封闭的系统可能是传达苹果知名的技术禅理的唯一途径。”

\begin{quotation}
5. 马库拉把自己的原则写在了一页纸上,标题为“苹果营销哲学”,其中强调了三点。第一点是共鸣(empathy),就是紧密结合顾客的感受。“我们要比其他任何公司都更好地理解使用者的需求。”第二点是专注(focus)。“为了做好我们决定做的事情,我们必须拒绝所有不重要的机会。”第三点也是同样重要的一点原则,有一个让人困惑的措辞:灌输(impute)。这涉及人们是如何根据一家公司或一个产品传达的信号,来形成对它的判断。“人们确实会以貌取物,”他写道,“我们也许有最好的产品、最高的质量、最实用的软件等,如果我们用一种潦草马虎的方式来展示,顾客就会认为我们的产品也是潦草马虎的;如果我们以创新的、专业的方式展示产品,那么优质的形象也就被灌输到顾客的思想中了。”
\end{quotation}
苹果的产品营销哲学:\emph{共鸣、专注和灌输(洗脑)}。
