\subsection{《国民党高层的派系政治》}

标签: 现代历史 \ 国民党 \ 蒋介石

作者:金以林

这本书的主要内容如书名所言,讨论的是国民党高层的派系之间的权力斗争,其主体内容是作者金以林的博士论文《国民党的权力重组——宁粤对峙研究》以及博士后论文《蒋介石的下野与再起——1930年代初的国民党派系纠葛》为内容,讨论了蒋在四·一二反革命政变后名义上统一全国后,各派系之间的明争暗斗,特别以中原大战等内战为由被各反蒋派系反对,蒋介石、汪精卫、胡汉民之间互相挟制,“拉一派打一派”,其中蒋介石扣压胡汉民(1931年2月28日),导致国民党汪派和胡派在广州成立政府,联合北方反蒋力量,与南京政府叫板,逼蒋下野,而蒋则依靠军权、财权在下野后通过暗中动作,分化各派系,特别是联合了汪精卫,获得权力,东山再起。

本书以时间顺序,将上述历史过程编入各章节中:

\begin{itemize*}
    \item 蒋、汪、胡分合回顾
    \item 蒋胡由合作到分裂
    \item 约法之争与胡汉民被扣
    \item 国民会议的本质
    \item “非常会议”与广州开府
    \item 北方反蒋的再次兴起
    \item 上海和谈
    \item 国民党第四次全国代表大会
    \item 蒋氏下野和国府改组
    \item 蒋汪合作
\end{itemize*}

这个过程基本上以蒋为主,以蒋的大战而结束。蒋介石在国民党内属于后辈,地位上不及汪精卫和胡汉民,其上位的过程主要就是依靠黄埔军校和孙中山的信任。在北阀中,他获取了权力,并迅速和江浙财阀(以宋子文、陈氏兄弟为代表)联结在一起,结合党内的反俄反共势力,发动政变,以血腥手段清党,并成立南京国民政府。不久汪精卫也加入到他的阵营里,两人合流。之后在南京,蒋介石由于在党内地位不稳,号召力不够,因此需要与汪、胡中的一个进行合作,他选择了性格迂直、态度激烈而品行清正的胡汉民,而胡汉民作为党内元老和实力派,想要“以党治国”,就得和有军权的蒋介石合作。在南京政府里,两人或者说两派肯定是有矛盾的,蒋想依靠军权夺取党权,实现个人独裁,而胡汉民想依党权扩大治权,收束军权,限制蒋介石的权力。两人的矛盾在“约法之争”中达到顶点,脾气暴躁的蒋介石扣压胡汉民,引发党内普遍的不满,胡派和汪派立刻纠集党内势力和军事势力,到广州建立新政府,并联合北方反蒋势力一起讨伐蒋介石。九·一八事变暴发,受到国际国内压力,两边政府在上海和谈,汪派和胡派裂痕渐起,但蒋还是下野,通过自己的势力操纵各派,拉扰汪精卫,从面逼胡汉民后退。

这段过程,是一种明显的国民党“无信仰”的派系争斗,由于军权与党权分裂,而各派立场也不够坚定,往往为一点私利而背叛本派系同志,内斗内行外斗外行不是假的。这样的党被共产党击败,真是必然的事情。

后两章介绍了国民党内的“省籍矛盾”和“政学系”。国民党的革命主力是广东人士,其对其他省份如江浙人较为排斥,这引起后者的不满,特别过分的是他们开会甚至只用粤语,完全不是一个全国政党的气象。对于后者,则是蒋在南京政府中重用了一批党外的学历较高、能力较强的人,这些人较多以个人关系依附于蒋介石,这有效分化了国民党其他派系对蒋的限权,扩大了蒋的实力。

本书史料丰富,内容翔实,是了解这段历史乃至国民党的重要著作。

评分:5/5。