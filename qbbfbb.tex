\subsection{穷爸爸富爸爸}

\subsubsection{一些标注}
\begin{itemize*}
	\item 在如今的美国,教师工会是所有工会中最大、最富有的一个。
	\item 如果想学习销售技能,最好进一家网络营销公司,也被称为多级营销公司。这类公司多半能够提供良好的培训项目,帮助人们克服因失败造成的沮丧和恐惧心理,这种心理往往是导致人们不成功的主要原因。
	\item 如果要破产的话,一定要在30岁以前,他的建议是“这样你还有时间东山再起”。贫穷和破产的区别是:破产是暂时的,而贫穷是永久的。
	\item 财富是将资产项产生的现金与支出项流出的现金进行比较而定的。财富就是支撑一个人生存多长时间的能力,或者说,如果我今天停止工作,我还能活多久?
	\item 害怕在公众面前说话是因为害怕被排斥、害怕冒尖、害怕被批评、害怕被嘲笑、害怕被别人所不容。简言之,是害怕与别人不同。这种心理阻碍了人们去想新办法来解决问题。
	\item 资产是能把钱放进你口袋里的东西。 负债是把钱从你口袋里取走的东西。
	\item 工作只是面对长期问题的一种暂时的解决办法。
	\item 一个人一旦停止了解有关自己的知识和信息,就会变得无知。
\end{itemize*}

\subsubsection{书评}
美国的所谓“畅销书”完全就是圈钱之作,写作质量令人作呕。

这本书作为一本投资理财的启蒙读物,还是有一些干货的,比如现金流和资金流,资产与负债的区别等,也给出了一种“成为富人”(并没有!)的方案,比如购置资产坐等升值并找机会赚一笔、捡漏、少用信用卡这样的负债手段、不要只安于“眼前的苟且”而不思进取等等。

但是,本书除了这些干货,完全就是一坨狗屎。作者以自己亲身经历说明,自己是如何“背叛”自己的高级知识分子和中产阶级的”穷爸爸“的老路并如何被同学的做生意的”富爸爸“来教导并成才的。作者出身于夏威夷,貌似是日本人后裔,但白手起家,身家到写书时至少为几百万美元。作者的经历完美诠释了美国人骨子里的实用主义和功利主义。可能美国人不全是拜金的,比如小所就裸捐,但很显然一大部分人(比如像作者这样)还是把金钱看成第一位的。

作者在字里行间表现出对”穷爸爸“的鄙夷与不屑。以金钱的观看来看,”穷爸爸“确实不值那么多金钱,但一个大学教授的价值能用金钱来衡量吗?或许作者的爸爸只是一个能力一般的教授,但传道授业解惑,为人师表,决不是几百万美元能够衡量的。作者表现出的对知识和精神追求的无知令人讨厌。相反的,作者洋洋自得地述说自己赚钱的生意经,似乎自己就是唯一成功的,并且似乎只有自己这一处成功方式,这也是我所不齿的。事实上,作者行文中的缺乏逻辑和写作技巧,让人怀疑这是他口述并让秘书写成的,并不是自己的认真的作品。作者在嘲笑有技能的中产阶级,但实际上,有知识有文化的中产阶级也会嘲笑作者的无知与狂妄。

总之这本书完全不值钱。如果你想知道这本书想讲什么,还是看一些关于投资启蒙的博客和书评更好些。