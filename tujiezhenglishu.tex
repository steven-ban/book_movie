\subsection{《图解整理术》}

这本书是日本人写的,主要内容是如何在工作中进行办公桌和信息的整理,包括文件的收纳、手账的用法、电脑的用法、笔记本的用法等等,侧重点是提高工作效率。涉及的工作形态是日本式的一般职员,有大量的符合日本特色的手账、笔记、文件夹的整理习惯,甚至办公桌都是日式的,似乎主要是2005年左右的工作习惯,与现在的工作有一点脱节和落伍。这本书一大优点是图文并茂,大量的图表和漫画来展现如何做整理,可读性很高,行文简洁详细明快。我买这本书主要是看看有没有家庭收纳的东西,结果则让我大失所望。这里的工作方法,事实在现在很多人已经在施行,除了那些落伍的和只适合于日本人的方式外,有价值的并不多。

工作文件的整理与工作信息的整理的关键是“分门别类”和“利于检索”,本书内容十分详细,可操作性很高,但方法可以举一反三。当然,现在我觉得文件信息的整理主要还是用电子式,例如各种笔记软件(有道云笔记、OneNote、DS Note等等),可以多平台同步,适合随时记录、时常整理,特别是用手机随时记录,用电脑做深度的编辑和整理,再常常用手机或平板电脑来看。为了最终形成的有价值的、系统化的内容,可以最终用\LaTeX 这样的软件进行保存,形成可打印的文件。当然,用手账和笔记记录和整理的方式具有直观迅捷的优点,是电子式笔记无法比拟的,并不算真的过时,现在仍然广泛采用,这个要看各人的习惯了。

评分:1/5。