\subsection{《女性主义》}

作者:李银河

这是一本科普读物,在书中,李银河以自己多年以来社会科学、特别是女性研究、性研究的成果为基础,简要介绍了两百年来女权主义的基本主张、发展过程、主要流派、主要论争、最新发展等内容,全面揭示了女权主义的内容。对女权主义感兴趣的人,应当好好读读这本书。

本书的内容并不算十分丰富,对各派及各阶段的论点、以及主要女权主义思想家的观点都是一笔带过,简要介绍。而且,李银河的治学态度并不算严谨认真,给出的各个论点并不是都有出处,还经常加入自己的理解,我也担心她会对一些内容进行不恰当的裁剪。特别是,李银河面对无法理解的女权主义论点,不是展现出认真阅读和独立判断,而是认为自己能力不够,无法领会,这就十分可笑了。一个观点,在你能够理解论者想表达的观点后,完全应当有自己的判断,因此不存在不能理解的情况。

另外,李银河对女权主义论题的介绍,有时过于简略,只介绍相应的论点,缺少对内部逻辑的解释。由于女权主义的时间较长,流派众多,各女权主义者的观点的根基都是不相同的。诚如李银河开宗明义地说,女权主义就是在全世界实现男女平等的思想。然而,除了这样一个相同点,任何观点都是女权主义的,我们很难去分辨一个自称女权主义者的人究竟是基于什么样的理论根基以及以什么样的逻辑来得出自己的论点的。如果仅仅是介绍主要论点,恐怕失之于肤浅。

大多数女权主义的根基,都是基于男女不平等的历史及现实。自然,在人类历史上,男尊女卑的现象是普遍的。但是,对于这种现象应当如何改变,以及应当改变成什么样的“平等”,各个流派的观点是几乎完全相反。自由主义者比较务实,比较保守,逻辑也相对严密,我个人比较赞赏;社会主义也相对温和;而激进主义以及基于福柯的重表述的文化女权主义则走上了反理性和自说自话的道路,甚至以历史上男人的方式来代替男人的“统治”,其偏激任性与启蒙主义以来的理性、平等、博爱观念背道而弛。要看到,历史上的男女不平等虽然存在,但这首先有生物学上的差异,而且这些差异在工业社会以前是无论如何也无法消除的,只有工业化才能渐渐消除这种差别,或者至少让这种差别无法引起明显的经济差异。另外,女性作为一个个的个体,被阶级、国家甚至家庭分割开来,甚至在一个家庭内部,婆婆和媳妇的利益都不是一致的,妄图让她们团结一致来反对一个虚无的“男权”是困难极大的。究其原因,“男权制”的构建就是只关注性别而忽视了人类历史发展的过程,忽视了社会实际上纷繁差异。男女之间组成家庭或组成一个公司、国家等共同体,合作的属性远高于对抗的属性。特别是家庭中男女的分工,虽然带来了人类历史上的不平等,但这种平等与否与当事人各自的权力对比有关,而且家庭与家庭也不相同。女性承担了生物学上的怀孕和哺育责任,因此从效益最大化的角度出发,必然是男性承担大量的事务性责任来负担女性的生殖,一味试图取消男女的差异并不明智。

2019年6月以后,女权主义在中国已经臭大街了,普通人对女权主义者完全失去了同情和支持,这和那些女权主义者不学无术、偏激任性有关,也与女权主义者没有明白自己作为一个独立的个体,性别属性只是一个不是那么重要的标签,更重要的是她的阶级、经济状况、受教育水平等,在具体的利益分配上,性别权重并没有那么大。认识不到这一点,就会陷入各种基于身份认同的矛盾中去。同时,那些自称为“真女权主义”者的人,面对那些下三滥手段没有去斗争,而是装聋作哑,试图投机取巧,这是表现出她们的虚伪性。

总之,为了了解女权主义,本书值得一读。但仅仅是了解而已,看完之后恐怕立刻会对女权主义失去兴趣,这东西完全没有研究的必要。

评分:3/5。