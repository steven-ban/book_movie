\subsection{十年日本}

英文片名:Ten Years Japan (2018)

由五个小故事组成:
\begin{itemize*}
	\item 75岁以上的老人,被政府强制安乐死。故事的基调很绝望灰暗,末路的老人们从相信到绝望,死亡的气息如此浓烈,特别特别丧气。
	\item 一个教育系统promise,将所有儿童耳朵上安装上耳机,教室、森林等各处均安装有摄像头来实时监控评估儿童的行为,一旦发现不符合该儿童未来的行为,就从耳机上语言劝阻直至发出刺耳的噪音。一个儿童偏偏不按设定的人生轨迹来,在他的影响下,两个儿童跟他一起把垂死将要处理的马放生,几人走进无人走过的森林,与马相会。但最终,promise监控系统重启,生活似乎退回到过去。
	\item DATA(电子遗产)。母亲死后,女儿获得母亲的电子遗产,有她生前拍过的花草、吃过的美食等等。女儿穿上母亲的连衣裙,去寻找母亲生前的经历。母女情感,细腻而可爱,父亲也很和蔼。整部片子很温情。
	\item 地面遭受核污染,人类转入地下,小女孩在不受人欢迎的女伴的影响下渴望看到地面上真实的阳光、风、雨,通过女伴挖的通道最终来到了地面。地下场景灰暗压抑,而地上世界湛蓝的天空让人迷醉,虽然只有最后一个镜头。
	\item 日本防卫省征兵,外包的公司认为征兵广告太老,不受年轻人喜欢,就让小职员去通知艺术家重新设计。小职员来到艺术家的家里,与她一起打电动,艺术家说图案与自己二战时的父亲有关,传达出反战的意愿。最后小职员离职,但征兵广告还是改成了直接的粗俗的样子。“谁来保护我们美丽的国家?”导演肯定有自己的想法,是反战的。
\end{itemize*}

几部片子题材和手法都不一样,以“十年日本”为题,可以代表了日本国内对将来政治、社会的展望,流露出一种无奈、伤感甚至绝望。