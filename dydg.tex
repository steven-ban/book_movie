\subsection{《大英帝国的崩溃与美国的诞生》}

标签: 历史\  美国史\  英国史

作者:尼克-邦克

正如作者在“序言”中所指出的,本书“将从英国的角度来说明伦敦当局是如何,以及为何会允许这样的悲剧(美国独立或“反叛”)发生”。也就是说,本书主要是从英国与美国间的关系来讲述美国独立前后的历史,对于美国的诞生背景和过程有启发意义,对于了解英国和美国历史及文化也有帮助。

史不在粗而在精细,只有精细的历史细节和梳理,才能形成清晰的脉络,才能形成正确的史感。现在从美国的角度论述美国是如何独立的历史书籍很多,站在美国人的角度,当时的英国人是压迫殖民地的自由的,因此独立战争是正确的伟大的,英国人整体上是邪恶的。然而,历史都有其不同的角度,站在英国人的角度上看待美国独立,可以为我们提供当时英美关系的视角,丰富我们对那个时代的认识,而本书正是如此行文的。正如作者所言,“这本书可以说是带着同情与体谅对失败的一方进行了探究,主要从英国政客和英国公众对这场失败的看法入手”。

\subsubsection{美国和中国}
在美国正要独立的时候,中国还处于乾隆的时代。这个时代对于中国人而言并不陌生:大量增加的人口、相对稳定的政局、赤贫的底层、日益腐败的官僚……然而在世界历史上进行评价,这个时候的中国是封闭的,专制的,落后的,压抑的。尽管如此,贸易依然让封闭的中国与快速发展的世界相联系。由中国出产的茶叶,经由广东十三行的行商,被穿越两个大洋运往英国,成为越来越多的人的奢侈品。茶叶贸易随着英国中产的形成越来越庞大,使得东印度公司这样的企业越来越激进,越来越冒险。他们拼命扩大茶叶的产量,订单都到了几年以后,远远超过了英国的实际需求。茶叶蕴含的危机成为美国独立的因素之一,这恐怕是当时的中国人想不到的。

同时,东印度公司攫取了印度的统治,作为一个商业帝国,越来越像殖民者本身。英国对本国的茶叶征了太多的税,使得茶叶的价格居高不下,鼓励了从荷兰和欧洲大陆其他国家的走私活动。东印度公司的茶叶越来越多地积压在本国,为了消除这些积压以及随之而来的公司财务问题,内阁想将这些茶叶卖往北美洲,并向美洲人消费者征税,以弥补长久以来北美殖民地漏掉的税款并缓解自己的财政危机。中国和美国,竟然因为一种商品,而奇特地联系在一起。这也显现出英国的两种属性:一种是政治的、官方的、带有国王议会和总督的,一种是商业的、私营的、非正式的,而前者的短视和后者的冒进,成为美国走向独立的条件。

\subsubsection{英国人眼中的美国}
在英国人眼中,美国的地位是和他们不平等的。它只是帝国的一个殖民地,需要国王和议会指派总督进行统治。作为殖民地,美国只不过是为他们的商业活动提供原材料、劳动力和市场,不能和帝国本身平起平坐。美国人采取的抵抗运动,在他们看来是暴民的行径,是对王权的蔑视。对美国的这种经验主义和利益主义的认识,却被政客用道德的(例如宗教的、王权的)辞藻进行美化,堂而皇之地体现在政令里。事实上从任何一个时代来看,基于海洋文明的英国人都是经验的,追逐利益的,而缺少整体性的哲学的反思。英国没有产生德国那样的哲学,没有产生法国那样的文学,他们过于依赖经验,使得他们的眼界永远缺少长远性和规划性。因此,当政客们遇见茶叶滞销的情况时,毫不犹豫地想到将茶叶运往美国进行销售以防止账务危机。当然,他们事实上有能力也有必要去真正倾听殖民地人民的呼声,合理评估此事带来的后果,在事态一步步扩大前的任何一个时刻,他们都应当停下来认真观察和思考。但是,此时的英国政客们陷于无休止的政治内斗中无法自拔,忽视了当时住在伦敦的富兰克林的文章。作者如此概括:英国政策的核心是空洞的。它本身就是无知和短视的产物,未能将美国视为一个整体。

\subsubsection{美国}
另一方面,美国则显现出与任何其他以前的国家均不相同的特点。移民美国的白人,并非没有任何文明的人,而是有着较高文化程度、有新教信仰的传承欧洲文明的人。他们在新世界扎根,类似的信仰和习俗以及较低的贫富分化程度使得他们可以采用契约的形式来结成社会团体,并采用一种自下而上的方式来进行社区治理。他们没有国王,没有贵族,也没有相互之间的压迫与剥削。当启蒙运动兴起时,美国人可以像在欧洲一样接受同样的教导,并可以在自己的土地上实施。他们认为,自己没有义务向国王纳税,也没有义务供养一个不经他们允许而设立的殖民地政府。这种诉求,在欧洲大陆任何一个国家都没有出现过,因此如果说英国人认识不到这一点就说他们短视,似乎是有点过了。究其原因,英国那样贵族林立的、基于土地所有权的体制,与美国那种平等的民主的体制,有着天然的不同,然而英国人看不到这一点,他们一再错判形势,认为美国人不会持续斗争,因此对美国人提出的请求视而不见,最终导致了美国的独立。

\subsubsection{书评}
本书的翻译相当、相当地烂,倒不是说语句的意思不对,而是完全采用了英语的句式,未转换成为汉语句式,因此读起来相当地别扭。译者在翻译上太不用心,降低了本书的可读性。

另外,即使是本书不考虑翻译问题,也仅仅是一些史料的简单的罗列,缺少作者对历史的洞察。这些史料之间缺少相互之间自然的联系,显现出作者的史学训练较差。这和很多干巴巴的历史论著的问题如出一辙。

评分:5/10。