\subsection{《癌症传》}

标签: 癌症 \  医学

作者:悉达多·穆克吉(Siddhartha Mukherjee)

\emph{文明并没有导致癌症,而是通过延长人类的寿命,暴露了癌症。}癌症的发现、治疗史,也是一部人类的医学史、技术史。

\subsubsection{治疗史}

\begin{itemize*}
	\item 1948年,法伯在儿童医院的地下开展解剖病理的工作。他采用朋友制造合成的人工叶酸拮抗剂抑制了白细胞繁殖,对治疗儿童白血病有效果。
	\item 18世纪中叶以后,随着麻醉和消毒的实施,医生开始用手术的方式切除肿瘤。
	\item 19世纪90年代到20世纪30年代,外科医生们追求癌症的病理性切除,手术方法十分激进:将想着的组织器官尽可能多地切去,以“根除”病灶。
	\item 19世纪末,X射线发现后,人们发现用X射线照射肿瘤会发生萎缩,但只对原位癌症有效,对转移性肿瘤无效或收效甚微。随后发现,X射线反而会导致癌变。
	\item 癌症的治疗重点和主线是白血病。
	\item 合成化工启示人们开展特异性药物的研究。二十世纪上半页,人们开始采用芥子气和嘌呤来杀死癌细胞,但收效甚微。
	\item 美国人建立癌症治疗与研究基金会,成为一种新的模式。他们建立癌症协会,由活动家牵头对国会开展游说。
	\item 美国癌症学会的战争式研究方式和“曼哈顿计划”类似,却与战后的国家自然基金那种无目的的研究不同。由著名的企业家和社会活动家拉斯克牵头,联合科学家、药物公司,他们在美国政治界形成了一个派系——“拉斯克派”,他们通过游说、募捐等多种方式扩大政治影响力,鼓动国家出资对抗癌症,提倡对癌症的全面的激进的治疗手段,企业在几十年内消灭癌症。然而,激进的手段并没有取得预想的成绩,外在的手术、化疗手法没有有效提高治愈率,同时令患者经受了更大的痛苦。80年代后,其影响力逐渐降低。然而,他们这种热情的、政治性的行动提高了人们对癌症的重视,也传播了癌症的知识,建立了关于癌症治疗、病友交流的“亚文化”,这种由民间发起的运动盛况在中国社会是无法出现的。
	\item 癌细胞对单一药物会产生抗药性,每种药能杀死固定比例的癌细胞,于是人们采用多种药物同时治疗,大大改善了治疗效果。
	\item 来自中国的李敏求发现了:癌症治疗必须在所有症状消失后坚持治疗,直至完全治愈,否则会复发。
	\item 睾酮和雌激素对前列腺癌,以及雌激素拮抗剂对乳腺癌有特定效果,因为某些癌变细胞依然保留着未癌变前的功能或激素受体,可被相应雌激素调控。但对于已经失去受体的癌细胞无效。这是一种简单的靶向药。
	\item 药物组合治疗开始和手术进行组合,以提高治愈率。
	\item 随着流行病学上统计和临床实践,吸烟与肺癌之间的关系越来越清晰,但是美国的烟草业采取了断然反对和不闻不问的态度。公众开始寻求联邦政府的广告管制和法律诉讼,而烟草业通过游说和诉讼胜利维持着自己的利润。烟草业开始向发展中国家扩散。
	\item 根据统计学的知识可知,想要研究常见因素与常见癌症间的关系很难,但研究罕见因素与罕见癌症间的关联则很容易。
	\item 化疗的剂量很小,因为脊髓能容许的剂量很小。因此有研究者认为可以先把脊髓抽出来,再加以大剂量的化疗药物来杀死足够多的癌细胞,最后把脊髓放回去。这样的治疗手段在开始时受到人们的追捧,美国甚至有患者打官司让保险公司出钱并且赢得了官司,但后来经证明效果有限,并且会带来并发的新的癌症。
	\item 20世纪早期曾有人认为癌症是由病毒引起的。随着逆转录病毒的发现,人们很快把癌症归因为逆转录病毒,但只有一种是由它引发的。相反,艾滋病是由HIV逆转录病毒传播并引发的。
	\item 20世纪末,已经有了很多靶向药物来对某种特定的癌症进行治疗,大大提高了患者的寿命,虽然会有抗药性的出现,但更换一种药物往往就会带来更好的效果。
	\item 21世纪初,人类基因组计划完成,随后进行了对癌症基因的测序工作。
\end{itemize*}

\subsubsection{科学事实}
\begin{itemize*}
	\item 癌症由癌细胞不受控制地繁殖引起,几乎所有癌细胞都来自于一个病变的细胞,并且每一代都会有变异,就像生物体会变异进化一样。
	\item 古代埃及的木乃伊骨骼上就发现生长有肿瘤。
	\item 中世纪的外科手术往往在理发店进行。外科医生们为了找到解剖用的尸体,常常需要去绞刑场和墓地。
	\item 可移动的肿瘤是早期的原位癌,不可移动的肿瘤是晚期浸润型的、转移的肿瘤。
	\item 癌症的难以彻底治愈,催生了安慰性质的疗养。
	\item 巴氏涂片可以在宫颈出现20前就发现有癌变征兆的细胞,可用于提前检测。类似地,乳房X射线摄影筛查可以早期诊断乳腺癌。但是这对治愈或存活没有直接的帮助。
	\item 九十年代之前的几十年内,很多癌症得到了有效的缓解甚至根治,但肺癌的发病和死亡率上升,导致实际上癌症整体的治愈率没有发生变化。
	\item 人类基因组里包含有\emph{原癌基因},在外界诱导下会使其发生突变而激活,从而引发细胞的恶性增殖,产生癌症。这种激活本质上就是基因的突变,包括两种:一种是DNA复制时发生的无序了、自发的突变,往往并不致命;一种是受外界的干扰而产生的关键性的、致命的突变。两者结合,造就了癌症。
	\item 慢粒白血病\footnote{参考\url{https://baike.baidu.com/item/\%E6\%85\%A2\%E7\%B2\%92\%E7\%99\%BD\%E8\%A1\%80\%E7\%97\%85/6147790}}是由于人类22号染色体和9号染色体一部分\emph{易位}而形成的(形成后称为\emph{费城染色体}),分别为src基因和abl基因,两者可合成一种激酶来促进细胞分裂,引发癌症。该激酶的表面有一些裂隙,通过寻找一种靶向分子,可与该激酶的深坑特异性结合,从而针对性地杀死癌细胞。然而在科学家在90年代初期找到这种分子并进行了初步实验后,诺华公司不情愿让它上市:临床测试工作需要1-2亿美元,而美国只有几千人患有这种病,难以抵消成本。经过科学家们的努力劝说,在1998年诺华才开始制造,称为“格列卫”(CGP57148,伊马替尼,imatinib)。格列卫大大\emph{治愈}了慢粒白血病,使患者的生命延长了30年。
	\item 然而,癌变的白细胞可能会产生突变,改变src-abl的结构,导致格列卫失效。通过另外的一种类似药物达沙替尼(dastatinib)可以靶向结合另一个裂隙,继续追踪并杀死癌细胞。
	\item 正常的Rb基因位于人类的13号染色体上,其会产生一种Rb蛋白,该蛋白形成一种“袋”将促使细胞分裂的蛋白质“囚禁”,当需要细胞分裂时该蛋白失活并打开,释放那些蛋白质。发生突变后(两条13号染色体均发生突变)Rb蛋白失活,因此细胞无法抑制分裂活动。人类很多癌症细胞中均发生了这种突变的基因。
	\item 人类基因组中存在着分裂-抑制分裂的机制,就像汽车的油门与刹车一样,这些机制在某些条件下(如激素、化学物质、其他基因的调控等)打开或关闭,保证人类细胞正常的分化。当这些关键的基因存在突变、特别是等位基因均发生突变后,这样的机制(特别是分裂的某些关键点上)就会产生错误,导致细胞的不受控制地分裂,同时产生的细胞不再具有正常的功能,即形成了癌症。癌症的治疗,也需要针对这样的机制进行。
	\item 癌症的产生、演化、加重到最终的全面爆发可能是缓慢的:一开始只有少部细胞发生癌变,但往往是良性的;随着环境的持续性刺激(如吸烟、X光照射等)突变会逐渐累积,当主导细胞分列的\emph{关键通路}的突变累积完成时,癌症才会全面显现。
	\item 靶向药物能针对癌症细胞进行攻击,并且对正常细胞不产生危害,是癌症药物治疗的重点。\emph{第一个靶向药物是80年代在中国上海的瑞金医院并由中国人完成的}。
	\item 癌症本质上是\emph{通路上的基因突变},目前的重点是针对通路进行靶向治疗。
\end{itemize*}

\subsubsection{癌症与社会}
癌症截止到目前依然整体上是一种无法治愈的疾病群,它是一类疾病的总称。20世纪的不少节点,人类似乎已经看到治愈癌症的希望,因此十分乐观地估计已经被人类击败,随之而来的是十分激进的治疗手段的流行:全面的扩大的手术性切除、加大剂量的放疗或化疗……这种乐观被癌症的强大一次次地击溃,医生面对病人也是小心翼翼、吞吞吐吐,病人饱受着煎熬,对医生的期待不断调整,有些病人变得不信任医生,有些病人丧失了对生活的希望。而作为医生,对这些病人也交织着复杂的同情甚至愧疚。

美国的拉斯克派建立了癌症研究所,在50年代向癌症发起了冲锋。他们游说国会,促进立法性的拨款,扩大癌症的研究。他们推广放化疗手段,宣传癌症的知识,使癌症进入普通人的视野并进行持续性的关注。这种方式,在中国可能根本无人会做,也无人能做,中国只能靠政府的关注和推广,相比这种方式滞后很多,也无力很多。截至目前,中国也没有广泛的癌症科普和抗癌行动的组织。虽然拉斯克派并没有显著地提高癌症的治愈率,但带来的社会关注为癌症的研究和治疗奠定了群众性的基础,而中国到目前还没有广泛地抗癌的风潮兴起。这或许是中国和美国在生命医学上差距扩大的一个因素。

抽烟大大引发罹患肺癌和其他相关癌症的风险,相关研究在50-60年代就已经完成,甚至烟草行业的内部研究也证实了这一点,但美国的烟草业为了行业的利益一再阻挠烟草广告,他们游说国会,迎击诉讼,即使是在败诉后也迂回地保护自己的利益。随着反烟理念逐渐深入人心,科学事实无法辩驳,美国的吸烟人口大量减少。利润就像人的鲜血,令资本的鲨鱼垂涎甚至疯狂。好在,美国已经立法禁止了香烟广告,烟草销售萎缩并稳定。相反,在中国,烟草承担了大量税收,政府的控烟似乎并没有那么积极,近年来的烟草消费不降反增,并且大量年轻人、女性也吸烟。一个触目惊心的数字是,\emph{中国烟草消费量占世界总量44\% ,吸烟人数5年增1500万}\footnote{参考\url{https://www.thepaper.cn/newsDetail_forward_1777028}},这对于癌症的防治真是百害而无一利,对于公共健康是一个沉重的负担。希望国家采取更有力的措施,遏制烟草消费,为人民的健康负责。

\subsubsection{如何面对癌症?}
癌症是如此可怕:它是由基因的累积性突变导致的,与很多病是由单一病因导致的不同,它是由多种因素缓慢而不可逆地共同导致的,这种过程可能持续好几代人,在不知不觉中将人类拖入死亡的深渊。细胞分裂的关键通路上的基因突变,导致了细胞分裂的不可控的恶性发展,最终成为肿瘤。当人们因为肿瘤增大而去求助医生时,癌症往往已经转移,进入晚期,此时无论是通过化疗、放疗、手术性地切割还是靶向药,都为时已晚。癌细胞可能会部分甚至绝大部分杀死,但也会大大损坏正常的体细胞,带来病人的虚弱;然而,少数的癌细胞会转移到安全的地方,被人体本身的安全机制保护起来,等病人结束治疗后再出其不意地反攻;即使病人是被治愈了,治疗手段可能会带来其他的癌变,依然会在病人多年以后再次给病人带来相同或不同的癌症。

癌症会进化:每一次的分裂后,癌细胞都会带来一些突变,这些突变可能给给予癌细胞抵抗特定药物的能力,导致药物治疗的失败。这种进化就像是生物种群通过大量的多次繁殖来提高适应环境的能力一样。对人类而言,这样的进化是消极的、不好的,然而对于癌细胞而言,这正是适应环境的一种表现,是生物的一种“本能”,也是生物得以生存延续的基础能力。癌症和生命一样,也是强大的。

人类在古埃及时代就发现了癌症的疑似病例,并在古罗马时代就提出的癌症的“病因”。然而,古代人们的医疗水平低,寿命也短,往往会死于其他的疾病。现在,这些疾病被人类一个又一个地攻克,人类的寿命大大延长,而大多数癌症会在中老年才显现,从而将人类终于带入了和癌症进行决战的主战场。

在20世纪上半叶,曾有科学家认为癌症是由病毒引发的。事后发现,这些能够引发人类病毒的致使基因,其实是人类基因组中部分正常基因的碎片的变体,这些病毒进入人体细胞后,会释放这些基因并合成蛋白质。可见,人类的基因组不仅是在人体内,还可能转化为其他的生物体,长途跋涉后再回到人体内兴风作浪。

本书写于2010年,截止成书到现在(2018年)已经8年多了,这8年多以来,癌症的治疗手段似乎并没有什么显著的变化,依然缓慢而没有疗效上的突破。当人们患上癌症后,依然要接受痛苦的放化疗甚至去除部分组织或器官的痛苦的手术,依然要服用费用高昂的靶向药,依然要和癌症进行多次的拉锯战,让自己身心俱疲,最终可能只会延缓几年的寿命而已。

在可预见的时间内,人类依然无法取得抗争癌症的胜利。这一代人的不少人,依然会在生命的某一阶段与癌症遭遇,并且进行长期的治疗。因此,这个时代的每一个人,都可能要被癌症陪伴。对此,我们似乎只能放平心态,祈求老天让自己幸运一点,要么不得癌症,要么得上那些容易治愈的、不会那么痛苦的癌症。万一真的被确诊,也只能配合治疗,告诉自己,生命就是这样,命运无常,人类的科技还没有那么先进,我们每一个人,其实都需要或明或暗地与癌症、与生命机制走完这个旅程。这或许是悲观的,但我们只能如此,让自己有限的生命更为有意义一点,把握当下,及时行乐,不要让自己的生命留下那么多遗憾。

\subsubsection{书评}
开卷有益,而读科普书更是益处多多。作为一本优秀的科普书,本书的内容相当丰富翔实,36万字的内容可以将任何一个读者带入癌症的奇妙世界,领略人类抗击癌症的波澜壮阔的历史。

本书的书写饱含作者的热量,他对于患者有着慈悲的心,对前辈科学家有着敬仰和钦佩,对癌症这一人类顽疾有着更为复杂的心态——无奈、惊讶、赞叹。如果说本书的干货很多就足以值得阅读的话,那么作者的这份医者仁心简直就是惊喜了。毕竟,罗列癌症的事实甚至能够说得有条理或有趣并不算太难,但结合自己的医患经历去诉说癌症对自己、对病人的影响以及随之带来的思考就很难得和可贵了。毕竟,到目前为止癌症整体上还无法治愈,看看那些患者如何与癌症抗争,更能激发读者对疾病、对生命、对自我的思考。

本书的语言流畅优美,与《基因传》一样,作者引用了大量的文学作品来抒发对癌症的复杂情感。同时,本书是理论与故事穿插,既不会让患者的故事压迫得人喘不过气,也不会让艰深的理论枯燥。这种写法,实在是很高明。

表扬一下本书的翻译:特别地流畅优美,翻译出了原文的神韵,丝毫没有翻译腔,读起来一气呵成,不会出现让读者停下来思考究竟整句话是什么意思的情况。

评分:5/5。
