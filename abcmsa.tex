\subsection{《ABC谋杀案》}

\subsubsection{事件}
署名ABC的人向波洛写信周五要有大事

亚历山大•波拿巴•卡斯特先生6月20号(周四)看列车时刻表,给名单打勾

阿谢尔,开烟草纸铺,老太太,被杀,脑后受到重击,最大嫌疑人是丈夫,关系不好,曾多次扬言杀她

以上是第一起事件,随后ABC连续向波洛写信进行了第二到四起谋杀,依次按照字母表在相应的地区杀害相应的人,并在尸体旁留下火车时刻表(字母缩写是ABC)。

\subsubsection{书评}

四起时间之间是连续系列杀人,似乎没什么特别联系,但事实上阿加莎在此隐藏了本作的基本构思(诡计):凶手通过连续杀人的幌子,将真正的目标隐藏其中,并栽赃陷害给看起来像是凶手的人来顶锅。这里阿加莎使用了双线叙事手法,给疑凶几段描写,故布疑云(事实上对于老读者而言,最像凶手的这个人肯定不是凶手)。

犯案动机上,是普通的爱情嫉妒心。凶手担心自己的哥哥与哥哥的秘书(自己的暗恋对象)结婚,于是编织了上面那个手法。凶手向波洛写信挑衅并采用连续犯案,显示出他有很重的童心,这成为波洛推理真凶的关键。疑凶患有癫痫,自卑且神经质,波洛认为他不可能挑衅大侦探。

推理手法上,依然是阿加莎擅长的层层剥茧式叙事,加上波洛强大细致的信息搜寻,黑斯廷斯的主观视角增添了案件的疑阵。

精彩程度低于罗杰疑案、无人生还和东方快车谋杀案,但作为开创这种诡计的鼻祖,阿加莎值得我再次献上膝盖。满分十分的话,给八点五分。

波洛的金句(不一定都在本篇小说里)
\begin{itemize*}
    \item 罪大恶极的凶手,所用手法一般都非常单纯。
    \item 法兰西的法庭对年轻貌美的犯人仁慈得很,对情杀案更是如此。
    \item 在雷诺先生所犯的罪行中,正如我强调的,凶手并非此案必不可少的关键因素,尸体才是。换言之,雷诺先生需要的是一具尸体,而非凶手。我们重新来梳理一下案情,试着从另一个角度分析。
    \item 一个命案中可能没有凶手,但如果有两个命案,那肯定会有两具尸体。
    \item 愚蠢——甚至糊涂,常常是与极度狡诈联系在一起的。
\end{itemize*}