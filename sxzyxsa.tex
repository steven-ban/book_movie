\subsection{《一桩事先张扬的凶杀案》}

标签: 马尔克斯 \ 文学

作者:【哥伦比亚】加西来·马尔克斯

\subsubsection{人物}

\begin{longtable}{p{0.19\textwidth} | p{0.35\textwidth} | p{0.4\textwidth}}

    \caption{《一桩事先张扬的凶杀案》人物表} \\
    \hline
姓名 & 特点 & 事件 \\
\hline
\endfirsthead

(接上表) \\
姓名 & 特点 & 事件 \\
\hline
\endhead

\hline
\endfoot
圣地亚哥·纳萨尔 & 牧场主 & 猥亵迪维娜·弗洛尔,后被杀 \\
普拉西达·利内罗 & 圣地亚哥·纳萨尔的母亲 & \\
易卜拉欣·纳萨尔 & 圣地亚哥·纳萨尔的父亲,阿拉伯人 & 已死 \\
维多利亚·古斯曼 & 纳萨尔家的厨娘 & 少女时被易卜拉欣·纳萨尔引诱,后跟他来此做厨娘 \\
迪维娜·弗洛尔 & 古斯曼和最后一个情人的的女儿 & 圣地亚哥·纳萨尔在死前那天调戏过她 \\
佩德罗·维卡里奥 & 凶手,当时24岁,双胞胎中的弟弟 & 感情用事,当兵造成他性格专断,喜欢替哥哥拿主意 \\
巴勃罗·维卡里奥 & 凶手,当时24岁,双胞胎中的哥哥 & 富有想像力,做事果断 \\
安赫拉·维卡里奥 & 佩德罗·维卡里奥和巴勃罗·维卡里奥的妹妹,漂亮 & 有种孤独无依、消沉萎靡的气质;并不喜欢巴夏尔多·圣罗曼的高高在上,不想嫁给他;凶案前一天婚礼的新娘,因不是处女而被退回 \\
克里斯托·贝多亚 & 医学院学生,圣地亚哥·纳萨尔的朋友 & 凶杀案时一时在寻找他,多年以后成为知名外科医生 \\
玛戈特 & “我”妹妹 & 不知道凶杀案的发生 \\
“我” &  & \\
路易萨·圣地亚加 & ”我“的母亲,圣地亚哥·纳萨尔的教母 &  \\
普拉·维卡里奥 & 安赫拉·维卡里奥的母亲,“我”母亲的亲姐妹 & \\ 
巴夏尔多·圣罗曼 & 凶案前一天的新郎,三十岁左右 & 难以被看透,喜欢聚会饮酒、劝架,痛恨打牌作弊,擅长游泳,很有钱 \\
庞西奥·维卡里奥 & 安赫拉·维卡里奥的父亲,失明,金匠 &  \\
普里西玛·德尔卡门 & 安赫拉·维卡里奥的母亲 & 婚前小学教师,婚后是家庭主妇 \\
阿尔伯塔·西德蒙斯 & 巴夏尔多·圣罗曼的母亲,黑白混血 & \\
佩德罗尼奥·圣罗曼将军 & 巴夏尔多·圣罗曼的父亲,保守派显赫人物 &  \\
希乌斯 & 鳏夫 & 房子漂亮,巴夏尔多·圣罗曼花大价钱买走,抑郁去世 \\
路易斯·恩 里克 & “我”的弟弟 &  \\
克洛蒂尔德·阿门塔 & 女店主,住在圣地亚哥·纳萨尔家门口 & 目睹了维卡里奥来店里杀人的过程,让人给圣地亚哥·纳萨尔报信 \\
拉萨罗·阿庞特上校 & 镇长 & 没收了维卡里奥兄弟的刀 \\
玛利亚·亚历杭德里娜·塞万提斯 & 老鸨 & 与“我”们中的很多人做过 \\
弗洛拉·米格尔 & 圣地亚哥·纳萨尔的未婚妻 & 事后绝望与人私奔,后被强迫卖淫 \\

\end{longtable}

\subsubsection{事件}
\begin{itemize*}
    \item 很多人都知道凶杀案将要发生。
    \item 凶杀案前一天,巴夏尔多·圣罗曼与安赫拉·维卡里奥举办了盛大的婚礼。巴夏尔多·圣罗曼晚上发现安赫拉·维卡里奥不是处女,就把她“退”回去,普拉·维卡里奥打了女儿,那一刻安赫拉·维卡里奥在思念巴夏尔多·圣罗曼。在哥哥们的追问下,安赫拉·维卡里奥说是圣地亚哥·纳萨尔让自己失去了童贞。
    \item 凶杀案当天早上五点,一个路过的女人告诉维多利亚·古斯曼,要发生凶杀,还说了原因和地点,她和女儿迪维娜·弗洛尔都知道这件事,但当时没有说出来(厌恶他)。圣地亚哥出门去迎接主教时,迪维娜·弗洛尔没有插上门闩以让他遇事可以退回来。有人从门下塞了一封信提醒圣地亚哥·纳萨尔会有凶杀,但他们都没有看见。
    \item “我”妹妹玛戈特邀请圣地亚哥·纳萨尔当天去“我”家吃木薯饼。
    \item 维卡里奥兄弟磨了刀,去圣地亚哥·纳萨尔门口的牛奶店里蹲点,对见到的人都扬言要杀了他(他们其实是想让别人阻止自己),镇长没收了他们的刀,他们马上又带来了两把。没收屠刀前是弟弟拿主意要杀人,没收屠刀后是哥哥执意要杀人。
    \item 圣地亚哥·纳萨尔从码头看完神父回来去未婚妻弗洛拉·米格尔家,后者很生气。他回家途中已经知道自己被维卡里奥兄弟追杀,拼命往家跑。普拉西达·利内罗听迪维娜·弗洛尔说儿女已经回到家,于是把大门关上,但儿子差几秒才能进大门,于是被追上来的维卡里奥兄弟残忍地乱刀砍伤。圣地亚哥·纳萨尔拖着身体捂着肠子绕院子走了一圈回到家里的厨房,倒地身亡。
    \item 医生不在,不够专业的神父负责验尸,把尸体弄得面目全非,草草埋葬。维卡里奥兄弟被投入监狱然后释放,他们家人都搬到了一个印第安部落。巴夏尔多·圣罗曼一蹶不振。目睹凶杀案的人和相关的人都像受到了诅咒一样伤病或者精神失常。
    \item 审判没有问出谁是那个侵犯安赫拉·维卡里奥的人,她只是坚持是圣地亚哥·纳萨尔。
    \item 安赫拉·维卡里奥一直思念巴夏尔多·圣罗曼,给他写信(一去不回),憎恨母亲。多年以后巴夏尔多·圣罗曼来见她,他一直没有打开那些信。
\end{itemize*}

\subsubsection{书评}
这本小说的故事其实很简单,涉及人物不超过30个,就是一桩凶杀案:女孩嫁入富人家,但是结婚当晚被发现不是处女,虽然这是一个天主教国家,但是不是处女的新娘多得很,很多女孩就伪装自己能“出血”来证明童贞,但是这个女孩没有;她当晚就被丈夫扭送着“退”到娘家,在母亲的逼问下说那个夺走她童贞的男人是另外一个阿拉伯人,于是女孩的两个双胞胎哥哥磨刀去杀掉那个阿拉伯人,全程都在张扬,其实是希望别人真的会阻止自己,但是见到他们拿刀的街坊邻居最多是给被害者通风报信,但并没有强力阻止他们,甚至公职人员也没有太把这事放在心上;阿拉伯男孩知道自己要被杀时已经晚了,他往家跑,母亲却因为厨娘和厨娘女儿的欺骗(阿拉伯男孩曾经猥亵厨娘女儿)而过早关上了门,阿拉件男孩被砍死在自己家门口;女孩在被退婚的当晚就爱上了丈夫,她家人在凶杀案发生后离开这个地方,她一直给丈夫写信,丈夫收到了信却没有打开,二十多年以后丈夫终于去见她。

这个故事仅仅只有九万字,但写得却十分好看,节奏控制得非常不错。故事中有“我”,但“我”仅作为多年以后采访各个相关人物的一个视角,真正的中心人物都是通过他人的陈述而渐渐丰满的。每一章都从一个视角把这个故事讲述了一遍,非常地“现代”。马尔克斯笔下的每个人都着着他惯常赋予的那种孤独感,他与人物保持着一种距离,没有掺杂过多的感情,但十分传神,短短几句话就把人物描写得活灵活现,非常有大家风度。

凶手是明确的,在第一章就已经明确了,而且凶手事先张扬了要杀人,很多人都知道凶杀案要发生,但是奇怪的是半天时间内竟然没有人可以真正阻止它的发生。一系列的巧合在让凶杀案发生。这种吊诡的“巧合”确实很魔幻。本书好的一点就是这种侦探式的层层录剥茧式的写法,每一章都是一个侧面,都是一个角度,而每一章都有时间上的跨度,给读者的感觉就是:前面渲染了那么长时间的凶杀案之前和之后发生的事情,到了最后一章才把血淋淋的事实和盘托出。

评分:5/5。