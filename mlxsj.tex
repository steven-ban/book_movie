\subsection{《美丽新世界》}
作者:赫黎胥

\subsubsection{设定}
反乌托邦小说。在未来的社会,受福特”流水线“工业方式影响的人类把这种生产方式扩大到了整个社会生活的方方面面。他们用”福帝“的称呼代替了上帝,人一生下来就按照不同的阶级和以后的工作进行分配,在受精卵的产生时严格进行数量分配,在人工的瓶子里的胚胎发育过程中使用不同的培养液,采用不同的心理暗示,在出生后就进行不同的职业培养和洗脑(每天重复播放社会准则,利用电击、强噪音等方式矫正条件反射和认知,防止出现不同的思想),成年后做一种工作,不结婚也不生孩子,在性上遵从分配原则,捐献生殖细胞给国家来统一生殖。他们整天都生活在满足里,一旦有了不好的情绪就吃”唆麻“(让人忘记痛苦,“恢复”快乐,有点类似于大麻),在药物的调节下忘记不快。以前的文学作品基本上被销毁或遗忘。在他们的认知里,一切违反上述动作方式的行为是怪异的、错误的、极端的。在这种整天快乐的刺激下,人们满足于自己的生活,维持着这种”正常”的秩序。
在文化上,所有关于父母和生育的词语都成为了禁忌,人们认为这是肮脏的、下流的、淫秽的,所有相关的书籍(如像莎士比亚的经典戏剧)都被遗忘,成为嘲笑的对象。
一切的感觉都是赤裸裸的感官刺激,取消了中间的人际交互,取消了“不必要”的追求过程。因此,人们不再需要爱情,也不再懂得爱情和浪漫,只把它解构成赤裸裸的得到和肉欲。
人的阶级分为十类,最高是阿尔法加,最低是爱普西龙减(按照希腊字母的排序)。
但这个社会里还包括一些印第安保留地,那里印第安人还保持着传统的生活方式。

\subsubsection{人物}
故事的主要人物是伯纳,他工作在人工孵化中心,是个阿尔法加。他对这个世界不满,脱离了主流习惯进行独立思考。
伯纳爱的人是列宁娜,她只是一个满足于自己生活的人,一个不会独立思考的普通人。两个人会经常性地约会并借助药物做爱。但是,由于这个社会没有婚姻,因此也没有从一而终的爱情。因此,两个人只是工作上的炮友关系,并非恋人。
约翰作为野蛮人,只读过传统文学中的莎士比亚,因此在表达被禁止的爱情、父母和憎恶情绪时,只会采用莎士比亚戏剧中的台词,这些词语都被当局禁止了。当局只鼓励“新”的东西。

\subsubsection{情节}
伯纳在孵化中心工作,反抗这么有秩序的世界,喜欢列宁娜。
伯纳和列宁娜到印第安保留地旅游,遇见孵化中心主任的情妇琳达和他们的私生子约翰(书中用“野蛮人”来指代他)。
伯纳将约翰带回“新世界”,琳达当众揭露主任的丑行,主任辞职。
约翰因为独特而出名,伯纳也因此成为名人,享受着人们的崇拜。列宁娜“爱”上约翰,约翰不知情。
约翰在歌唱家到来时未出席,沉醉于莎士比亚的“禁书”里,人们不再追捧伯纳。
列宁娜求欢约翰,约翰对她表现出兴趣。但当列宁娜按照新世界的规则脱光衣服时,饱受传统“束缚”约翰认为她淫荡色情,打她。列宁娜躲进浴室。
约翰找到临死疗养的琳达,琳达认不出他,只记得波培,约翰情绪激动扼死了她。约翰情绪激动,到别唆麻发放地点鼓励大家放弃这种廉价的开心,引发骚乱,警察过来喷唆麻,放洗脑演讲(为什么不要幸福和善良?),抓住了约翰、伯纳和赫姆霍尔兹。他们遇见了总统,总统向他们介绍了这个世界运行的原理和社会规则。随后约翰被流放到一个荒无人烟但景色秀丽的地方,孤独地与这个“文明”的世界抗争。他给自己抽鞭子,被猎奇的民众围观,拿他当猴子看。列宁娜也来了,他挥鞭向列宁娜抽去。故事的最后,他就在这个地方,孤独地存在于自己“野蛮”的世界里。

\subsubsection{小说}
情节枯燥,大段的内心独白,人物性格不鲜明,行为不合逻辑,难以捉摸,情节不吸引人。
故事的主旨是希望人不要被社会物化,不要丧失自己的灵魂,人不要成为社会达到某种目的(即使是幸福、快乐这样的褒义词)而被作为工具。人应当具备人本身的那些情绪和理想,而不是为了社会的目的而整齐划一。正如序言里所说的:
\begin{quotation}
人类是否可以这样转变,变得忘记自由的渴望、尊严、完整性、爱——也就是说,人类是否可以忘记他是一个人?或者人类本性是否有一种推动力,可以对违背这些人类基本需求的事做出反应,然后通过努力去将这个野蛮无人性的社会变成一个有人性的社会?
\end{quotation}
