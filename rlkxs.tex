\subsection{《人类酷刑史》}

1. 当人们出庭时需要举起手臂,展示他是否受过烙刑——这样,就有了宣誓做证时举起右手的惯例。

这算是一个冷知识了。本书的一大特点就是行文活泼生动,充满阅读趣味。

2. 英国政府对自愿去澳大利亚建设殖民地的人采取奖励措施:每人400英亩土地、40头牛、40个囚犯奴隶。

真羡慕大航海时代的西方列强。中国还在明清时的马尔萨斯陷阱里苦苦挣扎,列强们已经拓展了几倍几十倍的可耕种土地。

3. 司法机构开展的猎巫运动使得许多人丧生,至于死亡总数,最合理的估计是20万~100万人。与当时整个欧洲的人口总数相比会发现,这意味着几乎每200人中就有一人亡命于斯。

5. (北美移民)虽然移民们声称,自己离开不列颠和欧洲是为了躲避宗教迫害,不过,他们也仅在自己的小团体内倡导宗教自由;对有其他信仰的人则进行毫不留情地抓捕和折磨。

猎巫运动是北美的黑历史,更黑的是司法机构的参与。清教徒们自然比天主教更开放包容,比宗教裁判所要温和许多,但既然是一神论的宗教,天然对“异端”有着强烈的排斥,科学知识跟不上时,就会从上帝的误导中翻出只言片语来作为法律。托克维尔在《论美国的民主》对美国与清教徒的关系有深刻的分析,可作参考。

6. 直到1808年遭到拿破仑军队侵略的时候,西班牙仍然存在宗教裁判所。

7. 西班牙宗教裁判所自始至终都是对公正和宗教的荒诞嘲讽。

8. 从8世纪早期以来,西班牙大部分地区被信仰伊斯兰教的摩尔人(The Moors)统治着。

西班牙的宗教裁判所,是虐囚的集大成者。

9. 从1455年到1485年的30年间,英格兰贵族陷入内部争战,即著名的玫瑰战争(Wars of the Roses)。

玫瑰战争也是《冰与火之歌》的灵活源泉。

12. 英格兰自身社会并未孕育出封建制度,由于1066年的诺曼征服,封建制度才在英格兰普遍建立。


10. 19世纪中期之前的所有绞刑都是缓慢的、痛苦的、可怖的事情。

11. 无论对人类还是对动物的公开行刑,已经成为一种取悦观众的娱乐活动。

15. 任何一个国王,如果剥夺了臣民从偶尔的鞭刑和绞刑中寻求刺激及快乐的权力,将会面临被精通于如何取悦民众的下属贵族赶下台的危险。

21. 一个简单的野蛮行为与完全的酷刑之间的主要区别,就是有无更高权力机构的授权。



公开行使酷刑的社会学意义,一是震摄后来者,二是作为一种开恩的娱乐活动。行刑是由司法部位来完成的,代表王权和教权,是社会权力集中的体现,因此由之施行的公开行为是对民众“教化”的重要方式。人类自称富有同情心,然而对这种嗜血的公开表演充满期待,惨叫和鲜血刺激着肾上腺素,为平淡的生活注入调剂。人类的疯狂是文明和驯良下涌动的岩浆,平时不示于人,但潜意识里助长着公开行刑的政府行为。


13. 古代人未能认识到,对酷刑的等待与实际的折磨同样有效,而且比直接和无控制地使用暴力能够挖掘出更多的情报。

17. 在酷刑已成为常态的那些社会,每个人都清楚,如果他们被逮捕,肯定迟早会招供。

19. 酷刑的真实目的不是使真相大白,而是确保定罪。

20. 酷刑只有一个可能的结果——榨取信息或者施加制度所需要的惩罚。

酷刑与折磨的心理学探究。酷刑对人的压榨,不仅仅是折磨的进行时,还包括对折磨的期望水平。很多人在恐吓下就退缩下,很多人不怕恐吓而被渐渐加重的长时间折磨击垮,而有些人死不开口,在一次次加重的酷刑下身体被摧毁,直至死亡,作为警示案例让其他疑犯瞻仰,从而起到震摄作用。


14. 正是由于文明的原始性,汉谟拉比时期巴比伦的司法出人意料地不偏不倚。

16. 与一个不稳定的社会相比,稳定的社会往往更不容易产生酷烈的刑罚。

26. 正如本杰明·富兰克林(Benjamin Franklin)曾经说的:“不能从过去的错误中吸取教训的人注定会重蹈覆辙。”

什么样的社会会鼓励酷刑?答案是所有的社会。


22. 肉体残害,必须为达到一些特定的目标而施加时,才被当做酷刑。

23. 第13版的《不列颠百科全书》(Encyclopedia Briannica)这样解释酷刑:“酷刑(Torture),源于拉丁语‘torquere’(扭曲之意),是对变态的才智所设计的造成疼痛的众多方式的一种统称,尤其指被古代和现代的欧洲文明国家的法律所采用的。”

到底什么才是酷刑?作者做了一些探讨。酷刑的目的绝非痛苦本身,而是让对方屈服,并榨出他口中的招供。酷刑是手段而非目的。

25. 我们发现,过于冗长的序言本身就是一种对他人的最过分和不文明的折磨。

作者卖了个萌,然而我想说,本书写得很好,绝非任何折磨。