\subsection{《男人的一半是女人》}

标签: 知青 \ 劳改 \ 右派  \ 文化大革命 \ 伤痕文学

作者:张贤亮

这本小说以作者张贤亮自己的生活经历为原型,成书于1985年,主要是“我”在文革期间的(事实上是作者成书时的思想)劳改、结婚和思想状况。作者自己在1957年因写《大风歌》而被打成右派,被送到宁夏进行劳动改造,一改造就是二十年,本书的“我”也是这个背景。“我”在文革期间因为是劳改犯,不是知识分子这种“走资派”,反而少了很多运动被整的可能,只要在农场听管理人员的话,好好劳动就行。

劳动本身是苦的,特别是农场的工作包括插秧收割、放牛放马等工作,又脏又累,但“我”已经习惯,觉得比做文字工作好,后者显然更加“危险”(这自然是一种嘲讽)。但十多年的劳动改造让“我”和一众人无法接触到女人,因此本质上是性压抑的,对女人、对性都只有幻想,这种压抑伴随着文革和之前的左倾的政治上的高压环境,对人心理上产生一种折磨、囚禁和鞭笞。1967年左右,“我”在一次劳动期间在无人处偷看到一个女人洗澡,这种“新鲜”使得我默默地观看,没有去打扰她,“我”本身也是一种在欣赏美的心态在看,这和当时的压抑风气形成了一种强烈的对比。然而随后就被她发现,后来“我”知道,她叫黄香久,学历只是初中,但是也算一种知识分子了。之后“我”和黄香久在路上遇见,她悄悄对“我”说“恨不得杀了我”。

再之后“我”就再也没有见到过她,一直到八年后(1975年)“我”被安排到另外一个地方去放羊,才再次见到她,她也被支书安排到附近,“我”和她一起工作,一来二往就熟悉了,知道她结了两次婚,又都离婚了。在别人的介绍下,有了好感的两个人结婚了,“我”的生活被有了安稳踏实的感觉,她把“家”(其实就是一个小屋子改的)照顾得井井有条,“我”不用再去吃大锅饭,被子衣服也都经过她的修理而井井有条。但经过长时间的压抑,“我”竟然阳萎了!几次尝试后没有改善,之后两人就没有性生活了,心理上也日渐疏离,甚至出现了床上分边睡的情况。后来“我”在一次放马途中回家,发现一直对她“照顾有加”书记深夜里从家里出来,因此怀疑两人偷情(似乎本书并没有直言,但应该是偷情的),之后一直加重了对她的怀疑,但她离了两次婚,不想再离婚。一次抢险中,“我”身先士卒堵住了拦截洪水的大坝的漏洞,回家之后竟然不再阳萎,两人重新过上了“正常”的夫妻生活,甚至一次她给“我”在工作间送衣服送饭两人还在一起向恋爱的青年一样浪漫地相拥。

之前她攒钱给“我”买了一台收音机,“我”于是经常收听新闻。随着1976年地到来,周总理去世,两人以及其他劳改犯、走资派(黄香久不是劳改犯)经常性的写申诉材料的生活还是没有变化。“我”想起她出轨书记的事情,越来越觉得烦躁,两人感情和生活上吵架猜忌,忍无可忍后“我”去离婚签字。但是,黄香久已经爱上了“我”,她撕掉了协议书(其实两个根本就没有领取结婚证),两个过上了有希望的日子。

本书还包括了大段的思考,无一不是对当时时代的反思、控诉和嘲讽,对无情且僵化的政治话语的解构。作者对把他打成右派的政制体制、思想乃至社会主义、领袖等不乏批评。而且,本书里还借“大青马”之口,提出人就像生物一样,被驯化被压制了,被阉割了思想(就像大青马被阉割了一样),“我”也曾在发现黄香久偷情的深夜里和马克思等人对话,马克思说自己的思想被窜改了,可以说这里作者直指当时的左倾思想和毛本人了。同时,这种超现实也是魔幻现实主义的借鉴,可见八十年代马尔克斯对国内这些作家(例如莫言、陈忠实、贾平凹)的影响有多大。

总之,作者从反右起一直都是时代的亲历者和受害者,这些一次又一次的运动就是错误的(至少整体上是错误的,它可能有一定的必然性,当然这种必然性很小),一是严重伤害了知识分子的生活思想经济权利、对社会主义的认同感和积极性,严重阻碍了团结更多“先进力量”(工农个人本身并不是先进的,因为他们掌握了更少的知识)参与社会建设和治理,二是对社会主义思想本身也是极度的扭曲,二是在之后的一段时间内,也严重动摇了共产党和社会主义制度本身的合法性,河觞派、自由派公知等人自然有外界因素(如CIA的策反、世界范围内自由主义的传播等)的影响,有近代以来历次屈辱历史而面对西方文明产生“不自信”的心理,但最直接的,还是中国在建国后历次的错误。本书作为“伤痕文学”的代表作之一,现在看来对这个时代的反思也是基本正确的,后来人应当对当时的时代进行反思,防止同样的错误再次发生。我想最直接的、最大的反思,就是一个社会应当确定一条底线,就是对“法治”和“个人权利”的尊重,人不应当仅仅是政治动物,仅仅作为政治的一分子,人本身还有生物性、自由性,应当保证人的这一方面的自由。

本书的书名是“男人的一半是女人”,说实话我并有理解究竟是什么意思,作者也没有明说。就本书里涉及的男人女人之间的关系而言,仍然是比较传统的,女人(黄香久)对“我”而言有如下的含义:
\begin{itemize*}
    \item 性幻想的对象。“我”偷窥黄香久洗澡,是压抑时代下对美的欣赏,是发现了自己未曾接触过的“个人自由”
    \item 贤良的妻子。这仅仅是一种工具属性,但也被时代压抑了。
    \item 压抑时代下的个人自由,而这种自由首先就是指向性的。有了女人,“我”就有了合法的性生活的权利。但是,这种权利也是组织介绍和允许的。
    \item 阳萎下不甘心。阳萎可能代表了人无法掌握自己欲望的隐喻,是人失去自由和能力的隐喻。
    \item 个人财产的剥夺。是的,在“我”眼里,妻子与书记偷情,就是国家和政治权力对属于家庭财产或个人财产的夺取,妻子本身的不忠是因为“我”的欲望被夺走,因此不得不“求助”国家和体制。
\end{itemize*}

最后“我”和黄香久似乎可以过上正常的日子(婚姻生活的描写里依然是以“我”为主,或许是那个年代生活的枯燥,黄香久的人物形象也单一乏味,被作者强行赋予了一个“家”的含义),后来者也知道1976年马上文革就结束了,本书的结尾是满怀希望的。

本书的文笔只能说是还算可以,对人物心理的把握比较精准,而且景色描写大断的心理独白也属于作家中不错的。然而作为小说,本书最大的缺点是夹叙夹议的风格造成了割裂,让人分不清作者本身和“我”的区别,读者很难判断两人是不是同一个人?而且其他人物的形象都比较单薄。本书对“性”的描写可以说是比较直白的,也怪不得当时很多人传阅(可见大多数人还是喜欢窥私和情色的,是比较“三俗”的),但性描写本身的文学价值有限,而这有限的价值里,包括了对压抑的时代的隐喻和批评。因此,本书的价值,主要是政治上的,但这种价值也是比较“浅”的,直白的,并没有更加深刻的揭露。所以看到这类伤痕文学,邓小平说“哭哭啼啼,没有出息”也有一定道理的,这个时代本身有着巨大的错误,给知识分子带来了巨大的伤痛,但并没有产生伟大的能够匹配这种苦难的文学。可能的原因是前三十年本身社会环境就比较封闭,与外界交流比较少,知识分子的思想资源比较贫瘠。放在世界文学上,这种文字更不能入流。可见,建国前三十年的封闭对中国文学的影响是很大的,不仅仅是物质上的,还包括这种深刻的心灵上的。另一方面来说,改革开放四十年了,中国似乎也没有出现伟大的文学,伟大文学一方面靠时代,但更重要的是文学家本身。比如,对黄香久和“我”之间的感情互动就可以再细腻一些。因此,作者本人的文学能力,只能说是及格,但缺少文学的灵性。当然,作者张贤亮平反时,已经四十多岁了,人生最美好的年华已经丧失,而且在二十年间承受着常人难以承受的苦难,之后能保持五十年代的文学水准,还还是让人满意的。至于张贤亮之后成为商人,运营西北影视城,甚至一把年纪出现包养情妇的传闻(如果是真,那真是的对早年性压抑的补偿的,也没什么好指摘的),那就是另外一个故事了。张贤亮本身的遭遇,其实比这本书要精彩得多,其他中国任何一个经过过二十世纪主要历史阶段的人,他的人生都足够精彩到写成一本不错的小说,真是可惜了中国那个时代,也可惜了中国的文学。

评分:2/5。
