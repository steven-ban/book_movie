\subsection{《汽车是怎样跑起来的》}

作者:【日本】御堀直嗣

\subsubsection{一些标注}

为了提高汽车安全性,汽车越来越重,从而提高了油耗,减少了前方视野,反而更容易产生交通事故。\emph{汽车体积与碰撞能量成正比},因此需要开发出体积更大的车来确保安全。

汽车的五大要素:行驶,转向,停车(刹车),舒适性,安全性。

汽车大约由2万个零件组成。

汽车轮胎由黑色橡胶制成,受热时产生黏性,冷却时变硬。受热时变软,更容易贴紧地面,提高了抓地力。汽车行驶时轮胎与地面接触时产生弹性变形,摩擦生热从而产生抓地力。

发动机启动时,需要先用起动机促使曲轴旋转,带动活塞上下运动,产生四冲程。

空气与汽油理想混合比为\emph{空燃比},为14.7:1,此时汽油刚好燃尽。

进气阀位于气缸盖上,进气道倾斜以便混合气体(空气+燃油)流入气缸,这是为了使混合气体均匀、充分燃烧而设计的。

活塞升至上止点时,仅留有极狭小的空间,我们称其为燃烧室。燃烧室的容积和排气量的比,就是压缩比。压缩比越大,发动机做功越多,效率就越高。要增大压缩比,需要使用高辛烷值汽油。辛烷值是表示汽油抗爆性的指标,其值越高,汽油越难自燃。

排气量是发动机大小的指标,是指活塞上止点到下止点间的容积,也就等于气缸所能吸入的混合气体的量。活塞升至上止点时,仅留有极狭小的空间,称为燃烧室,燃烧室的容积和排气量的比,就是压缩比。

自动变速箱(AT)由液力矩变扭器、行星齿轮和油液循环路线等复杂的装置组合而成。

汽车加速时,后轮耗费了很大的驱动力,此时前轮上还残留着一部分抓地力,但后轮上的抓地力已经基本没有了。(这是针对前置后驱来说的)

摩擦大、黏性强的橡胶抵抗损耗的力也很大,因此能够增大抓地力。

汽车转向时,差速器用于改变左右车轮的转速,使里侧转速慢,外侧转速快。差速器的差速功能的发明人是汽车的发明人卡尔·本茨。

悬架由弹簧、减震器、稳定器、悬架摆臂和衬套组成,其中弹簧、减震器和稳定器负责在转向时调整车身的倾斜程度。

前置后驱的汽车,前轮制动器的制动负担更重。

轮胎不转动就无法制动,因此抱死后无法有效制动。

发动机制动时不给油,使发动机空转,降档(低速齿轮)下发动机转速增加,活塞的上下往复次数增加,摩擦阴力变大,汽车的速度就相应降低了。

NVH:Noise Vibration Harshness,指汽车的噪音、振动和声音粗糙,它们决定了汽车的舒适性。

发动机转速较慢时,吸入发动机内部的空气强度也较小,因此扭矩较小。随着转速增加,吸入更多空气,汽油量也随之增加,发动机的扭矩也随之变大,但当转速超过一定的数值时,气缸和活塞的摩擦等各种阻力就会增加,使得扭矩逐渐减小。

电动车里的锂离子电池包括多个单体电池,每个单体电池的电压为3.7V。

\subsubsection{书评}
十分详尽的汽车原理的科普书籍,浅显易懂,而且重点也很突出,对汽车有兴趣的人值得一读。很多开了多年车的老司机们,对这些基本的汽车的运行原理也缺少基本的认识,导致出现了“热车”“清洗油路”的反科学操作。同时,作为一种成熟的工业品,汽车其实是很耐“造”的,但需要好好保养,这也是很多驾驶员不太关心的地方。站在汽车工程技术人员的角度上来看待汽车,是正确使用汽车的前提。

评分:4/5。