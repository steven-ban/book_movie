\subsection{《沙丘》}

作者:【美】弗兰克·赫伯特

设定挺好的,很宏大,有想象力。虽然明显是世界各地宗教历史的简单拼接,但如果笔力雄厚是可以写出不错的篇章的。但显然,以文学成就而论,赫伯特的笔力并不够。情节的推进缺少人物本身的内在逻辑,基本就是靠作者这个“上帝”来强行安排。而且,堂堂皇帝竟然仓促之间败给一个相当于县级的叛乱武装,逻辑崩塌,简直就是笑话。

故事其实是单线进行,洋洋洒洒几十万字,情节却极其简单。小说充斥着装逼十足神神叨叨的对话,但核心就是龙傲天的自说自话。保罗这个主角的人格魅力为零,除了西方典型的血统论之外,根本就没有领袖能力,但强行被弗雷曼人供成神明,后期的战争也草草带过,跟过家家一样。

大概的设定是:
\begin{enumerate*}
    \item 在离现在(20世纪60年代左右)相距较近的时代里,人类废弃了计算机,并全面殖民银河系的各个星球;
    \item 人类政治上采用皇帝+总督模式,保留着中世纪的领主和血统形式,没有民主和选举;
    \item 某个充满沙丘、极度缺少水源的星球上,巨大的沙漠里有沙虫,沙虫制造了“香料”,这些香料具有一定的魔力,对于其他星都是有很大用处的,而香料的销售被某个势力很大的公会所垄断,缺少他们的支持,即使是皇帝也不能支持庞大的统治;
    \item 星球上的人需要穿着“蒸馏服”来保存水分,对于土著弗雷曼人,他们在缺水的情况下形成一种部落、宗教文明,对水极为重视;
    \item 人类分为不同的具有不同特性的人种,例如女巫、斗士等,女巫比较厉害,可以共享所有记忆(平时需要训练和吃香料),而不同的斗士具有精确的计算和格斗能力;另外,某些人还发展出敏锐的洞察力、意识读取能力、意识干扰能力等;
    \item 预言中的会在沙漠中成为王的人叫保罗,他是星上总督的儿子,母亲是一名女巫,犯下忌讳生下了他,而母亲的家族是他们的死对头,设计害死了总督;
    \item 保罗受到女巫和斗士的训练,在父亲死后与母亲一起流落沙漠,进入弗雷曼人部落,通过高超的格斗技巧成了首领,与一个弗雷曼女孩生下了儿子,后来在战争中死去;
    \item 保罗率领弗雷曼人反抗皇帝和总督(母亲的家族)的统治,在战争中获胜,杀死了杀父仇人,逼皇帝任命自己成为总督,并娶了皇帝的女儿以获取帝位继承权(此时一旁的弗雷曼妻子竟然同意了?)。故事结束。
\end{enumerate*}

虽然很多科幻粉一直强调科幻重设定而文笔是其次,但我并不认同这个观点。科幻小说也是小说,也要按照普遍的评价标准来看待。如果以小说的一般标准来看,《沙丘》在人物塑造、情节推进、细节描写方面只能算是勉强及格的小说。

科幻也好,奇幻也好,在我的评价体系里,优秀的标准是《冰与火之歌》。即使不算设定,就小说本身的文采来看,《冰火》也是极为优秀的。

另外,翻译也有一部分因素。我读的这个版本,翻译比较差,语气生硬,文字毫无美感。