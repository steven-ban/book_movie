\subsection{《最后的话》}

作者:瞿秋白

这是瞿秋白的散文集,分为《最后的话》《乱弹》《散论》。

先说《乱弹》《散论》。这是作者对当时(20-30年代中期)的中国国内社会的评论集,是杂文,从行文风格、遣词造句上来看是模仿了鲁迅杂文。从思想上来看,作者是典型的鲁迅式左派知识分子风格。对传统社会以及当时宣扬传统的人,作者是批判的;对胡适这样的自由主义分子,作者更是火力全开;对当时的国民政府,对其妥协政策,作者他是全盘反对。作者是站在左派、共产党和底层人民的立场,认为应当对社会予以改革,而不是温和的改良、妥协。作为认为,以当时中国的局势,受到帝国主义、国内官僚资本主义的人的共同压迫,中间道路是完全行不通的。从语言风格上上来看,他与鲁迅很相像(事实上与鲁迅也十分要好),行文短小,只求对社会问题奋力一击。当然,从思想和文艺价值上来说,是不及鲁迅的。

第一部分《最后的话》则是我最震撼和喜欢的部分。这是瞿秋白在狱中回忆自己短暂的一生的心路历程,包括以下几篇短文:
\begin{itemize*}
    \item 何必说?(代序)
    \item “历史的误会”
    \item 脆弱的二元人物
    \item 我和马克思主义
    \item 盲动主义和立三路线
    \item “文人”
    \item 告别
\end{itemize*}


他早年母亲自杀,自己跑到北京,学习俄文,本来是想考上北大当教师的,但是在1918年接触到了共产主义思想,去俄国留学,后来回国,当过陈独秀的俄文翻译,1927年成了党的关键人物。他对政治工作不算热衷和擅长,“独秀是事无大小都参加和主持的,我却因为对组织尤其是军事非常不明了也毫无兴趣,所以只发表一般的政治主张,其实调遣人员和实行的具体计划等就完全听组织部军事部去办”,之后一直推脱不做领导工作,从1925年到1931年他做了5年的中共领导,但是一直认为“像我这样的性格、才能、学识,当中国共产党的领袖确实是一个‘历史的误会’”。究其原因,作者认为自己就像一只羸弱的马拖着几千斤的辎重车,已经十分疲惫。他认为自己本质上是个文人、士绅,认同和接受了马克思主义,但是仍然无法与自己内心深处的认识和外在的行动力相匹配协调:
\begin{quotation}
马克思主义是什么?是无产阶级的宇宙观和人生观。这同我潜伏的绅士意识、中国式的士大夫意识,以及后来蜕变出来的小资产阶级或者市侩式的意识,完全是敌对的地位。……这两种意识在我内心里不断的斗争,也就侵蚀了我极大部分精力。我得时时刻刻压制自己的绅士和游民式的情感,极勉强的用我所学到的马克思主义的理智来创造新的情感,新的感觉方法。可是无产阶级意识在我的内心深处是始终没有得到真正的胜利的。
\end{quotation}
他批评自己:
\begin{quotation}
    读书的高等游民,他什么都懂的一点,可是一点没有真实的智识。
\end{quotation}
他开会时也是觉得“是替别人做的”,中央说什么就是什么,没有自己的主风、部署:
\begin{quotation}
    因为我在政治上的疲劳、倦怠,内心的思想斗争不能再持续了,老实说,在四中全会以后,我早已成为了十足的市侩——对于政治问题我竭力避免发表意见,中央怎么说,我就依着怎么说,认为我说错了,我立刻承认错误,也没有什么心思去辩白,说我是机会主义就是机会主义好了;一切工作只要交代过去就算了。我对于政治和党的种种问题,真没有兴趣去注意和研究。
\end{quotation}

瞿秋白说自己小时候,就想不明白为什么读书就要“治国平天下”,认为各人研究一门技艺或学问不就可以了?他说自己读马克思主义很少,一知半解,没弄明白,更没有去做把马克思主义和中国社会相结合的理论工作,但当时中国接触马克思主义正统理论的更少,于是机缘巧合成了中央领导。对于“盲动主义”“立三路线”,他承认自己应当承担责任。

最后,作者向世界告别,写得十分感人:
\begin{quotation}
    告别了,这世界的一切。
    最后……
    俄国高尔基的《四十年》《克里摩·萨摩京的生活》,屠格涅夫的《鲁定》,托尔斯泰的《安娜·卡列宁娜》,中国鲁迅的《阿Q正传》,茅盾的《动摇》,曹雪芹的《红楼梦》,都很可以再读一读。
    中国的豆腐也是很好吃的东西,世界第一。
    永别了!
\end{quotation}

矍秋白的一生,是一个中国式的文人在激烈的时代里接受马克思主义和共产主义,并投身革命的一生,他虽然有着文人的一些缺点,缺少行动力和政治上的洞察力,缺少领袖的精神素质,但是被时代推上了领导岗位,他努力工作,但是仍然无法胜任。从经历上来说,他可能并不是一个好的领导人物;但是如上所述,在个人层面,他是一个真正的共产党员,有理想,有激情,他可能更适合做一些文艺和宣传工作,写写杂文,宣传宣传左派的路线。他列出了那么多自己的缺点(其实未必就是缺点),剖析自己的内心,正说明他是一个勇敢的、磊落的人,这种人在人格上来说,就是一个不平凡的人。有这样的人参加共产党,我们才能发现,在那个共产主义屡屡受挫的20-30年代中期,这群人为什么可以在十多年后夺取全国的政权,为什么可以把新中国建设成强大的国家。总之,就像官方对他的正面评价一样,我个人十分敬佩这样的人。

评分:4/5。