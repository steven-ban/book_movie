\subsection{《我们》}
\subsubsection{一些标注}
5. 进了天堂就再也没有欲望,没有怜悯,没有爱。天堂里那些天使、上帝的奴仆……他们都是幸福的,他们都摘除了幻想(惟其如此他们才幸福)。我们已经追上了这个幻想,

6. 人们从小就祈祷、梦想、渴盼什么呢?就是希望有人能够确定不移地告诉他们什么是幸福,再用锁链把他们和这个幸福拴在一起。

7. (上帝)早在人还处于野蛮状态、全身覆盖着毛发的时代他就认识到,对人类的真爱、代数学意义上的爱,就在于残酷——残酷正是真爱的必然标志。 

11. 人只喜欢他无法占有的东西。

12. (我们中的音乐)那时合时分的无穷的行列发出的水晶般清晰的半音音阶,以及那泰勒[4]和麦克劳林[5]公式的整合和弦,那毕达哥拉斯短裤[6]、厚重的二次方全音转调,那衰竭震颤运动的忧郁旋律,那随着由许多个休止组成的夫琅和费谱线[7]而变换着的明快节拍——行星的光谱……气势多么磅礴!章法多么严谨!而古代人的音乐随心所欲,毫无规则,无非是一些野性的狂想,这种音乐多么渺小可怜……

\subsubsection{书评}
\begin{quotation}
坐在宽大的扶手椅里的她,真像一只蜜蜂——她身上既有刺,又有蜜。
\end{quotation}

这句话来形容爱情,真是太贴切了!太美妙了!