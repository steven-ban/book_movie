\subsection{《雅舍小品》}

标签: 散文 \ 梁实秋 \ 民国 \ 台湾

作者:梁实秋

这是梁实秋的四本散文或者说随笔集子,整体篇幅、字数都不长,内容是关于作者平时生活的感悟,没有什么高远的政治性的或者更为深刻的内容。这里的短篇涉及的生活琐事是一些对常见事物的议论,例如理发、猫、狗、烧饼油条等生活中常见的事物,有描写,有回忆,有议论,短小而随意。文笔上来说,属于文白夹杂,既有干净利落、三言两语的事物描写,也有“某某古书记载”这样的掉书袋,也有作者的一句两句的讽刺,整体风格属于恬淡、随意的流派,没有大悲大喜,没有深仇大恨。

作者长于北京,属于有产阶层,而且自幼读书,青年时代去美国留学,与胡适属于同乡(安徽绩溪)。他家里不算大富大贵,但也绝不贫穷,这种生活经历可能与他之后的文学追求有关。他不同于左派文学,认为文学只有人性的普遍追求,没有阶级性,因而不像鲁迅那样的文字笔下有火焰在燃烧。他有着古代文人士大夫的那种情趣追求,对于社会百态绝少关注与感怀,生活无非是吟风弄月,呼朋引伴,虽然写的很多是现代的事情,但真的没有现代人应有的那种平民视角。你看他写中年,“中年的妙趣,在于相当的认识人生,认识自己,从而做自己所能做的事,享受自己所能享受的生活”,这对于他这种有闲有产阶级,自然不错,但对于普通大众,即使是富商官僚,也不是容易取得的,中年有太多的艰辛和无奈,而又不能放下,像他这样闲云野鹤,无法多见。他观察生活,可谓能见到细微处,但却没有看到大局处,这可能是他这种文人的通病。当然,文人并不都是这样,像瞿秋白,也是自称文人,但是他一直在反省自己的文人习气,虽然不适合政治生活,但是依然对世界、对社会、对普通人饱含热情,“中国的豆腐是极好的,天下第一”,这一句话就把梁实秋这种单纯的文人墨客比下去了。

他翻译莎士比亚,属于翻译界的开流派之先的人。他笔下对于穷人,有一种鄙弃,例如对药店里的“乡巴佬”(虽然表现的不明显)、对北京的穷人等等,缺少热情,只是冷眼旁观,俯视芸芸众生。在《猫》这篇散文里,他记录了自己家的仆人在窗户上设置陷阱并在猫身上坠上一个罐子,自己也没有阻止,可见心肠并不算好。他晚年来到台湾,自然集子里也记录了台湾的生活,但并没有很台湾化,可见他们这些“外省人”,与本地的风土人情并没有十分契合,维持着一种“人上人”的生活。当然,作者自己是感受不到的。我们站在读者的角度上,似乎也不应当拿这些东西来苛责他,毕竟穷人的苦难、社会的百态不是他一个人造就的,他对于社会,只是一个隐士式的文人而已,有人喜欢这种情趣,但是肯定也有人很不喜欢,毕竟格局、胸襟小了些,不算是一流的文字。

评分:2/5。