\subsection{《代谢增长论:技术小波和文明兴衰》}

英文书名:Metabolic Growth Theory: Technology Wavelets and the Rise and Fall of Civilizations

作者:陈平

本书宗旨,是\emph{尝试以新的非平衡态物理学的方法,从技术进步作为根本动力学来重构经济发展演化的理论,以此来解释经济系统的复杂性和文明演化的多样性}\footnote{林毅夫}。从方法论上讲,是\emph{经济复杂性的生态物理基础}。代谢增长论在经济学上的贡献,在于\emph{把生态资源下约束下的技术竞争造成的产业新陈代谢而非资本积累作为经济增长的动力}。

亚当·斯密在《国富论》第三篇第三章提出\emph{斯密定理:分工受市场规模的限制,基础模型的逻辑斯蒂S形增长曲线}。代谢增长论提出\emph{一般斯密定理:劳动分工受市场范围、资源种类和环境波动的三重限制}。

“为什么资本主义和现代科学源于西欧而非中国?”(李约瑟之问)陈平认为,主要原因如下:
\begin{itemize*}
	\item 欧洲(包括传统的欧洲及地中海沿岸,显然也应包括地理大发现后的美洲)存在广阔的大平原,海运发达,人均资源占有量大,主要经济方式是农牧混合;反观中国则多山少地,运输不例,在东周时期便分解为小农经济并维持至20世纪。
	\item  欧洲航海需要确定经度,这比中国南北向的海岸线上只需要确定纬度相比难度更大,刺激了天文学、数学、光学等学科的发展。
	\item 欧洲农牧混合的经济生产方式下,肉类储藏需要香料,并且只能从印度、东南亚等地进口,存在拓展海外航线的动机。
	\item 中国基于小农经济,发展出庞大的专制帝国,反对一切形式的科技创新,一味追求政治经济上的稳定和可控,导致科技无法发展。因此,\emph{单一小农经济结构是我国动乱贫困、闭关自守的病根}。
	\item 日本在经济上更接近欧洲而非中国,依赖海运,同时政治上更接近欧洲的封建制,因此现代化转型的压力比中国小得多。
\end{itemize*}

作者猛烈批判了新古典经济学的肤浅。作者认为,新古典经济学的理性人假设、市场自动均衡理论过于简单,无法反映真实的经济现象,同时过度依赖数学模型和公式,背离了古典政治经济学的客观和实事求是。欧美新古典经济学派设计的“休克疗法”导致苏联旧加盟共和国经济一落千丈,损失超过二战。“华盛顿共识”强调最小化和中立化政府,认为政府不应当干预经济,这首先不是欧美国家的历史和现实(从资本主义兴起开始算,政府就一直在干预市场,“持剑经商”),其次也不能促进后发国家的经济发展。相反,中国的改革开放走的是一条独立自主、混合经济和逐步放开的过程,符合历史发展规律,高关税既保护和脆弱的国内企业,同时从沿海到内地的逐步开放也使得国内企业学习西方竞争对手的管理经验并学会竞争共存。从现在来看,中国的政策才是正确的。

索洛的外生增长模型基于规模报酬不变假设,预言经济增长的收敛趋势;罗默的内生增长模型基于知识积累的规模报酬递增假设,宣称经济增长有发散趋势。然而真实的情况要复杂得多,因此国家、产业和企业并非强者恒强、弱者恒弱,而是在不同时期表现出不同的增长速度。因此,应当采用\emph{有限增长的逻辑斯蒂模型}来解释经济增长,对于二次型情况,有
\begin{equation}
\frac{\mathrm{d}n}{\mathrm{d}t} = f(n) = kn(N^{*}-n)
\end{equation}
其中$n$是产出量,$N^{*}$是资源约束,$k$是产出的增长率。

在竞争状况下,假如存在两个竞争者(双物种竞争模型),存在Lotka-Volterra方程:
\begin{align}
\frac{\mathrm{d}n_1}{\mathrm{d}t} & = k_1 n_1 (N_1 -n_1-\beta n_2)-R_1 n_1 \\
\frac{\mathrm{d}n_2}{\mathrm{d}t} & = k_2 n_2 (N_2 -n_2 -\beta n_1)-R_2 n_2
\end{align}
其中$n_1$、$n_2$是竞争者的产出,$k_1$、$k_2$是各自的学习率,$N_1$、$N_2$是各自的资源限制或市场规模限制,$R_1$、$R_2$是各自的退出率,$\beta$是两者的竞争系数。上述公式可引入有效资源约束$C_i = N_i - \frac{R_i}{k_i}$来简化。

通过研究上述方程在不同情况下的解,可解释相应的竞争情况。例如,上述方程可用于描述新旧技术的竞争、同一市场下的竞争者的竞争等。对于前者,存在幼稚期、成长期、成熟期和衰退期(类型于专利分析中申请量的变化),不同的周期内政府需要进行相应的引导和监管。

集体主义的资源利用率高于个人主义的资源利用率,因此在产业扩张阶段,个人主义文化有较大优势,但对于资源有限的情况,集体主义可在不耗费过多资源的情况下存在。就在解释了为什么中国通过扩大人力来提高粮食产量,而欧洲同向外扩张殖民来养活人口。另外,两个个人主义集团可共存,一个个人主义一个集体主义集团也可共存并具有更高的稳定性(改革开放中混合经济的好处),而两个集体主义集团无法共存(中国历史上改朝换代的惨烈和大一统的习惯)。

这本书内容相当丰富深刻,可以看出作者对社会、历史和经济的洞察。这本书的数理可能偏弱,“代谢增长论”用于取代新古典经济学似乎有点粗糙。虽然可以解释作者提出的问题,但是否可以推广开来解释更多普遍的经济学现象还存在疑问,这可能有待于更多数据上检验。但陈样在本书中表现出来的利用物理学、复杂科学、生态学的手段“推广”或“借用”至经济学,显得大胆,这对于做学问是必须的,毕竟可以解释更多的现象。

评分:10/10。
