\subsection{《Python统计分析》}

作者:【奥】Thomas Haslwanter

单变量分布

\begin{table}[htpb]
\centering
\caption{单变量分布含义及对应的函数}
\begin{tabular}{r|l|p{0.5\textwidth}}

统计量  &  含义  &  计算方法 \\
\hline 
均值  &    & \verb | np.mean() | \\
中位数  &    &  \verb | np.median() |\\ 
众数  &  出现最频繁的数  &  \verb | scipy.stats.mode(data) | \\
几何平均值  &   &  \verb | scipy.stats.gmean(data) |\\ 
极差  &  最大值与最小值的差  &  \verb | np.ptp() | \\
百分位数  &  低于数据中特定百分比的数据的值  &  \\
标准差  &   &  \verb | np.std(data, ddof=1) |,为无偏估计(分母为n-1),否则为有偏估计(分母为n) \\
标准误  &  $SEM=\frac{s}{\sqrt{n}}$  &  \\
置信区间  &  包含参数真实值的范围  & \\

\end{tabular}

\end{table}

离散分布:
\begin{itemize*}
    \item 伯努利分布:$p_{head} + p_{tails} = 1$,产生方法:\verb | bernoulliDist = stats.bernoulli(p) |
    \item 二项分布:多次独立实验,产生方法:\verb | binomDist = stats.binom(num, p) |
    \item 泊松分布:$P(X=k)=\frac{\exp{(-\lambda)}{\lambda}^k}{k!}$
    \item 正态分布,产生方法:\verb | dist = stats.norm(mu, sigma) |
\end{itemize*}

假设检验:通常从一个无效的假设开始,根据问题和数据,然而选择适当的统计检验以及期望的显著性水平,然后要么接受检验,要么拒绝无效假设。

回归模型:选择数据可能的统计模型。

贝叶斯统计学:
\begin{itemize*}
    \item 频率学派:给定一个模型,找到所观察到的数据集的可能性。
    \item 贝叶斯学派:将观察到的数据固定下来,观察找到的特定模型参数的可能性。
\end{itemize*}


贝叶斯概率的形式:
$$P(A|B)=\frac{P(B|A) \times P(A)}{P(B)}$$
其中:
\begin{itemize*}
    \item $P(A)$为先验概率,是对A最初的置信度
    \item $P(A|B)$后验概率,是考虑了B之后的置信度,意为在B的情况下,A的概率
    \item 商值$P(B|A)/P(B)$表示B为A提供的支持


\end{itemize*}
