\subsection{《自由选择》}

标签: 自由主义 \ 经济学

作者:【美国】米尔顿·弗里德曼,罗丝·弗里德曼

如作者在“初版序言”中所言,本书考察了美国社会的现实问题,并以一种\emph{自由主义}或\emph{市场至上}的观点来为市场经济辩驳。自由主义者信奉充分竞争的市场能够合理调配资源,从而为市场参与者谋取最大程度的福利,以此出发,他们反对“过多”的市场干预行为,特别是政府的市场干预行为。

作为社会科学,自由主义者需要证明自发的市场竞争具有比计划经济及市场管制更有效,这当然需要充足的调查数据做支撑。然而,本书是通俗读物(甚至类似于电视节目讲解词),因此只能“形象”地展示这些论证过程。然而,第一章里对中央计划经济有害于人民福祉的论证简直是胡扯:首先,作者认为美国作为自由市场经济国家的发展水平高于发展中国家,而发展中国家多采用计划经济,因此市场经济优于计划经济,这完全是倒因为果,混淆概念;其实,作者比较了东西德的境况,论证采用自由市场的西德更“繁荣”,然而却忽视了美国的马歇尔计划和东德社会主义国家重点发展重工业的事实;更为荒谬的是,作者将二战后的印度与明治维新的日本相比较,认为前者过多的政府干预导致了其经济长期落后,却忽视了印度作为英国殖民地被长期掠夺和压制的事实。

自由主义在英国和美国渊源极深,很多知识分子坚持着极右的政治观点。但是,不谈别的国家,只谈论美国,极右的观点也是不足为信的。作者极为鼓吹市场的价值以及其自发调整作用,质疑政府、工会、公立教育、社保的正面作用,希望证明仅靠“小政府,低福利”就能实现社会的良性发展。对比美国几十年来社会保障的增强,我们会发现即使是美国人也不是对这套理论来者不拒的。自由主义的现实性原点,在于“生而平等”以及市场交易上的“选择自由”,但这两者在实际上都是不存在的。只有在一定的范围内,人才是“生而平等”的,比如20世纪60年代以前的黑人,无论如何也没有办法和白人平起平坐,因此也不存在平等的可能性,如果任由私立教育、低福利、无工会的发展,恐怕只会让他们更加处于劣势,并在市场中失去竞争力,被社会抛弃,这样一个贫富分化和种族主义问题严重的社会,必然会撕裂,甚至消亡。对于工会,虽然存在坐地起价的行为,但本质上仍属于市场行为,难道只允许资本家无限度地压低工资,而不允许工会提高工资吗?工人面对资本家,无论如何是缺少议价权的,如果不联合起来和资本家讨价还价,那还不会被剥削死?由于每个人、每个团体的起点不同,在市场交易时,他们的弹性也不是同的,因此事实上某些人天生就比另外一些人有更多的优势,有更多的选择,他们可以通过市场这种“公平”的方式来压榨那些选择更少的人。这种现象,在任何一个社会都不稀罕,在美国更普遍了。不知道作者是不是屁股坐到了资本家一方,假装看不见,还是故意忽悠读者。

总之,在本书里,作者的论证完全是意识形态化的,缺少基本的科学辨析精神,根本不值得作为一部经典来流传,这也和作者诺贝尔经济学奖得主的身份不符合。虽然本书写于40年前,很多经济现象比较古老,但其论述过于粗疏,放到任何一个社会里都不会是杰作。弗里德曼作为经济学家,想必是有专业的论文及著述的,但本书如此拙劣,纯粹是骗钱之作。本书在中国也很有名,曾多次出版,网上评价似乎比较高,但“盛名之下其实难副”,本书的写作质量太次了。

就市场经济和自由主义的价值而言,在四十年后再来看本书的内容,依然有其价值。市场经济的优点,对于进行改革开放的中国而言应该不需要多言了。中国借助市场经济完成了“富起来”和基本的小康生活这样的社会目标,这对于计划经济而言是不可能实现的。然而,作为后发国家,经济制度和政治制度不健全的情况下,政府是否像作者呼吁的那样“减少管制”呢?是否像作者呼吁的那样应当取消关税呢?答案恐怕是否定的。市场经济的建立并非自发的,无序的竞争秩序并非市场经济,中世纪文化需要政府确定基本的现代经济和社会制度才能逐渐被取代,落后的产业也需要选择地保护以防止国际上优势企业的倾销和不正当竞争。即使是美国,在面对中国的竞争下,也进行了严格的限制,如华为和中兴在美国的禁售令。可见,在全球化已经几十年的今天,打着自由主义旗号的美国也开始关起门来搞封闭了,何况是处于不平等国际秩序下的后发国家呢?

本书作者是站在美国的角度上研究美国及部分其他国家的问题的,很多经验对于美国适用,但对于不同文化和制度的国家而言则不适用,这一点可能需要读者去辨别。另外,如我在其他的书评里所说的,西方普通民众和知识分子如非该领域,对于东方国家的历史和文化缺少了解,往往基于错误或部分的事实得到错误的结论,这一点在本书里也大量体现,如前面的印度与日本的比较就是一例。

书的内容可以探讨,但出版社出于“意识形态”方面的考虑私自删除本书的一些内容就很过分了,参考译者的评论:《自由选择》译者的话 \url{https://book.douban.com/review/1592980/}。

评分:1/10。