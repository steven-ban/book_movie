\subsection{《今日简史·人类命运大议题》}

作者:【以色列】尤瓦尔·赫拉利(Yuval Noah Harari)

英文名:21 Lessons for the 21st Century

本书是以色列著名的青年作家尤瓦尔·赫拉利的最新作,是他饱受好评的《人类简史》《未来简史》的姊妹篇。书中的内容和观点,很多来自于作者在2017-2018年为New Yorker等写的专栏和自己的博客,与时俱进,以2018年下半年看也丝毫不过时。

本书按章节分别对全球目前的下述议题发表了看法:
\begin{itemize*}
	\item 理想的幻灭。实际上讨论了目前全球各个主要经济体面临的主要思想问题。
	\item 就业
	\item 自由。由大数据、人工智能、互联网信息垄断者导致的自由丧失。
	\item 平等。谁该拥有数据?
	\item 社群
	\item 文明
	\item 民族主义
	\item 宗教
	\item 文化认同
	\item 恐怖主义
	\item 战争
	\item 谦逊
	\item 神
	\item 世俗主义
	\item 无知
	\item 正义
	\item 后真相时代
	\item 未来不是科幻小说。从科幻小说中反思人类命运
	\item 教育
	\item 意义
	\item 重新认识自己
\end{itemize*}

对以上任何一个议题,作者都是在全球层面上进行讨论的,跨越了国家、文化、民族和宗教。作者有着较好的全球视野,既没有中国普遍的民族主义和共产主义观点,也没有欧美那种自由主义的倾向(相反,作者对这两点均进行了批评和反思),更没有宗教化的偏见和臆断。作者依靠自己的观察(特别是最近几年的科技和互联网的冲击)对上述问题进行了整理与解答,这些内容基本上都是最新的,在十年前的书上,肯定看不到类似的观点。

关于自由主义。国内的知识分子,如果摆脱了单纯的共产主义意识形态和民族主义,很容易落入自由主义的窠臼。共产主义曾经做的恶,在不少国家都成了历史,因此无论东西方都对其开展了批判。而民族主义由于容易被操纵,也容易使人变得狭隘,也很容易被知识分子看清面目并进行批评反思。自由主义现在成了世界上最主流的意识形态,它被好莱坞电影一次次以娱乐的形式散布在世界每个角落,同时由于西方长久以来的文化强势,也为很多人所接受。自由主义的核心,在于每个人都可以依靠自主意志来决定自己的生活,可以进行创造。然而,作者指出,人并非都有”自由意志“,这或许是人在商品经济或其他潮流下的被动模仿而已。人的主观的感受,并非都是真实的,也并非都是可靠的。作者对自由主义所做出的反思,在我读过的西方作家(包括自由主义经济学家如弗里德曼,其经济思想也都源于自由主义的上述假设)里很少见到,而作者进行了针锋相对的解析。自由主义并非稳固,现在随着人工智能和大数据的发展,人们的心智越来越多地被数据所捕捉和利用,自由意志实际上被大公司甚至政府所控制,更不要提在此基础上做出的决定是否真的反映自己的想法了。2018年的这个时代点上,自由主义正在经受着考验,可以想象接下来的几十年内世界主流意识形态会相对于工业革命以下发生较大的变化。

关于民主。如上所述,如果自由主义的”故事“崩塌,那么选民所做出的选择无法代表自己真实的想法和维护自己真实的个人利益,基于此的普选还有什么正当性和普法性呢?2018年,民主制度也经受着巨大的考验:美国人已经选出了一个不靠谱的总统,他反对经济全球化和自由贸易,向世界慢慢关上美国的大门,走向19世纪的孤立主义;韩国总统陷入腐败和邪教的漩涡匆匆下台;以”民主自由“引领华人社区的台湾在民选的领导人带领下经济越来越坏,渐渐丧失合法性……这一切均表明,传统的民主制度在经受着巨大的考验。相对地,以前总是给人以威权主义甚至专制的中国经济相对稳定,也更加开放和包容,这或许会给世界上不同文化经济状况下的其他国家带来新的启示和经验。

关于失业。人工智能的发展,使得越来越多简单和重复性的工作被机器人取代,甚至以前那些认为只有人才能做的事情(如作曲)也被证明机器人可以做得不错,因此人们渐渐担心不久的将来自己会被机器人取代而失业,变得对社会无用。很多人开始号召要进行持续的学习,养成开放的创造性的思维。但这样何其难也?谁也不知道下一个失业的是程序员,还是销售代表?

关于宗教。作者在书里对自己国家的宗教”犹太教“进行了嘲讽,同时也涉及基督教等。这些宗教都产生在人类文明的前中期,是农业甚至部落时代的特产,只适合于一时一地,但都在历史上造成了大量的破坏。在21世纪,很多宗教依然维护着两千年前的先知的教导,背离了这个时代开放包容的宗旨,反而越来越孤立。宗教本来很多教人向善,但却同时惩罚异端甚至异教徒,两相矛盾,而深陷宗教叙事的人们却认识不到。显然,作者倡导一整片世俗主义的生活方式,包容不同信仰和习俗的人们,使全球因为宗教问题导致的冲突减少到最小。相对于这些人,中国人可能经历过佛道扰乱社会秩序的惨痛,自宋以后就渐渐走向世俗主义,这正好符合作者的倡导,这符合社会的发展。

关于恐怖主义。恐怖主义来源于极端者的绝望。他们无法以战争和大规模冲突的方式让相对者满足自己的政治诉求,进而以暴力的方式否定政府的合法性。作者认为,现代政府的合法性来源于政府垄断暴力后的和平的承诺,当这个承诺被恐怖主义者打破,政府的合法性也就会减弱。于是,各国政府对恐怖主义进行了过激的反应,大力围剿,甚至像美国那样不惜发动战争(至少以这个名义来进行),然而这反摧毁了当地的社会秩序,导致更多的宗教和经济问题,扩大了滋生恐怖主义的土壤。恐怖主义造成的绝对伤害很小(不及车祸和慢性疾病),更多地是在民众心里埋下不信任的种子。各国政府应当避免过度反应,才能稳定人心。

本书的内容十分庞杂,但议论都很有见地,没有老调重弹,给人以耳目一新之感。同时也没有”西方中心论“的倾向,有利于开拓思路,因此读者不可不读。本书的翻译也相当流畅自然,译者十分用心。

评分:10/10。