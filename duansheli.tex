\subsection{《断舍离》}

作者:【日】山下英子

读书是为了获取新知,与作者交流。读书能够获取的东西,以知识为上,情感为中,消遣为下。很不幸,这本书属于下品。

所谓“断舍离”,只不过是作者为了方便收纳、清静生活而自己造的词。按照作者的说法,”物品要用才有价值——是为断。要意识到物品的数量和质量,要斩断用不完的物品、多余物品的流通。物品在此时、当下,应出现在需要它的地方——是为舍。即使是以前用过的东西,如果现在已经不需要了,就不该怀着‘说不定将来还有一天能用得上’的心情,随随便便地保存、保管、收纳它,而是让它们去到此时此刻最需要它们的地方,要有意识地不断把物品送出去,‘舍弃’掉。物品处于恰当的位置,才能展现美感——是为离。不断地重复拷问物品,也拷问自己的‘断’与‘舍’,挑选出与当下的自己最相称的物品。经过精挑细选筛选出的少量物品,能够各得其所,物品仿佛是回归了属于自己的的空间一般的离开。“

这几句话基本上就可以把断舍离说清楚,顶多展开为一篇博客文章,然而作者竟然扩充成了一本书,絮絮叨叨讲了又讲。更有甚者,这本书仅仅是作者拿来走穴演讲培训捞钱的一个副产品,作者自己在日本各地开展讲座,宣传自己的主张,书中不少案例就是讲座时的学生的状况。然而,这些案例仅仅是拿来论证作者自己某些零碎观点的工具,案案本身很是单薄,缺少丰富的细节,以致于并不具有普遍性。因此,这本书是一个纯粹的商业产品,本身并不包含值得人们学习的知识。

不过,就断舍离的理念来说,还是值得商业社会的人去思考的。现代人往往纠结于商品的价格低、自己惜物不想扔掉旧物等思想,在家里屯了不少平时用不着、将来了不会用的东西。对于这些“食之无味,弃之可惜”的东西,确实应该按照作者的说法,统统扔掉。特别是日本和中国城市这样的小户型来说,家里的收纳空间本来就不足,因此稍稍有了一些废旧杂物,就会显得十分拥挤,而且会吞噬掉需要物件的空间,用户也会因为家里拥挤杂乱而不再购买真正有用的东西,影响平时生活的心情。当然,如果家里空间实在大,可以把杂物放在专门的杂物间里,或者直接买个仓库来储存,也就不存在这样的问题了。

那么如何解决呢?无非就是开源和节流。后者比较直接,就是好好规划家里的收纳空间,把各种东西分类储存摆放,按照使用的频率、类型、体积等等分门别类。这当然也包括在装修时设置更多更科学的收纳空间,特别是考虑到几年之后的情况。日本流行的各式“收纳术”即是指此而言。而作者强调的断舍离则是往开源方向上下功夫,减少家里杂物的数量,因此也降低了合理收纳的要求。对于物品,作者持有的是一种类似于极简式的、禅式的“身外之物”的思想,对于大部分无用、不用之物,即使它看似有用,也都扔掉。物品不仅仅是物品,还是人对于世界的一种“羁绊”,舍弃掉无用的羁绊,才能集中精力在真正有用的人-物关系上。这就好比水杯,装满水就不能再装新的水了,仅在它不装水的时候,才是对接下来的装水而言最“有用”的。这种思想其实与现在流行的极简风格相近,属于同一种哲学。我觉得这本书最大的用处,就是讲清楚了这一点。

本书的缺点,除了上面说的语言太啰嗦之外,还有就是论证缺少逻辑,举例太少,真正“术”的内容太少,很多人可能看完除了学习作者的生活哲学之外,扔东西的本领上去了,对于如何收纳整理还是一头雾水,这恐怕也是很多冲着收纳术的人来说有点失望。因此,本书之外,还是需要一些收纳案例的学习,才能搞定家里的收纳。

评分:3/5。