\subsection{《贸易与理性》}

标签: 国际关系 \ 中美关系 \ 民族主义 \ 改革开放

作者:郑永年

本书的内容主要是作者对2019年以来中美关系的一些观察和思考,涉及贸易战、新冠肺炎疫情、中美各国国内对对方国家的思潮等等。作者是著名的国际关系学者,这些思考也都是比较严肃的,与充斥各国国内的一些发泄情绪的言论有本质上的差别。

本书的一些我认为有价值的论断:
\begin{itemize*}
    \item 只要中国坚持开放,就不会被孤立。
    \item 洪特:判分古今政治的界线是否将经济、商业视为核心政治事务。
    \item 中国自古以来一直有三个市场共存:顶层永远是国家资本;底层是自由民间资本,中间层是国家资本与民间资本互动合作的部分,有合作也有竞争。通过三层资本结构,政府维持这种平衡,履行经济管理的责任。
    \item 中美之间三个主要的互动领域:跟特朗普团队主管商贸的人谈贸易摩擦,这是贸易领域;美国国会两党主导的、要跟中国共产党发起“技术冷战”的领域,这是国会主导的;传统安全领域,军工系统主导,也就是美国鹰派,想把中美关系引向传统美苏式的冷战。
    \item 中国新时期的外部风险就是“修昔底德陷阱”。
    \item 民族主义往往高估自己的实力,而低估其他国家的力量。
    \item 美国是靠“危机感”驱动的社会,一定要给自己找一个敌人;中国是靠“危机”驱动的社会,没有危机就没有前进的动力(后半句我持人否定观点)。
    \item 熊彼特认为,企业家是不断在经济结构内部进行“革命突变”,对旧的生产方式进行“毁灭性创新”,实现经济要素创新组合的人。
\end{itemize*}

然而我在本书中,并没有发现很能体现学术性的思考,相反,本书充斥着作者的仓促的论断,没有实例去佐证,有些论断甚至没有展开说明。更糟的是,可能是由于本书简单地把他这两年的一些博客文章或者讲话等文字拼凑在一起,导致很多内容反复在各章出现,缺少系统性,这也让这本书的可读性大打折扣。

作者对中美关系(乃至中国与西方的关系)的思考,相对而言是比较全面的。作者认为中国与美国的关系进入到一个新的阶段,在这个阶段里,美国对中国怀抱着越来越大的敌意,视中国为战略上的对手(毕竟官方都在正式场合讲了很多次了),不断以贸易战、中国南海、台湾问题等对中国展开打压行动,并且这也是美国两党上下乃至整个精英集团的共识;就中国而言,也是在面对美国的咄咄逼人的态势,开始了外交上的调整,一方面温和理性,更一方面则在民间甚至甘休某些官方场合展示出强硬态度。

作者认为,中国两个世界上最大的经济体,不能脱钩,否则进入“冷战”状态,中国会被美国乃至整个西方国家孤立,重新进入苏联的状态,从而丧失发展机遇,而这正是美国内部军工集团所希望看到的。与之相反,中国与美国应当寻求共识,增强贸易上的关联性,中国自身要进一步向西方国家开放市场(例如加快构建海南自贸区、大湾区建设等),防止被孤立。这一点上,作者的想法和官方是一致的,中国也是这么做的,我们可以看到在贸易战的背景下,中国坚持了更高水平对外开放的y主政策,开放力度不断加大,2020年与美国及其他西方国家的贸易量反而有了很大增长。这一点上,我是认同作者的判断的。

对于国内“民族主义”的声音,作者虽然认为这不全是坏事,但是仍然持一个担忧、反对的态度,认为官方某些人的民族主义意味太深厚(可能是赵立坚等外交部发言人在外网上与西方媒体展开论战)。作者害怕这种民族主义的声音过大,导致中国人骄傲自满,认不清自己的地位,从而产生民粹思想,并且给官方压力(幸好中国不是西方的那种民选政府,即使社会上有一些思潮也不会对官方的战略产生直接的影响),强行与美国脱钩。这一点上,我认为作者是有点杞人忧天了。中国的民族主义,主要是近代以来的屈辱、反抗以及国家建设建立起来的,近年来的民族主义高涨,主要是中国的发展确实很不错,很多人也享受到了发展红利,同时西方国家在2008年金融危机以后应对失策,失误连连,暴露出他们制度和社会上的很大问题,同时官方也有意进行了宣传引导(例如“开学第一堂课”、正式场合升国旗唱国歌等),这实质上是社会整体对2010年以前自卑、仰望西方的一种纠偏,也没有什么过火的地方。\emph{一个社会或者国家在自卑的时候反而更容易进入一种非理性,而在自信的时候,会更加理性,也会更认真正面地对待自己的缺点。}吊诡的是,作者在之后的某篇讲到,西方对中国的攻击是没有道理的,这就掉进一种谬论里:既然错在对方,那么要不要反击呢?显然一味挨打挨骂是不行的,而现在这种反击更多是民间和互联网上的声音,官方还是保持着极大的冷静与克制的,因此这种“民族主义”是没有必要担心的。这也可以看出本书作为一种类似讲话集合的出版物,本身的严谨性是很差的,作者似乎在不同场合也是根据不同的受众来输出观点,这大大降低了本书的价值。

作者反复强调,中国不应该“过度”展开自信,不应该与西方秩序发生冲突,甚至不能让西方“联想”到中国对现有秩序的挑战。然而,这种想法显然已经与现在的情况不符了。无论中国做什么,只要中国在发展,西方就会找中国的麻烦,中国的崛起在西方国家看来,本身就是对现有规则的挑战,因此会下死手打压中国。因此,中国不应当顾忌太多,而是习惯与西方斗而不破,以国家的长远利益为追求,该和气就和气,该斗争就要斗争。以历史经验来看,西方是欺软怕硬的,你弱了他就欺负你,你强了他就抱你大腿。这事实上是中国的发展有没有自己独立的规划的问题。

另外,对于国内经济问题,作者认为应当限制国有企业,鼓励民营企业,特别是在贷款上面要加大对民营企业的扶持力度,这当然是正确的,并且近些年中央也是这么做的。然而,国企的地位是应该加强还是应该削弱?与作者观点相反,我认为是应该做大做强,当然其手段与“纯粹”的市场经济手段不同,应该是行政和经济手段双管其下。中国正在爬坡过坎的关键时期,这种情况下最容易引发系统性风险,因此要牢牢把住自己的经济命脉,防止内外某些因素联合导致“崩盘”或类似风险,这其中国有企业就是经济的基本盘。并且从现在的实践上来看,国企的垄断是有充分的监督的,而民企的垄断则更加无所忌惮,例如百度、阿里和腾讯作为民企垄断互联网企业的代表,其种种行为都指向了利润而非社会责任,这已经引起了很多网民的不满。中国还有很多民营企业,其出身就不正当,要么是制假贩假,要么是国有资产流失,问题很多,对它们的监管丝毫不比国企更严峻。这里作者的呼吁显然没有认清现实。更有甚者,作者认为企业和政府间的“政商关系”不正常,那么应该怎样正常呢?作者认为现在的企业“讨厌”这种关系,这显然有失偏颇,因为那些没有“背景”的企业会是讨厌它,但有背景的企业肯定想保持现状,这应当分开来分析才对。作者讲美国能打败苏联,原因在于苏联是计划经济,只有国有企业,没有私营企业,美国则相反,这完全是归错了因,很难想象这话是一个国际关系学家说的话。苏联的失败是多方面的,计划经济只是原因之一,美国的“成功”也有一定的偶然性,不能得出计划经济就一定是错的,市场经济一定是正确的结论。

作者还认为中国多商人而少企业家,缺少企业家精神。不过,我认为,中国不缺少企业家和企业家精神,特别是近些年来互联网领域和很多民营领域都出现了不少不错的企业家。只要有市场经济,就不会缺少企业家。

作者反复强调中国目前的创新太少,而且集中在应用层面的创新,没有原创性的创新。确实如此,但以此否定中国已经取得的创新贡献也是不可取的。近些年,基础科学的进步非常缓慢,甚至有停滞的迹象,即使是美国也很少拿出水平很高的原创性贡献了。至于科学上的创新,这本身就是全人类共享的,应该看到中国的高校和研究所在近些年有着非常多的原创性学术成果,丝毫不比哪个西方国家差。至于产业应用上的创新,中国也做得比较快,能快速将技术应用到市场里。中国应用层面的创新本身就是社会和产业进步的标志,其贡献是很大的,虽然整体水平比不上美国,但已经比很多发达国家强了。因此,需要担心的并不是目前中国创新水平低,而是如何保持在产业上的敏锐嗅觉,并投入足够的资源来源源不断地产生创新,提升产业水平。在这一点上,我其实是很看好中国未来的发展的。

此外本书的错误论断还有很多,比如“除了亲西方的自由派以外很少有人理性讨论中美关系”等,这些评论显然是不负责任的。

整体上来说,这本书写得是很烂的,很多话题都是浅而又浅,有些论断也是错误的。本书展现出来的论证水平,连张维为都不如,张维为写《中国崛起》这种书,还是有实例有论证,而且论证也比较严谨可靠,而本书则是直接给一些结论和建议,根本没有展开。所以,本书强烈不推荐去阅读。

评分:0/5。
