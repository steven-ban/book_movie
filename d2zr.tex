\subsection{《第二自然》}

作者:(美)Gerald M. Edelman

\subsubsection{一些标注}

人的大脑重约1.5千克,它是宇宙中目前所知的最为复杂的事物,它的连通性让人惊叹:大脑褶皱的皮质上大约有300亿个神经元细胞和1000万亿条神经突触连接,这样的结构中可能的活动通道的数量远远超过宇宙中基本粒子的数量。

证据表明,意识来自皮质区域和丘脑间的折返活动以及皮质与自身和皮质下结构的交互。理论认为,随着丘脑皮质系统进化得越来越完善,特定丘脑核的数量也随之增多,大脑皮质也不断增大,使得原始意识出现。这一系列进化事件可能始自爬行动物向鸟类进化并在大约2.5亿年前分化出哺乳动物的时候。

高级意识必须等到真正的语言出现才能全面发展。这个时候,意识的意识才成为可能。指向可以构成词汇,词汇的标识可以通过语法连接起来。关于过去、未来和社会自我的丰富概念涌现出来,意识不再局限于当下记忆,意识的意识成为可能。

作为选择性系统,大脑的运作显然不是基于逻辑,而是基于模式识别。大脑的动作是将其非线性的变化能力与外部世界和自身信号提供的偶然、新奇和非线性的事件进行选择性匹配。

虽然我们必须承认进化和神经元群选择机制提供了获取知识的基础和约束,却是历史、社会文化和语言因素设定了真理的规范准则。关键是这些准则可以通过这些方法以自然化的方式建立。

感知系统以及对意识的考量如何会发展并成为信念的基础?首先,基于脑的认识论以物理学和进化生物学作为其主张的基本平台,因此它排斥理想主义、二元论、泛精神论和任何不是根据大脑结构提出的心智观点。……概念是从大脑本身的感知映射匹配发展出来的,从而概念最终是关于世界的。……思维能在没有语言的情况下产生,最初依赖于隐喻方式(image schemata,意象图式),这些隐喻活动受到大脑中冗余回路连接能力的有力支持。……根据基于脑的认识论,逻辑和一定程度上数学的成就都依赖于高级意识,而高级意识本身的充分表现则有赖于真正的语言的获得。……受限于进化的身体特征以及选择性的大脑,显然只允许对世界事件进行有限的采样,而事件的数量是无穷的。……在正常的大脑的运作中,不存在绝对正确或绝对错误的精神状态,我们甚至还会有现象状态的错误——有内容却没有对象的幻觉。……大脑活动具有寻找完整性的倾向,会进行填充完形,有必要时甚至进行虚构。不仅如此,我们还具有某种必要的幻觉。……所有的这些特征的背后是大脑折返丘脑皮质系统或动态核心的活动,其复杂的整合神经模式导致了意识。加上潜意识系统的活动,从而产生出学习、记忆和行为。……从高级意识的神经基质中涌现出来的是艺术创造、伦理系统和将我们置于万物之中的科学世界观。

\subsubsection{书评}

这本书内容并不多,是作者作为一个脑科学学对“意识”的一些论点,关键在于作者提出的所谓“神经达尔文主义”。作者在各章中对意识的形成原因、历史进行了辨析,内容比较抽象。不过,对于唯物主义者而言,意识是大脑的功能这一观点现在已经被大多数受过教育的人所接受,关键在于研究具体哪些生理活动造成了意识,以及在进化中这些功能是如何一步步出现并增强的,这中间有一个客观和主观的鸿沟。不过,除了这些内容,本书并没有过于深入地讨论,比如没有介绍哪些具体的研究得出了哪些重要结论,而是仅仅就几种观点做出的展示。说实话,具体的内容我没有看懂。

让人无法忍受的是本书的翻译特别差劲,特别是前几章,除了直译外,连语序都没有改,完全就是谷歌翻译的水平。这本书属于《第一推动》丛书,是湖南科学出版社的著名的出版项目,之前出的《宇宙的琴弦》等现代物理学的科普著作翻译流畅,内容充实,是打开我对自然科学兴趣的钥匙。相比之下,本书连个像样的翻译都不请,翻译本身就很敷衍,出版方包括编辑也绿灯通过,真是砸自己的招牌。

评分:1/5。