\subsection{《论美国的民主》}

这本书在最近几年很火,常常被人提起,另一本类似的书是托克维尔的《旧制度与大革命》,温总理还说他经常读,我准备以后抽时间也读一读。

本书是托克维尔在19世纪30年代根据对美国的考察写的书,阐明了自己对美国社会政治和人文制度的思考,并与法国和英国的旧式制度进行了比较。托克维尔从美国形成时期的宗教、阶级等方面出发,论述了美国政治制度形成的原因、影响因素和基本特点,展现了他眼中的美国社会。

全书是事实与观点的堆砌,基本上就是为了阐述观点而写的,缺少比较严密的分析和推理。这种写法类似于勒庞的《乌合之众》,读完之后我就对这种写法产生了质疑:只能够满足阐述自己的观点,但反对者很容易进行攻击,因此这些观点也难以真正让人信服。本书其实也有这样的问题,比如关于平等的推行与专制的讨论中,虽然作者的眼光极其敏锐,但有些推论还是和实际情况差别很大。因此阅读本书只当成是与作者进行交流的方式就可以了,如果想深究,还得参照更严谨的著作。

托克维尔生活的英法及其他欧洲国家,大多数还是君主制,或刚刚建立起共和制不久,社会还是区分着严明的阶级的,旧的思维和统治方式还在发挥着作用(不仅是在具体事务上,也是在人的思想和文化整体上)。在前言中托克维尔也指出,\emph{本书中的“民主”,其实多数时候指的是“平等”},本书更应当称为“论美国的平等”。

美国真是一个幸运的国家。一群清教徒,带着刻苦和冒险精神来到这片大陆,没有阶级,没有国王,大多数依靠了在母国里宗教环境下的自治传统,慢慢发展成一个一个社区和国家。美国土地辽阔,人均可耕种面积多,在可预见的时间内(事实上在历史上也是这样)没有进入马尔萨斯陷阱中;同时其他自然资源丰富,气候适宜,更何况没有外敌,因此社会可以维持一种松散的组织形式。人与人之间如果对抗,还不如去开拓新的天地更好,因此社会矛盾也比较少。这样的优厚条件,去实施社区自治并建立参与政治的热情再合适不过了。美国之后的政治制度,也是因为有了这样的好条件才顺风顺水的。

相反,欧洲国家则不幸多了。欧洲国家多,面积小,人地矛盾突出,同时国家之间距离近,容易发生冲突和战争,因此不得不保持一种备战状态,只有以君主制为代表的中央集权才能协调大量的资源,因此无法产生美国那样的社区自治。启蒙运动后,美国相当于同时接受了这种思潮,产生了《独立宣言》,而在法国则产生了惨烈的大革命,在保守的英国则没有兴起什么浪潮(英国在忙着搞工业革命)。因此,因为美国社区自治和民主制度搞得好而鼓励其他国家照搬它的制度也完全不可行的。同文同种的英国都不行,更何况文化和政治完全不同的亚洲国家了。

\subsubsection{民主与庸俗政治}

在书里,托克维尔对美国的社会形态进行了预测。美国建立在一个新大陆上,缺少欧洲大陆那种沉重的历史负担,借工业革命之势而起,天然具有平等的精神。美国的民众更关心自己社区内的生活,而对中央政府以及其他社区(如其他州)的生活缺少兴趣,因此这将可能造成一种庸俗的政治形态。其实从后来的历史来看,美国人的地理很差劲,对其他州简直是漠不关心。当然,这也算是美国的一种幸运吧,在其他历史悠久的国家简直是不可能的。

\subsubsection{民主与阶级}
本书写作时,共产主义运动还没有大范围兴起,反乌托邦主义更是遥不可及。但是,托克维尔在讲到工业发展对人类社会造成的影响时,还是展现出敏锐的观察力。作者说,美国工商业的发展和对工作效率的追求,使产业工人越来越专精于一种技艺,因此越来越疏于其他安身立命的方法,长此以往会限制自己的眼光和能力;而资本家由于工作的关系,会越来越具备综合能力。因此,两个阶级的差别会越来越大。当然,由于美国比较平等,因此资本家本身并不会局限于固定的人,而是有一定的流动性。这和美国之后的发展相符合:美国确实产生了专精于一项技艺的庞大的“中产阶级”和流动性极强的富人阶层,两个阶层之间的流动性还是挺大的。

\subsubsection{美国与宗教}
美国源自移民的新教徒,与天主教不同,宗教在政治动作中几乎没有任何影响力。但是,宗教在人民的生活中扮演了不可或缺的角色。基督教伦理下的对个人幸福的追求催生了美国的资本主义工商业的兴起。同时,美国人对于追求财富和享乐丝毫不避讳,所有人都“平等”地追求着个人世俗意义上的“成功”。

另外,宗教在社区的组织中扮演着重要角色,对于匡扶人心也有重要作用。想像不到一个如此追求个人世俗价值的国家,如果没有宗教的制约,会产生多少丧尽天良的事情。

\subsubsection{民主和集权}
在书里,托克维尔对民主与集权的关系进行了论述。由于本书里,民主更多的指平等,因此他其实讨论的是近代以来大众化政治与中央集权的关系。由于欧洲与美国的政治生态不同,因此民主的推进与中央集权的关系也不同。

在欧洲,民主(或平等观念)是法国大革命以后才大范围深入人心的。中世纪和君主制下的政治,呈现出贵族政治的特点,平民缺少与贵族相平等的权利,资产阶级的兴起有利于平等观念的扩散。当较少受到教育的底层人民决定争取平等权利时,就必然从君主和贵族手里夺取。法国大革命以及伴随拿破仑军队踏遍欧洲的过程,就是一个暴力和兵荒马乱的极端的过程。普通民众在政治上缺少联合,因此需要更多地依靠君主(如拿破仑)和中央政府才能保持权利,中央政府借此获取了大量的权力(相对于分散的中世纪庄园经济)。平等越推进,中央政府的权力也就越大,最终可能走向极权。欧洲各国间战争的可能性太大,集权的政府更容易在这种强敌环伺的环境下存活,同时民众也更欢迎这样的本国政府来保护自己。托克维尔在书里的这种预言,简直不要太有前瞻性。

在美国,情形则完全不同。美国先天就具有平等的基因,个人主义盛行,其平等思想的推行不需要借助中央的权力。相反,美国人民对政府有种天然的不信任,并且建立了精巧的三权分立体制来防止政府权力过大和过于集中。美国联邦政府的权力较小,对普通人影响最大的还是他所在的社区和州。因此美国的平等可以与类似小政府并行不悖。因此,不得不再次感慨美国的幸运。

托克维尔担心在一个平等的社会里,个人的能力弱小,同时缺少贵族社会那种由贵族、庄园主、教廷那样的“中间阶层”来调节个人和国家之间的矛盾,原子化的社会形态使专权者更容易侵害个人,而个人无处求助。人与人之间也缺乏同情和互助风气。所谓”中间阶层“,可以理解为独立于中央政府的其他自治性组织,如NGO、在野的政党、宗教团体、独立的科研机构和大学等等,它们起到了在个人与政府间协调的作用,遏制了政府对权力的垄断,防止出现一家独大的情况。前些年中国也在讨论”有国家无社会“的问题,希望多一些这样的”小共同体“来对政府权力进行制约,提高社会治理水平。但最近几年,似乎这样的呼声低了很多。不过,政府一家独大的情况在美国没有出现,却出现在二十世纪的共产主义国家和反乌托邦作品里。

托克维尔十分担忧在一个平等的国家里中央政府权力过大,但其实对于俄国、日本、韩国或中国这样的落后国家里,在向现代化和工业化转型的过程中,必然需要一个强大的中央政府。它必须首先在理念和组织上是现代化的,它必须有实现工业化和现代化的决心,同时与旧制度和既得利益者做严酷的彻底的斗争,摧毁农业社会里政治、经济和文化上的一切阻碍,才可能初步建立一个现代化国家。在这个过程里,这个组织会越来越庞大,也必然在建立政权的过程里采用历史上最集中、最严格的权力形式。在俄国和中国,是列宁式的共产党;在日本,是倒幕运动中的下层军官和随之而来的国家主义;在韩国,是采用独裁手段的李承晚。至于之后的慢慢开放,则需要这些政党的自觉和人民的斗争了。这种现代化路径的不同,恐怕是托克维尔写此书时没有想到的。
