\subsection{《塑料王国》}

标签: 记录片 \  环境污染  \  禁片

来自美国加利福尼亚、德国、澳大利亚、日本、韩国等发达国家的塑料在中国港口入关,随后被运到山东某农村被人工分拣被打碎做为塑料的原料,使用当地的地下水进行清洗后无所顾忌地排到河里,不能分开的东西则被烧掉或废弃在农田边上……当地的水源已对被严重污染,村民不得不买外地的水;黑烟滚滚充满刺鼻的气味,很多人都认为自己活不久了,但不得不靠这种方式挣钱。钱大部分被开厂的人挣走了,他们即使是夫妻双方同时挣钱,一个月也只能赚两千多。

更可怕的是,这些人手工分拣,根本没有有效的隔离措施,经常手工接触不名物质,被针扎入皮肤。废弃医疗用品被随意接触并丢弃,甚至被小孩子拿来做玩具,用嘴去当气球吹。孩子们在河里摸鱼,在垃圾堆里玩耍,婴儿被母亲喂奶,脸上爬满苍蝇。很多适龄孩子已对辍学,在父母眼中表现得很聪明能干。

村民们并不是不知道这样污染了当地环境,不是不知道这对健康的危险,但他们口中,钱更重要,当下的生存更重要。他们不是没有去反映问题,但已对对政府绝望。全片充满着作者的讽刺:垃圾入关时“永远跟党走”,片尾处“我们的祖国是花园”,配以垃圾遍地的镜头和孩童,让人心痛到极点。

本片反映的这种现实,不知道现在治理了没有。如果没有而仅仅是记录片被封禁,那就太搞笑了。
