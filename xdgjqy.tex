\subsection{《中国现代国家的起源》}

标签: 汉学 \  近代史 \  政治史 \  思想史

作者:【美】孔飞力

翻译:陈兼,陈之宏

\subsubsection{中国前现代化的历史挑战}
清人入关以后,维持了和明朝相同的治理方式,但世界大势在悄悄改变社会。从美洲来的玉米、红薯、烟草开始大范围种植,极高的产量促进了人口的大爆发。在18世纪,中国人员从1.3亿增长到4亿,人地矛盾突现,失地农民开始频繁起义,而人均耕种面积并没有增加,大部分农民的生活水平依然很低甚至有了下降。激增的人口为社会治理带来了挑战:耕地的增加导致水土流失加剧,长江黄河连年发水,政府由于财力有限经验有限无力有效治灾,这导致了更多流民;县级以下的人口增加,但基层政府还是那么多官员,导致治理能力跟不上人口增加。到了乾隆中期以后,整个帝国就进入了一种没有生机的状态。整体而言,从国家层面上来看,中央政府难于直接向农民征税,必须通过官僚体系,而后者治理能力的低下导致税源不稳定甚至下降。

同时,科举制度没有得到改革。本来读书人口就少,但官职更少,很多读书人被科举制度刷下来没有机会做官,只能做疍吏或幕僚。清政府由于少数民族统治而天生的自卑和恐惧使得文字狱愈演愈烈,知识分子噤若寒蝉,不敢言政,造成言路闭塞。

\subsubsection{魏源}

魏源没有动制度的大体,他借由《诗经》这样的儒家典籍,引申出应当鼓励知识分子言政,并由中央政府择优采纳,使国家富有生机。魏源是小心谨慎的,他不能得罪皇帝和整个国家的意识形态,这里的“知识分子”是包括自己在内的类似举人这样的“高级知识分子”,而非读了书就可以。统治者采纳言论是为了广开言路,增强社会治理能力,而非为了“言论自由”甚至“议会民主”,他甚至认为只要国家的目的是高尚的正确的,则不必苛求其手段。可见,魏源的思想完全是“体制内”的,是为了维护这个体制,只是通过扩大**尽量少**的政治参与来提高其治理水平。他这种思想并不鲜见,在一个威权或专制的社会里,能出声的也只有这样的言论,看看今天就明白了,甚至魏源的境况还更严苛一些,因此他这是顶着乾隆反对的“朋党”“妄议”的危险来提出的。

\subsubsection{冯桂芬}
冯的思想相较魏源进了一大步。他不仅提供学习外国的科学技术,还从社会构建的方面进一步从西方观点来启示中国问题。在官员的选拔上,他提倡下级官员选举上级官员;在地方治理上,他也主张地方民众选举地方官来代替目前的胥吏。这种选举的思想,完全是受了外国人的影响。

但在戊戌变法中,光绪皇帝拿出冯的著作让大臣发表看法,这些士大夫发出了几乎一致的反对声音。冯的思想与中国正统的儒家思想不相合,因为在儒家看来,只有有德行的人才能去治理社会,而社会上大多数人没有经受圣人的教育,德行上是不够的,那些经过科举考试的读书人才能做到德行的完善。即使是读书人也并非德行完美,普通民众更是愚昧且自私的,因此他们无法做出自下而上的“正确”的选举行为。

孔力飞笔锋一转,开始就“公众利益”进行了美国建国者和中国当时建制者之间的比较。前者认为,人性的固有的善依靠“看不见的手”可以开展符合公众利益的合作,德行完备的人可以经由民众的推举产生;后者对人性抱有悲观的态度,因此他们认为,公众利益必须有一个代表性的团体来完成,这些人只能是士大夫,因此只能建立并强化中央权威才可能实现。

\subsubsection{土地革命}

与魏源和冯桂芬对知识分子政治参与的强调不同,本章提示了中国明清时代社会如何在中国现代化过程中反复出现并最终解决的总是。作者认为,这些问题在整个现代化过程中,并不是消失了,而是反复出现并贯穿于革命和改革的整个过程。魏温柔的和冯桂芬是改良主义者,但他们的政策并没有真正被实施。

明清时代,国家的税源主要在于农民,然而明代中后期土地开始大规模集中到大地主和皇亲国戚手里,而这些人不用纳税,失地农民不得不成为佃家,他们需要承担田租和国家税收,因此极度贫困,不得不将纳税“包揽”到地主手中,甚至卖地以躲避税收,而国家也越来越难以从基层收税。明朝的瓦解,和这个原因不无关系。

清朝统治者对于居于国家和农民之间的中介——地方胥吏和生员——并没有明朝那样的感情,他们拼命压缩这一阶层以从这些中介手里夺取收入。然而,由于人口的激增和地方统治能力的低下,他们不得不依赖这些人来取得越来越少的收入。雍正的“官绅一体当差一体纳粮”和“摊丁入亩”相对于这个广阔的阶层而言并不明显,同时也没有继续很长时间。这些中介占据了地方上的很多事务,在具体的一些事情上甚至敢于和中央政府叫板。基层官员为完成纳税的考核,也需要看他们脸色。随着太平天国运动,地方上的中介机构开始招兵买马并掌握地方权力,从而从根本上瓦解了清政府的统治基础。清末实行宪政和地方自治,这些人也逐渐坐大。民国政府虽然试图加强对基层的控制,但实质上的不统一和抗战压力大大拖延了这个过程。

共产党以一个革命者的姿态出现,它彻底摧毁了地方上的中介阶层,控制了土地并进行相对平均的分配,当看到建国初年贫富分化已经有了苗头并且很可能再次形成富农阶层时,共产党开始了农业集体化,彻底夺取了土地所有权,农民成为国家的佃农,同时也取消了乡村的市场经济,国家第一次对于土地和税源有了绝对的控制。依靠这样的控制能力,共产党积累了原始资本开始搞工业化,但同时也为农民造成了巨大的损失。费孝通在1957年就看出来,仅仅二十年过去,“江村”已经相对于原来变得很衰败,农民的实际生活水平出现了下降。这或许为20年后的改革开放埋下了伏笔。

\subsubsection{现代化之路}
魏源和冯桂芬关于扩大文人参与的建议,在他们生前并没有被施行。只有在中国多次遭受侵略后的甲午战争后,这样的“参与”才以“公车上书”的方式显现。然而,仅仅扩大文人政治参与的想法,刚一冒头便已过时。这里的中国东南各省实质上已经半独立,不听中央号令,而中国政府的威信也在改革失败和革命倒逼下渐渐丧失。清廷于20世纪初年的进行宪政并没有施行多少年就被革命派消灭。在这个过程里,梁启超和章炳麟的各自的观点对中国现代化的基本进程做出了不同的猜想。梁认为,中国应当进行地方自治运动,在这个过程中会自下而上建立起富强的国家;而章认为,中国的现代化应当是自上而下的,只能依靠一个强大的中央政府,才能消除地方上的自私和局限。很显然,他们所建议的事情都需要做,但并非同时做。北洋政府做的是中央层面的事,国民党想先做地方自治但发现不成功后也开始加强中央权力,而共产党是完全相反的路子。

中国作为一个整体,在近代史中时刻有着巨大的生存压力,内部战乱也不停歇,这对强大的中央政府有着巨大的渴求,因此在操作层面上,先集中再放权似乎成了唯一合适的道路。1949年后的集体化和改革开放即是这样的过程。

\subsubsection{书评}

中国的现代化,并非是照搬西方制度和文化的过程,实际上是内生的,它在为解决以往社会问题的过程中不断对自己进行反思,不断借助新的手段来创造解决问题的可能性。本书的主旨,也是以这样一种思想来剖析现代化的过程:中国本身蕴藏了现代化的资源和需求。

评分:10/10。