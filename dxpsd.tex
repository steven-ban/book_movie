\subsection{《邓小平时代》}
\subsubsection{感想}
高晓松说八十年代是一个大师的时代。如果他说的是真的,那么李光耀、撒切尔、老布什肯定算是大师,并且与他们比肩的邓小平也绝对没有任何理由不被称为大师。

作为中共元老,邓小平参与了中共1925年以来所有大事,从江西瑞金到晋冀鲁豫根据地到四川到中央,再到江西的下放和重回权力中心,邓小平见证了中共从一个城市革命党,到分疆裂土的诸侯,到夺取全国政权的执政党的历程,从毛时代的革命至上到江时代的铆足劲儿改革赚钱。邓干过中央,干过地方,带过兵打过仗,从资历上无可挑剔。

邓三起三落,但在瑞金时代因祸得福,站队到了毛泽东,以至于赢得了毛泽东的持久信任。

大起大落虽有多次,但都不是很致命,1975年被毛靠边站,但没有开除出党,何况毛时日无多,下一届的政治格局已悄然形成。邓只需要等待时势。

邓充分利用了毛的信任,即使在文革后期也锐意纠偏改革,在毛的监视和自己的政治诉求间做好了平衡。邓的这种动作为几年后重回权力中心和重启改革积蓄了能量。

邓是做事的人,在改革上是个猛将,是个布局者。

关于八十年代的改革,邓小平无愧于“总设计师”的美誉。虽然邓并非事必躬亲之人,但所有事情他都拍板,为改革派提供大量的权力资源。

邓是坚定的民族主义者和共产主义者,他对“自由民主”的看法实际上到现在还是官方所坚持的信念:可以能社会以活力以自由,但不能危害到共产党的统治。

\subsubsection{一些标注}
邓小平说出了他的离别寄语:一定要让群众和外国人明白,中国领导人将坚持对外开放,这一点十分重要;他的接班人要维护党中央和国务院的权威,如果没有这种权威,中国就无法在困难时刻解决问题。[22-13]

曾庆红的父亲曾山在党内多年从事组织和安全工作,曾庆红通过他知道了很多党内的人事内幕;他母亲邓六金是延安幼稚园的园长,许多现在的领导人都是当年从那里出来的孩子。

只有通过缓慢地建立基础,通过改善更多人的经济生活,通过加深人们对公共事务的理解,逐渐形成对民主和自由的经验,才能取得进步。

赵紫阳的支持者和对手或许都想引导示威学生,但事实上他们都无法做到。学生们踏着自己的鼓点前进。甚至学生自己的领袖也只能鼓动他们,却不能控制他们。

邓小平没有指责赵紫阳闹派性,但是他说,赵紫阳和胡耀邦一样,都只与一个小圈子的人共事。[21-28]

与胡相比,赵紫阳则参与过阴谋和整人。

冲突在6月4日达到顶点,军队在这一天向北京街头手无寸铁的平民开枪,恢复了秩序。
十三大的中央委员会的产生,是在中共历史上第一次实行了差额选举,

邓小平希望通过高度积极的人员提高管理效率,赵紫阳则想更大范围地减少党在经济和社会单位中的作用。

政治局实际上以那些老干部的更年轻、教育程度更高的追随者取代了他们本人:

导弹、卫星和潜艇的研究在文革期间普遍受到保护。

1987年邓小平放弃了党的副主席和副总理的职务,但他一直保留着中央军委主席一职,直到1989年秋天才把它转交给江泽民。

葡萄牙在1967年和1974年曾两次提出将澳门归还中国,北京已与葡萄牙达成协议,大体勾画出了归还澳门的方案。北京担心这个决定会对极不稳定的香港民意造成负面影响,因此一直对协议保密,公开的说法是还没有作好收回澳门的准备。

1984年10月,在经过试验后,政府在全国范围内实行新税制,用缴税的方式取代了原来的利润上缴(“利改税”)。

通产省一方面会为大公司提供鼓励和支持,但同时也会让他们为争夺市场份额而相互展开激烈的竞争。

会后东欧的学者们去中国各地考察,开始同意中国东道主的观点,即一次性大胆进行改革的东欧模式在中国行不通,因为中国太大,各地情况千差万别。中国唯一切实可行的道路是逐步开放市场和放开价格,然后再进行渐进式的调整。

在向发达国家学习现代化的秘诀这一点上,除了日本和南韩外,没有任何发展中国家能够在广度和深度上与邓小平领导下的中国相比。而由于中国庞大的人口基数,它在学习外国的规模上很快就超过了日、韩两国。

邓小平用他的改革方式又一次赢得了胜利:不争论,先尝试,见效之后再推广。

邓的小儿子邓质方曾对一个美国熟人说:“我父亲认为戈巴卓夫是个傻瓜。”在邓小平看来,戈巴卓夫从政治体制改革入手,分明是误入歧途,因为“他将失去解决经济问题的权力。经济问题解决不了,人民会把他撤职的”。[15-1]

盛田昭夫曾说过,一般而言,没有现代工业的国家,其官僚机构也效率低下,但是一旦采用了现代工业的效率标准,这些标准会逐渐渗透到政府之中。

在一地进行试验、成功之后再加以推广的思想,早就是中共惯有的智慧。[

自中共建党以来,一旦一方在争论中获胜并巩固了权力,其领导人不但要选拔高层干部,还会开展吸收新党员的运动,使符合其标准的人进入党内。

社会主义也能运用现代管理,共产党也可以提倡现代管理。

第一次世界大战期间,很多法国青年上了战场,一时造成工厂劳力短缺,于是招募了15万中国劳工赴法打工。

China Vitae是一个有关在世的中国官员的十分有用的英文网站。

1920年代邓小平在巴黎和上海从事地下工作时,就学会了完全依靠自己的记忆力——他身后没有留下任何笔记。文革期间批判他的人想搜集他的错误纪录,但没有找到任何书面证据。为正式会议准备的讲话稿均由助手撰写,有纪录可查,但其余大多数谈话或会议发言都不需要讲稿,因为邓只靠记忆就能做一个小时甚至更长时间条理分明的讲话。

本书的翻译原则是在“中文化”和“陌生感”之间寻求有效平衡,即在符合中文读者阅读习惯的大前提下,适度保留直译元素和翻译色彩,以使中文读者得以相对直接地分享作者特有的概念、分析思维和学术视野。例如:将“the radicals”直译为“激进派”,而不采用中文特定语境中的“极左派”;将“the builders”、“the balancers”直译为“建设派”、“平衡派”,而不译为“改革派”、“稳健派”。

我在书中记述了邓小平的积极贡献——他努力让所有中国人过上富裕生活,维持与其他国家的良好关系,大力削减军费,增强法律的作用,扩大普通民众公开表达意见的机会等;