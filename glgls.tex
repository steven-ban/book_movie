\subsection{《古拉格·一部历史》}
\subsubsection{书评}

美国人写的关于古拉格的历史性调查,获得了普利策奖。

古拉格,苏联的特殊机构,中国人不会不熟悉——劳改营。大量的囚犯、政治犯并非关押在传统的监狱里服刑,而是以囚犯组织的形式,遍布在苏联广阔土地上的各个角落里,作为廉价劳动力来为经济建设添砖加瓦。苏联不缺少未经开发的土地,因此古拉格可以出现在任何角落。

沙俄时代的知识分子,很多都被流放。对于苏联时代的知识分子,古拉格类似于流放。

在古拉格,犯人们除了和常规的犯人一样缺少自由外,还需要进行强度很大的劳动。古拉格有着监狱里类似的看守、犯人之间的社会形态:犯人与看守之间的奴隶与奴隶主之间的关系,犯人间的等级分化,缺乏的食物,恶劣的环境……犯人们还有着特殊的劳动制度,加上匮乏的食物供应,造成了更加恶劣的生存状况。职业罪犯常常欺负政治犯。

古拉格的管理并不规范,苏联的官僚体制并不高效。像苏联的所有管理体系一样,古拉格对犯人的统计是错误的,接到上级命令后也不会百分之百地执行,充满了瞒报和敷衍塞责。

可是,即使是在如此恶劣的环境下,苏联人仍然坚持了下来,这中间有着闪光的人性。政治犯之间会相互帮助,有些看守还会帮助囚犯摆脱因饥饿和过度劳累造成的死亡。医院是这些囚犯少有的慰藉之处。

更为可贵的是,在领袖和国家意志下受苦的政治犯,虽然有“看清”苏联当局丑恶嘴脸的人,但也有很多依然在内心深处保持着对党对国家的忠诚,一旦出狱会很快投入到对德作战中。现在把这种现象称作“洗脑”,作者作为美国人也似乎不太理解这种行为,但对于中国人,这样的感情应该不陌生。虽然对他们的境遇表示深刻的同情,对当局的行为十分反感,但我对于这些人是尊敬的,对国家深沉的爱对于苏联和东亚国家来说不陌生。对于这些政治犯对国家(甚至是对领袖)的无条件的深沉的爱,我表示尊敬。

劳改营是在斯大林的支持下才完善和建立的,在大清洗时用于关押大量的政治犯,在二战时曾用于关战俘。斯大林死后,劳改营由于管理成本高、经济效益差而渐渐瓦解和消失。

对待劳改营的态度,往往体现了苏联当时领导人对待政治的态度。斯大林一死,他的副手就立刻着手放人,赫鲁晓夫否定了斯大林的个人崇拜,劳改营成为他反对斯大林的工具,随后的苏联领导人为了防止苏联历史被群众知道而抓了大批“异见分子”(当时已经有人权组织在苏联活动了),戈尔巴乔夫着手苏联的改革,也废除了劳改营制度。

对待自己的历史成为苏联的心结。不反思历史,不废除毫无人性的劳改营,苏联将在错误的道路上越走越远;然而错误已经那么深刻,如果全面把历史还给民众,则会造成人们对苏联本身的巨大不信任,政权可能不保。苏联领导人采取了慢慢解冻的办法,也依然无法阻止人们获知历史后的震惊,也无法阻止加盟共和国对苏联的离心离德。戈尔巴乔夫的改革推了最后一把力,激烈的民主化彻底粉碎了苏联政权。

具体到每一个人,曾经的政治犯很多去写回忆录来揭露劳改营和苏联那段黑暗的历史,如索尔仁尼琴、金斯堡;很多人回到了现实里,发现几十年的关押后,已经物是人非,难于融入新的社会;很多丈夫得知劳改营的妻子归来,强烈怀疑她在劳改营里的不忠,感情再难弥合;很多人迫使自己遗忘这段岁月;甚至还有很多人,无法脱离劳改营,认为出了劳改营也无法存活于世,于是就在劳改营附近生活(借用《肖申克的救赎》里的一句话,他们已经被“体制化”了)。

作者采访、调查和写作时,苏联解体还不到10年,那时的俄罗斯人,已经不愿意再去回想这段黑色的历史了。他们对作者寻根究底的行为表现出冷漠甚至敌意。苏联虽然做恶,但它曾经强大,俄罗斯人把这种自豪感与民族的自豪感联系在一起,不愿再去谴责故去的恶魔甚至帮凶。这种思想,我们中国人应该也会比较熟悉吧。

苏联虽然解体,那个体制上曾经的螺丝钉后来在其他共和国里担任着要职,他们不愿去回想自己曾经在那个体制里做过的恶,这会影响他们的仕途和利益,于是刻意少提这些事情,这成为对苏联反思不力的一个重大原因。

或许正如作者所说的那样,古拉格从未消失,它依然囚禁着很多前苏联解体后的国民心里。

本书内容翔实,引用的个人回忆录、苏联解密的官方档案以及作者自己的实地调查采访很丰富,对官方管理制度、上层态度、生产活动、囚犯管理制度、不同囚犯的类型(政治犯、恶性罪犯)、爱情、同性恋、生育、养育儿女等进行了详细的分章介绍。从这一70万字的历史中,我们既可以从宏观层面看到古拉格的产生、壮大、衰败和消亡,也能看到这个体制下众多囚犯、看守的个人生活。

本书作者的态度,是完全站在一个美国人的角度上来看这一制度的,不免带着美国人的感情色彩。作者不理解俄罗斯那种沉重、深厚的国民性格,也不理解这个民族为了崛起而愿意付出怎样的努力,更不理解在这种历史下每一个国民对待苦难的真正的态度。在每一个调查方面,均有一些空白,此时作者使用了大量的推论和猜想。对于关押人数、死亡人数等,恐怕还需要更加详细的调查和专业的研究,我信为作者在本书里采用的数字未必可靠。总之,我对于这种写作态度持一定的怀疑态度。恐怕这种立场限制了作者对古拉格更深层次的思考。

评分:5/5。

\subsubsection{标注}
1. 三十年代后期和四十年代所进行的大规模逮捕在一定程度上可能也是为了满足斯大林对强制劳动力的需要,而不是—像大多数人一直想当然地认为的那样—为了严厉打击他心目中或者潜在的敌人。

2. 没有苏维埃当局,只有索洛韦茨基当局”这句话将一遍又一遍地反复被人提到。

3. 第一部布尔什维克刑法典中的某些内容使西方最激进的刑事司法制度改革人士兴高采烈。其中,这部法典规定,“没有个体犯罪这种事情”,因此,刑事判决“不应当被视为惩罚”。7

4. 在德国,你可能死于残忍;在俄国,你可能死于绝望。在奥斯维辛,你可能死在毒气室里;在科雷马,你可能冻死在雪地上。你可能死于德国的森林或者西伯利亚的荒原,你可能死于一次矿井事故,你也可能死在运牛的火车上。

5. 在苏联,害人者自己也有可能成为受害者。古拉格的看守、管理人员、甚至秘密警察的高级官员也会被捕然后发现自己也被判处在劳改营里服刑。没有哪一棵“有害的杂草”总是有害的,换句话说,没有哪一个苏联囚犯群体始终生活在不断产生的死亡预感之中。55

6. (阿伦特)她接着写道,“极权主义国家警察的任务不是发现犯罪,而是随时准备在政府做出决定时立即逮捕某一类人”。50问题再次回到了原处:人们被逮捕,不是因为他们做了什么,而是因为他们属于某一类人。

8. 布尔什维克的西伯利亚经历为他们提供了一个作为基础的早期模式—以及一个应当汲取的教训:惩罚制度必须特别严厉。


