\subsection{《呼兰河传》}
\subsubsection{评论}

《呼兰河传》从严格意义上很难说是一部小说,因此它没有一气呵成的故事和完全意义上的“主角”。如果说全书是萧红对自已童年的回忆录,那很难解释开篇前几章对呼兰河地理和社会风情的描写。不过,书里的几个人物,如祖父、“我”、团圆媳妇、团圆媳妇的婆婆、有二爷等等,写得确实好。他们(尤其是那些大人)对世界没有明确的仇恨,也没有热烈的依恋,只不过是像牲口一样地活着,间或有生活的甜蜜,但更多的是生活的无奈;他们隐忍,他们谦卑,他们似乎只是为了活着。团圆媳妇是悲惨的,生了病,愚昧的乡民们把她放进滚烫的开水里去邪,活生生地折腾死了一个年轻鲜活的生命;团圆媳妇的婆婆,当听信江湖骗子的话,拿出自已不多的几十块钱积蓄时,那种纠结、心疼和痛苦,真是活生生的贫困人的写照。几十块钱对她来说,是养小鸡时对生活仅有的一些希望,但当遭遇天灾,那仅有的一抹亮色也变成对生活的屈服。我觉得这是这本书里写得最好的一段\footnote{本段写于2014年}。



萧红的文字大气,很耐看。

团圆媳妇,令人悲伤,令人绝望。

磨坊里的歪嘴

贫穷封闭的生活让人变得麻木冷漠

\subsubsection{一些标注}

他不但没有感到绝望已经洞穿了他。因为他看见了他的两个孩子,他反而镇定下来。他觉得在这世界上,他一定要生根的。要长得牢牢的。他不管他自己有这份能力没有,他看看别人也都是这样做的,他觉得他也应该这样做。

等我生来了,第一给了祖父的无限的欢喜,等我长大了,祖父非常的爱我。使我觉得在这世界上,有了祖父就够了,还怕什么呢?虽然父亲的冷淡,母亲的恶言恶色,和祖母的用针刺我手指的这些事,都觉得算不了什么

是凡在太阳下的,都是健康的,漂亮的,拍一拍连大树都会发响的,叫一叫就是站在对面的土墙都会回答似的。

人若老实了,不但异类要来欺侮,就是同类也不同情。

温顺也不是怎么优良的天性,而是被打的结果。甚或是招打的原由。