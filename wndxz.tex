\subsection{《无奈的选择——冷战与中苏同盟的命运(1945-1959)》}

英文书名:The Cold War and the Fate of the SINO-Soviet Alliance 1945-1959

作者:沈志华

本书是作者沈志华对中苏关系的力作,主要是梳理了中国、西方和前苏联的史料,特别是苏联已经解密的史料,详细阐述了中苏关系从1949年-1959年十年间的变化,91.8万字,十分详细认真。作者是一个真正的历史学家,对历史采取了客观的、超脱于意识形态的冷静,对于有哪些史料、不同史料之间的分歧、由于史料不足而进行了必要的推理或者猜想都有明确的展示,足见作者的态度与功力。阎明复在自己作的序里说:
\begin{quotation}
沈志华与李丹慧的新著有三个突出特点:一是引用的史料来源丰富且极为扎实;二是对这段历史所涉及的诸多重大事件的过程,特别是对其中一些关键细节做了重新考证,厘清了许多史实,并有不少新的发现;三是以开阔的视野对中苏关系演变的历史做了新的诠释,书中既有对全球冷战史的观照,又有对中苏双方,特别是中国内政与外交相互作用的探究,还有对中苏关系特殊结构的分析。
\end{quotation}

按照章节顺序,中苏关系在1949-1959年的演进如下:
\begin{description}
    \item[第一、二章] 二战结束时,中苏与苏联之间并没有相互结合和依赖的意向,斯大林要实现苏联在亚洲的战略利益,必须依靠和蒋介石的合作,而毛泽东欲在中国为共产党谋得一席之地,希望能得到美国的帮助。斯大林劝说中共与国民党和谈,并进入国民政府,毛则有意利用天时地利依靠武力壮大共产党的实力。苏联为保证其特殊利益而在东北问题上左右逢源,中共依靠苏联的帮助在东北和华北建立起革命根据地。毛背水一战,在国共内战中逐步取得优势(苏联援助并非决定因素),斯大林担心中共势力进一步发展引起美国介入,对扩大援助中共犹豫不决。此时,苏联的冷战战略不具进攻性,且希望避免和推迟与美国的直接冲突,特别是在亚洲,斯大林此时有种矛盾心理:既想全面详细了解、全面掌控中共,又不便直接接触、公开支持。毛泽东急于前往莫斯科争取苏共的理解和支持,斯大林出于种种顾虑谨慎地回避直接与毛见面,在中共取得最后胜利的关键时刻,斯大林有意出面斡旋国共和谈遭到毛的强烈反对。米高扬秘密访问西柏坡取得积极成效,中共中央一再表明追随莫斯科的政治立场和各项方针政策。刘少奇秘密访问莫斯科收获颇丰,斯大林全面满足中共的各种要求,毛则宣布新中国向苏联“一边倒”的既定方针。
    \item[第三章] 主要是1949年底-1950年初中苏明盟条约谈判中双方的利益冲突及其结果。苏联第一个表示承认新中国,但斯大林有意回避重新讨论中苏同盟条约问题。毛为维护中共的威信和地位,坚持必须签订一个新的中苏同盟条约,斯大林被迫同意中国的主张。莫斯科为维持1945年条约的基本内容,精心设计了中苏同盟新条约及其协定,毛推翻苏联关于中长铁路、旅顺港和大连的协定,让周恩来重新起草新的协定,提出中国至迟在两年内收回苏联在中国东北的全部特权。迫于国际形势的微妙变化,担心美国在中苏之间加入楔子,斯大林不得不再次做出重大让步,按照中国的主张处理中苏之间的利益关系。苏方在中苏后期外交谈判中斤斤计较,迫使中国接受苏联的苛刻条件,斯大林则独辟蹊径,把目光投向朝鲜半岛,寻找能够保证苏联在远东和太平洋安全战略继续实施的替代方案。
    \item[第四章] 主要是中苏同盟在朝鲜战争中的巩固和发展。中苏同盟条约的签订和整体上对稳定苏联的东方战略以及中共政权的巩固和发展具有保证作用,但谈判过程在斯大林和毛心中都留下了阴影,同时对于中共解放台湾、统一中国有间接的阻碍作用,是当时最根本的原因。斯大林为了保证苏联在远东的出海口和不冻港,改变同毛之前商定的帮助中共解放台湾的主意,转而支持金日成通过军事手段实现半岛统一,毛不得不接受。朝鲜战争初期,毛积极准备,主动表示愿意出兵帮助朝鲜,而斯大林则力主支持金日成单独作战,以保证苏联对朝鲜的控制,对毛的暗示和金日成的请求置若罔闻。美国仁川登陆后,形势急转直下,斯大林不得不要求中国出兵,毛力排众议决定出兵。在苏联一再退缩的情况下,毛和中共的行为感动了斯大林和莫斯科对中共的看法,决定派出空军作战,对中国提供了大量军事和经济援助\footnote{苏联当时向中国提供的军事贷款为62.88亿卢布,占1950年代全部贷款的95\% ,抗美援朝贷款总计32亿卢布。},并在军事战略方面与中国保持了高度的协调和统一,巩固和发展了中苏同盟。
    \item[第五-八章] 全面讲述了1954-1957年中苏关系处在最亲密、最友好的蜜月期的情况。赫鲁晓夫主动提出全面发展中苏关系,大力增加对华援助,退出股份公司,归还旅顺基地,毛在中国掀起全面学习苏联的高潮。在苏联的大规模援助和苏联专家的直接参与下,一五计划顺利完成,经济发展迅速。苏共二十大提出的一系列方针,促使中共对苏联模式进行了思考和反思,秘密报告中关于批判斯大林的个人崇拜现象使毛和中共感到轻松愉快,但秘密报告引发的冲击使得毛担忧,认为有责任帮助苏共。中共八大提出一些新的思路和路线,开始以苏为鉴思考自己的发展道路,取消了“毛泽东思想”的提法,吹响了与苏共二十大的双重奏\footnote{从20世纪80年代至今中国的经验看,否定斯大林模式必须过两道关口,即经济体制的市场关和政治体制的民主关,而赫鲁晓夫和毛当时的思考都在这两道关口面前止步了。}。波兰-匈牙利事件使赫鲁晓夫焦头烂额,急需中共的帮助,而中共为解决波兰危机提出了批判苏联的大国沙文主义和巩固社会主义阵营的两在原则,支持了波兰的独立立场,而认为匈牙利发生的市民武装暴动到后期已经是反革命事件,坚决要求苏联进行军事镇压。波匈事件的余波未来,赫鲁晓夫难以处理,周恩来应邀帮助莫斯科,在苏、波、匈之间协调开展穿梭外交,中共在社会主义阵营的威望、影响和地位空前提高。共产党情报局解散以后,苏共希望再成立一个国际机构以加强各国共产党之间的联系,毛则主张开会解决问题。中共在莫斯科六月事件后公开表明支持赫鲁晓夫,赫鲁晓夫则投桃报李,突破以往的禁区,向中国提供核援助\footnote{还在苏联第一次核试验之前,中共就知道莫斯科掌握了核技术,甚至提出参观苏联的核设施。}。世界共产党莫斯科会议成为中苏两党两国紧密配合的经典之作(此时中共和毛在社会主义阵营的影响和地位已经超过了苏共),毛在会议上意气风发,指点江山,但发言是即兴的,引起争论,两党出现了分歧的迹象。
    \item[第九-十章] 分析了1958-1959年中苏在内政外交上的分歧。毛发动“大跃进”和人民公社运动\footnote{毛认为中国当时在领导能力上已经超过苏联,之所以不能做共产主义的领导是因为经济比苏联差,因此必须把经济赶快搞上去才能具备领导资格,并表明中共的发展模式好于苏联,这是这场运动的主要动机。},客观上使苏联的经验、技术、苏联专家在中国受到冷落。莫斯科对“大跃进”从热情支持转向谨慎反对,对人民公社则始终表示沉默\footnote{下层比较积极,上层则始终保持谨慎态度,苏联公开反对则是在大论战之后的事情。},很大程度上刺激了毛,庐山会议期间赫鲁晓夫在波兰的讲话授人以柄,毛决心向赫鲁晓夫和其他怀疑和反对人民公社的人宣战。苏联为实现中苏军事合作提出在中国建立长波电台和苏联潜艇部队停靠中国口岸的建议,毛对此十分反感,严厉斥责苏联大使,赫鲁晓夫为此秘密赶往北京登门解释。毛借题发挥,公开了赫鲁晓夫的行程,并劝说赫鲁晓夫签署中苏联合公报,随后却下令解放军炮击金门,希望仅仅通过炮轰的方式来封锁金门,造成一种气势和压力,迫使蒋介石放弃金门,从而实现收复全部沿海岛屿的既定军事战略和安全战略,引发了远东危机。在美国公开表示支持蒋介石并威胁北京的立场后,苏联不得不发表声明为中国大陆提供核保护伞,台海危机显示出中苏之间对国际局势认识和应对的分歧,赫鲁晓夫对毛的举动既不满又担心,随即决定停止对中国的核援助。中国镇压西藏叛乱引起印度的反对和不安,引发中印边界纠纷,莫斯科不满中国的强硬立场,在冲突中采取了中立的立场,中共认为苏联参与了帝国主义和反对派的反华大合唱,毛对赫鲁晓夫的访美示好、和平主义活动不以为然,主张利用紧张局势加强与美国的斗争,中苏两国领导人在会谈中发生了激烈争吵,双方的感情均受到伤害。此后,中苏分歧公开,双边关系逐渐恶化。
\end{description}

本书对这段历史的解读,贯穿了这样的框架:\emph{社会主义国家之间的关系在结构上不同于一般意义上的现代国家之间的关系,由于受到意识形态和历史传统的影响,其内部运行的政治准则有某种特殊性,中苏关系又相对于普通社会主义国家之间有着特殊性,即两个社会主义大国为争夺国际共产主义运动主导权暗中较劲}。基于此,作者提出了相对以往不同的一些看法和解释:
\begin{enumerate}
    \item 从二战到冷战这段时间内,中共与苏共之间既没有结盟的愿望,也没有结盟的行动。毛开始设想的是依靠美国的帮助,而苏联则欲与蒋介石合作。之后美国对国民党的支持和蒋介石的反共倾向,才让斯大林和毛走到了一起。某种程度上讲,中苏同盟和美蒋同盟都是他们在最后关头迫不得已做出的选择,意识形态并并非这种转变的根本原因。
    \item 1944-1946年,毛对美国确实心存希望,但1949年时情况已经发生了根本性的转变,中共和之后的新中国和美国之间已经不可能再有“机会”形成外交伙伴了。
    \item 导致朝鲜战争爆发的根本原因,不是中苏之间合作的前景(斯大林并不认为中共足够强大,也不认同中共的军事经验),而是在同盟建立中双方利益冲突的结果。
    \item 毛力排众议坚持入朝作战的动机是什么?作者认为,不仅仅是国家利益,而是毛在考虑亚洲革命和世界革命,认为中共和中国对朝鲜革命对责任。当时能够保证政权稳定的前提是中苏同盟,而为了保证中苏同盟必须得到斯大林的支持(毛看出来斯大林对中共和他本人的不满和不信任),因此必须听从斯大林而出兵。这个判断是正确的。
    \item 赫鲁晓夫上台后,执政经验不足,客观上需要中共的帮助,而中国的经济发展则需要苏联的援助,两国是相互需要的关系。
    \item 之前很多人认为苏共二十大是中苏同盟破裂的起点,而作者认为不是,苏共二十大和中共八大在对内发展经济、对外寻求和平方向是完全一致的。只能说,苏共二十大造成的混乱为之后的分裂埋下了伏笔。
    \item 1957年的莫斯科共产主义大会不仅是两国两党合作的顶点,但分歧也就此发端。引起苏共和东欧各党不满的地方主要是毛以国际共产主义运动领袖自居的傲慢态度和对世界大战和核战争前景的“耸人听闻”的表达方式。
    \item 中苏同盟的破裂最初不是在意识形态和国家利益方面,而是1958-1959年对内外政策的不合。两党在意识形态和国家利益事实上并没有实质性的分歧和冲突(即使有也可以进行沟通和妥协),争的是“正统”和主导权,这是无法妥协和调和的。\emph{根本原因就在于,20世纪40-70年代,中苏关系和社会主义国家之间的关系始终处于一种不正常、不成熟的状态,存在以党际关系掩盖甚至替代国家关系的现象,这是造成这种分裂的结构性原因。}
\end{enumerate}

前述20世纪40-70年代社会主义国家之间的\emph{结构性失衡}具体表现在:
\begin{enumerate}
    \item 主权观念不明确,表面为国际主义理念与民族或国家利益诉求之间的矛盾,以意识形态的同一性替代或者掩盖了国家利益的差异性。
    \item 平等意识不清晰,表现为同盟内部领导与被领导的组织原则与各国享有平等权利的准则之间的矛盾,以社会主义阵营的统一领导排斥了同盟各国应享有的平等权利。因此,党的关系的破裂就意味着国家关系的破裂。
\end{enumerate}

五十年代后期为什么中苏之间出现矛盾和分化?作者给出了十分全面和充足的理由(参考了阎明复的总结):
\begin{enumerate}
    \item 两国的过去历史、革命斗争性质和形式、社会经济发展水平、社会主义建设所处的阶段不同。苏共已经执政多年,走过一些弯路,并对此进行了总结和反思,并且苏联已经是两极世界的一极,已经融入和主导了国际秩序,需要制订和遵守外交规则;中共则相反,执政时间短,国家贫穷落后,被西方孤立,并且对苏联之前的弯路体会不深。与此同时,两国对政治方针和措施的评价标准、领导干部的文化特点和民族性格,也都有差异。
    \item 苏中两国所处的国际地位有很大差别,由此产生双方优先考虑的对外政策重点有很大不同。这使双方领导人在国际局势的估计、对外政策策略、军事政治措施和反帝宣传等方面,都出现了重大分歧。
\end{enumerate}

总的来说,五十年代末,苏联已经开始改变其对内对外方针,而中国则受客观条件的限制,仍在继续沿着先前的方向走,因此两国产生和利益、目的、对外政策方针和行动上的区别和矛盾,对立和冲突。开展论战、扩大分歧的主要动作是由赫鲁晓夫来进行的。

本书对于中苏历史爱好者、中国现代史爱好者而言是一本难得的好书。中苏关系在1960后之后的发展,则是李丹慧(作者妻子)的《无悔的分手》,应该也值得一读。

评分:5/5。