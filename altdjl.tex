\subsection{《阿勒泰的角落》}
\subsubsection{感想}

1.母女两人在荒凉纯朴的新疆阿勒泰,随着牧民的迁徙而游走,卖小商品,给人裁衣服……

2.看到哈依娜一节,李娟写到她和哈依娜由陌生到朋友的转变,以及成为朋友的那种“尴尬”,突然明白了李娟田园牧歌逐草而居的生活,和喧嚷嘈杂的都市生活的区别在哪里?在哪里呢?在于她的生活很空旷,星星点点却都是珍贵的东西,她处在自己喜欢和欣赏的物体的包围里。而都市人呢,处在各种喜欢的不喜欢的层层叠叠水泄不通的包围里,都市人离不开这种包围和它带来的物质,离不开这种丰富带来的享乐。他们希望自己身边不喜欢的那些,要么通通消失,身边只剩下让自己开心的那些东西,要么成为自己喜欢的,所以他们羡慕李娟的快乐和单纯。但是,物质的丰富和物质的俗厌从来就是硬币的两面,不可能去除一面而完整保留另一面。所以,李娟的生活再美好再单纯,也不过是满足心理的空虚罢了。但是,能在复杂的生活外阅读这样的文字,对生活本身而言,已经是非常好的弥补了,这大概也是读书的价值所在吧。

3.李娟的妈妈和姥姥都是很精悍的女人,特别是李娟的妈妈,保留着巨大的童心,爱折腾,又是修窗户又是养金鱼,冒着严寒大老远带来花去养。但她不会养,能力不高,虐待金鱼,这点让我讨厌。

4.李娟在“新版自序”中说,她写下这些纯真快乐的文字时的感情,在后来的生活里已经失去了很多,所谓的“纯真与朴素”,已经慢慢失去。李娟说,“刻意地保持纯真,这本身就不是一件纯真的事吧?”难得的诚实。如果在纯真和诚实之间选择,我更喜欢后者的李娟。

\subsubsection{一些标注}

在阿克哈拉恋爱多好啊!尤其在秋天,一年的事情差不多已经忙完,漫长而悠闲的冬天无比诱惑地缓缓前来了……于是追求的追求,期待的期待……劳动的四肢如此年轻健康,这样的身子与身子靠在一起,靠在蓝天下,蓝天高处的风和云迅速奔走。身外大地辽阔寂静。大地上的树一棵远离一棵,遥遥相望。夕阳横扫过来,每一棵树都迎身而立,说出一切。说完后树上的乌鸦全部乍起,满天都是……在遥远的阿克哈拉,乌伦古河只经过半个小时就走了,人活过几十年就死了,一切似乎那么无望,再没有其他任何可能性了。世界寂静地喘息,深深封闭着眼睛和心灵……但是,只要种子还在大地里就必定会发芽,只要人进入青春之中就必定会孤独,必定会有欲望。什么原因也没有,什么目的也没有,我妹妹就那样恋爱了。趁又年轻又空空如也的时候,赶紧找个人和他(她)在一起——哎,真是幸福!

我妹妹刚满十八,已经发育得鼓鼓囊囊,头发由原先的柔软稀薄一下子变得又黑又亮,攥在手中满满一大把。

缺乏野外捕食的经验,加之天气一天冷似一天……后来第一场雪下了,第二场雪也下了……看不到一个人,得不到任何救助,然后就什么也不能明白地死去了!\footnote{可怜的狗!——读者注}

这样的山野里会有什么毒物呢?这开阔的,清新的,明亮干爽的,高处的……一眼望过去,万物坦荡,不投阴影。\footnote{有毒的很多哦!——读者注}

当他们喃喃自语地在草丛里寻找什么东西,当他们把一颗完全能够一口就吞下的糖分成无数次耐心吮完,当他们互相之间有条有理地谈论着在我们听来乱七八糟的话题……小孩子的幸福多么宽广!

其实我们这里的所有孩子都会弹电子琴的,他们好像天生就对音乐、对音阶高低的细微变化敏感异常,刚刚听完一首歌,顺手就可以在琴上完整地敲出来。