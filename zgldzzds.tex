\subsection{《中国历代政治得失》}
1. 钱穆痛切警告:国人懒于探寻国史真谛,而一意据他人之说,肆意破坏,轻言改革,则自食其恶果。

2. 刚日读经、柔日读史

3. 我们天天说我们的法不够,其实不够的不在法,而在才。这也不是我们之无才,乃是我们的才,不能在我们的法里真有所表现。

4. 皇室的权,总是逐步升,政府的权,总是逐步将。这也是中国传统政治上的大毛病。

5. 若说政权,则中国应该是一种士人政权,政府大权都掌握在士------读书人手里,从汉到明都如此。

6. 尤其是清代,可说全没有制度。它所有的制度,都是根据着明代,而在明代的制度里,再加上他们许多的私心。这种私心,可说是一种"部族政权"的私心。一切有满洲部族的私心处罚,所以全只有法术,更不见制度。

7. 春秋时代有封建贵族,东汉以下至中唐时期有大门第,晚唐以下迄于宋明,社会大门第全消失了。农户散漫,全成一新形态。

8. 汉代培养人才的是掾属。唐代培养人才在门第。宋代培养人才在馆阅校理之职。到明清两代,始把培养人才的机构归并到考试制度里。

9. 西方社会有阶级,无流品。中国社会则有流品,无阶级。

10. 行省是一个行动的中书省。

11. 宋代根本无地方官,只暂时派中央官员来兼管地方事

12. 宋代制度之缺点,在散,在弱,不在专与暴。

13. 在西方现行的所谓民主政治,只是行政领袖如大总统或内阁总理之类,由民众公选,此外一切用人便无标准。这亦何尝无毛病呢?所以西方在其选举政治领袖之外,还得参酌采用中国的考试制度来建立他们的所谓文官任用法。

14. 汉宰相是采用领袖制的,而唐代宰相则采用委员制。换言之,汉代由宰相一人掌握全国行政大权,而唐代则把相权分别操掌于几个部门,由许多人来共同负责,凡事经各部门之会议而决定。汉

15. 中国魏晋以下门第社会之起因,最主要的自然要追溯到汉代之察举制度。

16. 中国历史上的土地政策,一面常欣羡古代井田制度之土地平均占有,但一面又主张耕者有其田,承认耕地应归属民间之私产。
