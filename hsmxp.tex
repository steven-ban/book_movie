\subsection{《海上明信片》}

作者:【英】Victoria Hislop

\subsubsection{一些标注}

希腊人认为有各种各样的爱,一个“爱”字概括不了。这些“爱”之间的界限不甚清晰,但大致是这样的:agapi是对神或家庭的爱(或许是最理性的),filia是对朋友的爱,erotas是因性而产生的爱。

\subsubsection{书评}

这本书是一本短篇小说集,但套着一个壳子:在英国伦敦工作的艾丽每天疲于两点一线的枯燥办公室生活,生活不富裕,每月担心自己的房租;一段时间里,她一直收到一些从希腊寄来的明信片,甚至一本笔记本,寄信者要寄给另外一个女人,他爱上了那个女人,但不知道自己寄错了人;他每天思念着她,从一个地方到另外一个地方,与当地的人交谈,把自己的听到的故事和感想写下来;希腊的风景让艾丽着迷,于是她请了假到希腊旅行;故事的最后,她遇见了那个男人,他试述了他与那个女人的爱情,其实很简单,就是他单方面的迷恋,那个女人是个富家女,有品味,喜欢艺术和博物馆,也喜欢希腊,他们在伦敦的酒吧认识,男人想让她求婚,但是没能如愿;失意的他在希腊写了一张张明信片想寄给她,但那个地址是那个女人编造的;艾丽本来要回去,但被这个男人感染,于是决定留在希腊,给他出版那些明信片和笔记本上的故事,她虽然接到了那个女人从伦敦打来的询问是否有明信片的电话,但没有告诉她是艾丽自己收到了(那个男人让艾丽说的)。

而里面的一个个短篇,则是关于希腊不同地方的小故事\footnote{这15个短篇的简要内容,可以参考\url{https://m.douban.com/book/review/10395516/}},我不知道这些故事是如何创作出来的:是作者实地采访出来的,确有其事?还是就是纯粹地编造出来的?这些故事里,有外地小夫妻在封闭落后的小村庄被囚禁,男人被当成囚犯烧死的;有一个男孩为了赎“罪”(其实就是青春期时被哥哥捉弄,给当地地妓女打电话手淫而被父亲发现,暴打一顿)而做了教士,与世隔绝长时间而无法释怀,为一个少妇忏悔而无法忘怀少时的“罪”,于是跑上山自杀;有年青的跳水者被同辈嫉妒而背上几十年谋杀的罪名……这些故事里,希腊并不“美好”,而是贫穷,落后,封闭,甚至有一些愚昧。希腊遭受了债务危机,经济摇摇欲坠,人民生活水平并没有提高,然而与之相对的,是这个国家有着欧洲最为久远的文明遗迹,有地中海的漂亮的自然景观,这些吸引了艾丽这样的大都市的人,他们不远万里来到这里,释放生活的压力,追求自由恬淡的生活。但是,一旦与这个国家的近距离接触,除了异国风情的滤镜外,你会发现这个国家、这个民族与世界上任何地方的人并没有实质性的不同,他们也要为生计奔波,也要为现代化付出代价,人性也有贪婪、自私的一面。像艾丽、写明信片的男人、一个故事里放弃德国银行业高薪工作的本地女人这样的“文青”的那些人,在希腊追求的更多是一种慢下来的前现代生活,但这显然是有前提的,那就是他们要么已经有了财富积累,不用为生活奔波(毕竟希腊无法提供那么多高薪职位),要么随时有退路,可以回到原来的都市生活里。很难想象,他们真的能够忍受一个国家经济慢慢地凋谢,犯罪率上升,被国际主流抛弃。其实这些文青,和那些对西藏一无所知,却执意去西藏“洗涤灵魂”的人没有什么区别,如果他们不仅仅是为了晒朋友圈,吸引他人的眼光,而是真的对异域风情感兴趣,对前现代的相对于现代社会的诸多“不同”感兴趣,甚至于对神秘主义、宗教感兴趣,而放弃了快速变化的现代社会,那简直就是愚蠢至极了。

因此,我并不喜欢这本书。虽然它确实是对希腊的近况,特别是2008年以来的状态有过准确的把握,对希腊的独立乃至影响了西方社会的神话、城邦有过描写,但是我对希腊仍然爱不起来,原因已经在上面说了,这是一个没有希望的国家。作者强调希腊的历史,说它坚韧不拔,然而这样的坚韧在中国历史面前就是小巫见大巫,我说这些并不是无谓的骄傲,而是觉得一个国家一个社会最重要的仍然是当下。当然,这本小说的壳子,就是那个男人的迷恋一个女人又放弃的故事,核心就在于他在希腊找到了“自我”,这种庸俗的自由主义叙事实在是平庸且无聊。

评分:2/5。