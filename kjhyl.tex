\subsection{《苦妓回忆录》}

作者:马尔克斯

自从看了马尔克斯的《霍乱时期的爱情》,更对他的更多作品感兴趣。之前虽然读过《百年孤独》,但吸引人的更多是写法和对拉美几百看历史的深度浓缩,而《霍乱时期的爱情》则是把目光投向普通人的喜怒哀乐,以及马尔克斯自己柔软细腻的内心。这本《苦妓回忆录》是他生命最后的作品,是一部中篇。故事讲的是“我”是一个刚过90岁的记者,仍然按时给报社写文字,终生未婚,之前经常光顾妓院。他曾经有过一个未婚妻,但是她对生育的热情阻止了“我”继续与她生活。“我”虽然垂垂老矣,但是性能力依然不错,就连妓院老鸨也常常以此打趣。老鸨按照“我”的要求,推荐了一个处女,她是一个未成年女孩,白天辛苦地做工,晚上来到妓院,“我”则几乎没有和她有语言上的交流,而是止步于观察裸体、打扮房间、为她读书,调整自己的心情,而她在绝大多数时间则是睡觉,只会在“我”睡着时在镜子上留下一些字。“我”甚至为她买了用于通勤的自行车。后来妓院发生命案,死者是权倾一时的富商,老鸨逃走,女孩也不知所终,“我”则时时想念着她,担心她死去。后来老鸨回来,又把女孩找回来,但看到被老鸨浓妆艳抹的女孩,“我”感受到强烈的醋意,知道真相后才气消。在一年后的九十一岁生日时,“我”进入到一种新的生命阶段,感受到幸福和满足。

本篇的主题依然是生命和爱情,特别是老人的爱情。“我”虽然年纪很大,不曾结婚,但对爱情有着敬畏和尊重,对那个未成年的女孩也保持着爱,这种爱并非占有,而是欣赏与陪伴,并且有着老年人和文艺人士的单纯和敏感,就像是一种生命的底色。自然,这样的爱情,包括嫖娼本身,在中国社会肯定是不伦的,有着恋童的不洁,但作者似乎很喜欢刻画老年人在生命快要达到终点时的那种对爱情的珍惜(在《霍乱时期的爱情》里也有,比如阿里萨随着老年的接近,突然感受到自己等不到费尔明娜丧夫而自己上位的那种震惊和慌乱,以及他们是生命最后阶段重新在一起的美好),这不同于年轻人爱情的热烈奔放、占有肉体的欲望、对未来的不确定(老人没有未来,或者他们的未来没有任何的不确定,也没有这种不确定带来的心态上的失衡,他们只有回忆)和各种由于年轻带来的花花肠子。如果说“我”在90岁和91岁生日时有什么不同,那就是经历过一年的与女孩的爱情后,特别是经历过得到与失去她的那个过程里,“我”通过内心的反省,对死亡已经不再有任何惧怕,对爱情或者说自己爱的能力有了足够的自信,可以坦然面对生命到死亡的最后一段旅途,事实上“我”认为自己会到100多岁时依然有爱的希望。或许我现在还无法懂得老年人的这种心态,是不是这种心态下面对爱情的感受才是更加纯粹的?毕竟作为一个年轻人,我害怕衰老带来的痛苦和不便,更认为衰老会带来爱情上的无情(并不是性无能)。或许这样的书,我在四十岁或者五十岁之后才能读懂吧。

评分:5/5。