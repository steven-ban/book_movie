\subsection{《秘密》}
\subsubsection{一些标注}
\begin{itemize*}
	\item 写实与无情是体现作家东野圭吾创作理念的关键语。
	\item 他嫉妒重走青春的直子。他嫉妒能与直子一同享受青春的年轻男孩。同时,他也憎恨自己不能对她产生爱情与情欲。
	\item 了解他之前所不可能知道的事情。 原来是这样啊!从藻奈美口中说出许多连她原本不可能知道的事,像是平介与直子的第一次约会…… 果然,与其说藻奈美的性格变得与“直子完全相同”,还不如说是直子的人格附在藻奈美身上,这样比较容易让人接受。 平介快速地浏览了这本书。结果,他发现在“附身”这个单元之后,便是“多重人格”了。他仔细阅读了一下,其中提到几个实例,都是说明当这种现象无法用心理学的角度来解释时,就
现在的社会已经趋向高龄化了。到时候退休年龄可能已经提高到六十五或七十岁了呢!
\end{itemize*}

\subsubsection{书评}
一个看似很感人,实际上充满了人性自私的故事。不过,大概所有的感人都是人的自私在作祟。 
在一次司机疲劳驾驶的回乡之旅中,平介的妻子女儿发生事故,奇迹的是妻子直子虽然去世,但灵魂转移到女儿藻奈美体内。度过了一段不适应后,直子做了一个决定,以女儿藻奈美的身体生活,在外(主要就是学校)扮演努力学习成绩优异的藻奈美,在家扮演妻子,让自己有一个不后悔的人生(直子是家庭主妇,只上过社区大学,应该是个学渣)。平介当然是支持的。 
但随着直子升入私立女中和高中,平介无法忍受这既是女儿又是妻子、既不是女儿又不是妻子的直子了。他身体有性欲,但无法通过外貌是女儿的妻子来解决。为了事情不败露,他又没有找妻子的打算。但是,刚过四十的平介显然感觉到了自己的衰老,恐惧和孤独不断腐蚀着他的内心。他嫉妒直子,为什么她就可以“抛弃”丈夫重获青春活得有滋有味而自己却要独自承受这种人生的荒谬呢?终于,这种不满在直子将要和高中学长出去私会时达到高潮。平介探知他们的行踪,在家里装窃听器来偷听他们打电话,并最终在他们约会地点阻止了这场幽会。 
另一方面,平介也通过自己的渠道来获取真相。司机多年前有段婚姻,但婚姻里的儿子不是自己的。司机爱着这个不属于自己的儿子,虽然离婚再娶,但当这个儿子要上大学需要花钱时,通过加班和疲劳驾驶来多挣钱并汇钱给他们,终于酿成苦果导致一车人丧生,除了直子。真相很简单。 
绝望的直子又扮演起了藻奈美的角色,平介也决定以女儿的心态对待直子。但当他第一次叫直子“藻奈美”时,女儿的意识复活了。似乎直子和藻奈美共享了藻奈美的身体,五年后女儿的意识突然复活,每次醒来就要轮替意识。之后,藻奈美的意识越来越长,而直子的意识越来越短。最终,在初遇的公园里,直子的意识完全消失。 
结尾处,藻奈美和司机的儿子结了婚。失去妻子和女儿的平介想打司机儿子来泄愤,但却蹲下号啕大哭。实际上,通过藻奈美戴上了直子的婚戒,平介已经知道了其实直子没有消失,而是继续扮演着藻奈美的角色生活了下去,这成了他们两人之间的“秘密”。

其实关于换了意识的“藻奈美”究竟是直子还是藻奈美,很多人有不同看法。两者都能说通,都震撼人心。但我更喜欢“直子扮演藻奈美”的猜想。处在两难境地的平介,其实是这一群着最痛苦的人。他需要养活直子,但得不到直子的身体,因为那个身体是藻奈美的;当藻奈美“复活”,他了同时失去了女儿。平介嫁女时的心态,有女儿的父亲们应该都心有戚戚,女儿也是父亲的小情人啊!

不过,我想吐槽的是,其实需要这么痛苦吗?有着熟女内心和萝莉身体的“直子”,不是所有男人的梦想吗?这简直是通往新世界的大门啊!难道不能在一起做那种事情吗? 
而且,即使有痛苦,不能和直子坦诚交流吗?为什么苦着自己呢?

电影版由广末凉子主演,把藻奈美从小学直接搬到了高中,也取消了司机后妻的设定,直接让藻奈美遇见了司机的儿子。故事变得更简单了,取消了悬疑元素,几乎成了一部纯感情片,与原著相距甚远。因此,电影版的情节变得有点支离破碎,内心戏大幅减少,表演流于表面。可是,广末凉子好美啊!