\subsection{《历代经济变革得失》}

\subsubsection{一些标注}
(分税制)中央将税源稳定、税基广、易征收的税种大部分上划,消费税、关税划为中央固定收入,企业所得税按纳税人隶属关系分别划归中央和地方;增值税在中央与地方之间按75:25的比例分成。

国民党把它的宏观经济治理模式称为“统制经济”,即有计划的商品经济,或者说是国家资本主义。

(费正清)“中国的传统不是制造一个更好的捕鼠机,而是从官方取得捕鼠的特权。”
桑弘羊之问”:如果不执行国营化政策,战争的开支从哪里出?国家的财政收入从哪里得?地方割据的景象如何化解?而这三项不正是治国者必须面对和解决的最重要课题吗?

1973年8月,毛泽东创作《七律・读<封建论>呈郭老》一诗,将孔孟儒学贬为“秕糠”并公开替秦始皇“焚书坑儒”翻案,全诗曰:
\begin{quotation}
劝君少骂秦始皇,焚坑事业要商量; 
祖龙魂死秦犹在,孔学名高实秕糠; 
百代都行秦政法,十批不是好文章; 
熟读唐人封建论,莫从子厚返文王。
\end{quotation}

在中国的统治术里,貌似水火不容的儒家、法家其实谁也没有淘汰谁,在很多朝代,实际上呈现出“半法半儒”、“儒表法里”的景象。

汉娜・阿伦特在《极权主义的起源》中总结了极权主义的三个特征,即“组织上国际化、意识形态全面化、政治抱负全球化”

在诸国中,齐国是食盐、冶铁以及丝绸的输出国,是税率最低的自由贸易区,是粮食产销最稳定的国家。

\subsubsection{历代变法}
\paragraph{管仲}

管仲是公元前七世纪的凯恩斯,颇有海洋文明的贸易思维,他本人就是商人出身。变法的内容:
\begin{itemize*}
	\item “士农工商”分离
	\item 放活微观,管制宏观
	\item 盐铁专营
	\item 鼓励奢侈消费
	\item 以商止战,玩商战
\end{itemize*}

\paragraph{商鞅变法}

变法内容:
\begin{itemize*}
	\item 土改:废井田,开阡陌,全民农业化
	\item 打击商业:控制粮食买卖,土地国有,重税,户籍制度,禁止迁徙,取缔货币,以物易物。
	\item 打击贵族,推行军功爵制,全民“平等”
	\item 统一度量衡,推行郡县制
\end{itemize*}

\paragraph{汉武帝变法}

文景之治:放权让利,无为而治。后果:商人崛起,地方坐大,官商勾结。 
桑弘羊改革:
\begin{itemize*}
	\item 产业国营:铸钱,煮盐,冶铁,酿酒
	\item 流通改革:均输(统购统销)、平准(物价管制),构建计划经济
	\item 税收改革:算缗(向有产者一次性征收10\% 的财产税),告缗(挑动群众告发群众)
\end{itemize*}

\paragraph{王莽革新}

变法内容:
\begin{itemize*}
	\item 全面回复盐铁专营和均输、平准二法
	\item 计划经济
	\item 铸多种新币,币制武断混乱
	\item 恢复土地国有,平均分配给农民
	\item 禁止奴隶买卖
\end{itemize*}

\paragraph{唐前期}

措施:
\begin{itemize*}
	\item 科举制兴起,打击豪门世族
	\item 降低工商税
	\item 盐铁经营回归民间
	\item 精简吏治,降低官员数量,向富人征税供养官员
	\item 藩镇养兵
\end{itemize*}
安史之乱以后,刘晏以桑弘羊之法扩充中央财政,盐民种、政府统购统销,向富人征收资产税

\paragraph{王安石变法}

又一次国家主义尝试,失败以后,中央再也没有进行整体性的配套改革

\paragraph{明清}

闭关锁国。稳定压倒一切,也压垮一切。

朱元璋鼓励水稻和棉花纺织,中国人以小农生产的方式实现吃穿的自给自足,这是经济自我封闭的前提条件。

总结:\emph{中国历史的经济政策,都是在收和放的循环中交替进行。放任的政策促进工商业发展,但引起贫富分化、土地兼并和地方坐大,于是不得不“改革”,中央收权,打击地方势力,产业国营,提高税收(往往是提高富人财产税,薅富人羊毛),土地国有并试图“平均分配”,为穷人减税,这往往会打击经济自主性。}

虽然历次改革均号召土地国有平均分配,但在大多数时间里,中国土地是私有的,可以买卖,因此中国古代是“古典的市场经济”。

\paragraph{洋务运动}

由地方汉族实力派发起,缺乏顶层设计,中央政府承担后果却无获益,动力不强。

局限在几个工业上,缺少政治制度和下层配套的跟进。

缺少更广泛和深入的西方学习。

\paragraph{北洋政府}

极度自由,民营经济空前强大

地方自治,互不来往

\paragraph{国民政府}

官僚资本主义,国家资本主义,挤压民营资本。1945年后昏招频出,通货膨胀,失去底层支持。

\paragraph{改革开放}

1994年分税制前后,中央与地方力量的消长

\subsubsection{作者对当下和将来经济的看法}
\begin{description*}
	\item[前提] 统一文化(传统)是一切自由化的边界
	\item[两个永恒的主题] 分权与均富
	\item[三个最特殊的战场] 国有经济、土地和金融业
	\item[新势力] 互联网、NGO、企业家和自由知识分子
\end{description*}

\subsubsection{书评}
本书条理框架还是比较清楚简单的,至于是否准确还是由方家来说吧。

作者将一半的篇幅是介绍新中国成立以后的,特别是改革开放三十年来的改革路线,可以由此看出作者的立意:以史为鉴,为当下的政治经济改革建言献策。

叙述还是比较精练的,如果需要了解更多,需要去读更详细更专业的书籍。

我个人还是对80年代以后的改革路线感兴趣,特别是90年代的过程。因此朱容基的传记可以看看。