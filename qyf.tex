\subsection{《去依附:中国化解第一次经济危机的真实经验》}

标签: 温铁军  \ 三农  \ 土改 \ 民国 \ 经济危机

作者:董筱丹 \ 温铁军

本书是《八次危机》的姊妹篇,是对前者的增补。《八次危机》中的每一次危机,是新中国一五期间形成的,而本书将目光聚集在1949-1952年间的国内经济、金融和财政波动,提出其属于“第零次”危机,但这次经济危机的原因要上溯到民国在30年代提出的法币改革。

萨米尔·阿明提出第三世界的“依附理论”,温铁军团队在此基础上指出:任何发展中国家只要通过“非暴力”谈判形成国家主权,从制度经济学理论看,就是一个与宗主国的“交易过程”,由于双方信息不对称和实力不对称,导致收益格局结构性失衡,往往是被殖民国家出让经济主权以获取原宗主国对其独立的政治认可,导致了“主权负外部性”,不得不在经济和政治上继续领附于原宗主国,后续进入发展陷阱。

中国如何“去依附”?作者认为,中国是依靠了“土改红利”(占人口88\% 的农民在参加土地革命之中构建起了史无前例的、严密的农村组织体系,这是宏观上稳定城市通胀的主要工具)和“政府理性”,化解了民国以来长期的经济问题。本书的主要论点:
\begin{itemize*}
    \item 中国在20世纪二三十年代的十年黄金增长期,得益于西方当时还没有兴起全球金融化浪潮,依靠不同币制规避西方经济危机。当时中国的银本位制在国际市场上属于法定货币贬值,避免了经济危机向中国传导。1930-1931年,中国出现了近半个世纪难得的国际收支盈余,1931年进口白银4545万两。但1931年英、日、印相继放弃金本位制,美国于1933年也放弃了金本位制,使得银本位制下的中国出现货币升值,在世界主要国家渐渐迈出大萧条时,中国却进入了衰退。
    \item 1934年6月美国实施《白银收购法案》,大量收购国际市场上的白银,导致国际白银价格持续走高,中国发生输入型通货紧缩,农村尤其严重,农村经济体大量破产。
    \item 为应对经济危机,民国政府进行法币改革,将银本位制改为现代化纸币制度。但是法币绑定美元,是一种外汇本位币制,具有严重的外部依赖性:如果出口少,则外汇不足,难以调控国内经济和货币政策;如果出口过多,则相当于将本国劳动创造的财富以货币税的形式让于外币发行国,国内经济对外依存度过高,容易导致输入型经济危机。现在中国的情况属于后者,而民国的情况属于前者,并且由于战争,需要外汇购买军火,外汇不足,对国内的调控更加乏力。
    \item 同期的解放区,则部分实行“物资本位制”。
    \item 国民党于1935年进行\emph{依附型币制改革},导致之后十多年的高通胀,制造业利润跟不上通胀,迫使产业资本家析出资本进行投机领域(工商业96\% 的资本用于投机,脱实向虚)。1949年国民党败退台湾,带走大量黄金,客观上杜绝了新中国实行黄金本位制的可能。
    \item 共产党接手的烂摊子下,战争和新政权的稳定需要财政开支,延续了更为严重的通胀,需要发行新货币稳定经济。依靠长期战争形成的集中体制,形成巨大的中央权威,并实现了大规模的跨区域物资调运能力,以此作为后盾和信用基础发行货币。新发货币进行“三折实”,即政府发行折实公债、公职人员改放折实工资、银行举办折实储蓄。同时实业无法获利,因此客观上需要进行国家层面的动员和统购统销的国有经济。
    \item 因此新中国必须重建经济主权和金融主权。
    \item 土改下农民需要扩大土地再生产,因此节衣缩食向国家交公粮,并将新币的40\% 用于储蓄,使得这些货币没有进入流通领域,没有造成进一步的通胀,但阻碍了城乡之间的工业品与农产品的交换。当然,这也加快了农村经济的货币化和两极分化,随之引发合作社运动及相关争议。
    \item 私人资本囤积物资,试图哄抬物价,延续民国以来的投机行为。
    \item 新政权加大赋税(均田不免赋,什一税对于农民来说比较容易接受,20\% 则属于高税赋),获得大量物资,有策略地投放到城市,打击私人资本的投机行为(“三白”战役,即白米、白面、白布),使投机资本落败。资本生产的假冒伪劣、质次价高商品在朝鲜战争中表现出来,于是政府开始“三反”“五反”。这属于一种“政府理性”,进行了“逆周期调节”。
    \item 高通胀之后必然是萧条,随着通胀结束,城市开始出现产品积压、商品没有销路、市场成交量远低于上市量的现象,1950年春夏之交出现超过20\% 的失业高峰,农村的工业产品却难以推广使用,这是由于工农业之间具有剪刀差,农民由于历史经验倾向于自给自足,因此形成了小农经济与城市资本之间的矛盾,客观上需要国家力量介入。
    \item 改革开放初期债务危机,事实上也是经过再一次给农民分地(家庭责任承包制),通过三农领域而再次化解。
\end{itemize*}

这本书延续了《八次危机》的分析风格,不是意识形态(包括官方和新自由主义)的预设立场,而是科学的中性的分析。这本书关注的是建国初期化解经济危机的手段,有利于认识新中国建立这一历史过程。

评分:5/5。