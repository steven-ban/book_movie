\subsection{《说中国》}

作者:许倬云

书名为《说中国》,实际上论述的是中国这一文化共同体的发展变化。本书是许倬云结合人类学、考古成果和中国基本历史(包括政治制度演变、经济重心转移等)对中国历史的梳理,内容不多,篇幅不长,却值得一读。

中国不仅仅是一个地理概念,更重要的是一个文化概念。在中国这片土地上,自新石器时代起,地理范围和文化内涵就在不断变化,两者是一个相互牵制、相互影响的过程。史前时代华夏大地的不同部落,经过长期的战争、迁徙和融合,不断扩大“华夏”这一范围的边界,并最终形成现在的广大农业区。这一农业区的边界,是牧业区(北部)或地理阻隔。于是在这样一个土地上,华夏文明生根发芽,不曾中断。并且,华夏文明与周边的文明进行碰撞、交融,但没有失去自己的主体性和延续性,这在人类文明史上是独一无二的。

在这种流变中,文化(统一的文字、历史记忆、儒家学说等)与经济相互牵制将广大地域和不同文化习俗捏合在一起。一旦有了外族入侵(如三国后期),联结在一起的各个区域之间由于有经济上的依赖性,因此无法分割开来,使得割据一方成为不可能。汉朝奠定了华夏文明的疆土和文化认同,唐朝有了进一步发展,这在许倬云看来都是十分耀眼的,是中华的“正统”。宋代并不占有全部的“天下”,而元明清三代的奴化压迫较为严重,在许倬云看来失去了华夏的气度。这里许倬云明显受到了日本人近代的影响,并且有一定的台独倾向,将元明清(特别是明朝的改土归流)与殖民等同,其史观令人惊讶。沿着他的说法,日本殖民中国的正当性就有了,台湾独立的合法性也有了,这是十分错误了。在我看来,元明清虽然有其弊端,但整体上都是华夏文化的传承,虽然后来越来越封闭,但这没有背离华夏文明的传统,因此对其贡献应当予以承认。

因此,这本书前半部值得一看,后半部便失了水准,读者应当对其内容选择性和批判性吸收。

评分:3/5。