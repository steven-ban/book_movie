\subsection{《说中国》}

作者:许倬云

\subsubsection{标注}

1. “中国”这个观念维系力量有三,一是经济网络,二是政治精英,三是书写文字。

2. 特别强调不同的生产方式和生活方式,如何使不同的族群与文化逐渐杂糅、融合与交错。

3. 如果说《万古江河》重点在讨论中国的“历史”和“文化”,《我者与他者》重点在讨论历史与文化中的中外关系,那么这本《华夏论述》重点就是在讨论历史与文化中“中国”之变动。

4. 欧洲人从新大陆取得巨量白银,其中至少有三分之一甚至一半流入中国。

5. 我(作者)将“中国”看作是一个复杂体系的共同体。

6. 我习惯上使用系统论的方法分析历史现象,尤其所谓“大历史”的研究方法,不能从单独的事件着眼,必须从各种现象的交互作用来观察整体的变化。

7. 蒙古和满清,两次征服中国全部地区,在中国历史上留下深刻的烙印:最沉重之影响,应当是完全倚仗暴力压制的统治形态。于是,中国传统的“天命”观念,及“天命”应建立在“民视”、“民听”基础之上的相对性,经过上述全盘暴力镇压的残酷现实,竟从此再不能支持百姓对绝对皇权的抵抗。

8. 定期察举,等于将全国人不断地周转,不使任何地方独占权力,也使全国的信息因为人才流转而流转,全国的政策不至于有地方性的偏差。

9. 华”是华美,“夏”是伟大——华美而伟大的文化,就是“华夏”,

10. 台湾原居民的语言是南岛语系的源头。由此可以推知,南岛语系的祖源其实就是包括所谓“百越”在内的东南人群。

13. 离现在一万年前左右,黄牛第一次被驯养为家畜。

\subsubsection{书评}

书名为《说中国》,实际上论述的是中国这一文化共同体的发展变化。本书是许倬云结合人类学、考古成果和中国基本历史(包括政治制度演变、经济重心转移等)对中国历史的梳理,内容不多,篇幅不长,却值得一读。

中国不仅仅是一个地理概念,更重要的是一个文化概念。在中国这片土地上,自新石器时代起,地理范围和文化内涵就在不断变化,两者是一个相互牵制、相互影响的过程。史前时代华夏大地的不同部落,经过长期的战争、迁徙和融合,不断扩大“华夏”这一范围的边界,并最终形成现在的广大农业区。这一农业区的边界,是牧业区(北部)或地理阻隔。于是在这样一个土地上,华夏文明生根发芽,不曾中断。并且,华夏文明与周边的文明进行碰撞、交融,但没有失去自己的主体性和延续性,这在人类文明史上是独一无二的。

在这种流变中,文化(统一的文字、历史记忆、儒家学说等)与经济相互牵制将广大地域和不同文化习俗捏合在一起。一旦有了外族入侵(如三国后期),联结在一起的各个区域之间由于有经济上的依赖性,因此无法分割开来,使得割据一方成为不可能。汉朝奠定了华夏文明的疆土和文化认同,唐朝有了进一步发展,这在许倬云看来都是十分耀眼的,是中华的“正统”。宋代并不占有全部的“天下”,而元明清三代的奴化压迫较为严重,在许倬云看来失去了华夏的气度。这里许倬云明显受到了日本人近代的影响,并且有一定的台独倾向,将元明清(特别是明朝的改土归流)与殖民等同,其史观令人惊讶。沿着他的说法,日本殖民中国的正当性就有了,台湾独立的合法性也有了,这是十分错误了。在我看来,元明清虽然有其弊端,但整体上都是华夏文化的传承,虽然后来越来越封闭,但这没有背离华夏文明的传统,因此对其贡献应当予以承认。

因此,这本书前半部值得一看,后半部便失了水准,读者应当对其内容选择性和批判性吸收。

评分:3/5。