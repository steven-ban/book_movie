\subsection{Wonder Woman}

女主很美,各种扮相都很美,但这是本片唯一区别于其他电影的地方了。
神奇女侠们生活的地方叫天堂岛,是个女儿国,国内没有男人,女侠们个个身经百战刀枪不入,虽然只用冷兵器但战斗力超过了一战时的热兵器水平。主角是个天选之女,遇到逃跑的男主,救下男主,和男主一起上一战时的德国前线。有人说这是女权主义电影,然而本片只能算是女性视角,在没有男人的国度里长大的女主进入了以性别划分身份的一战时的西方世界,心态是好奇的。
情节上没有什么剧烈的冲突,故事推进就像是温吞水。
电影中对一战时的伤亡描述很多,对战争的思考深度却很浅,没有超出“战争摧毁双方的生命”。
正派和反派都很刻板印象,没有什么深度,基本就是幼儿园级的打架水平。
豆瓣有7.4的高分,显然是刷成了这样,其实如果是中国人拍的,五分顶天了。

当然了,我上面是看了一半时写的,大家看看就好。
后半段随着剧情的深入,特别是真的反派BOSS的出现,意义升华。善与恶并非分别存在于不同的个体,人类本身就是善与恶的混合,是为善还是做恶取决于自身的选择,而神也好,超级英雄也好,代替不了人的选择。
女主和反派BOSS属于神仙打架,和希腊神话有点渊源。
但是上述情节和主体情节还是脱节的,而且轻轻松松打败BOSS也比较草率。
那么就给七分吧。