\subsection{《宋明儒学论》}

作者:陈来

本书的作者是中国哲学研究学者陈来,1976年毕业于中南矿冶学院(中南大学前身)地质系,1985年北京大学哲学博士毕业(由工转文),主要研究方向为先秦儒学、宋明清理学、现代研究哲学。

本书的内容为陈来先生的六篇论文,为:
\begin{enumerate*}
	\item 《李延平与朱晦庵》,主要论述朱子与延平授受渊源与思想关系。
	\item 《朱熹淳熙初年的心说之辩》,研究淳熙初年朱熹与同时学者吕祖俭、石礅、方士繇、吴翌、游九言、何镐等人以书礼相往来,辨析“心”之学说,并借此讨论朱熹对“心”的看法。
	\item 《王阳明哲学的心物论》,介绍王阳明哲学中的心物概念关系。
	\item 《王船山晚年的思想宗旨》,介绍王阳明晚年的思想变化。
	\item 《宋明儒学与神秘主义》,介绍宋代和明代理学家们对打坐达到物我合一主观体验的论点。
	\item 《王阳明与阳明洞》,考证阳明洞的地理位置,应为会稽山阳明洞(阳明结庐之侧)。
\end{enumerate*}

程朱理学与阳明心学是儒学的重大发展,对后世影响很深,其强调的个人修养至今仍对人有所启示。作为传统哲学的大课题,一本书六篇论文是难以详述的,仅是从几个方面来探究这两门学说的几个课题。对于不了解的读者,本书阅读时肯定如坠五里雾中,不知所云,但本书作者陈来治学严谨,论述严密,从中可窥见他研究之深刻。读者可先读中国哲学思想史,再略读程朱及王陆入门通论,再涉猎本书,方得其中滋味与乾坤。

评分:4/5。