\subsection{萨利机长}

标签: 美国电影 \  灾难 \  英雄 \  空难

英文名:Sully

2009年1月15日,一架全美航空的客机正常起飞,但还没到达预定高度,就被一群飞鸟迎面撞上。机长Sully经过检查,发现飞机两侧的发动机都失灵了,备用系统也损坏,这时要么飞回机场或找另外一个机场降落,要么……正如他随后所做的,经过35秒钟的分析,他和副飞升一起将飞机降落到了哈德逊河。机组人员经过指挥,让乘客系好了安全带并放低身体,飞机成功降落,救援也快速行动,仅使用了24分钟,机场上的全部乘客和机组人员就被转移到了救援船上。

这是一次成功的降落,机长Sully也随后被各大媒体报道,成为民众口中的英雄。然而,航空公司、保险公司和飞机公司联合交通运输部组成的委员会对这件事进行了调查,认为机长当时应当将飞机飞回机场或找附近的机场降落,在哈德逊河降落的决定严重威胁到了乘客的安全。委员会对当时的情况进行了计算机模拟,认为当时绝对有条件到机场降落。

机长Sully即将接受问询,他的生活也变得紧张。他夜不能寐,常常做飞机失事的噩梦。他在脑海中一遍遍回忆当时的情形,推演各种可能的情况。他回忆起自己当上机长时师傅的教导,以及自己曾经处理飞机失事的经验。他去公共场所,所有认出他的人都把他当作英雄,而他则有口难言。

在问询会上,看到模拟结果时,他站出来为自己辩护:计算机模拟虽然精确,但忽略了机长是个人这样一个因素。在当时的情况下,机长需要考虑各种情况,需要有时间进行决策,而这需要一段时间,计算机模拟仅就一旦出现状况可以返航,而没有计入机长决策的环节;当计入这个环节时,飞机已没有时间降落到机场了。现场的模拟支持了他的结论,听证会上的所有人也随之接受了他的意见。

英雄事迹需要被质疑,当质疑被事实消灭后,英雄更显得可贵。

本片一开始即是降落成功后,主线是Sully被接受质询的过程,中间穿插了飞机失事的过程以及Sully的回忆。

评分:8/10。
