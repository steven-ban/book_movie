\subsection{《羊毛战记》}

\subsubsection{情节}
无非是个反乌托邦故事,不算是科幻(算不算蒸汽朋克?)。

未来的美国人生活在50个地堡里,
\begin{quotation}
整个地堡分为三段楼层,每一段四十八楼
\end{quotation}
里面有农业区、工业区,有保安官,有首长,但最核心的是有一个资讯区,他们有三层楼的空间,有很多服务器,用来保存内部历史信息。奇怪的是人们通讯大多通过邮差来上下楼梯送信,而不是电子邮件。电子邮件太贵,其目的也是为了不让人喜欢通信。

地堡有个奇怪的规定:不允许人们出去,谁一旦有了这种想法,当局立刻将其判处死刑,并拉出去清洗镜头。外面的世界是荒漠和海市蜃楼一般的城市,但城市都是断壁颓垣。出去擦镜头的人都需要穿上厚厚的防护衣,头盔里有一个显示屏可以“看到”外部影像(当然,影像是合成的“春意盎然”的假象)。他们需要使用“羊毛”来擦洗镜头。这些人每天在地堡里看到外部的影像是荒凉的(但却是真实的),而在头盔里看到的却是有生机的,因此他们乖乖擦完镜头后立刻迫不及待想到翻过镜头看不到的山丘去生机勃勃的城市。但资讯区的人在防护衣上做了手脚,密封胶带是劣质的,很容易漏气,因此这些人没走几步都会因吸入毒气而死去。

对此人们当然不会坐视不管,
\begin{quotation}
平均大概每隔二十年就会有一次大规模的暴动。
\end{quotation}
但每次暴动之后资讯区的人都会删除所有历史资料,因此后来的人们难以获知这些秘密。

地堡之间会通过资讯区的头头之间联络,以获悉哪些地堡会废弃。资讯区头头会有一个秘密房间,这个房间有所有人类以前的历史和五十个地堡的地理位置。地堡实施学徒制,资讯区头头也会到一定时间选择一个“可靠”的人来做学徒(竟然不实行我党的历史审查和档案制度,真是二!),并把一本“地堡运行指南”让他读,
\begin{quotation}
有好几个章节是在解释什么叫作“说服群众”,什么叫作“思想控制”,什么叫作“恐吓式教养的效果”,另外还有一大堆关于人口控制的图表……
\end{quotation}
(从上文就可以看出来作者根本就对共产党和极权主义缺乏了解,这些手段相对于真实的历史真是小儿科,根本就不管用。好好看看苏联和我党历史吧少年!)

地堡竟然实施选举制而非威权和极权体制,这让我很惊讶。这快死了还搞个啥民主?正如作者所说,
\begin{quotation}
“如果你不想失去孩子,那你就必须对他们残酷一点。”
\end{quotation}
所以搞点血腥政治,搞点互相检举揭发啥的很必要啊!

第一章霍斯顿夫妻俩前仆后继出去擦镜头的情节很好看,紧凑合理,扣人心弦。

第二章里詹丝的心理活动就差远了,很难想象这种随波逐流缺乏干练和决断力的思考会来自一个首长,会来自一个老年人。这让人觉得,地堡并不需要一个首长,她就像是一个橡皮图章,缺乏一个领导人应该有的那种权威和大权在握。

自从茱丽叶被送出去之后,这部小说对读者就没有什么秘密了,完全成了拖沓的拼凑故事。

茱丽叶和卢卡斯的生死之恋莫名其妙。

暴动就是走过场,没啥亮点。

十七号地堡里的故事也很乏味。

整个故事架构构造还不错,前期故事很有张力,但bug太多,无法解释。

坑爹的小说,只配两星。据说是美国去年销量很高的小说,对此我只能说,读书一定要远离所谓畅销书。

\subsubsection{白瑞德}
为什么单说这个人?这个人的行为完全反映了作者坑爹的人设。卢卡斯被任命为学徒(其实就是指定接班人)前,这个人的形象完全是反面的狰狞的,害死茱利叶前男友、害死詹丝、强势夺权、试图害死茱利叶等等和后面的反转虽说不上突兀,但反派突然正面化就使全书缺乏一个刺激的愿景,读者开始无所适从。白瑞德对卢卡斯谆谆教导颇有长者风范,系万民生死于一身让充满好感,但手下反水放茱利叶十八号地堡立刻被压权,丝毫没有反手之力,虎头蛇尾。

\subsubsection{茱利叶}
主角身上有强烈的光环,鬼挡杀鬼佛挡杀佛,大难不死,福寿安康。作者一介绍茱利叶,特别是詹丝一死,读者立刻会知道茱利叶才主角,肯定死不了(KINDKE的X-RAY里能看出来)。茱利叶能修发电机能修机器,能潜水,意志力坚强,体力旺盛,都不像个女人。

坑爹的是茱利叶卢卡斯间的爱情,看了一次星星突然就发生了,用电话聊了几次天突然就成生死至交了,茱利叶一回十八号地堡几乎就要结婚生子了。前景铺垫太少,后面就太突兀。

\subsubsection{卢卡斯}
这人没能力没皮相没魄力,却突然被指定为地堡实际的继承人,突然让白富美茱利叶一见倾心抱得美人归,突然就策动政变端掉了自己老板,名利双收,比主角的光环都亮。

\subsubsection{老沃克}
毫无性格,在机电部却会折腾无线电,竟然能听到别的地堡的信号。隐忍多年救下出去擦镜头的茱利叶,按说挺有智慧,眼睁睁看着机电区的人出去暴动(送死)。