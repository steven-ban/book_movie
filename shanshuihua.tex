\subsection{《山水画是解读中国文化的密码》}

这本书是广西师范大学“人文讲谈录”中的一部分,收录了青年画家、作家韦羲的《照夜白》一节,陈丹青演讲稿《陌生的经验》一节,内容是两位画家对中国山水画的一些观点,十分短小。我于山水画一窍不通,因此无法评论内容质量,但韦羲文风做作,陈丹青流里流气,都不是我喜欢的类型。

不过,书里陈丹青一句“我们尊崇一流,忽略二流,最终恰好是困居三流,因为,经由二流去一流的路,断了”我倒是心有戚戚。大众往往只是关注行业顶尖的创作,然而这仅仅更多是依靠名气而非行业眼光,对于当世作品就是流于一时名气,对于历史有定论的作品则流于表面。虽然大众不大可能真的去某些行业里做一个或半个专家,但也应当提高艺术类作品的鉴赏能力,全社会形成学习讨论的氛围,显然当前社会这是比较困难的。即使不讨论一流二流的定义总是,所谓二流,往往是开创了某些一流的先声,其意义未必比一流作品本身更低。