\subsection{自杀小队}
DC的漫改电影,几位“乱序中立”的超人们为了老婆孩子去阻止邪恶巫师毁灭世界,场面宏大,特效逼真,典型的美式爆米花电影。

出彩的是小丑女,把不良少女演绎得惟妙惟肖,与小丑间的爱情可看性也很强。

黑人女领导既蠢且死硬,让人讨厌。美国种族平等的政治正确让人讨厌。

满分10分的话,值6分。

\subsection{肉与灵}

一个屠宰厂,来了一个性格孤僻、行事刻板、能记清自己哪天来月经和得水痘的怪异女人,她长相清秀,严格照章分类牛肉品级。她与屠宰厂主管(是个瘦削的老头)做着同一个梦,梦中有雪地、树木和溪水,公鹿和母鹿在相互交流、注视和觅食。

厂里的催情剂丢了,心理女医生逐一访谈厂里的员工,问梦到了什么,此时发现女人和主管的梦境不谋而合。两人随之开始交往流。随着交流的深入,两人逐渐突破了自己的心理防线,逐渐放弃了自己的交流障碍。特别是女人,从刻板的生活直到会听恋爱音乐,脸上也逐渐有了笑容。当听到男方不想和自己约会时(其实就是不让她去他家而已),就割腕自杀,但中途又接到男方电话,于是两人约会,开始了恋爱生活。

这是一部很唯美的爱情电影,高光下女人很美,大眼睛黄头发,以及白色纯美的皮肤。剧情推进很慢,但很好地诠释了人们恋爱时从相识到渐渐将对方放进心上的过程,以及爱情中受挫时的惨烈。这部电影的男女主人公都是社交障碍者,所以电影其实大大强化了上述过程。

满分十分的话,本片值7分,值得一看。

\subsection{绣春刀2·修罗战场}

商业片就应该是这样,剧本经得起打磨,能击中受众的心,表演合格偏上。

\subsection{敦刻尔克}
诺兰在2017年的大片,讲述了敦刻尔克大撤退时法英普通士兵和英国平民的遭遇。

对这段历史不熟悉,也不是很感兴趣。三十三万人的转移确实很宏大,但相比于法国的渣战斗力和英国政府前期的自私,这点规模的战略转移根本不是主要问题。

\subsection{白百合}

日活的经典情色片翻拍,同性之爱的电影。陶艺家登纪子有个住家的学生小春,小春对登纪子特别敬仰和热爱,两个保持着身体关系。后来,登纪子朋友的儿子来到了陶艺教室,登纪子立刻爱上了这个男人,小春的妒火立刻中烧,请求登纪子不要和男人交往,但被登纪子拒绝。男人的女友找来男人,但男人拒绝和她回去。登纪子撞见小春和男人接吻(其实只是在解释和引诱),恼羞成怒辱骂小春,并让小春和男人在自己面前做爱,两人就范。做的正激烈,男人的女友回来,看见这一幕,激动的拿起水果刀,争执中刺伤小春。小春出走,几个月后回来,这时登纪子没了小春的陪伴已经失魂落魄,陶艺教室也关闭,两人再度共赴云雨,之后小春飘然而去。

画面比较美,不错的百合画面。

\subsection{英伦对决}
成龙饰演的关是伦敦一家餐馆的店主,只有一个女儿,他的妻子和另一个女儿早年在泰国被杀,唯一的女儿眼睁睁死在恐怖分子手里,被炸死,于是关独自走上复仇之路,从副部长入手解开了爱尔兰恐怖分子利用副部长情报开展恐怖袭击的阴谋……副部长的妻子的哥哥以前是恐怖分子,妻子被仇恨蒙蔽出卖了情报,而情妇才是真正的袭击者。爱尔兰恐怖组织只是背锅的。

说实话情节太乱,人名太对也记不住。

满分五分的话,给三分。

\subsection{出租车司机}
全片2小时10分钟,片长并不算长,但内容相当丰富,节奏明快(“明”是指全片亮丽的色调和利落和高潮迭起层层推进的叙事节奏),无愧于韩国本土要房和人气第一的殊荣。

金四喜是首尔一个普通的出租车司机,妻子早死,留下妻子生前坚持买的跑了60万公里的出租车以及一个11岁的女儿。他们住在破旧的租房里,生活拮据,欠下10万块房租,女儿只能将普通鞋子当拖鞋穿。金四喜有着“底层”人民常见的斤斤计较,但也有着善良人民所有的“不忍”之心。别的出租车司机不会拉孕妇去医院,他会,而且乘客夫妇忘记带钱也没有追着人家要钱(其实也未必要得回来)。他对于走上街头游行的大学生的态度是“给自己拉客带来麻烦”以及“上大学不好好学习却不务正业”。为了欠下十万块钱的房租,他顶替一个工友去光州拉一个外国记者往返。彼时正是1980年5月中下旬韩国光州事件正盛之时,学生走上街头反抗军政府的专制统治,而军政府残忍地对学生痛下杀手。他为了十万块钱,冒着危险带着外国记者进了光州,随后就想开溜,后来遇见了一个爱唱歌的大学生。他目睹了群众流行以及军民冲突,但还是不想掺和进去。后来,出租车坏掉,金四喜、大学生和德国记者住进了光州出租车司机家里,几人更加熟悉,但金四喜依然挂念着自己的女儿,第二天一早就溜出来回家。没出光州,他看到了韩国电视上扭曲事实的报道,联想到光州人民的遭遇和几位新朋友的处境,毅然返回光州并决定带德国记者回首尔。在当天晚上,他们受到军政府的追捕,危急之下大学生舍身取义拖住官员献出了生命,第二天,金四喜和德国司机在医院看到了大学生的尸体。他感受到了自己的愤怒,决心和光州人民站在一条战线上。他同其他出租车司机一起,在枪林弹雨中救出垂危的人民,并带着德国司机冲出光州回到首尔。很快,光州事件的真相被曝光,成为韩国民主化进程的重要事件,德国记者也因为报道该事件获得了韩国的新闻大奖。在颁奖典礼上,德国记者说希望再次见到金四喜,但直到他去世,也没有再见到他一面。

作为历史剧情片,本片剧情紧凑,没有一处闲笔,同时演员演技精湛,几乎没有短板。如果非要说有短板的话,可能反派人物比较脸谱化,均是穷凶极恶的专制分子。回首尔时的哨卡那儿,一个军官发现了金四喜的伪装,但依然将他们放行,这算是调和了这种一致性的脸谱化了,也说明专制的军政府不得人心,即使是手下也依然不全是冷酷的杀人机器。

本片的民主化事件,与中国89年的事件较为相像,其实也没有相像,只是类似而已,但豆瓣上的词条被下架,知乎上也搜不到参与的人主题,只有在其他网站上还保留着零星的影评。中国政府声称的“文化自信”“制度自信”此处曝露出虚假和软弱。其实即使民众看了本片,恐怕也不会产生对89事件的过多联想,中国目前文化政策的闭关锁国与自欺欺人,可见一斑\footnote{这无意于批判中国整体的政治制度,这是我2017年的感想,从现在的2021年看来,中国的文化审查部门,确实长期以来在泛意识形态领域比较保守,处于守势,希望这种现象以后会有比较大的改观。}。从电影质量来说,中国近几年的电影,恐怕没有一部可以和这部相比,中国的电影从业者,真该羡慕韩国同行一辈子。

\subsection{末化皇帝}
本片用将近四个小时讲述了溥仪一生的重大时刻:离家受位、登基、宫中学习、结婚、逐出紫禁城到天津、日本支持在东北称帝、狱中十年改造、文革中去世……他的一生可谓跌宕起伏,处处被人利用引诱,被囚禁毫无自由,没有子嗣。他一直在挣脱束缚,但只有最后魔幻地变成蟋蟀才真正解脱。

电影配乐大气,画面自然,演技更是朴实深沉,无愧于大量手笔奥斯卡最佳电影的殊荣。

\subsection{猩球崛起3}

讲的是猩猩,其实还是后启示录的人。

情节:凯撒的妻子孩子被上校杀害,小儿子被掳走,于是凯撒走上复仇之路。猩猩被上校关在武器库做苦工,做完就会被杀掉。凯撒联合猩猩展开了一场越狱和逃亡。最后,凯撒溘然长逝。

特效惊人,动作捕捉惟妙惟肖,表情栩栩如生,技术实力相当高。

剧本流于套路,槽点颇多。比如那么多猩猩关在围栏竟然没有一个守卫日夜看守,外面的守卫也形同虚设,让哑巴小女孩大摇大摆进进出出,不被灭才怪。

因此看完之后,我也没有再去看第二部的冲动了。

\subsection{非正式特工}

上来就黑中国的电信诈骗?开玩笑吧?电信诈骗台湾才是大陆的老师,为啥不黑台湾?韩国电影的思想性其实是不足的,看似批判社会黑暗面,其实是很标准的类型电影,这种民族主义情绪流露的电影更是扯淡。

\subsection{秒速五厘米}

第一话《樱之抄》里电车和纷纷扬扬的大雪让人印象深刻:雪的飞扬、静止飘落。贵树与明理小时候就互相通信

第二话《cosmonaut》里的风和草、虫鸣,人生如纸飞机般在风中飘零。有人喜欢上了贵树,想表白却发现贵树的温柔不是对她,于是暗暗留在心里

第三话《秒速五厘米》里,明理要结婚了,她通知了贵树。但两个人一千次的短信,也只能将距离拉近一厘米。即使身处一个城市(东京),即使常常擦肩而过,但仍然隔着一个天涯的距离。

环境声效很逼真,情绪是新海诚很标准的身在东京大都市的飘零感。
