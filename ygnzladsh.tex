
\subsection{《一个女子恋爱的时候》}
1.富家女父亲猝逝,欠下一屁股债,昔日父亲好友垂涎美色,先是温情款款让她来家居住,后来藏掖不住威逼利诱不还钱就必须嫁给她。富家女愤然离开,一直瞒着未婚夫,当掉父亲遗产住在小旅馆,找中介投资试图以三百块一年内赚十万还债(牛市冲天),发现中介也是图谋美色,又愤然离开当游轮女主人,到古巴见到父亲又一好友,回来后没钱还债,和未婚夫的感情眼看将要破解,正要被迫嫁人突然父亲好友拜访说你父亲有股份给你留了十五万,后来几位蜀黍当着坏叔叔的面揭穿了阴谋,钱还是富家女的。富家女喜见未婚夫并成婚,皆大欢喜喜大普奔。按情节就一不入流小说。

2.教训:(1)男人要慎重交友,有人对你好不一定是真的好,还有可能看上了你女儿。(2)股市有风险,投资需谨慎,一不小心就可能把女儿卖了。(3)女孩纸要谨防各种蜀黍。(4)奋斗什么的看起来很燃很鸡血,但最后的钱,还是亲爹的。

3.邹韬奋翻译不错,没有翻译腔,流畅自然,撮其大意,前几回的译后注有个人体会挺好看。

4.译后记:会说话的人能轻声讲重话,不会说话的人一出口便闹。诚哉斯言!

5.丁恩阴险,但并不猥琐,起码给了富家女贞丽起码的尊重。如果本故事走暗黑系套路,那么贞丽将死无葬身之地。

6.“在此僵局之下,尼尔珠莉和她自己三人皆感着苦痛,与其如此,不如让珠莉与尼尔能成眷属,自己宁愿牺牲;三人同苦,不如让两人快乐而只留一人痛苦。她想到这样的意境,妒意为之消除殆尽,心神反为之一爽。”从来对这种女性对情敌谦让的心态无感,如刘若英唱那句“看着她走向你,那幅画面真美丽;如果我哭泣,也是因为欢喜”,太假,太做作。

7.译者言“西谚谓“天助自助者”,我请改一字,说“人助自助者”,必先努力自助,而后人助乃得加入,若自暴自弃之徒,旁人见之,只有“爱莫能助”。”确实如此,虽然贞丽庸碌大半年啥也没干,但起码撑下来了。“机会诚若可遇而不可求,但只有最能努力者始能利用,则断然无疑。”诚哉斯言!
