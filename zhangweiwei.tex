\subsection{《中国触动》《中国震撼》《中国超越》}

这本三书相当于是一个三部曲,而主要内容也是作者张维为对中国发展过程中与世界、特别是与西方的关系的思考。三本书的内容其实重合很多,作者的观点也能仅从一本书中得出。《中国触动》写于2008年,之后在2012年左右有了部分修订,《中国震撼》写于2012年左右,《中国超越》写于2014年。总体上来看,这三本书虽然是一个时间上递进的关系,但考虑到作者的修订,其实内容比较雷同。

\subsubsection{作者其人}
作者张维为是翻译出身,80年代中期曾担任过邓小平等国家领导人的翻译,随同见证了他们与外国政要的交流,这属于第一手的资料,虽然都是公开资料,但由经历者本人谈起还是给人以更好的亲切感,这恐怕也是作者在论述中国模式的优点之一。

作者声称自己走过所有的西方国家,所有走访的国家超过100个,同时也在国外生活了20年,因此对世界整体上都有比较深入的了解。就这一点来看,由他来讲中国与西方国家、与第三世界国家之间的区别还是很有说服力的。毕竟不论人立场如何,在事实这一层面上是很难产生太大异议的,最多不过是双方强调的事实不同,但如果已经将事实摆出来,那么读者也可以有自己的思考,而不是被作者的观点带着走。

2019年初,东方卫视和观视频有一档节目很火,叫《这就是中国》,主讲人就是这位张维为,他现在是复旦大学中国研究院的院长,在节目中也是从各个维度讲解中国与各个国家的区别,讲解中国模式。从视频的内容来看,作者的观点一以贯之,和这三本书的内容基本上相同。作为一个教授,作者在节目里侃侃而谈,风度很好,我也是被节目吸引而关注这三本书的。

不过,在中国知网上搜索他发表的论文,发现基本上都是给各个报纸写的评论文章,没有学术性的研究。考虑到他80年代就开始工作,这么多年做到研究院的院长,不知道都做了哪些学术性的工作,是不是都是保密性的?这一点比较疑虑。从这三本书的内容上来看,作者主要是随性的论述,没有艰深和严谨的学术性内容,因此也看不出来他水平是真的水,还是不愿讲自己究竟做了哪些研究。

从立场来看,作者明显属于“左派”,与政府走得近,说的主要是政府的好话。他讲这些中国崛起或中国模式的内容,主要是2010年或更早,那时候中央层面还很少提这些概念,而在2012年习主政后,关于中国模式、理论自信道路自信以至于全球治理的话就提了很多,不知道作者是否对这些政策有贡献?如果有的话,那说明作者和主政者走得是相当近了。

\subsubsection{标注}

2. 美国民主党竞选的主要财源是华尔街和好莱坞,共和党主要财源是军火工业和其他传统工业,而奥巴马当选总统很大程度上靠的是华尔街的巨额捐款,随后就有奥巴马用几千亿纳税人的钱去“救市”。

3. 即用中国话语来论述中国和世界,只会随着中国的崛起,而越来越具有生命力,

4. 中国与西方,特别是美国之间的问题,不只是一个意识形态之争的简单问题,而是一个国家利益的问题,一个地缘政治、地缘经济的问题。

5. 大概是经历过“文革”的缘故,我对从意识形态出发而大大简化历史的论述总抱有深深的疑虑。

6. 邓小平做大的决策,总是首先把可能出现的负面影响估计透,估计到最坏的局面,然后再看中国有没有办法处理。如果可以,他就拍板了。

7. 法治是对付腐败最好的办法。

8. 美国的内部凝聚力极强,特别是在危机时刻。

9. (本书)一条主线贯穿其中:中国人要用自己的话语来解读中国和世界。

10. 中国人要用自己的话语来解读中国和世界。

11. 印度有大量的无地农民,约占印度农村人口的一半以上。

12. 八大中国理念是:实事求是、民生为大、整体思维、政府是必要的善、良政善治、得民心者得天下与选贤任能、兼收并蓄与推陈出新、和谐中道与和而不同。

13. 稳定优先不是回避或掩盖矛盾,而是通过稳定来创造条件,从而更加有效地解决矛盾。

14. 中国没有西方意义上的神学传统,中国今天的实践理性背后是中国文化的世俗性。

\subsubsection{主要观点}

这三本书主要是论述性的,不是学术性的,内容也都很浅显。概括一下,作者的主要观点如下:
\begin{itemize*}
	\item 中国是一个“文明型国家”,有着悠久的历史传统,政治上有很强的延续性,同时疆域辽阔,各地发展极不平衡,与单一的欧洲民族国家有着很大的区别,因此中国不能也没有必要照搬“西方模式”。
	\item 福山所谓的“历史终结论”是错误的,历史并没有终结,西方制度并非完美。
	\item 西方所谓的一人一票选举领导人的民主有着很大的缺陷,容易陷入纷争,严重者甚至要撕裂社会。这对于已经工业化完成的西方国家而言问题可能不大,但对于穷困的第三世界国家则是个重大的陷阱,在人民生活水平不高、中产阶级未壮大的国家实行民主,容易导致民粹,延误发展经济的机会,导致社会长期停滞不前。
	\item 西方所谓的人权缺少对人生命权的重视,因此西方社会特别是美国指责中国人权状况差枉顾中国几十年前消除贫困的巨大努力,是不客观的。
	\item 中国在2008年之后,经过改革开放三十年的努力,在发达版块整体上已经达到了西方平均水平甚至更好。
	\item 西方社会意识形态先行,为发展中国家以及前苏联国家开出的民主化药方在事实上都已经失败,导致这些国家经济和社会的全面下滑。相反,中国改革开放是在邓小平等人的高瞻远瞩下渐进式地改革,先经济后政治,努力发展经济水平,提高人民生活水平,敢于试验,成功后逐步推进,这比其他国家的结果好很多。
	\item 中国在产生领导人方面实行“选拔加选举”的方式,克服了单纯一人一票的盲目性和民粹化,选贤任能有着中国几千年前的智慧,超越了西式的对抗式民主。协商式民主可以使在野者的智慧同样体现到政策中,而不是为反对而反对。事实证明这种方式是适合中国的,其他发展中国家也可以借鉴。
\end{itemize*}

作者在论述过程中,穿插着他本人的经历、当地人的观点等,但明显这些话语都是选择性的,是为了作者宣扬自己的观点服务的。在这一点上,这三本书有着巨大的缺陷。虽然作者力图证明中国模式的正确,也会顾及中国存在的问题,但都是一笔带过,没有深入的论述。相比之下,秦晖在《共同的底线》中的论述就严密和精准得多,这也可以看出虽然作者有着较为广泛的观察和比较,但在学术化语言组织上与科班出身的学者还是有着巨大的差别,这种差别阻止了这三本书成为学术性较强的著作的可能,只能成为观点简单的铺陈。

当然,作者的观点本身在事实上是没有太大问题的。如果说十年前的2008年国内自由主义氛围较为浓厚的话,那么到了2019年的今天,即使是严格的自由主义者,也要面对民主制度和自由主义思源在全球的低迷和困境。无论是台湾、韩国的民主乱象,还是民主典范的美国重新走向孤立,都受困于民主制度的民粹化倾向。相反,2008年经济危机以来各国经济普遍受到不同程度的冲击,对经济全球化失去信心,逐渐走向孤立,无论是德国、英国还是美国,都是保守主义甚至是民族主义抬头。《全球通史》的作者拉赫利也承认自由主义受资本和技术的发展而变得岌岌可危。这种背景下,本来就缺少自由主义传统的中国社会渐渐接受自身政治传统也不足为奇了。这些年中国的发展还是不错的,官方也有意为地培养国家意识和民族意识,对自由主义观点进行了一定程度的禁言,民众无论是从自身还是从环境出发都更加地保守,因此接受张维为在十年前就坚持这种左派的思想就不足为奇了。可以预测的是,将来的一段时间内,这样的思潮只会更加流行。

从最低的意义上来说,这三本书至少提供了一种中国官方的视角来解释中国这些年的发展,并且给出一种区别于西方的价值观。其实,西方那些“中国崩溃论”“中国威胁论”的论述方式,并不比张维为更加高明,也是选择性地拿一些事实来“证明”自己的观点。在这一点上,中国完全可以把张维为的这些观点包装一下输出到其他国家,借助自己的媒体和其他文化产品与西方价值观相抗衡。毕竟,在“一带一路”的战略构思下,仅仅像以前那样最韬光养晦是不够的,还要选择性地为自己的战略利益选择盟友。

这三本书里的观点有赖于中国模式的真正成功,也就是中国的真正崛起。因此,即使这些观点有需要丰富的地方(张维为一直在强调这一点)甚至有错误的地方,类似的论述和研究肯定会越来越多,从不同的侧面来支撑中国崛起道路上的经验。如书封面所说,张维为倡议的“中国人,你要自信”,会越来越多地成为中国人的真实 心态。