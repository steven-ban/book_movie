\subsection{《大明王朝1566》}
\paragraph{海瑞}
海瑞刚直不阿,容不得丝毫蝇营狗苟,无愧于“海青天”的美誉。想想看,大明虽然武功不如大清,但政治气候可就活泼得多了,有清一代也没有见到一个海瑞,没有见到半个海瑞。

但海瑞刚猛(号“刚锋”)正直之外,其实是很有才干(清流中的才干,不是严嵩、吴宗宪那种实干)的人,也是很聪明的人。他对业务很是精熟(如“大明律”),对政府工作流程很熟悉,也知道有些事情现在不能碰(很有心学的意思)。如果不是这样,很容易就会被人抓了把柄治罪。但海瑞能屹立官场三十多年而不倒,恐怕和他的聪明和自保有关系。

海瑞无父,只有一母和妻子,曾有一女还溺亡。因此,海瑞有退路,在忠和孝上有选择。他可以专治于忠而对孝的标准要低于父母双全之人。海瑞由母亲教育成人,读《孝经》,对母亲百依百顺,母亲甚至强势,造成了他”无君无父“的性格。他不会屈于权威,恐怕和缺少父亲的教导有关。

海瑞成就了自己的清名,成就了自己的理想,虽然这样的理想最终并没有在大明王朝落地生根。其实,明朝的问题,并不能由海瑞解决。海瑞想让每个官员都成为清官,想让皇帝成为明君,认为这样就天下大治了。其实,大明王朝的问题,并不在昏君和贪官上,而在小农经济本身的问题上。黄仁宇认为明朝没有实现”数字化治理“,财政年年吃紧,积弊难除,积重难返,其实任何一个农业时代的朝代都是这样。

当然,从明朝的政治体制上来讲,权力高度集中,官员的权力比前代大大减少,不设宰相,党争严重。最重要的,是明代官员的工资太低,官员的灰色收入和黑色收入占了绝大部分,因此官员不得不贪,否则养不了家孝顺不了双亲。

\paragraph{海母,海妻}
海母强势,海妻软弱。海母的强势造就了海瑞在外刚正而在家恭谨的性格。其实,对海母这样的女人,我虽佩服,但并不喜欢。她对海妻的压制正是两人同房较少而不育的原因,也是海家”绝后“的原因。可是她懂不了。

最苦命的是海妻。男人做事,如果有妻子的牵绊,大半做不成。海妻地位的低下,正是海瑞压榨她,才能在外为官清廉,否则后院起火,考虑太多,也做不了清官。海妻一生劳作,无儿无女,多次流产,这样的苦命不是谁都能任劳任怨的。

\paragraph{嘉靖}
嘉靖很聪明,但也很贪婪,他贪婪的不是金钱美色,而是长生不老。他二十多年不上朝,都交给严嵩父子来协理天下。这样的君主,其实就是昏君,海瑞上疏骂他,是有道理的。到了晚年,为了给儿子接班腾位,又署名严嵩父子,节制清流势力,事情做得很漂亮。

明朝的君主,在敬业方面大大不如清朝。嘉靖不上朝,但朝中的事情都是他做主,他掌握着官员的行动,掌握着财政。

\paragraph{严氏父子}

\paragraph{吴宗宪}

\paragraph{王用汲,李时珍}
