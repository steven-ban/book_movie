\subsection{切尔西:白人特权观察}

英文名:Hello, Privilege. It's me, Chelsea (2019)

豆瓣页面:\url{https://movie.douban.com/subject/34806927/}

切尔西是个犹太人,她自己十五六岁时搬到一个有色人种多样的社区,交了一个黑人男朋友,他抽大麻并且贩毒,被拒绝大学申请,随后入狱多年。有趣的是,切尔西本人当时就揭发了他,多年之后的她对此事似乎不是很上心,可以看出此人凉薄寡情。现在,为了做这个记录片,她与当年的男友重聚,了解他之后的生活,似乎与之前达成了一种和解(并没有)。

本记录片聚焦在平权运动半个多世纪后的美国的种族问题,白人在教育、执法、找工作时依然享受着相对于有色人种的种种特权。切尔西采访了不少学者和黑人,她语言直接,性格率真,单刀直入,展示出赤裸裸的现实。

其实对于任何一个理性的人来说,美国的种族问题根本没有表面上那么和谐,虽然黑人受到了不少照顾(不少白人似乎不喜欢这样,但绝不会在公开场合说出来)。

有特权者感受不到自己的特权,而没有特权者则处处感受到歧视和不公,这是一种普遍的社会现象。具体到美国的种族歧视问题,黑人自不必多说。他们的社区有着更高的犯罪率,开车时更容易被警察询问,甚至会被警察开枪杀死(新闻里这样的事件已经不少了)。他们一旦在年轻时被卷入吸毒、贩毒和暴力事件,就一步错步步错,上不了大学,找不了好工作,收入低福利差,代代如此,形成恶性循环。而白人虽然也有穷有富,但相对而言这种麻烦就少了不少。那么,为什么在21世纪的美国,无论是法律还是政策都给了黑人不少照顾,却仍然存在这样大的种族偏见呢?唯一的解释,就是社会的政治和财富资源仍然掌握在白人手里,而政治本身是一种圈子政治,讲求“他者”和“自己人”,黑人被完全排除在权力中心之外。毕竟黑人人口不算少,从自下而上的自治和民主政治制度的逻辑来讲,他们完全可以形成自己的圈子,与白人分庭抗礼。

白人这种对特权普遍的视而不见,事实上是美国社会分裂的一个写照。这说明,美国仍然是一个杂拼式的社会,而不是一个充分融合的、同一的社会。这样的社会相比中国这种一统式的社会是好是坏,很难简单地判定,但就文化本身的凝聚力和社会整体的向心力而言,肯定是较差的。

不过,从切尔西的采访中也能看到,美国社会不少人,特别是白人,对这个问题还是敢于直接面对和分析的。但是可惜,这样的种族融合过于缓慢,反映到政治和文化上就更加缓慢了。这可能是美国社会将来的一个重大隐患。
