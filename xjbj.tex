\subsection{《乡建笔记——新青年与乡村的生命对话》}

标签: 乡村建设 \ 三农 \ 温铁军 \ NGO \ 大学生

这本书的组织形式与温铁军其他的偏学术化的书籍不同,是由一篇篇的心得体会、讲话稿组成的,有一些还是由乡建中心的人采访整理而成的。这本书涉及的乡建,主要是由梁漱溟乡村建设中心、社区伙伴等组织发起和组织的。

中国的农村将向何处去?这是近代以来无数学者(经济学家、社会学家、革命家等)和普通民众都一直在思考的事情。传统中国的乡村是由乡绅来治理的(南北方有不同),存在着宗族权力,并且乡绅往往是读书人,是一个小社会。进入近代以后,特别是科举废除以后,乡村精英和知识分子流向城市,乡村渐渐衰落,并且出现了比例更高的土豪劣绅,贫富分化。民国时期出现了梁漱溟、晏阳初这样的“乡村建设”派,他们深入农村去革除陋习,组织农村进行自我提升,但整体的社会环境并不利于大规模推广。新中国以后,中国普遍完成了土地改革,农民都分到了土地,但也同时摧毁了地主阶级和其上的知识精英,农村完全成为国家统治的基层单位,在工业化进程中成为“剪刀差”的一方,为城市贡献部分劳动力、粮食和工业品市场,由于户籍制度的限制,城乡差别很大,农民的生活水平一直低于城市居民。虽然进入21世纪后,国家基于农村官民矛盾大、农民生活苦、农业艰难的情况,进行了三农建设,向农村进行转移支付,特别是免除农业税、村村通、消除绝对贫困等工作,确实让农村的面貌发生了重大变化,但城乡差距依然很大,农村的年青人特别是知识分子依然流向城市,农村只剩下“老弱病残”。

当然,这种趋势,是工业化的必然。中国如果不走西方和一些后发国家那样的强迫工业化的道路,不产生南美、印度等国那样的城市贫民窟,要稳住农村,保障地有人种并且农民的基本生存,只能逐步进行城镇化,经过几代人时间,农村人口渐渐下降,可能才能达到发达国家那种城乡差距比较小的情况。在这样的背景下,农村应该怎么办呢?任由它慢慢被资本(国家资本和金融资本)侵蚀、凋零吗?农民应该怎么办呢?只能出去打工吗?只能将老弱病残放在农村,亲人两隔吗?从农村出来的大学生,应该怎么办呢?只能在城市背负房贷,过着紧巴巴的、背井离乡没有父母亲戚的生活吗?只能融入城市的格子间,成为机器化大生产上的螺丝钉吗?

我之前一直认为,农民必须转变成市民,必须合村并居,工作上进厂只保留较少的农民,也就是变成市民,而农村土地在保证农民生活水平的前提下进行集中,搞机械化大农业(特别是华北平原这种),这样农民的生活水平才能提高。毕竟现在的农村居住分散,连下水道都没有(铺设维护成本比集中居住高),旱厕问题无法解决,关系到人的尊严。也就是说,农村本身必须被消灭。

但是乡村建设学派却不这么认为。他们认为,农村应当保留,农民应当生活在农村,农业才是人与然和谐共处的典范,自己种自己吃也是健康天然的,相比城市的那种人在工业社会里的“异化”要更和谐自然;农村的资本和人才不应该像现在这样流入到城市,那只会让城市中的金融资本获利,农民得到的很少,相反应当让这些资源“在地化”,让当地的农民获利;从农村走出来的那些人,未必都喜欢城市里的钢筋混凝土,自己种地种菜,自己有自己的院子,可以朝起暮眠,看阳光雨露、雨雪雾霜,是一种诗歌田园式的美好,生活节奏慢对心理和生理上也有好处,没有那么多的压力和焦虑,农村的生活可能清贫,但并不是一无是处,甚至一些城市人(大多数城市人也是两代甚至更是时候从农村迁过去的)也有意在农村养老;当然,基于温铁军一贯的观点,农村的生活具有稳定性,是城市资本周期性危机的泄洪区,农村可以承接城市的危机让经济“软着陆”,否则会产生重大危机导致政治体制和经济上的重大损失;农村的集约化可以通过他们乡建团队的合作社,利用绿色生态农业的方式与城市居民连接,这样农民可以获得收入,市民则可以吃到安全放心绿色的农产品,另外还可以搞乡村旅游。

这些乡建团队成立了NGO,举办一期一期的培训班,这个培训班很规范,会进行两次的集中理论学习和两次的下乡实践(这中间会筛选掉那些知难而退的人),集中学习会进行晨练和朝话,并需要结业里撰写报告,教会学员如何在农村生活,如何进行绿色农业生产,如何建立合作社、文化社等。在农村,他们组织起农民,进行文艺演出、办合作社、经营农场养殖场,甚至帮助农民上访。他们的学员来自各行各业,主要是大学生、开公司的、农民工、一些有一定文化的农民等。他们的共同学习生活在回忆里是快乐的温馨的,实践活动是辛苦但甜蜜的,有一些年轻人还在这样的天南海北的下乡活动中结为恋人,甚至有着异地恋的阻碍也初心不改,他们举办过两次集体婚礼,没有世俗的大操大办,大吃大喝,但是足够温馨感人。当然,他们在农村的活动也存在各种挫折,例如农民的不理解不配合、人心不齐、资金断裂、个人婚姻生活压力等。整体来看,他们对抗中国城镇化、资本化的努力似乎有点螳臂当车的意味,因为这种零散的小规模的合作社与资本对抗是力量不足的,组织能力、资金、管理水平与成熟的商业大资本运营有天壤之别,这从多个合作社资本规模只有几万、农民不信任这些组织者、经常退出等就可以看出来。当然,这些理想主义者,我是极为尊敬的,他们在用自己的方式来改变社会,并且让社会(至少让农民)过得更好。而且有一些组织者,确实在开展商业化的运营,相对而言也比较成功,比如定点供应蔬菜、举办儿童游学等。另外一点让我感慨的是,很多大学生的学校并不算好,不是什么211学校,而是很普通的像师范类的这种学校,这些学生坚持思考、脚踏实地的行为让人尊敬,而不像某些好学校的学生,一心只想着挣更多钱,更“成功”(虽然这不是什么不道德的事情,人各有志),他们对社会的思考可能是稚嫩的不成熟的,但是那种坚持理想的做法更接近大学教育的本意。

对于乡村建设的某些论点,我是反对的。例如农业,农业产生于一万年前,本身就是对生态环境的改造甚至破坏(当然没有工业化那么大),而且农产品的“绿色”往往是刻意追求不使用化肥农药,这需要投入更多的人力物力,经济上并不划算,而且抗拒现代农业手段并不一定能让农产品质量更好(例如口味未必更好,营养成分未必更优)。另外,乡建是围绕着农村和农民来展开的,但其实是连接了城市资本、城市市场、城镇人口与土地。城里人与山川农田相隔离,为了体验农村风光,就只能借助于这些媒介。我感觉未来的乡村建设,也是要不断加强对城市一边的吸引力,打造现代化的旅游、农业产品,尽量让城市人参与到农业生产中去,与儿童教育相结合,提供沉浸化的体验,提高服务品质,不能一提乡村旅游就是粗犷风格,而是可以精细化,这样才是后现代的三农形象。温铁军等人推崇的日本农协的模式,似乎也不太受社会认可,这种农协垄断农产品的最大问题是推高价格,进而推高CPI,城市人肯定不答应。

现在看来,乡村建设除了在一些农村(例如山西蒲韩)有了稳固的发展模式之外,在主流的乡村中仍然是边缘化的。返乡大学生和农民工整体上还是比较少,农村依然在凋零,并被城镇化所取代。但是,无论怎么说,我还是很敬佩这些人,他们在一无所有的情况下,不依靠政策,而是依靠自己的力量,去一点一滴改变农村的生活和生产,这为后来者(无论是新的乡建人员还是政府)都提供了很好的经验借鉴。整体来看,乡建事业相对于现在的官方农村政策而言依旧是小众,但相比从前,乡建的力量本身也在不断地增强。

评分:5/5。