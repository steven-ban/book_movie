\subsection{《1493:物种大交换开创的世界史》}

标签: 世界史 \ 物种交换 \  美洲史

作者:【美】Charles C. Mamm

英文名:1493: Uncovering the New World Columbus Created

作者是一个资深记者,并非专业的历史学家,因此本书决不可能写成一部教科书,而是一本对1493年以后随着欧洲人对美洲的开发而带来的跨越大洲的物种、人种交换的简单介绍(是的,虽然本书篇幅很长,厚厚的一大本,但每个主题下都没有足以作为教科书那样的深度)。

在”发现“美洲以前,美洲与其他大陆(特别是欧亚大陆)之间是隔绝的,然而随着地理大发现,美洲的生物与人种与其他大陆之间产生了交换,地球进入一个称为"同质世"的世代。在这个世代里,各个大洲的生物进行了融合交换,产生了新的物种与文化,这种交换称为”哥伦布大交换“,本书的主要内容,也就是这种大交换对各个大洲的近代化的深远影响。

对美洲而言,欧洲人带来了天花、疟疾等疾病(一个不完整的列表:天花、流感、肝炎、麻疹、脑炎、病毒性肺炎、肺结核、白喉、霍乱、斑疹伤寒、猩红热、细菌性脑膜炎等),使得本地人口锐减。来自欧洲的白人、来自非洲的黑人在美洲混合,与当地的印第安人的各个国家、部落之间通婚,形成了美洲现在众多的人种与文化。欧洲人带来了种植园经济和奴隶制,为黑人和美洲土著带来了深重苦难。同质世伴随着工业革命和现代化,然而这种”进步“的果实大部分被欧洲人抢走了,留给美洲(主要是拉美国家)的是印第安文化的消失、贫穷落后和内战频仍。欧洲人为美洲带来了蚯蚓,这之前美洲的落叶形成了厚厚的一堆一堆的小生态系统,蚯蚓进入后分解了这些落叶堆,使得赖以生存的林下植物、动物消失,林地更加开阔,深刻改变了美洲的生态结构。疟疾的输入使得本地人口减少,即使是欧洲人也难抵挡,而非洲人对恶性疟疾对抗性,这客观上促进了黑奴贸易。这里作者提供了一个极有意思的观察:密西西比河流域的美洲土著人是蓄奴的,而北方则不蓄奴,这对于之后美国奴隶州分布于南方似乎有一定的影响。美国存在所谓“梅森-狄克逊线(Mason-Dixon Line)”,它区分了美国的南方和北方、奴隶州与非奴隶州、恶性疟疾与非恶性疟疾的分布,有恶性疟疾的地方,更容易产生经济上的两极分化,为奴隶制制造的土壤(并非原因,显然作者并没有否定奴隶制由殖民者主导的罪恶历史)。制糖业需要大量且密集的人力,白银提取需要汞,这种既危险又劳累的工作给美洲土著和飘洋过海的黑人带来了无尽的痛苦,也间接导致拉美政治上的无能和社会的动乱。全球各地人在拉美聚集,与西班牙的天主教文化融合,互相通婚形成了复杂的血缘和种族,也形成了光怪陆离的拉美文化。在亚马逊河流域,黑奴暴动和逃亡后建立了很多社区,与殖民者斗智斗勇,诞生了不少奴隶英雄。

对欧亚大陆而言,美洲的作物经过改良,提高了粮食单产量,为欧亚大陆各个国家(从英国到中国,从西班牙到非洲)的人口增长提供了条件。在1550-1750年这段时间内,北半球经历了一次小冰期(各地时间不一),这些作物显然防止了大量的人口衰退。这本书还引用了弗尼吉亚大学的气象学家拉迪曼(William F. Ruddiman)的说法,小冰期的另一个解释是当时美洲土著人口锐减,之前的“焚烧农业”受阻,大气中的二氧化碳减少,因此引发气候变冷(存疑)。土豆为欧洲人提供了难得的可以度过“小冰期”的口粮,防止了人口衰退,是“农业革命”的一部分。另外,秘鲁海岸的海岛上还有“鸟粪肥”,是现代化肥之前的重要农业产品,欧洲殖民者采用奴工来开采,为欧洲农业提供肥料。然而,欧洲土豆的基因单一,导致了1845年之后的土豆病和饥荒。美洲为将来的工业革命提供了橡胶这一重要原料,这一作物在东南亚、云南省等地方随着殖民化而被推广。

同时,这些作物的引进,也促进了人们将农业引入山地,农业作物的扩张带来了滥砍滥伐和水土流失,又从而带来了大量流民和政局动荡,导致了中国的衰落(作者的这个论断没有给出出处,我也是第一次看到,正确性本身值得怀疑,显然对于中国衰落的解释过于简单,这是大部分西方人对中国的研究总是带着先入为主的偏见和傲慢所致)。清朝封关禁海,将沿海居民内迁,这些人中有一些在山地群居(“棚民”),广泛种植美洲作物,政府无法统计人口。中国增长的人口并没有 让自己逃脱出“马尔萨斯陷阱”,清代中叶之后的事情就很广为人知了。


美洲发现的白银,作为一种硬通货,被殖民者拿来与中国进行交换。毕竟西方对中国的瓷器、丝绸、茶叶的需求量很大,而中国相比之下对西方的货物缺少兴趣,造成中国长期的贸易顺差。白银的输入,造成了中国内地商品经济的繁荣和通货膨胀。中国人在当时就广泛参与了全球贸易,中国人仿制的欧洲产品价廉物美,走私商与西班牙的商人合作,逃过总督的监管,在马尼拉做交易,形成了巨大的中国人社群,即使在遭受军队屠杀之后仍然死灰复燃。中国人这种早期的对于全球化的参与很有意思,从这里可以看出中国人确实聪明,商业嗅觉很灵敏,中国近些看的崛起显然与这种民族素质强相关。

总之,本书内容十分丰富,对于大多数人来说,肯定会对美洲历史和世界历史的整体感觉产生震动,值得一读。但是具体的细节可能会有错误,还需要读者予以辨别。本书的翻译十分低劣,与前作《1491》相比简直是同出一手,中信的这种外文译作对翻译也太不重视了,白白浪费了这么好的书和选材。

评分:4/5。
