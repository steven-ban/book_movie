\subsection{《我们台湾这些年》}

\subsubsection{一些标注:}
有人说岛民的性格是“浅碟”文化,就如同碟子一样浅浅的没有深度。

马库色在《单向度的人》里提到的,以市场经济为主的新极权会提供人民一切的要求,满足他们而让人民失去独立思考的反抗精神。

整个90年代,就是省籍问题不断被激化,但又被消弭的过程,其实也是台湾人一次次在学习互相包容的过程。

社会越来越开放,越来越不知道该反对些什么,导致愤青们很失落。

那时候珠算的神话就是,珠算比电脑还快。

因为“乖乖虎”苏有朋在当时就读于全台湾第一重点高中——建国中学,后来也考上台大机械系。

早期这些人士的“台独”思想还是比较具有理想性的,不像现在,只变成一种政治语境加以操弄。

现在有许多人开口闭口总是“蒋经国在位的时候……”似乎那个年代比现在更好。我想,其实可能是怀念当时经济正在全力发展,人人埋头苦干、同舟共济的充实感吧!现在大家都富了,反而一点共同的目标都没了,有些空虚。

过去台湾的中学生有一定的仪容规定,在发型方面,男生一律都是三分头,女生则一律都理着耳下一公分,俗称“西瓜皮”的发型。

三台分属于党(中视)、政(台视)、军(华视),

台湾可以说是一个移民社会,几百年来不断有人移入,主要有四个族群,闽南人(73.3\% )、客家人(12.0\% )、外省人(13.0\% )与少数民族(1.7\% )。

其实国民党要的只是政治的绝对权力,至于基层社会,个人和传统的空间并没有被消灭,而是换了另一种形式管理。比如在台湾的各乡镇,就算是偏远地区,都可以看到国民党的“民众服务社”,其实说白了就是乡党部。平时服务些什么不知道,但大家到了选举时期,这里就变成了各乡镇动员、固桩,甚至买票的基地。

台视长于新闻,现在很多有线新闻台的主管最早都是台视出来的。中视长于戏剧,台湾第一部连续剧《晶晶》就是中视推出的,许多大陆朋友熟悉的琼瑶剧,如《梅花三弄》、《还珠格格》也都是中视播出的。不过这几年很多剧都外包给大陆,或为了节省成本,索性直接买大陆剧来播,算弱掉了。华视虽然是军方所有,但长于综艺节目,从前到现在一直如此,许多名主持如张小燕、胡瓜等人,也都是在华视发迹的。

\subsubsection{感想}

看《我们台湾这些年》,发现两岸虽然政治制度不同(这种不同并没有很多人想象得那么大,戒严以前的台湾和改革开放以后的大陆都是威权体制),但教育领域的相同点太多了:体罚、应试教育、政治挂帅、补习班、分快慢班、文理分科……作者很多场景都会让曾经年轻的人会心一笑。

台湾选举乱糟糟,让我去台北当中共台北市委政法委书记可好?

台湾的新闻媒体为了收视率,可以说是不择手段,丧失了新闻本该有的独立客观,拼命挖材料博眼球,显现出自由浪潮下的失控。

不喜欢作者描写的台湾兵营生活,枯燥。

作者的大陆游记某些还可一读,但更多的是流水账,走马观花,感觉是在凑字数。
台湾社会经过蒋介石时代的高压、蒋经国时代的威权、李登辉时代的开放和陈水扁时代的党争,普通人的生活似乎改变不大,社会在有限的进步。但更多的政治生活内容,本书并没有介绍,这也是我想读到的。