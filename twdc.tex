\subsection{《台湾单车旅行笔记》}
\subsubsection{一些标注}

上海的街道就是用中国省份和都市来命名的:南北纵向用省份,东西横向用城市。

因为“走路太慢,开车太快,而骑单车速度刚刚好……”

看北京夜景的最佳位置是香山山顶,空气清透的日子,等太阳落山,看二环、三环、四环、五环的路灯逐格点亮,

将台湾原住民分为9个族群,分别是阿美族、泰雅族、赛夏族、排湾族、布农族、卑南族、鲁凯族、邹族和达悟族,后经台湾当局修订,再加上撒奇莱雅、噶玛兰、太鲁阁、赛德克和邵族共14个族群,人数约40多万,占全台湾人口2\% ,在花莲和台东县,便有8个原住民部族。

十里青山行画里,双飞白鸟似江南。

挑战乳酸的感觉,有如吸毒一样,让你欲罢不能。

视野开阔的人,从不沉湎于痛苦或者其他糟糕的情绪,因为他能看到影子的边界,知道没有永远停留的绝望。

浊水溪,不只是一个地理名词,它不但分隔了台湾北部和台湾南部,还分隔了政治上的泛蓝和泛绿,分隔了“本省人”和“外省人”,分隔了国民党和民进党,也分隔了台湾的政声人心。

现在的台湾,铁路分为台铁和高铁两个相对的概念,前者包括了7条环岛干线(纵贯线、台中线、屏东线、宜兰线、北回线、台东线、南回线)和4条支线(平溪线、内湾线、集集线、沙仑线),后者仅有一条,即台北至高雄采用日本技术的“新干线”。

全世界第一位以双脚徒步、骑车,完成环球壮举的勇士,是中国的潘德明。

壮游要具备:明确地自我期许,要有超人的智慧。

\subsubsection{读后感}

《台湾单车旅行笔记》,文笔很稚嫩,脱不了学生腔,特别是“眼睛默黑得像……”和“看上去很平和”这样的描写不知所云,完全没有必要,根本没有传达给读者有用的信息。

鸡汤太多,没有来由的感叹太多。 鸡汤不是不能喝,而是里面要有肉,有干货。《台湾环岛旅行笔记》里,让人羡慕的不是作者随处而发的感想,而是里面的真人真事,是各人之间实实在在的温情付出。

我关心的是事实,是真实的见闻,并希望作者以直接翔实的语言表达出来。

突然对骑行有兴趣了,以后搞搞吧。让人对旅行感兴趣,或许是一本成功的旅行书的目的。