\subsection{《21世纪资本论》}

作者:【法】托马斯·皮凯蒂

就像《资本论》一样,本书是一本“政治经济学”著作。它不同于“纯粹”的以数理模型和基本假设为基础的新古典自由主义经济学,后者把精力放在简单并可以错误的假设之上,对现实缺少关注,而本书则把眼光投向资本主义的经济不平等上面。近些年来,特别是2008年经济危机以来,政治经济学再次进入人们的视野,事实上本书也是2013年左右在回顾了经济危机以及几百年来的不平等而写作的。新古典自由主义经济学越来越受到人们的质疑和否定,平等再一次激发起人们对资本主义制度的反思。

对于新古典自由主义经济学,作者不乏批评。本书一开始,他就说:
\begin{quotation}
坦率地说,目前的经济学科不惜牺牲历史研究,而盲目地追求数学模型,追求纯理论的、高度理想化的推测。这种幼稚的作法应该被摒弃了。经济学家们往往沉浸于琐碎的、只有自己感兴趣的数学问题中。这种对数学的痴迷是获取科学性表象的一个捷径,因为这样不需要回答我们所生活的世界中那些更复杂的问题。在法国做一个理论经济学家有个很大的优势:在这里,经济学家并没有受到学术界以及政界、金融界精英的高度重视,因此他们必须撇开对其他学科的轻视以及对于科学合理性的荒谬要求,尽管事实上他们对于任何事情几乎都一无所知。……事实上,经济学并不应该试图与其他社会科学割裂开来,只有与它们结合起来才能获得进步。社会学科的共同特点是知之甚少却把时间浪费在愚蠢的学科争吵之中。如果想要进一步了解财富分配的历史动态和社会阶级的结构,我们必须采用一种务实的态度,利用历史学家、社会学家、政治学家和经济学家的研究。我们必须从基本的问题开始,并试图去回答这些问题。学科争论和地盘之争是没有意义的。在我眼里,本书是部经济学作品,同时也是一部历史学作品。
\end{quotation}

在本书的最后,作者又一次说道:
\begin{quotation}
我把经济学看作社会科学的一个分支,与历史学、社会学、人类学和政治学并列。我希望本书能够让读者明白我的想法。我不喜欢“经济科学”(economic science)这一表述,为其中的极端傲慢感到震惊,这是因为它暗示经济学获得了比其他社会科学更高的科学地位。我更喜欢“政治经济学”(political economy)这一表述,它可能显得有些过时,不过在我看来传递了经济学和其他社会科学的唯一区别:其政治、规范和道德目的。
\end{quotation}

本书基于发达国家(主要是法国、英国、美国、德国、日本的数据,毕竟这些国家的资本主义制度更为完善,经济数据更加全面,特别是法国,其保留了大革命后连续的收入和纳税数据)的经济数据,建立了“世界顶级收入数据库,一个关于收入不平等演变过程的最大的历史数据库”,并得出了本书的主要结论。因此,本书的优势是历史长度下的真实数据,具有较高的价值。

本书的主要结论:
\begin{itemize*}
	\item 财富分配的历史总是深受政治影响,是无法通过纯经济运行机制解释的。
	\item 财富分配的动态变化表明,有一个强大的机制在交替性地推动着收入与财富的趋同与分化,那些长期存在的促进不稳定和不平等的力量并不会自动减弱或消失。趋同的主要力量是知识的扩散以及对培训和技能的资金投入。
	\item 导致收入不平等加剧的最主要因素是\emph{从长期看,资本的收益率大于经济增长率,即$r>g$}。
\end{itemize*}

本书首先给出了几个概念和定律。
\begin{theorem}[资本主义第一基本规律]
$\alpha = r \beta$,其中$\alpha$是资本收入占国民收入比,$r$是资本收益率,$\beta$是资本/收入比。这是一个会计恒等式。
\end{theorem}

经济增长主要来自两个贡献:一是人口的增长,二是人均产出的增长。公元元年到1700年间人口和经济的年增长率均低于0.1\% ,其中人口增长率为0.06\% ,人均产出增长率为0.02\% ;1700-2012年全球产出的年均增长率为1.6\% ,其中人口和人均产出的增长率均为0.8\% 。

从历史上看,英法两国的资本/收入比经历一个U型曲线:19世纪该值很高(保持在7左右),20世纪上半叶出现急剧的下降(2-3左右),但在20世纪后半叶又出现回升(5-6左右),21世纪将会进一步提高。德国的曲线类似,而美国则比较平缓(得益于较为平等的政治制度),但现在美国的值较大。

\begin{theorem}[资本主义第二基本规律]
$\beta = s/g$,其中$\beta$是资本/收入比,$s$是储蓄率,$g$是经济增长率。从全球来看,21世纪资本/收入比也会不断上升。这是一个长期趋势,短期内的资产价格波动不会影响这个趋势。
\end{theorem}

资本收入占国民收入比在19世纪很高,20世纪上半叶下降,20世纪下半叶再度提高。当然,相比于$\beta$,$\alpha$值的变化相对平缓。在21世纪,$\alpha$的值大约为30-40\% 。

在资本主义社会,收入呈底大顶小的金字塔形结构:收入最高的1\% 、收入最高的10\% 以及最低的50\% 呈现出巨大的悬殊。2008年金融危机的出现,就在于长期积累的收入断层线。2010年以来,在大多数欧洲国家,尤其在法国、德国、英国和意大利,最富裕的10\% 人群占有国家财富的约60\% 。

“超级经理人”是英美国家的特殊现象,其处于收入的最上层1\% 水平,但主要来自于运气因素,是公司治理的内在缺陷。

由于$r>g$是资本主义制度自发的、内在的机制,因此无法自发调节,除非遇到大的外部冲击(如两次世界大战)。由于继承财富的存在,使得贫富差距会越来越大,21世纪会有更广泛的“拼爹”现象。

资本的积累使得少数富裕阶层越来越富,而平民的生活水平没有大幅提高,造成了不平等。21世纪的国家,应当在维持民主制度的基础上对财富征收\emph{累进税},甚至国际社会应当联合征税,防止一国资本外逃避税。当然,这一点上作者显得太幼稚了。现在的情况表明,不仅各国不会联合征税,而且会提高关税壁垒并降低本国税收以吸收外资。国家间的利益仍然是最重要的格局,明显比西方左派相像得复杂。

本书的数据比较详实,特别是法国的数据,并且作者还举例了19世纪巴尔扎克和简·奥斯汀的小说来证明上流社会需要多少财产(比平均国民收入高多少倍)才能维持阶层的生活,这一点挺有意思的。

本书对资本主义的揭示是深刻的,对理解资本主义非常重要。不过似乎本书受到自由派经济学者的批评,这一点作为外行,我无力评判。不过,如果本书对资本主义基本运作方式的揭示是正确的,那么21世纪里,放任自由的经济政策恐怕会越来越少地使用,而社会主义、大政府的方式会得到越来越多人的认可,特别是后发国家,与其接受“华盛顿共识”,不如向中国学习(能不能学得来不一定)。因此,政府更多地参与财富分配会成为越来越多人的共识。

评分:10/10。
