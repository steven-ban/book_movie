\subsection{《雍正王朝》}
剧本、演技、布景什么的就不说了,绝对的国内一流。
 
关于立意,主创张黎说这是一个“当家难“的故事。做皇帝看似高高在上万人景仰,但清朝的皇帝并不好当。阿哥们五岁起就要到南书房学习,课业负担不比现在的学生差,还要面临时时刻刻的考核。选太子就不说了,肯定要德才兼备的,长大后都基本上要做事,比如到户部当头头(就是皇帝和六部的联系人或耳目),要办事。竞争是一方面,事务繁琐也是一方面。像雍正这样立志有为的皇帝,要改革,就得破除各种阻力,甚至是康熙留下的积弊。但雍正是坚定的,把”摊丁入亩“严格贯彻了下去,而”一体当差一体纳粮“则受到的反对太多。这种大力改革的姿态似乎迎合了九十年代的那播领导,当时搞下岗搞军队不得经商,也是先要得罪一下既得利益者。还好,当年的长者和以前的雍正都基本成功了。作为一部“洗地剧”,也算是比较成功了,达到了目的。 

八爷逼宫那两集太精彩,雍正心力交瘁乱中取胜,张廷玉老成持重激辩群雄,十三爷鞠躬尽瘁 死而后已,还有八爷的成竹在胸野心昭然,飙戏很过瘾! 

不过该剧有很多历史错误,细节的不清楚,但是关于八爷十三爷的死期就跟历史不符。