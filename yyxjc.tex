\subsection{《营养师手册》}

膳食建议:
\begin{itemize*}
	\item 米面混食
	\item 豆谷混食
	\item 吃带馅食品
	\item 豆腐与鱼合吃
	\item 土豆和牛肉合吃
	\item 胡萝卜要炒着吃
	\item 炒豆芽放醋
\end{itemize*}

炒菜时要急火快炒,即用高温短时间炒,这样可以大大减少维生素的损失。炒菜时不要过早放盐,否则,菜不仅不容易热,还会流出较多的菜汁,一些维生素和矿物质也会同时流出。一般肉类也要急火快炒。菜快出锅时加入酱油,略经烹炒后即出锅。

做汤时要水开后下菜。

原则上食物只要做熟,则加热时间愈短愈好,空气愈不流通愈好,所以烹调时尽量盖上锅盖。

水溶性维生素流失,洗切时的损失,因此蔬菜在烹调之前,必须清洗。

如果用碱类发面剂(如苏打)制作馒头,能破坏面团中大部分维生素B1。煮稀饭时加些碱,虽然能使米饭烂得快,稀饭黏性好,但是维生素的损失可达75%。做油条等油炸食品时,因加碱或高温油炸,可使面粉中的维生素B1全部破坏,维生素B2和烟酸各损失50%。

北方人爱吃捞面条,大量的营养素会随面汤的弃去而损失掉。一般可损失40%的维生素B1,57%的维生素B2,22%的烟酸,蛋白质和矿物质的损失也比较严重。

淘米时,用大量的水反复搓洗,维生素B1可损失29%~60%,维生素B2和烟酸可损失23%~25%,矿物质损失70%,蛋白质损失15.7%,脂肪损失42.6%,糖类损失2%。米越精白,搓洗次数越多,淘米前后浸泡时间愈长,淘米用水温度愈高,则各种营养素损失愈多。

油炸食物时,许多氨基酸会因高温而破坏;水溶性维生素会因水洗而大量流失。

成人每日摄入5~10g食盐为适宜,WHO推荐的含盐摄入量为6g。夏季排汗多,失去的盐分也多,摄入量以维持在10~15g为好。若已经是高血压患者,食盐的摄入量以低于5g为宜;如是肾病患者,就应以其他咸味调料代替食盐。

蒸、煮、急火快炒的菜及馅料不宜多用味精,以免加热过程中使味精变成焦化谷氨酸钠而产生毒性。

不能在含碱或小苏打的食物中使用:因为在碱性溶液中,谷氨酸钠生成有不良气味的谷氨酸二钠,失去其调味作用。

水温为70~90℃时,味精的溶解度最高,当受热120℃以上时,味精中的谷氨酸钠就会变成焦谷氨酸钠,不但失去鲜味,而且有一定的毒性。

哺乳动物刚出生时,乳糖酶活性一般都很高,随着年龄的增长,该酶的活性越来越低,到成年后,活性已降到很低的水平了,所以婴幼儿不耐受乳糖的人很少,而成人常年不喝牛奶者容易发生。因此,如果平时很少喝牛奶,在某些重要时刻来临前(如学生高考)不宜喝牛奶,以免因乳糖不耐受导致腹泻发生,而造成不良后果。

生食鸡蛋对人体没有好处。蛋在储存时,要防潮,不能水洗或雨淋,否则蛋会很快变质腐败。蛋壳外表呈霜状,无光泽,表明蛋是新鲜的,如无霜状物,且油光发亮不清洁,说明蛋已不新鲜。

有些水产动物感染肺吸虫和肝吸虫,特别是小河或小溪中的河蟹,常是肺吸虫的中间宿主,如未煮熟即食,可能致人患病。所以,在烹调加工时要注意烧熟、煮透。

从维护维生素的角度,肉类食品宜炒,不宜烧、炖和蒸、煮。

膳食中动物性蛋白质,至少要达到总蛋白量的10%以上。

由于胡萝卜素是脂溶性维生素,烹调胡萝卜时加些油或肉类,可增加胡萝卜素的摄入。

马铃薯在烹调前应先削去皮,挖去芽眼,有芽的地方要多挖去一些,削去发绿或发紫的部位,用清水泡2~3h,充分烧熟再吃,龙葵素呈弱碱性,烹调时加点醋如醋熘土豆丝,可以破坏龙葵素。

辣椒是蔬菜中含维生素C最多的蔬菜之一,含量为89~185mg/100g;含胡萝卜素仅次于绿叶菜,而高于其他蔬菜;其他成分也不少,是一种营养价值很高的菜。

最好是用石膏做凝固剂,因为石膏是一种钙盐,可以增加豆腐里的钙质。蛋白质的含量以大豆为最高,一般可达35%~40%,大豆蛋白质为优质蛋白质,含赖氨酸较多,是谷类蛋白质理想的互补品。

糙米碾磨程度越高,维生素含量越少,易消化且好吃,但是蛋白质、脂肪、矿物质及维生素都有很大损失。谷粒外层的蛋白质含量较里层高。

在我国老百姓的膳食中,有70%~80%的热能和60%左右的蛋白质由谷类食物供给,同时谷类食物提供的矿物质和B族维生素,在膳食中也占相当比重。

过食动物脂肪和植物油均会对身体造成不利影响。

体表面积大者,散发能量也多,故同等体重者,瘦高者基础代谢高于矮胖者。

每克糖类的产热量为16.7kJ(4.0kcal),每克脂类的产热量为37.6kJ(9.0kcal),每克蛋白质的产热量为16.7kJ(4.0kcal)。