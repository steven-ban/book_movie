\subsection{《消失的女儿》}

标签: 小说 \ 乡村 \ 拐卖 \ 郑小驴

作者:郑小驴

五个短篇:

\begin{description}
    \item[消失的女儿] 守林员的女儿被拐,村里之前出现一个城里来的摄影师,他拍过女儿的照片。摄影师十多年后结婚生子,但被人跟踪(应该是守林员),多年后被拐的女儿回到林场。
    \item[骑鹅的凛冬] 转业的老军人,打死了自己的一个儿子,孙子是个傻子,喜欢骑着鹅,而老军人的叔叔则卷入到村里一家人女儿的强奸事件,被活生生打死。老军人为儿子复仇,杀死了凶手,并像杀羊一样把他分尸。整个故事透露着悲凉、黑色与绝望。
    \item[盐湖城] 南方的一个镇上,一个老实巴交甚至有点窝囊的人因为征地拆迁纠纷,为了自己邻居兼乳母出头,去青海买枪欲杀掉征地的恶人(那人是自己的同学,长期欺负别人),但刚买到枪就被发现,判了刑。五年后,他出狱回家,发现妻子已经对他疏远,他要迁回此地,但出狱证明被他丢了,派出所所长与恶人同学官商勾结,借故不给他上户口,他只能回监狱补开证明,回来后发现还是被为难,自己多年前的户口材料有问题,身份证也丢了很多年,他发现自己的老婆与恶人同学有染,而且孩子也不是自己的,与儿子登山途中被恶人同学搸了一顿。他想再次买枪杀人,于是再次回到了青海,他狱中曾经和一个黑帮头子交好,被告知可以在哪儿买枪,并且还有一个在外的情人,对她一往情深。他到了青海买到了枪,也发现了那个情人,但情人已经做了娼妓,他开枪杀死了那个情人(我理解是自己委屈大,但为狱友了结了仇恨)。
    \item[私刑] 藏区的一个复仇故事,一个青年因争夺牧场捅死了另外一家的儿子,被判入狱,杀人者父亲也予以赔偿,但受害者家庭不接收,把钱物放在家中显现的门梁上,等罪犯出狱了,纠集家族中的人去逼他,欲杀人报仇。罪犯悔罪自杀,但仍然没有消除受害者家的仇恨。
    \item[大罪] 小马是一个基层警察,他有过女朋友但分了,他没有过手什么大案,于是想干点大事,闹点动静。当地中学校长蒋清泉与开发商勾结在一起很多年了,征地拆迁,民怨很大。后来蒋清泉的头被发现在江里,而开发商老总的无头尸体则被发现在水稻田里。侦察工作开展,小马是一员。他们找到了两个死者的老婆、情妇,将犯人锁定在一个赌徒身上,他是退伍军人,多年前强征地时父亲被压断腿,如今缺钱被蒋清泉夫人雇佣找小三,小马与赌徒在天台上搏斗,被赌徒打中脑袋,抬进ICU,不久死去。赌徒坚称人不是自己杀的,后来在刑讯逼供下改口。故事结束,但不难猜出真正的凶手是警察小马,他把现场处理得很巧妙,再栽赃给赌徒,只身去找赌徒搏斗也是为了立功。他真的想干票大的,弄出点声响,他成功了,但阴差阳错死去了。
    \item[枪毙] 可以说是微型小说了。一个赌徒因欠赌资而拉走了爷爷的名贵电器,这电器是赌徒父亲在外地挣钱买给爷爷的,爷爷被绑在椅子上死去(怎么死的我没有看明白),于是赌徒被警察抓住,执行死刑,弟弟从人群中冲出来,拿玩具枪给他开了一枪。很奇特的阅读体验。
    \item[雨赌] 三个小男孩,去山上玩诈金花的游戏,无知且愚蠢,赌的是谁输谁喝水,结果一个叫二墩子的经常输,喝了很多水,活活胀死,而另外一个叫范范的小男孩则认为二墩子的父亲偷了他家的钱去买了贵州老婆(二墩子的后妈),在死前欺负他,死后两人将尸体放好回家,却发现二墩子的父亲死去,贵州老婆被绑在床上(可能是打她,也可能是在玩SM)。这个短篇干净利落,我觉得比之前篇幅更长的好。小孩的残忍绝对不输大人。
    \item[蓝色脑膜炎] 一个叫黄秋的女孩,得了脑膜炎(被她捉弄的女同学传染的,她得病拉到县城,似乎是死了),她喜欢蓝色,她的母亲因为生了她(二胎女,一胎女)还想再生男孩,被强制引产。她喜欢(并不是男女之间的那种)她的语文老师,他来自省城师范,酷爱读书,女友因为阻隔时间长而离开了他,他失意中与一个教数学的寡妇老师叮当在一起,在带同学看望完黄秋回去的路上,过河时发生山洪,老师和另外一个女学生被冲走。黄秋的父亲是个木匠,他在给黄秋打完一个漂亮的棺材、把她入殓后,收到村支书委托,再为老师和女学生打两个棺材。这部短篇的节奏感和语感也很好。
\end{description}

这几个故事都算是短篇,题材都远离城市和现代文明,而是山村、郊区,涉及农村的愚昧、无法制、仇恨、拐卖、家暴、继父母、计划生育等主题,黑暗而压抑。语言较为简练而凌厉,远观的态度,绝不加以议论。人性则冷漠、疏远、自私,没有大爱,安于现状,在命运面前无能为力。而体现国家力量的警察、政府,则要么贪污与当地豪强沆瀣一气,要么不讲法治搞刑讯逼供,要么毫不作为对底层人士冷眼旁观,要么无力改变当地的落后与愚昧。读下来,说实话,很难受,很痛苦,虽然某种程度上很“享受”这种剖析现实的残酷美感。风格上,则是谜语式的,不是把事件原原本本地展现出来,而是将事件的几个切面展现给读者,加以剪辑组合,读者需要自己还原真实事件,类似于悬疑电影的剪辑手段。这种阅读体验,有点像《冰与火之歌》的POV式写法(事实上就是POV,只是不像后者篇目就写是谁的POV)。文字上,则凌厉简练,作者不抒发什么乱七八糟的评论,全由人物视角来带动故事走向。不过,我觉得在具体的人物视角上,似乎还不够贴切,例如乡村人物并不会有那么多文绉绉的感叹,这一点上似乎还需要作者打磨。当然,作为小说,这几篇是较为偏向情节走向的,而非人物形象,人物是为情节和作者观念而生的,更像是一个个符号而非栩栩如生的有血有肉的人。这一点上应当如何评价?我其实并不是很喜欢。我个人认为,好小说的一个重要标准就是人物是否立得住,情节可能是易逝的,但人物特别是写得好的人物,永远是不朽的。这一点上来看,作者似乎更想做一个剪辑师式的小说家,而非古典式的小说家。当然,据说现代文学已经超越了“讲什么样的故事”这样的阶段,而是专注于“如何讲故事”,这就不是我能够评价的了,因此我对20世纪以后的文学,较少关注。

作者的文学素养显然还是比较高的,比商业性写作的作家强,很多当代作家称赞他。但是,作者似乎是个80后,写90年代的乡村故事是否有事实依据,我是有些存疑的。首先农村是否这么黑暗我不清楚,这几篇不少发生在90年代甚至2000年后,甚至还有2010年后的事情,但个体是否这么无助呢?我不是很清楚,但直觉告诉我,他的很多事故逻辑是编的,例如警察不敢干涉藏区的仇杀、藏区抢夺牧场出了人命后“一命还一命”、征地拆迁时过于赤裸裸的官商勾结、从远方嫁过来的女人明目张胆与人偷情……这仅仅是为了形成一种故事背景和矛盾冲突,真实性上应该很打折扣。作者似乎想表达的是一种个人面对命运的反抗、无助和毁灭,整体上都是一种悲剧性的氛围。

不过,这几个短篇里,越是篇幅短的,节奏感、语言越是优秀,而较长的篇幅则未免有些割裂感,欠锤炼,可见仅从这几篇来看,作者的功能在于短篇而非长篇。至于他长篇写得怎么样,我不清楚,有机会可以读一读。话说回来,写作似乎是一种天赋,而郑小驴在这方面显然是有写作天赋的人。假以时日,相信他能写出更好的作品。

评分:4/5。
