\subsection{《如何欣赏一部电影》}

作者:【美】托马斯·福斯特。此人是美国一位饱受欢迎的公开课教授,同类作品有《如何阅读一本小说》和《如何阅读一本文学书》。

1. 我们一直生活在伟大的改编电影的时代。

2. 当人们一起唱歌时,表达的意思就是‘我想和你做爱’,而当他们一起跳舞的时候,那就是在做爱了。

3. 一部电影由时长不同的三幕组成,第二幕的长度是第一幕或第三幕的两倍。

4. 几乎所有西部片中都有一种无意识的潜台词:枪是性的象征。

5. 每一部小说都是关于如何阅读那本小说的一堂课,而这一课在最需要它的地方是最有力的,那便是开头部分。电影也是一样。

6. 黑白电影在呈现荒凉风景方面有优势。它们几乎可以呈现噩梦般的不真实。

7. 电影艺术唯一的和最严重的灾难性的改变,就是黑白电影的死亡。

8. 默片的制作方式存在于任何一个电影类型中,存在于几乎每部电影中。

9. 电影当初是没有声音的,于是导演和摄影师学会了将图像运用到极限,而我们全都从中受益,电影的历史进程也因而确立。

10. 一个镜头会传递出视觉的(也可能是听觉的)信息。

11. 一个段落运用更大段的故事——那些场景——来形成一个故事框架,不管它有多简单,也总是包含开始、中间和结尾。

12. 一个场景会包含一段故事,它是由摄影机拍摄的一段段视觉信息即镜头组成的。场景是一个完整的行动,没有固定的时长。

13. 电影中最基本的单位不是场景,而是镜头。

14. 这就是电影不言自明的特点:它需要观众的合作。

15. 演员们接受的训练和直觉都植根于戏剧表演,那是一种关于呈现的艺术;而镜头本身和在剪辑室里进行的选择与演员的劳动是分离的。

16. 电影的语言和戏剧的语言不同。它的语法是选择,而不是展现。相比之下,戏剧没有选择,只能时时刻刻在展现舞台上发生的一切。小说在选择叙述视角方面与之相似,这种选择决定了文本将传达何种信息,以及作者通过叙述者来分享多少信息。

电影的本质是“motion picture”即“运动的画面”,基本单元是一个一个的镜头。

本书从场景构成、历史上的默片、亮度、形象、镜头深度、人物、取景框、类型电影、改编、配乐等方面来介绍了欣赏电影的要点,是作者的讲课记录(比讲义丰富),因此具体内容上有点乱,并且也不够深入,可作为电影欣赏的入门书。