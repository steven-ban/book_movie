\subsection{《张居正讲解《大学》《中庸》》}

中国人需要读四书以明事理。儒家的东西,虽然已经有两千多年,但其蕴含的民主主义、世俗主义、自我修养等内容具有深刻的现代性。这本书是张居正写给未来的万历皇帝的对《大学》《中庸》的解注,同时加入了编译者的现代解释(没什么价值),可以看到明朝主流对这两本书的认识。

1. 君子之所不可及者,其唯人之所不见乎。

2. 君子之道,淡而不厌,简而文,温而理。知远之近,知风之自,知微之显。可与入德矣。

3. 好学近乎知。力行近乎仁。知耻近乎勇。

4. 知则明睿,所以知此道者。仁则无私,所以体此道者。勇则果确,所以强此道者。

5. 大贤则以师傅待之,小贤则以朋友处之,

6. 夫孝者,善继人之志,善述人之事者也。

7. 君子戒慎乎其所不睹,恐惧乎其所不闻。

8. 中是无所偏,庸是不可易。

9. 仁者以财发身。不仁者以身发财。

10. 与利不可并行,民与财不可兼得。

11. 财聚则民散,财散则民聚。

12. 所恶于上,毋以使下。所恶于下,毋以事上。所恶于前,毋以先后。所恶于后,毋以从前。所恶于右,毋以交于左。所恶于左,毋以交于右。此之谓絜矩之道。

13. 君子有诸己,而后求诸人。无诸己而后非诸人。

14. 这格物、致知、诚意、正心、修身,是明明德的条目;齐家、治国、明明德于天下,是新民的条目;人能知所先后,而循序为功,则己德明、民德新,而止至善在其中矣。《大学》之道,岂有外于此哉!

16. 明德了才可新民,便是明德为本,新民为末,恰似树有根梢一般。

