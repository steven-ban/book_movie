\subsection{《白夜行》}
1.恶只能产生更大的恶。如果不是桐原他老爸有特殊嗜好,如果不是他把魔掌伸向贫穷的雪穗母女,便不会产生这一系列令人扼腕的悲剧。

2.桐原他老爸把魔掌伸向雪穗,因为她们家穷,因为她们家软弱,但当时觉得软弱的人,多年之后变得异常强大——不,那个时候雪穗就已经很强大了。今天软弱,不代表永远软弱。

3.雪穗母亲没有尽到母亲的义务,本质上是个皮条客,拿女儿身体挣钱。家长失职,带给孩子的是永远的伤害。

4.桐原亮司的强大令人钦佩,其早熟也让人惊讶。杀父后桐原已经是一个大人了,似乎有某种更深刻的意味。

5.桐原亮司虽然自尽,但他是带着他的“白日”离开这个世界的。

6.桐原亮司一直在为被他杀害的父亲赎罪。可是,为了赎一个人的罪,把伤害带给不相关的人,是更大的恶行,这罪愆越赎越多。所以,我丝毫不同情他。

7.雪穗的犯罪“技术”丝毫不比桐原亮司弱,她善于利用人,这是犯罪最好的技术,也是这个世界最强大的技术。

8.桐原亮司死后,雪穗会有何种结局?所有的线索都已经断开,老刑警不能再追查下去,当事人死的死伤的伤,大概雪穗会继续开店,控制筱冢康晴,并最终控制筱冢家族?不过筱冢一成是她的老对手,对她的了解相当深,而她自己又失去了桐原亮司这个重要搭档,遇到拦路虎谁给她除掉?顶多也就是总裁夫人这个结局吧?自己的店倒是会开下去,搞不好会成为女装大鳄。不错的结局。

9.什么是真正的优雅?什么是真正的贵族?雪穗的优雅陶冶,骗过了这个世界的所有人,却独独骗不过真正高门大户的筱冢一成,为什么?如果说家庭的陶冶能让人从细节甚至直觉上区分别人和自己是不是一类人,那一成的堂兄康晴为什么没有这个能力?筱冢家族的其他人为什么没有这个能力?可见在东野笔下,这是某个人的一种特殊能力。一成能从雪穗的猫眼中看出她身上不属于贵族的那种狡黠和狠毒,算是这部小说的一个后门吧?果然还是富贵人看富贵人比较准,但等等,为什么康晴没有看出来呢?果然一成是雪穗的知音啊!
