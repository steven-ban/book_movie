\subsection{《你好,之华》}
岩井俊二用中国的演员和场景,讲了一个日本故事。

早恋,情书,手写的信件,邮箱,落魄的作家,倏然而逝的女人……这一系列的符号中国不是没有,但不是这样的呈现方式。

每个人都像一座座孤岛,唯有多年以后才忍心拂去时光的尘埃,通过写信这么间接的方式追忆自己的青春。这太日本了。

中国的故事都是有或多或少的烟火气的,纷繁杂乱,就像王安忆《长恨歌》一样,即使追忆也要团团坐下,在深夜里点一盏暖灯炖一锅豆腐。而日本故事,就像一盘生冷的寿司,拿起来是凉的,吃起来是凉的,咽到肚子里还是凉的。

但这是一个好故事,电影的细节也做的相当好,喜欢岩井俊二的观众肯定会喜欢这部电影的。

何况,周迅的演技那么棒,冲着周迅也值回票价了。