\subsection{《毛泽东品评四大名著》}

作者:盛巽昌 \ 李子迟

这本书如书名所言,是收集了一些毛泽东关于四大名著的点评和历史小故事,史料来源于毛泽东身边的人的回忆录以及毛本人对四大名著的讲话、评点等,是中国社会里“毛泽东热”的一个表现。作者盛巽昌是中国人民大学历史档案系毕业,主要写一些历史性随笔,研究三国、毛泽东等零星的历史事件,而李子迟是中国人民大学中文系毕业,写一些文学类随笔。这两个人并非专业的历史研究大家,本书也仅是收集一些历史小故事。因此,解读这本书,可以从毛泽东本人与四大名著的关系,以及本书的写法两个方面来谈。

毛泽东本人是解读二十世纪中国历史和社会无法跨越的人物。从圣人的立身、立功、立言来说,他都达到了极致,放在中国乃至世界历史上绝对算是伟人。无论是崇拜他,赞同他,还是反对他,都无法否定,作为一个人来说,他是一个伟大的难以超越的人。那么,什么造就了他的伟大?或者具体地说,什么造就了他在社会急剧变革时期的那种革命斗志、宽广心胸和高超手段?我们从他的成长史来说,他并非一个学者型的人才,小时候接受私塾教育,看的书也是经史子集,但正如他自己所说,他读“经”但不喜经,而是喜欢四大名著、说岳、说唐这种小说类书籍。在接触马克思主义和加入中国共产党之前,他的世界观、认识论和知识主要即来源于这些,可以说在那个时代的知识分子当中平平无奇。他从小即喜欢斗争,与父亲斗争,因看杂书而与教书先生斗争。他关注社会,关注农民,他家里虽然是富农,但对贫苦大众则有着同情,这和那个时代想要改变中国命运的知识分子是有很大共同点的。他豁达有气魄,不拘泥于细节,这是个人气质,也是他虽然爱读书但并不读死书的一个重点。因此,他不是一个学者,而是一个天生的政治家。加入中共之后的事情,中国人都很熟悉了,这里就没有必要展开。值得说的一点是,他的学习能力很强,虽然没有上过军事学校,但在战争中学习战争,比那个爱读经的军官学校学习过的对手蒋介石强的多了;他没有专门学习过马列经典著作,但运用辩证法则是十分纯熟,比中共一些理论家看得更远。他喜读这些旧书,一直到老年依然常常翻阅。

他从四大名著里看的是什么呢?主要还是对旧社会的一些认识、对阶级斗争的学习等等,除了宏观上的认识之外,还有微观上的细节上的认识。

\begin{quotation}
1936年在延安时,毛泽东就曾对斯诺说:“我读过经书,可是并不喜欢经书。我爱看的是中国古代的传奇小说,特别是其中关于造反的故事。
\end{quotation}

对三国演义,毛泽东十分熟悉时面的人物和故事,经常拿来与实际问题相结合(其他名著里也有,但都没有三国里多)。他不仅读演义,还读三国志,对那段历史的认识很深刻,把这段历史不仅当成真实存在的历史读,还当成军事教科书来读,对里面的战例研究很多。他高度评价曹操,认为他是伟大的军事家和诗人,抑制豪强,发展生产,这一点与传统上把曹操看成白脸奸臣是不同的,这可能也是领袖之独到之处,不过于计较人物的道德,特别是在封建伦理下的道德,而是看他如何改造了社会,如何影响了当时及后世。他认为张鲁借道教与为群众治病,属于阶级阶级斗争行为。他最喜欢的还是开头那句“天下大势,分久必合,合久必分”,认为揭示了治乱循环的现象和规律。

\begin{quotation}
1918年,毛泽东在杨昌济讲所授修身课的教材《伦理学原理》(德国泡尔生著)上,做了 12000字眉批。其中涉及三国史论,称“吾知一入大同之境,亦必生出许出(多)竞争抵抗之波澜来,而不能安处于大同之境矣。是故老庄绝圣弃智、老死不相往来之社会,徒为理想之社会而已。陶渊明桃花源之境遇,徒为理想之境遇而已。即此又可证明人类理想之实在性少,而谬误性多也。是故治乱迭乘,平和与战伐相寻者,自然之例也。伊古以来,一治即有一乱,吾人恒厌乱而望治,殊不知乱亦历史生活之一过程,自亦有实际生活之价值。吾人揽(览)史时,恒赞叹战国之时,刘、项相争之时,汉武与匈奴竞争之时,三国竞争之时,事态百变,人才辈出,令人喜读。至若承平之代,则殊厌弃之。非好乱也,安逸宁静之境,不能长处,非人生之所堪,而变化倏急,乃人性之所喜也”。(《〈伦理学原理〉批注》,见《毛泽东早期文稿》,湖南出版社 1990年版,第 185— 186页。)
\end{quotation}

对红楼梦,他主要是从封建社会的百科全书、阶级斗争来看的,把它称为中国文学的顶峰,足以与世界名著比肩,十分敬佩作者曹雪芹。他认为全书的关节是第四回对四大家庭的描述,他同情里面的底层,鄙视为官为富的人,但认为贾宝玉和林黛玉是有反抗精神的,而对于其他人物,他则发表的内容较少。毛泽东多次讲,红楼梦应该看五遍才有发言权,这一点我是赞同的,因为这本书的内容是如此丰富,情节是如此绵密,不看五遍则难以理清人物关系和情节走向。建国后他对红学研究的指示,影响了红学内部对阶级斗争的观点,他对李希凡、蓝翎观点的支持,对俞平伯观点的反对,影响了建国后一代红学研究的走向。这种影响十分深远,似乎在1987版的电视剧中都有一些体现,比如对林黛玉”一年三百六十日,风刀霜剑严相逼“的渲染,对东风西风互相斗争的渲染,对封建制度要走向衰落的揭示等等。这一点我觉得毛对红学研究本身是有过的,或者在那个时代里,领袖对专业的红学研究讲话定调子,是影响学术研究的行为,不利于百家争鸣的学术氛围。红楼梦主要还是一部悼亡之书,哀怨追悔之书,虽然看出了社会的腐朽,但并不“反封建”,事实上作者执笔时,对传统社会的主流价值观还是认同的。

\begin{quotation}
我认为聪明、老实仁义,足以解决一切困难问题。这点似乎同你谈过。聪谓多问多思,实谓实事求是。持之以恒,行之有素,总是比较能够做好事情的。
\end{quotation}

对水浒传,毛也是从阶级斗争的角度来看的,比如梁山泊是农民发起的阶级革命,但宋江是投降派,一百单八汉里大部分是反封建,很多是底层人,要反对上层统治阶级。“《水浒》这部书,好就好在投降。做反面教材,使人民都知道投降派。”

对西游记,毛也是赞同孙悟空对玉皇大帝既有封建秩序的破坏,喜欢他的反抗精神,这一点在之前的评价中是少有的。

再说本书的写法上来说。本书主要是毛的一些讲话,大部分是涉及了该部书中的人物或事件,但是也有一些其实和书无关,例如一些成语,那本书里可能有,但实际上是社会上通常的用法,作者竟然也罗列出来,简单就是凑字数。本书没有注释,而是把所有注释如毛的讲话里涉及的人物和事件都随手写在这段话之后,有很多重复,感觉是为了凑字数。更有甚者,对毛的一些讲话里涉及好几部书的,在这几部分书的分章里竟然都重复出现了,这更是凑字数的行为。总之,这本书只是将毛对四大名著的一些史料(一些史料的价值并不大,很多属于借题发挥的即兴式表述,对四本书本身的看法特别是有学术价值的看法并不是很多)的堆砌,本身的写作难度不大,价值不高。

评分:2/5。