\subsection{《东方快车谋杀案》}

作者:阿加莎

\subsubsection{人物角色关系}
\begin{longtable}{p{0.18\textwidth} | p{0.13\textwidth} | p{0.45\textwidth} | p{0.2\textwidth}}

    \caption{《东方快车谋杀案》人物表} \\
    \hline
姓名 & 角色  & 特征 & 行李 \\
\hline
\endfirsthead

(接上表) \\
姓名 & 角色 & 特征 & 行李 \\
\hline
\endhead

\hline
\endfoot
赫尔克里·波洛 &	主角,侦探	& & 旅行袋里有数位证人所说的睡衣,猩红色,薄丝绸,上面绣着龙 \\
布克 & 火车公司董事,波洛朋友 & & \\					
康斯坦汀 & 医生 & & \\					
塞缪尔·爱德华·雷切特(美),真名卡塞蒂 & 死者 & 阿姆斯特朗绑架案的罪犯,事前知道自己有危险,死在开着窗户的车厢房间内,被十二刀刺死,有几刀比较浅,有几刀很用力;只抽雪茄,但尸体旁边有烟斗通条,损坏手表指向一点一刻 & \\
赫克托·麦奎因 & 死者曾经的秘书 & 抽香烟,六号卧铺,二等厢,午夜至一点半(阿巴思诺特上校作证;一点一记得至两点,列车员作证)	& \\
爱德华·亨利·马斯特曼(英) & 死者仆人 & 四号卧铺,二等厢。安东尼奥·福斯卡雷利作证午夜至凌辰两点。不太可能会说法语  & \\
皮埃尔·米歇尔(法) & 列车员 & 十二点三十七分,有声音从雷切特房里传出来时,波洛在过道见过他。一点至一点十六分,其他两个列车员作证 & \\
卡罗琳·玛萨·哈巴特(美)老太太 & & 死者车厢隔壁,三号铺,头等厢,哈德曼和施密特的证词可证明有个男人在她房间 & 帽盒,廉价手提箱,装满东西的旅行箱 \\
格丽塔·奥尔松(瑞典)太太 & 都会学校护士长 & 十号铺,二等厢,帽盒中不见了一些铁丝,午夜至凌晨两点,玛丽·德贝纳姆作证。她是最后一个见到死者活着的人 & \\
娜塔丽亚·德拉戈米罗夫公主 & 家住巴黎 & 十四号铺,头等厢,午夜至凌晨两点,列车员和女仆作证 & \\
希尔德嘉德·施密特 & 公主的女仆 & 八号铺,二等厢	深蓝色法兰绒睡衣,卷起来的褐色列车员制服,第三颗纽扣不见了,口袋里有万能钥匙,声称不是自己的,波洛推断是碰见的列车员放的;看到列车员,但不是已知的一位:又小又黑,长着一撮小胡子,声音柔弱,像个女人;	午夜至凌晨两点睡觉,大约十二点三十八分被列车员唤醒去公主那里,列车员和公主作证 & \\
安德雷尼伯爵(匈牙利) & 有外交护照 & 十三号铺,头等厢,午夜至凌晨两点,列车员作证,不包括一点至一点十五分这段时间 & \\
埃伦娜·玛丽亚·戈尔登贝格(原姓),伯爵夫人	& 20岁 & 十二号铺,头等厢,箱子上的标签都湿了,午夜至凌晨两点服台俄那(安眠药),睡觉,药瓶在她橱柜里,一块海绵,面霜,香水,贴着台俄那标签的小瓶子 & \\
阿巴思诺特上校(英) & &十五号铺,头等厢,抽烟斗,午夜至凌晨两点,和麦奎因谈到一点半,回房后没有离开过,麦奎因和列车员作证 &两支很重的皮箱子,一包烟斗通条(与死者身旁的相同) \\
玛丽·赫迈厄尼·德贝纳姆小姐(英) & 家庭教师 & 十一号铺,二等厢,在叙利亚的车上和阿巴思诺特上校说“不是现在,不是现在,等一切都结束了,等事情过去了”,拒绝解释原因,在斯坦布尔车站晚点,很不安,但因为谋杀时的晚点却镇静自若;午夜至凌晨两点,格丽塔·奥尔松作证 & \\
赛勒斯·贝特曼(B)·哈德曼先生(美)	& 41岁,假装打字机带推销员,侦探	& 死者让他陪同坐火车,抽香烟;可信;十六号铺,二等厢;提供证据:凶手是小个子,深色对皮肤,说话声音很尖细;午夜至凌晨两点从未离开过包房,列车员作证,除了一点到一点十五这段时间 &	几瓶烈酒 \\
安东尼奥·福斯卡雷利(意大利,美籍) &	福特汽车公司的代理人	& 抽香烟,五号铺,二等厢,午夜至凌晨两点,爱德华·马斯特曼作证,凶器的特点可能符合他的性格 & \\
\end{longtable}

\subsubsection{情节}

有疑点的事情:
\begin{enumerate}
    \item 有字母H的手帕是谁的?首字母涉及H的有:哈巴特太太、德贝纳姆小姐、公主女仆希尔德嘉德·施密特。
    \item 烟斗通条?
    \item 穿猩红色睡衣的那个人是谁?
    \item 假扮列车员的人是男是女?
    \item 为什么手表指向一点一刻?
    \item 谋杀的时间?是一点一刻吗?是更早还是更晚?
    \item 凶手有几人?
    \item 对刀伤如何解释?
\end{enumerate}

我看这部小说属于本格推理,于是将人物关系和线索都列了出来,结果最后其实是没啥推理,一车人都是凶手 ,感觉有点跌眼镜。不过,如果是第一次看到这样的故事,恐怕会为作者的狡诈而折服。
\begin{table}[tpb]
    \centering
    \caption{《东方快车谋杀案》时刻表}
\begin{tabular}{r|l}
时间 & 事件 \\
\hline
九点一刻 & 火车开出贝尔克莱德 \\
大约九点四十分 & 男仆准备安眠药,离开雷切特 \\
大约十点整 & 麦奎因离开雷切特 \\
大药十点四十分 & 瑞典太太看见雷切特(最后一个见到雷切特的人,雷切特在看书)\\
零点十分 & 火车开出温科夫齐(晚点) \\
零点三十分 & 火车陷进雪堆 \\
零点三十七分 &	雷切特的铃响了,列车员应门,雷切特用法语说“没事,我按错铃了” \\
大约一点十七分 & 美国老太太认为有个男人在她房间里,按铃叫列车员 \\
\end{tabular}
\end{table}

\subsubsection{同名电影2017}
电影很好,但失去了小说阅读的迷雾和刺激。