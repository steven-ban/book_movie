\subsection{《告别百年激进》}

作者:温铁军等

这本书是温铁军在不同场合下的演讲稿,时间跨度大概是从90年代末到2013年左右。书中的主要观点,和《八次危机》一脉相承。所谓“告别”百年激进,大体是讲中国百年来的革命和改革是激进的,比如从近及远,80和90年代激进的大下岗、建国30年年内的激进的引进苏联和美国资金、上山下乡,民国时期的激进的法币改革等等(这些内容均可以在《八次危机》《去依附》里看到),而现在我们提倡三农,提倡乡村振兴,则需要考虑之前“正确”的经验(即增加三农投入,增强农村这个劳动力蓄水池和社会稳定器的能力),作者给出的药方是“乡村建设”即梁漱溟、晏阳初的农村合作社化的道路,也是他们现在正在做的道路。

我不喜欢看演讲稿,因为一般来说内容虽然经过前期的准备和整理,但演讲的场合决定了其深度不如单纯的出书,更多是一种思想和观点的展现,未必有充足的论证。这本书里,作者的观点也都是他学术研究的展现,相对而言不是那么严谨。

温铁军是拿脚做学问的人,实地跑了很多地方,没有盲从国内外的意识形态说辞(主要是国内改革开放后的以及西方自由主义经济学),而是根据一般的经济学理论(如交易成本理论、成本转嫁理论、经济周期理论等)来分析国内外的经济问题,这在之前的《八次危机》里已经见识过了。这本书里的观点,也是反复在强调他的结论都是根据基本的经济学理论和实地的考察得到了,去除了意识形态化的说教。不过,去意识形态化能够更清晰地看出事件的脉络,看到意识形态和道德政治宣讲下的经济社会学的主要矛盾,但抛开意识形态可能也会带来简化和细节的缺失,因为有些政策,很可能就是在意识形态的影响下提出并实现的。

中国的发展,放在世界近现代史上来看,就是典型的后发农业国实现工业化和金融现代化的过程。目前,中国面临着西方发达国家在金融霸权上的打压。

对于三农问题,作者在不断地强调说中国的工业化过程是以城市为代表的工业化力量在通过投资等过程后,积累了经济和社会矛盾,通过向农村转移而度过了危机,否则就需要城市自己承受损失,不得不启动改革。农村经过二十世纪前半叶的土地革命,使得农民成为有产者(小资产阶级),能够承接城市转移来的风险,这是是中国能够实现工业化的关键。其他前殖民地国家则不存在这样的条件,因此始终无法发展起来。当然,东亚和东南亚的一些经济体也实现了土改,因此也有这样的条件来不断工业化。

现在有一个问题。那就是站在作者形成主要论点的2010年前后来看,下一步的经济三农改革应该如何实施呢?作者也承认,农民之前为中国的工业化做出了很大的贡献,承受了巨大的损失,但存在着巨大的农村-城市剪刀差和贫富差距?应该如何办?农民就得一直这样穷苦下去吗?作者给出的答案,是“乡建”即乡村建设,特别是借鉴日本农协那样的模式,保留农村的基本经济形态,通过合作社来掌握农村的资源(主要是土地资源)并进行经济开发,不能转让给城市大资本,以此以实现农村社会的稳定,防止农民变工人后形成失业隐患。作者还举例说,他在浙江、海南等地方开展了试验,借鉴了欧洲的“市民农业”模式,与大城市中的中产阶级结对子,吸引后者去农村包地种菜,这样干净卫生环保。我对这样的模式表示怀疑:这样小规模的所谓生态农业,究竟有多大的市场容量?能吸引多少城市中产?所谓的农协能不能像市场经济中的优秀企业那样准确地抓住市场痛点?分散的这种小投资有没有规模优势,会不会因此造成资金浪费?特别是所谓“农村生态旅游”,从这些年的操作来看根本就是伪概念,因此城市中产没有时间和精力去鉴别这些,农村的那些经营也是充满了落后观念,无法让城市人接受?甚至出现高价宰客、以次充好的问题。事实上,这些问题在之前的农村是很常见的。依靠这种所谓的乡建,实际上还是没有办法改变农村普遍的贫困,现在在看来,赶农民上楼、把农民改造成工人似乎是最不坏、唯一可行的方法。现在的城镇化,实际上是逐步地减少农业人口(中国将来不需要那么多的农民),把他们变成市民。至于温铁军担心的以后的经济危机没有了农村怎么“转嫁”的问题,在我看来这似乎是个伪命题:真到了那时候,就是得改革,把危机转移给其他国家(有难一起扛),在这个过程里要像发达国家那样不断提高社会保障,加强中央层面的调控。毕竟,90年代国企大下岗再惨,也惨不过农村的绝对贫困。

中央提倡生态文明,绿色文明,提倡环保,温铁军说这和世界上的农业生产是契合的,因为农村和农业生产就是生态的,就是和当地的自然地理有机结合的产物。对此我也表示反对。相比于渔猎时代,人类的农业生产,本身就是集约的、规模性质的获取食物和其他商品的方式,需要对大自然中原有的动植物进行育种,并且单一化地生产,这本身就是对生态环境的“破坏”。比如人类因为过度扩张农田,会导致其他植物和动物的生长环境缩小,甚至导致物种灭亡。之所以大家都觉得农业时代造成的生态灾难小且少,那是因为当时人类的生产水平较低,一旦人类真的铆足了劲去搞农业,那滥砍滥伐、水土流失都是家常便饭。比如某些文明的沙漠化、黄土高原的荒废、华北平原的水土流失等。农业生产相比于工业生产,对生态环境的破坏力是小了很多,但这绝不代表现在的农村那一套生产方式是生态的,是符合有机绿色的。解决农业爆产能危害环境,在我看来还是要靠工业化,比如大规模的水培、细菌产生蛋白,以此来减少对传统农业的依赖。

评分:3/5。