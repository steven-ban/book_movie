\subsection{《第二性》}

作者:波伏娃

\subsubsection{女性是“他者”吗?}
姑且不论“他者”在波伏娃所笃信的存在主义中是什么意思,对于每一个男人而言,女性就是无法到达的彼岸,是他者,这是无疑的。女性是男人出生的地方,也是男人为之纠缠一生的地方。男人或许永远无法理解女人,而男人很容易理解男人。从一出生开始,男人女人就因为生理性别的不同而被区别对待,长此以往,在他们发展出独立意识之前,就已经被塑造成不同的“物种”了。男人和女人在爱情里因为性别而讨价还价挑挑拣拣,在婚姻里把这种社会角色传递下去。

我是个男人,而且是个直男。直男不是贬义词。我当然可以用“社会词汇”来与女人交流,但我无法因为一个女人是女人而站在她的立场。对我而言,美丽的女人有柔美的身材,有飘逸的长发,有姣好的面容,有漂亮的衣饰,有化妆品和香水的香气,有人畜无害的行为。这自然是我追求的“他者”,是我的自由选择。泼妇和无趣的妇女,自然不被我喜欢,虽然她们也不会在意我是否喜欢。我不需要借助女人来实现自己的价值,因此爱情对我而言可有可无。爱情对于大多数男人而言,只不过是个责任,家庭的责任,这种责任是无趣的,甚至是沉重的。男人追求的不是女人,是权力,政治上的高位是权力,财富也是一种可以变现的权力。男人通过获取权力来获取女人,而难以依靠自身来获取女人和爱情,这或许是男人的可悲之处。男人在人生的长途跋涉里,始终是独自一人前行。女人可能会给予男人温柔和体贴,但给不了男人更多东西。

女性,或者说美丽的女性,在我心里似乎是至美的代名词。女性于我而言,是彼岸,美的彼岸。在潜意识里,我认为自己与美无关,甚至是丑陋的卑微的,于是永远无法抵达的唯一其他的性别成为我对美的心理上的投射。当然在事实上,女性并非都是美的,甚至说绝大多数女性与美根本无缘:她们肤色暗沉,头发发黄,身上有赘肉,大腿很粗,有口臭,举止粗俗……存在一个我想象中的至美的女性吗?当然不存在。这种美只存在于二次元,存在于图片中,它们展示出了美的一个方面,隐去了让人不适的丑的一面,因此成了躲避现实、躲避自我的对象。基于上述分析,问题的根源在于:为什么“我”不能是美的呢?为什么我只能仰望美欣赏美而不能占有美、并让美成了自我的一部分呢?我觉得,这当然是可以的,问题的关键在于理清我与自我的关系,接纳自我,并让理想中的自我实体化,让他一点点改变增加作为客体的一切优点。

\subsubsection{存在主义与人}
波伏娃在论述人的“超越性”“内在性”等一系列本质时,采用的是存在主义的观点和方法。她认为,男人有超越性,而女人没有,男人和女人组成的社会整体上也固化了这种印象和“规定”。我对存在主义没有什么了解,对于这种论述方式也无法评论。不过,波伏娃的基本观点是明确的,即使不借助存在主义的框架,也能归纳出来:社会意义上的女性是被构建出来的,不同的社会、不同的作家会有不同的构建,而这种构建都是男性视角的,都从男人的需求出发,但没有以女性为主体来进行。

第二章里论述了自然界里不同生物中雄性和雌性的差别,之后论述了原始社会、奴隶社会、工业社会(资本主义社会)中女性地位的演变。我认为从动物的雄性和雌性来类比和推导人类男女关系有失偏颇,因为尽管很多动物的“两性关系”与人类相类似(比如雌性都是负担大量的生育任务,雌性身体更弱,实行一夫一妻制等等),但动物之间相差巨大,花这么多篇幅来介绍动物在本书中并无必要,试图从动物的“两性关系”来研究人类也是缘木求鱼的行为。关于不同社会阶段里男女关系,我觉得东西方的差别可能比波伏娃认为的要大,而波伏娃只研究欧洲一部分国家的风俗,诸如女性有继承权这样的社会习惯在中国就不存在,因此这部分的论述虽然重要,但看看即可,切不能急切地推广到一切社会形态中。

之后是对神话和文学作品中女性形象的评论,用到了很多弗洛伊德的精神分析手法,对此我认为这只能作为一家之言,结论并不科学,看看即可,不能太当一回事。当然,波伏娃的分析还是很锐利的,往往能一针见血直击要害。因为这一点,波伏娃不去做文学评论真是太可惜了。

\subsubsection{宗教}
基督教对欧洲的影响就不需要多言了,本书里也涉及很多宗教与女性关系的论述。这些描述,自然有其局限性,因为世界上的宗教很多,伊斯兰教和犹太教对社会规范和对女人的定位就与基督教不同,像东亚文化圈里对女性的规范也与强宗教国家不同。

女性对天主的虔诚,与女性对情人的感情,在波伏娃笔下有着很大的相似性。波伏娃认为,女性其实是在情人身上寻找一个天主,从而满足自己的依附性,随之而来是大量对女性爱情心态的描写。

\subsubsection{爱情}
波伏娃对女性的爱情观进行了详细的论述。在她的时代,波伏娃认为女性没有独立经济地位,没有实现超越性,因此对男性是绝对的依附。女性在男性身上寻找超越性,从而成为这个人的奴隶。这种现象在现代以前是普遍的,无需多言。

那么如何打破这种关系呢?波伏娃认为,女性应当有独立的经济地位,有独自的人格,像男人那样实现自我的超越,,像男人那样获得主体性,以一个平等的姿态去和男人开展爱情关系。她这样的想法,是很正确的。当然,我们在18世纪以来的欧美文学里,她常常会看到这样的情节,如《简爱》里简对情人的拒绝。

让女性获得独立的经济地位,在波伏娃的时代还不算广泛,二战以后欧洲的女权主义进程,大大加速了女性加入职业大军的速度,以至于现在男女从业比例基本相当。甚至是东亚国家受到女权主义或共产主义的影响,男女平权也大大加速,中国女性的就业率就相当高了。其实现在在中国,女性的就业虽然有各种歧视,但基本上实现了同工同酬,男女可以平等地进行爱情和婚姻关系了。

\subsubsection{精神分析}
《第二性-II》完全是用精神分析的方法来分析女人不同年龄阶段和不同角色的女人心理的。

关于女孩(指从出生到进入青春期时),她们意识到自己没有阴茎,产生男女有别的心理;来月经时的不安与羞耻,对男人的憎恶和躲避……形成了一种矛盾的心理:接收自己与否,长大的期望。

波伏娃在认证女性的心理变化时,大量采用了文学作品和精神科医生提供的案例。这些案例往往隐含着“不正常”与“病态”。我认为,使用这些不够严谨的论据建立起来的理论无法有较大的普适性,特别是极端的案例都在“佐证”波伏娃采用精神分析手法确立的结论。因此,从这个角度上讲,《第二性》更像是一本文学作品,而非社会科学作品,它带有明显的作者的个人痕迹,不够客观。

当然,波伏娃的观察是细致的,即使是基于有限的样本的心理体验,这种观察也是直击人心的,最起码,作者给出了女人很多不幸的现象和原因,够大多数女性去发现自己,作为一面镜子来反省自己的内心(即使自己的情况与波伏娃描绘的情形不同)。

\subsubsection{女性的角色}
波伏娃花了大量篇幅来介绍女人的家庭生活,特别是作为女儿、妻子和母亲的心路历程。在波伏娃的时代,大多数西方国家的女性都被困在家里,缺少职业生活,因此她们的生命就处在一种重复的封闭的状态。从这种男女社会分工的角度出发,很容易就得出波伏娃纵贯全篇的男女之间“本质与非本质”“超越性与???”的区别。因此,从波伏娃的时代来说,女性必须走出家庭,像男人一样直接与社会接触,才能解放自己。从战后欧美社会的发展来看,女性确实取得了这样的进步。

对女性认知更普遍的一种偏见是,女性蕴含着“母性”,女性因为母性而伟大,而得以和男性(至少)等同。在《第二性II》的“母亲”篇里,波伏娃详细讨论了女性在受孕、分娩和作为母亲的一系列的心理变迁,得出结论“不存在母性的‘本能’”,母亲的态度是由她的处境来决定的。女性和母性,本来就是两种东西。这种见解振聋发聩。毕竟在中国社会,目前还是把母性捧得很高,借助讴歌母性之美,或明或暗地把女性区别于男性的。

波伏娃的意见,男人女人看了都会有启发,但是更重要的是,女性要主动去看,去接受,去反思,去争取。女性至少要存在心理上的自觉,才能在目前提倡“女德”的环境下,保留自己的一份精神独立。

\subsubsection{“第一性”}
这是一本讲述女性的书,然而世界上只有两种性别(暂时不讨论双性人等极少数群体),只谈女性不谈男性是谈不好女性的。这本书里,按照波伏娃的存在主义和精神分析手法,同样对男性也有相对应的分析。

然而,相对于波伏娃对女性分析的细腻和繁复,对男性的分析只过于简单和浅薄。波伏娃采用了大量的“男性是主动的、内在的、本质的”这样简短的语言来概括男性,缺少对男性不同类别和不同阶段的详细剖析。波伏娃也对男性的心理进行了论断,如男性的自我实现、对本质的追求、对世界和女性的绝对操控、性行为和性心理等,但每个方面都是一带而过,因此几乎所有的上述论断都失之于偏颇,甚至是错误的。不知道她是否了解男性也有很多种,也有不想追求本质的,也有拒绝实现自我的,甚至有拒绝成为男性的。这些人其实还很多。男人在和世界的对抗中,放弃那些追求而彻底堕落和自暴自弃的绝对不是少数,但波伏娃似乎没有看到这一点。很多男人对身边的女性是爱慕的甚至乞求的,对于“第二性”他们渴求她们的理解和接纳,甚至因此而放弃自己的惯常的性别认同。如果像波伏娃这样粗暴地认定男性是第一性而女人是第二性,依据同样的逻辑,男人里也区分着“第一性”和“第二性”,同样的,女人也可以有着传统价值里男人的那些特质。事实上,简单粗暴地对男人和女人下定义,除了生物学意义上的可以被广泛地接受,其他任何的基于已经构建的社会学理论的区分都不可能是准确的。

这本书里,波伏娃创造了一种男性与女性的区别与对立,并且极其粗暴地把男性和女性描绘出完全不同甚至是相反的两个社会生物,从而创造出一种主体与客体、内在与超越的对立关系。在她的这种描述中,男性与女性之间如同仇雠,势同水火,永远没有和解和包容的可能。作为一个接受过科学训练的人,我对波伏娃这种论据均源自文学作品和”我有一个朋友如此说“的叙述方式深感抵触。本书中的实例即使都是客观的和真实的,但通篇下来都是女性的悲惨和男性的跋扈,看不到男女之间的理解和相互接纳,也看不到女性的快乐和男性的谦卑,更看不到作为”第一性“的男性在这个世界上踽踽独行的悲壮。如果女权主义想成为严格的社会科学,那么就需要像普通社会学那样,有充足且全面的调查,有量化分析,有统计数据,而不是拿一些片面的事实来论证自己的观点,否则只会越来越狭隘,堕入浅薄的泥淖。从这个意义上来看,《第二性》只是一种文学性和观察性出彩的个人随笔,而不能称为严格的社会学和心理学著作。多说一句,《第二性》里展现出来的自以为是和狭隘,和现在网络上上蹿下跳的”中华田园女权主义者“有异曲同工之妙。

\subsubsection{女性化与女权主义}
波伏娃在《第二性》里反复提及的女性的解放,首先是女性要以独立、职业的面貌取代以往附庸、主妇式的面貌,像男人一样通过工作来认识和改造世界,从而取得平等的地位。这一点我也认识到了,事实上这正是百年来女权主义发展给女性带来的巨大进步,在这个社会,取得六十年前波伏娃的上述认识,并不需要费很大力气。

但是,是不是女性取得经济的独立与平等就可以获得解放了呢?我之前较少地思考这个问题。我倾向于把这种独立当作一种个人的自由选择。当一个女性选择了工作和经济独立,那么她势必需要放弃一些东西,如传统上对女性的扶助、过于”女性化“的妆扮等等。我考虑到,现代的资本主义宣扬的消费主义、娱乐主义和享乐主义正在物化和诱惑女性,使她们大大提高了自己的物质和享受欲望,从而明显高于自己的能力,因此很多女性不得不顺从自己的欲望转而求助于男性,成了新的一批附庸。鉴于此,我认为女性的解放应当摆脱这种诱惑,重新回到内心,合理支配自己的欲望。

但我没有从一个女性的角度上思考。波伏娃的论述让我吃了一惊:走了上面第一步(即工作和经济独立)的女性的身份是害死的,她们的传统角色与个人的新角色并不统一,但女性并不必然放弃自己的传统形象(如美丽、女性化等)而成为无性人,女性可以兼得两者。这一点我实在没有想到,或许我低估了女性自由选择的能力和自制力,她们可以在这个资本主义的时代活得既独立又漂亮,同时又可以是一个独立的自由的人。她既可以享受资本主义带来的持续快感,也可以冷静时做形而上的超越。似乎这样的女人,才应当是大多数。那么我在”要求“女性独立时,同时也可以允许她们接纳传统了,两者并不是二选一啊。

\subsubsection{波伏娃其人}
在本书里,波伏娃用深邃的笔触书写了女人的精神史,这个精神史有两层含义:既有一个女人从出生、童年、少女、婚姻、生育各个阶段的精神脉络,也有女性整体从生物、原始社会、庄园时代、资本主义时代的整体变迁史。波伏娃展示了一种对女性进行精神分析的框架,为女性画像和解剖。

女人苦,这在不少女人那种都有强烈的共鸣;男女不平等、不公平,也是很多女人和男人的共识。但是,为什么女人会成为这个样子,为什么男女的分野会这么大,仅仅稍作思索的人并不能给出详尽的答案。庆幸的是,波伏娃给出了一个相对比较圆满的解释。
波伏娃的笔端带着一股怨气,它反映出波伏娃内心对女性身体的厌弃(我觉得波伏娃自己讨厌成为一个女人,甚至憎恶自己作为女性的身体)和对男性的反攻姿态。她恨男性(本书的这个形象似乎让我想起萨特和她之间的关系)给女性的冷漠和无视。抛开波伏娃自身的体验,这种情况即使存在,但也没有波伏娃描绘的那么普遍(波伏娃把它描述成一种共性即女性的本质)。在本书里,女性似乎天生委身于男性并受困于男性,但波伏娃显然漏掉了男性对女性的追求(不知是否因为她没有被人追求的经历)和奉承,也漏掉了女性在上而男性在下的情感和婚姻中的权力关系。比如,中国人里常见的“妻管炎”“惧内”情形,波伏娃就没有做出有力的论述和分析。我隐约觉得,男女之间的“第一性”“第二性”关系,在更为本质的层面上,需要让位于男女之间的权力关系。

因此,我认为每一个女人,都要读一下这本《第二性》,尽管它有一些晦涩,有一些狭隘(特别是对于男性的分析,我觉得女人如果相信波伏娃在这本书里对男人的分析,有很大概率成为“仇男主义者”甚至“直女癌”),从中发现自己未能体会到的女性背后的社会和精神渊源。读完本书的女性,未必会立刻投入到男女平等和女权主义的大军里,但她们至少已经有了一种觉知,能够看清这个社会依然借各种“女德”“母性”等温柔面纱下对女性的桎梏,也能够认清消费主义和享乐主义横行的今天为女性设置的各种陷阱。至少,本书的女性读者能够活得更明白。从这个角度来看,本书称为“女性的圣经”一点不为过。

同时,男性也应当看看这本书。男性与女性之间,始终有一条巨大的认知上的鸿沟。女性之间有着更为类似的生命体验,因此女人之间有着更好的相互理解的基础,但男人没有。在目前的社会里,男性对女性的无知(如果不是傲慢的话,这根本不值得一说)可谓巨大,他们之间虽然具有语言、爱情、婚姻和性方面的交流,但男性似乎更关注女性的语言和物质诉求,听信女人的唠叨,却没有兴趣探究女人的精神。很多情人只能满足女性的物质需求(并且很多女性更看重男性的物质能力、权力、男子气概和温柔、体贴等少数几种性格),很多丈夫认为对妻子提供一个物质充裕的环境就是对女性的回报,他们在认知上堕入“直男癌”的深渊。而媒体和一些“情感专家”、公共知识分子似乎也愿意鼓吹男女的差别而非共性,为男人和女人贴上鲜明的标签,为男女不平等、不公平的社会和家庭环境戴上温情的面具,将女性继续引诱向“第二性”的道路。以上这些行为未必是哪个男性做出的,而是男性和女性的合谋。借助理解和沟通,男性和女性之间应该达成一种包容和和解,而非激烈的对抗。

波伏娃在《第二性》中大量解读了文学作品,剖析了这些文学作品中的女性观和女角色的心理;另外,波伏娃还剖析了大量的女作家。对女作家这个群体,波伏娃整体上评价不高,认为她们过于自恋,只是把艺术作为一种消遣和展现自我的工具,而非一种沉重的追求;由于受视野所限,女作家们的文学成就远远比不上那此最伟大的男艺术家们,实际上世界上最伟大的艺术家也都是男性。女性作品流于表面,多是”畅销书“。这一点我比较认同。波伏娃之前的女作家也好,女艺术家也好,水平确实都非一流,比如之前读到的凯特-肖邦的女权主义小说,包括《她的国》,虽然有一些灵光一现的想法,但写作技巧非常差劲,思想深度也不高,只是作为一种文本具有一定的艺术或思想上的有限的意义。可见,波伏娃是一个有深度的文艺评论家。

本书的主要内容,可以从《第二性II》最后的译后记来快速了解。

总之,本书采用存在主义和精神分析的工具,搭建了女性心理分析的框架,虽然细节上有很多争议之处,但仍然足够细密精深。

评分:5/5。