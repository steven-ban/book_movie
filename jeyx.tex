\subsection{《饥饿游戏》}
\subsubsection{关于原著小说}

1.革命不是请客吃饭,也不是真人秀,但在小说里却是。小说意在讽刺电视娱乐,顺便黑了一下革命宣传,前者在第一部和第二部里是主线,后者在第三部里才图穷匕现。前两部里我期待革命者会发动苦大仇深的人命推翻凯匹特统治然后建立人民民主专政政权实现共产主义……哦不,是资本主义,但其实革命者和反革命者的套路相同,宣传工具和手段如出一辙,只不过凯匹特用恐惧的杀人游戏,革命者用高大全的革命文艺。到了凯匹特的总统府对决,革命者不惜炸死总统府前的儿童并嫁祸总统来达到自己的“革命”目的。一切革命都是反革命。看来科林斯不仅反对娱乐至上,更反对乌托邦,甚至反对一切政治,把所有革命宣传事业黑了个底朝天,这可是青少年文学啊,美帝的花朵看这种东西真的没关系吗?还能开心地投选票吗?

2.因为是给青少年看的,所以情节比较简单,也比较拖,但第二部太行文太仓促。心理描写一大段一大段,反复无常,看得人昏昏欲睡。

3.情节漏洞太多,不知道作者是不是对社会不熟悉还是故意要把社会描述地简单一些,整个帕纳姆的经济不合理,科技很发达但产业很落后,比如都能用基因工程改造动物了还用得着用人力搞农业渔业和矿产吗?

4.恋爱就是请客吃饭,但凯特妮丝和皮塔的感情,一直没看懂。皮塔为什么喜欢上凯特妮丝就不说了,凯特妮丝对皮塔的感情让人琢磨不透,凯特妮丝你真的这么高贵冷(leng)艳(dan)吗?另外对盖尔的感情让人缺乏共鸣。其实,与书中政治的弱智相比,几个主人公的年龄写得都有点大了:盖尔才多大啊去当游击队长(二区鄛匪司令?)?

5.翻译显的语言显得稚嫩,好多地方直译,不知道是不是也是要给青少年看?

\subsubsection{关于改编电影}

大表姐的表演还算可圈可点,第一次看是2013年年底在北京看第二部,特别惊险刺激。 
2017年3月8日终于刷完了最后一部,把这个系列划上了句号。 
电影不错。 
可惜电影最后生生把Snow洗白了,成了Kitness的知音。杀死了总统于是就实现了民主,这设定也太幼稚。