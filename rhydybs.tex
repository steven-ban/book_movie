\subsection{《如何阅读一本书》}

标签: 阅读技巧

作者:莫提默·J·艾德勒 查尔斯·范多伦

本书初版于1940年,再版于1972年。

\emph{阅读越主动,效果越好。}阅读的目的可以仅仅是\emph{获取资讯},也可以是更深层次的\emph{获得更好的理解}。

阅读具有不同的层次,努力越多,效果越好:

\begin{longtable}{p{0.08\textwidth} | p{0.3\textwidth} | p{0.3\textwidth}}

    \caption{阅读一本书的层次} \\
    \hline
序号 & 层次 & 含义 \\
\hline
\endfirsthead

(接上表) \\
序号 & 层次 & 含义 \\
\hline
\endhead

\hline
\endfoot

1 & 基础阅读(elementary reading) & 认字 \\
2 & 检视阅读(inspectional reading) & 这本书在讲什么?包括哪些部分? \\
3 & 分析阅读(analytical reading) & 通盘阅读,追求理解 \\
4 & 主题阅读(syntopical reading)或比较阅读(comparative reading) & 架构出书里相关主题的分析,可能涉及多本书 \\
\end{longtable}

这本书之前一直被人推荐,于是买过来打算认真看,然而看了两三章就觉得啰嗦乏味。这本书里讲的阅读的不同层次,其实就是我们中学学习的语文和英语课里的速读或者精读、提取段落大意这种事情,只不过作者用西方人的方法又讲了一遍。任何一个受过九年义务教育的人,应该都具备这种能力。当然具体的阅读习惯和阅读对象可能存在差别,比如现在的信息过载、垃圾信息太多、碎片片信息太多等等,读者往往是一目十行,并没有耐心去深究背后的意味。从网上反馈的情况来看,很多买了这本书的人也觉得这本书的内容十分啰嗦重复,没有什么有价值的信息。于是,我放弃了这本书的阅读。

评分:2/5。