\subsection{《只爱陌生人》}

作者:【英】伊恩·麦克尤恩

这本书是作者刚出道时的“力作”,根据本书的译后言,它也是作者的短篇代表作。然而,这本书的水平,实在是一言难尽。

这本小说本质上讲了一个“猎奇”的故事:
\begin{itemize*}
    \item 威尼斯的一对夫妻罗伯特和卡洛琳是外交官的后代,他们之间是SM的关系,丈夫毒打妻子,打断了肋骨,打落了牙齿,甚至丈夫要杀掉妻子;
    \item 丈夫罗伯特偶然之间发现了前来旅游的情侣科林和玛丽,其中科林长得有一种阴柔美,被罗伯特看中,于是他疯狂地跟踪他,拍了无数照片,并贴满在床头的墙上“欣赏”;
    \item 罗伯特接近科林和玛丽,在两人迷路瞎逛的情况下引导到自己的酒吧,向他们讲述自己的故事;
    \item 据卡洛琳事后对玛丽说,有夸大,内容大致是罗伯特生活在一个外交官家庭中,父亲严肃古板,不让儿子吃甜食,不让女儿化妆,一旦发现就对他们进行毒打,这其中罗伯特告发了姐姐们化妆,而姐姐们则设计陷害他吃下甜食巧克力和催吐效果的饮料,并把他骗到父亲引以为生命和尊严的书房,事后罗伯特被毒打,罗伯特年龄不小了还和母亲睡,这被姐姐们嘲笑,而被初次见面的卡洛琳赞同,于是罗伯特对卡洛琳十分认同,两人很快结婚;
    \item 罗伯特把科林和玛丽诱骗到自己家里,在他们熟睡时欣赏科林的“美体”,事后带他们欣赏自己引以为傲的“博物馆”(神父和父亲留下的物品),这其中他偷伯科林的照片被玛丽见到,但未能识破。卡洛琳的腰背部有被打痕迹,被两人注意到。罗伯特打了科林一拳,科林没有反抗,似乎罗伯特默认了科林的“抖M”潜力。
    \item 回到旅馆后科林和玛丽似乎受到了某种刺激(未必是直接的发现了罗伯特和卡洛琳的SM,而是他们夫妻那种特殊关系流露出的SM独有的情色意味),一改之前性爱上的乏味与无趣,做爱更加热烈,两个内心似乎诞生了某种SM情结:玛丽想把科林四肢砍掉供自己淫乐,而科林则想把到丽束缚住并用性爱机器人一直抽插到死。
    \item 玛丽从旅馆外望向旅馆阳台,突然想起罗伯特那张照片拍的就是科林。但两人似乎没有怀疑夫妻俩有什么更深的图谋,还是在一天回到了罗伯特的住所。罗伯特支开科林,和他一起去酒吧,路上拉着他胳膊,当成自己的同性恋人并向熟人们展示。玛丽则被卡洛琳下药,卡洛琳向她展示夫妻偷偷拍下的科林的大量照片,玛丽发现真相,但因下药无法反抗而睡着。
    \item 从酒吧回来后,罗伯特和卡洛琳在住所当着虚弱的药劲没下的玛丽割了科林的腕,卡洛琳把血涂到科林的嘴唇上,罗伯特去吻科林的嘴,两个最终将科林杀害,似乎还奸了他的尸体(暗示)。
    \item 玛丽被送到医院,后来指认了科林的尸体。
\end{itemize*}

总之,故事情节本身没什么意思,品味低下,充满猎奇和窥探的恶俗气息,完全配不上这本书和作者的名头(与《水泥花园》真是差太远了)。虽然有一定的悬疑意味,但手段并不高明,特别是故事线主要由后半段推进,科林和玛丽“二进宫”之后有点仓促,失去了前半部那种迷离的颓废的美感。至于作者这种对SM、同性恋有描写有多少“社会学”意义,或者多么大胆,在现在看来根本就是商业上的媚俗而已。更何况罗伯特对卡洛琳更多是一种家庭暴力,而非现在认可的点到为止、不能对身体产生伤害、要签协议的合理的SM。作者还穿插了女权主义、科林和玛丽吸大麻的情况,带有明显的六七十年代的嬉皮士意味,很有年代感。科林和玛丽对女权主义的“批评”现在看来则是很一针见血的。

这本书最让我喜欢的,还是前半段故事没有明显推进时科林和玛丽在旅馆做爱和在威尼斯闲逛时的百无聊赖感,此时他们在一起已经七年,玛丽有前夫,有两个孩子,两人未同居,只是来威尼斯旅游,他们的感情处在一种疲惫、烦躁和不安中,貌合神离,一个说东一个说西,谈话浅尝辄止;威尼斯他们不熟悉,走来走去找不到吃饭的地方,特别是那段玛丽走很长时间渴得要命、在咖啡馆里服务生也不上水的描写可谓犀利。麦克尤恩最让我称道的,就是这种对生活的锋利的解剖,对大量的密不透风的生活细节的罗列。没错,这本书的精化并不在上面那种猎奇情节以及“社会意义”的揭示,而是这对情侣在旅游时的分裂状态。

评分:2/5。
