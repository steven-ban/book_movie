\subsection{《男人这东西》}
\subsubsection{感想}

1、看完渡边淳一的这本《男人这东西》,深深为日本男人感到悲哀:男主内女主外的经济状态很难改变,工作压力极大,与妻子的交流产生了莫大的障碍,导致性功能低下。

2、作者描述的是日本,想必是对日本社会、尤其是婚姻家庭有大量的观察总结才做出的,即使作者没有做严谨的社会学调查和数据统计分析,但身为日本人,又是作家,对日本社会的观察应该还是站得住脚的。但是,这种观察结论能否适用于中国社会呢?我想这是需要打一个问号的。比如,日本普遍(根据作者的语气,应该90\% 以上)男主内女主外,家庭主妇相当普遍,但在中国,男女共同工作(早些年的“双职工”)现象相当普遍,因此男女经济社会地位相对而言更对等,因此像日本的那种夫妻双方分工明确的现象没有那么严重。

3、日本男人可怜的自尊心:日本男人认同更强,因此拼命想做到更强,但竞争激烈,大部分人仍旧是平庸的,这产生了大量的沮丧情绪,而日本男人在家里又想维持一种高高在上的心态,对妻子强迫命令,这各矛盾心态造就了日本男女之间那种尊卑关系。虽然日本经济发达,女权运动比中国要早,但男女截然不同的经济命运和强大的心理传统造成了男女之间的差别越来越大。

4、作者认为,男人的生理结构决定了只有在“高高在上”的情况下才能勃起并有性能量,而现代社会的竞争机制使以往男女之间的能力差异迅速缩小,甚至在少年时代被女生吊打,因此男人的自尊心受到极大的“损害”,从而导致性功能障碍。失意的男人为找回“自尊”,在婚恋时更希望找到愿意做家庭主妇的人来弥补这种挫败感,因此男主内女主外的传统更难打破。

5、在作者笔下,男人是脆弱的可怜的,特别是处于激烈竞争环境下的男人,需要和女人做心理上的比较和对抗,忙于工作,逃避家庭,任凭自己在酒精的刺激和烟花女子的抚慰下堕落。对此我只想说:你们不能改变吗?这种稳定的日本社会不能发生自发的变化并朝向对男女双方都有利的方向做出修改吗?作者的答案似乎是:不能,我们日本人难以改变。

6、日本的女权状况不会比中国更好,这不是由经济发展水平决定的,而是由强大的传统和习俗决定的。同样,本书的分析未必适用于中国社会。中国的职业状况,会朝着西方学习,而不会退回到男人必强女人必弱的旧时代去,因此中国相比日本,更接近西方。

7、在作者笔下,日本男人尽管可怜,可又相当固执,非要在女人面前保持好胜心,用比较的方式来证明自己的优越和对方的差劲。亦即,日本男人的心态是:你有各种各样的缺点,因此就别在我面前逞强了,乖乖臣服于我的命令吧。当然,可以想见,日本女人也会有同样的心理。这样,男女双方都在一种”比比谁更差劲“并且以羞辱对方并打击对方自尊心的方式来交流的。想想这样的交流方式其实相当可怕。我不知道日本社会人与人的关系是否都如此,然而根据作者描述,日本的男女关系就是对上述心态的演绎。因此,日本男人要么不是直男癌,如果是,一定比中国男人更接近晚期。

8、作者鉴于男女关系在一夫一妻制对男女双方个性的压制,极力推崇日本平安时代的”走婚“形式,即女性保持多个男性伴侣的婚姻。虽然作者也说日本的社会不允许这种婚姻,但在我看来,即使允许,恐怕也不能解决男女关系的种种问题。走婚并不是多么稀罕的东西,中国西南某少数民族即有这种形态,但婚姻不仅仅是男女关系的问题,其实是一种经济形态,是在保证血缘关系前提下最灵活同时最稳定的经济小集团。走婚,或者不婚,虽然可以保证男女在年轻时的性欲望得到最大满足,但一旦老去,将难以解决晚年的生活问题。概而言之,只有一夫一妻制,才能以最小的成本来保证双方一生的生活水准。

\subsubsection{一些标注}

希冀爱情的人最好对结婚打消期待,而希冀结婚的人最好对恋爱打消期待——

男人和女性属于不同的群组,从本质上说是无法真正相互理解的,

在近代小说中,恋爱的精神性被推到了最前沿。

近现代的男女之爱表现出浓烈的禁欲主义色彩,

(日本)即使是因妻子感情出轨而离婚,丈夫也不得不支付高额的赡养费,而如果因丈夫有外遇而离婚,妻子则只需给极少的赡养费。

女性的筑巢本能强于男人

夫妻之间保持一定的距离感、紧张感,有利于彼此保留一份新鲜感,双方互需互求,这样或许可以减少男人的花心。

没有血缘之亲的父母和子女之间,永远都不可能真正理解,

但对男人来说,他们所期待的家庭无非是一个不必设防、能够彻底放松身心的场所,失意时能够得到抚慰,屈辱时能够得到鼓励,使其第二天继续鼓足干劲儿投入到工作中去。

男人本身是一个非常社会性的存在,

女人想最终称心如意地得到心仪的男人,最有效的办法莫过于抓住他的心理脆弱之处。
男人如果失去友情便意味着被男性社会屏弃在外,最终成为一只失群的孤狼,这也意味着男人将彻底被这个社会抛弃。男人早在少年时期就开始体会到这种感觉,所以在男性社会中确保自己的位置对他们而言,同珍视自己的妻子或女友一样重要,绝对是不可或缺的。 这种倾向在日本的男性社会里尤其明显,如果对男人的这种感觉不能理解,也就无法真正地理解男人和女人。

对那些依靠女人养活的倒贴男人,社会上一般都称之为“吃软饭的”,竭尽蔑视,而反过来对于依靠男人养活的女性,则呼之为“妻子”,正儿八经地予以认可。

与女性相比,男人在性爱中所得到的快感是一种浅层次的快感,而且不可能随着精神之爱的逐渐深化而加深或变得更强烈,故此对男人来说,同某个特定的女性的性爱在他们身上很难刻下深刻的烙印。