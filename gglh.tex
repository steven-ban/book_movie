\subsection{《两个故宫的离合:历史翻弄下两岸故宫的命运》}

标签: 故宫 \  文物 \  台湾 \  政治

作者:【日本】野岛刚
该作者是日本《朝日新闻》驻台湾的记者,经常采访台湾高层,因此得以获知比较真实的信息。

《两个故宫的离合:历史翻弄下两岸故宫的命运》封面:\url{http://collection.sinaimg.cn/cjrw/20140422/U5566P1081T2D149736F7DT20140422074105.jpg}

\subsubsection{文物中的近现代史}
故宫文物的变迁其实是中国近现代史的演变。故宫本来是皇帝的家,在明清时代普通人不能进入,宫廷的文物是皇帝的私人财产,不可能开放给普通人参观。然而随着辛亥革命的到来,帝制终结,共和到来,皇帝不再高高在上。作为皇帝优抚条件和革命不流血的条件,民国决定保留皇帝和紫禁城,同时给予生活费。但毕竟帝制已经终结,皇帝的开销依然很大,于是皇帝身边的工作人员甚至皇帝本身开始变卖文物,故宫文物开始流向民间,有心的欧美人和日本人开始收集。后来溥仪被冯玉祥赶出紫禁城,他之后投靠日本人,生活条件更加窘迫,于是变本加厉地变卖这些文物。1925年故宫博物院成立,故宫成为普通人能够进入的景点,文物也随即开放给公众。

1931年日军开始侵华,抗日战争爆发。为保护文物,工作人员开始将文物打包转移。在整个14年搞战中,这些文物被运往南京、武汉、西安、成都、乐山等地,辗转万里,文物保护人员克服了无钱无粮的艰难局面,较为完整地保存了这些文物。这一段历史看得人热情澎湃,我想每一个中华儿女都会通过这些文字激发对国家民族的热爱,以及对这些文物保护人员的崇敬。

1945年抗战胜利,文物开始返回南京,随之而来的并非和平,而是国共内战的爆发。随着共产党的节节胜利和国府的节节败退,蒋介石开始转运文物至台湾。1949年新中国成立,定都北京,文物大部分运住北京故宫博物院,有两千箱留在了南京博物院,南博多次以多种理由留下这些文物,不愿给北京故宫,两者矛盾甚至闹到了中央出手。随着各地考古发现以及其他国定级文物的出土,故宫博物院充实了馆藏,不再局限于皇帝私藏内容,而成为中华文明的综合性博物馆。

\subsubsection{台北故宫的改变}
台北故宫的藏品来自于1949年国共内战中国民党势败,于是蒋介石将大批藏于南京的文物用军船运到台北,等待借着“反攻大陆”的时机并“还于旧都”。可惜,天不遂人愿,蒋到死都没等来这一天。蒋借着这些国宝,试图确立自己才是“中华正统”的地位。作者在本书中也不断强调这一观点:对于中国人而言,文物不仅仅有历史意义,还有政治意义,统治着需要凭借这种正统来确立权威。随着中国大陆“文化大革命”的进行,国内文物受到破坏,蒋介石反其道而行之,更借着回归传统的明义来增强“正统”地位。他更多是把文物保存起来,而非让大众去观看学习。

台北故宫:\url{http://www.emswxw.com/UploadFiles/20111215/20111215155821731.jpg}

随着两蒋时代的结束和台湾民主化的进行,“正统”的思想越来越淡薄。国民党更多是靠着这些文物来确立自己是中华文化传承者的地位,而民进党一直想独立,因此在意识形态上拼命去除“中华正统”的观念。民进党认为,中华文化顶多是台湾文化的一部分,不应强调台湾应当“中国化”。陈水扁上台后,几任台北故宫博物院的院长都淡化中华文化色彩,试图走“多远化”“亚洲化”路线,甚至想在南部建分院。2008年马英九上台,立刻中止了这种行为。

总之,台北故宫博物院的变迁,是台湾政治化变迁的一个缩影。

台北故宫珍品:清代玉雕“翠玉白菜”:\url{http://1804.img.pp.sohu.com.cn/images/blog/2013/2/5/17/24/u79583889_13d6d8a3e5fg86_blog.jpg}

\subsubsection{文物回流}
80年代以来,中国国力逐渐增强,相应地国内政府和民间开始考虑通过政治和经济方式讨要或回购文物,本书对这一方面也进行了相应叙述,但干货不多。

\subsubsection{两岸故宫的交流}
2008年,马英九上台,开始与大陆方面进行频繁的对话交流,其中也包括两岸文物界的交流。新上任的台北故宫院长与北京故宫方面开展非正式和正式的接触,商谈文物共同展出的事宜。

在这一方面,本书作者作为记者,开始采访两方负责人。两岸政治上存在相当敏感的地方,因此发言谨慎甚至有话不说的状况很容易理解。但本书作者脸皮很厚,故意问一些让双方不宜公开商谈的话题(如“两岸故宫合并”),显得很龌龊。

日本人很希望两岸文物能联合在日本展出,但遇见了政治上的阻碍,至今未能成功。\emph{司马辽太郎}于90年代劝李登辉同意NHK记录片不出现“国立”二字,终于成功,使得同时出现两岸故宫的记录片得以播出。200X多年日本的\emph{平山郁夫}力促两个故宫的文物同馆展出,大陆方面比较务实,同意了这个请求,但台湾方面依然没有同意。同时,日本国内也需要通过文物展出相关法律来打消台湾方面的疑虑,他们担心大陆向日本方面提出返还文物的要求。但是日本政局在2010年后出现了振荡,相关法律未能及时通过。

\subsubsection{书评}
本书书名是“两个故宫的离合”,内容也主要是故宫文物的变迁。但本书真正的干货来自于作者亲身经历的采访,这些采访也是主要针对台湾方面负责人的,很少涉及大陆方面,更少涉及历史上文物变迁的第一手资料,因此内容并不充实。而在于作者采访的台湾方面的内容,也是局限于微观层面,较少猛料,阅读乐趣并不大。

本书的语言还算流畅自然,这大概和日语翻译界普遍的水准较高有关系。

评分:6/10。