\subsection{《沉默的大多数》}
\quote{缺少科学知识,没有想象力,这都是中国出不了科幻片的原因},这么多年过去了,王小波的说法还是很正确。

尤其是理科的男学生,肯定希望在校园里出现一些表演系的女生??这很有必要。

中国传统的士人,除了有点文化之外,品行和偏僻小山村里二十岁守寡的尖刻老太婆也差不多。

中国有种老女人,面对着年轻的女人,只要后者不是她自己生的,就要想方设法给她罪受:让她干这干那,一刻也不能得闲,干完了又说她干得不好;从早唠叨到晚,说些尖酸刻薄的话--捕风捉影,指桑骂槐。

第三个假设是凡人都喜欢有趣。这是我一生不可动摇的信条,假如这世界上没有有趣的事我情愿不活。有趣是一个开放的空间,一直伸往未知的领域,无趣是个封闭的空间,其中的一切我们全部耳熟能详。

一部《情人》曾使法国为之轰动。大家都知道,这本书的作者是刚去世不久的杜拉斯。这本书有四个中文译本,其中最好的当属王道乾先生的译本。我总觉得读过了《情人》,就算知道了现代小说艺术;读过道乾先生的译笔,就算知道什么是现代中国的文学语言了。
作者看这部小说说不错,我觉得的一般。

真正有出息的人是对名人感兴趣的东西感兴趣,并且在那上面做出成就,而不是仅仅对名人感兴趣。

人和人其实是很隔膜的。有些人喜欢有趣,有些人喜欢无趣,这种区别看来是天生的。