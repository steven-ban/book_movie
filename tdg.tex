\subsection{《她的国》}

作者:(美国)夏洛特·铂金斯·吉尔曼

写于1915年

\subsubsection{人物}
\begin{table}[htp]
    \centering
    \caption{《她的国》人物表}
    \begin{tabular}{l|p{0.4\textwidth}|p{0.3\textwidth}}
人物 & 特征	& 事件 \\
\hline
特里-欧-尼克森(老尼克) &	有钱,喜欢探险,多才多艺,擅长机械和电力等客观事实,拥有多个船和汽车,好的飞行员 & 认为漂亮的女人才是男人追求的对象,根本不应考虑长相相互平凡的女人,在她国里依然试图证明男性的优越 \\
杰夫-马格瑞沃 & 天生是个诗人或植物学家,医生,喜欢”科学的奇迹“,擅长生物领域 & 总是把女人理想化成南方好女人的形象 \\
范戴克-简宁斯(我) &	社会学家,兴趣广泛,语言能力好	 & \\
吉尔玛 & “我”的恋人和妻子,和“我”回到外部世界 & \\
\hline
    \end{tabular}
\end{table}

\subsubsection{事件}
\begin{itemize*}
    \item 主人公“我们”三人进行生物探险,进入”她的国“
    \item 她国的植物都受到很好的照料,环境整洁优雅
    \item 我们被看护起来,学习她们的语言,并教她们我们的语言。我们用床单做绳子从窗户里逃走,经过长途跋涉找到船,但被她们缝住。上校赶来,我们又被捉了回去。
    \item 她国里只有女性,两千年前曾经有男性。
    \item 她国驯化的猫不会叫,用来捉老鼠。
    \item 她国的人有雅利安人血统,是白人,因为战乱死掉大量男人,疆域收缩。她们重视教育,十分聪明,虽然与外界缺少交流,但科学发展良好。一个女人突然生出五个女儿,而且很长寿,女人也只能生出女儿,渐渐的这个国家的人都成了她的后裔。
    \item 她国没有竞争,没有战争,只有协作。
    \item 为了防止人口无尽膨胀,她们进行生育控制,女人培养自己的生育激情,如果能生育就好好生,如果不能则转移到别人的孩子,“幼吾幼以及人之幼”,所有人都是这个国家的“母亲”。
    \item 我们在她国出了名,全国都在研究我们的社会和心理。
    \item 我们分别与三个女孩恋爱结婚,我和女友交流她国和现实社会里教育、宗教、历史等,比较两个社会的不同。
    \item 特里承受不了他的妻子没有女人味,试图施暴,被抓住和受审。
    \item 我们离开,带着艾尔玛,保证不透露她国的位置。
\end{itemize*}

\subsubsection{书评}
书里的女人是漂亮的,可爱的,这本身就是男权社会(如果用女权主义者的话语来讲的话)里女性的投射。然而,经过波伏娃们的阐述,女性本身的上述特质只是男权社会里对女性的物化,是被塑造的,女性本身并不一定是美的,女性只是女性。因此,本书的这种架空除了让读者对现实社会中男人女人的关系进行反思(通过一种激烈的倒错的形式),很难有更多现实上的意义。

如果真的是为现实世界里两性关系做出反思,那么即使女性真的代表了一些美好情感(如平和,理性,不冲动,不恶性竞争,”母性“),把她们从两性地位不对等的关系中抽离出去,真的会保持住特性吗?我认为完全不能。现实世界里两性的性格、职能的分工互补,一旦脱离了对方,则在一个新的社会中,还是会需要这种分工,只是这种分工可能没有那么明确地按性别划分而已,而是通过具体的人来区分。因此,如果真的存在一个完全由单性生殖和女性组成的社会,那么也应当是有人性之恶,有仇恨和嫉妒,有利益划分,有冲突和歧视……现实世界里的那些恶性,在这个社会中依然不缺。这一点,本书的作者没有想到,但读者不能不想到。

我对小说,有一种现实主义的审美观念。小说应当反映现实,即使完全是想象,也需要依据现实的关系。而本书的写作,基本上和现实主义手法无关。本书采用了大段的”描述“,这种描述是抽象的和分立的,属于干巴巴的抽象,没有让人有身临其境的现场感。因此,本书的人物,无论是主角们,还是他国中的主要女性,都没有可信的性格,而是充满了作者的预设。作为小说来讲,本书并不成功。鉴于作者的观念,我觉得写成论文似乎更了一些。

另外,小说里还存在着一些只为猎奇心态而存在的情节,特别是三人在她国恋爱结婚怀孕的事情,仅仅是为了展现她国里女性的一系列观念,为”我“和她国女性进行沟通交流做铺垫,同时为真实世界里两性关系做出反思。我感觉完全没有必要,这种三流的写作手法大大降低了本书的品味。

刚进入她国时,三个男人的潜意识里认为女人是漂亮的年轻的,这大概是男权社会里对女性的投射。女性一旦年老,失去了对男性的性吸引力价值,便不再是”女性“了,而成为了和年轻女人不同的一种物种,因此这种想法完全是男人们将女性作为客体和工具时的观念。当然,这种想法在男人中间(即使是现在)是普遍的现象,很多男性像书里的特里一样,将女人当作玩物或猎物;相反的,很多男性(如周国平)将女性当作”女神“,认为女人的”母性“充满着神性。这都是极端的想法。

总的来说,本书出现在19世纪末端20世纪初,对于当时的大多数人来说,还是很震撼的。它促使人们重新审视作为家庭主妇的女人的命运, 以一种旁观的心理再来审视男性与男权社会的各种观念。

然而这本小说的缺点和优点一样明显。首先在设置上,单性生殖在生物上就不可能,同时近亲繁殖会大大降低个体差异,导致免疫力低下,疾病盛行。其次,封闭的环境下不可能科技昌明,反而会因为缺少竞争而日益停滞和落后,闭关锁国的中国就是例子。再次,她国主张社会化生育和教育,如果母亲有缺陷就“教育”母亲放弃养殖,这本身就不人道,人会犯错,国家就不会犯错了吗?最后,全书充满了十九世纪末对科学的傲慢,比如作者认为她国是雅利安人后裔而“血统高贵”疾病较少,这就是种族主义的毒瘤。如果读者恰好读过反乌托邦的《美丽新世界》和《1984》,就会对这种消除个性和家庭的社会充满恐惧。

更为深刻的是,作者对她国的政治描述仍然是模糊的。我们不清楚她们的权力结构,不清楚她们的人性之恶在社会中的表现。她国的女人似乎都是性格平和思想睿智的,没有丑恶和任何负面情绪,这似乎是作者对和蔼母亲和温柔大姐的投射。

总的来说,作为一部乌托邦小说,本书出现在20世纪似乎有点落伍。本书是完全架空的,主人公三人性格模糊,除了特里表现有点直男癌外,另外两位没有什么差别,“我”只充当了记述者的角色,完全可以合并成两个人。情节简单,完全架空,充满了猎奇写作的痕迹和不真实感,废话连篇,阅读体验较差。

至于本书被后世推崇的作为女权主义文学的“思想性”高度,我大大表示怀疑。本书使用架空手法与现实社会做比较,但由于对她国的描述太不真实,因此不能作为社会比较的样本。作者似乎更感兴趣的是女性接受教育和进行社会化生产的可能性以及对这种可能性的美好憧憬,但没有再深入下去做更好的挖掘。

评分:2/5。

\subsection{《黄色壁纸》}
《她的国》的附录的一篇小说(或者是散文),分为一和二两部分。家庭主妇喜爱写作,家里在维修,租房的地方让她不舒服、压抑、烦躁。黄色壁纸让她想入非非。丈夫是个医生,对妻子体贴温柔,但常出门工作。最后家里修好了,主人公撕掉了壁纸。