\subsection{《虐恋亚文化》}
\subsubsection{一些标注}

西方的第一次性革命是性的解放,而第二次性革命却是从性中解放出来。

女性的虐恋幻想多为屈从和降服,而男性的虐恋幻想富于迎接苦难、自我放弃的味道。
女性的被动性心理机制是缘于其缺少适当的器官,无法发泄攻击和占领外在世界的冲动,只好将这种冲动保留在其机体内,将其自身作为攻击对象。换言之,向外在世界施虐的冲动在不可能得到发泄时,转向了内在世界。

由于权力使人感到害怕,不平等关系使人感到羞辱,所以是性感的。不平等关系能使人产生热情,因而人类性行为本身就具有权力的内涵。

欲望是人所固有的,它似乎是人的某种本质,它决定了人的某种身份;而快乐则是人的选择。虐恋到底是人的欲望还是人的选择?在福柯看来,它是人的一种自愿的选择,是人对快乐的追求。

女人比男人的内心矛盾更强烈:她们既需要解放,又需要庇护所;既需要自由,又需要压抑。

性的满足植根于权力之中。

在他的作品中,女性被残害、被强奸,为男人的快乐服务。他还曾说过,由于女性的性快感是可以假装出来的,而疼痛是不能假装的,因此对于女性来说,性活动的最高形式是痛苦而不是快乐。

权力不是一种外在的有形的机制,而是监控人的内心机制。它早已被人内化在自我之中,使人的自我成为自身欲望的劲敌。它使人害怕自己的某些欲望,甚至害怕自己的好奇心。它将人的精神锁进牢笼,不允许它自由地感觉,自由地宣泄,自由地满足它的欲望。

虐恋亚文化的出现应追溯到17世纪和18世纪,

端庄在一种文化中地位越是重要,丧失端庄、受到羞辱在人们心目中就越是可怕,从而引起过多的焦虑。虐恋中的羞辱因素是对端庄的补偿,或者说是对过度强调端庄的反动。

下山时他不能不想着爬下一座山的辛苦;而当他辛苦地往山上爬时,心里充满对下山的预期和快感。这就是受虐倾向的感觉,他在一切顺遂时感到压抑,在经受磨难时才感到愉快。

双性恋者和同性恋者参与虐恋活动的比例大大高于异性恋者。

\subsubsection{感想}

1.这本书是对SM的一个深入全面的科普对SM的形式、历史、各方关注焦点和文化社会意义进行了介绍。内容上主要是李银河引用资料(而非做实地调查),如SM的主要作者、心理学家、文化学家的论述,一些受访者的口述(当然是第二手资料),女权主义和SM爱好者的论战等等。全书是典型的学术型结构,比较好读。

2.SM不是少数爱好者的怪癖好,而是参与者的性游戏和性角色扮演。从生理上讲,鞭打拳交捆绑等等看似受虐的行为为在下着带来痛感,同事大脑分泌力比多使人产生快感。而SM更重要的是心理上的,它使受虐者在羞辱和束缚甚至奴役的状态下体验权利剥夺感,将自我心灵完全交给在上者或施虐者,体验在社会现实生活中完全体验不到的感觉,从而获得性快感。具体的心理机制是复杂的,可能跟基督教赎罪文化、抛弃自我及义务、恋物等有关。因此,SM本质上是受虐者的游戏,是施虐者服务受虐者的。

3.SM渊源极远,它和人类文明里奴役压迫禁锢相伴生,然而比较显著的影响是一些文学作品,比如萨德和马索克,比如英国维多利亚时期的地下文学,比如性革命后的广泛地下实践。不得不说英国人严肃绅士的面孔下,对鞭笞竟然有如此大的爱好。人不能太严肃,否则极易“变态”啊!

4.女权主义对SM有完全分裂的观点。激进女权主义将性固定为霸权,所以只要和性有关系,必然拖入政治领域来审视。然而,SM是私人领域的游戏,不应当拿公共领域的标尺来衡量。激进女权主义中强烈丑化反对SM的人根本不了解SM,认为它和公共领域的奴役和暴力没有区别。推崇SM的激进女权主义则为SM加入了过多进步意义。我同意SM(包括大多数不影响他人的性)和政治无关的观点。

5.从我个人来看,我是有强烈的受虐倾向的(不过SM从整体上来讲,本来就是受虐者的天堂,施虐者只不过是一个服务者,或者纯粹发泄内心愤怒的人,但他的行为应当是严格受控的。)。我有恋物癖,我喜爱丝绸,喜爱头发,喜爱皮肤,为了这种癖好,我想我可以接受轻微的虐待,比如鞭打,捆绑等等。但是,我是有女性崇拜的,必须由我崇敬的女人来打我,她至少应当强势,或者应当漂亮。我真希望我的女友能够满足我的这种癖好,否则出去找这种服务对我而言很困难。