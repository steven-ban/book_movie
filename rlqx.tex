\subsection{《人类群星闪耀时》}
作者:茨威格

\subsubsection{主要内容}
\begin{itemize*}
	\item 巴尔博亚,殖民者,人类(欧洲人)中第一个同时看到大西洋和太平洋的人,1513年9月25日,手段阴险。
	\item 马霍梅特,攻占拜占庭,1453年5月29日。
	\item 亨德尔,50多岁时中风,经历病痛的折磨,依靠自己的意志力熬了过去,之后经历了一段苦闷的生活。在尖刻的评论家寄来《弥赛亚》的启发下,三个星期连续工作,写下同名歌剧,随后无论是怎样的挫折都不会再把他打垮。
	\item 马赛曲,鲁日-德-利勒,法国大革命时的低级军官,受市长邀请写了《马赛曲》,当时他所在的城市是战争的前线,他是在炮火声中的深夜里写成的。写成之后被市长和市长夫人演唱,但反响不好。在法国大革命的风起云涌里,这首歌成为法国人革命热情的象征,从士兵、军官中开始传唱,风靡法国,后来成为法国的国歌。但是,鲁日之后开始对革命厌烦,与这个革命得有点疯狂的时代格格不入,最终藉藉无名,死后多年才能人怀念。这大概是历史与历史人物之间的悖论吧!
	\item 滑铁卢战役,格鲁希,拿破仑的元帅,习惯了听从命令,在拿破仑做出最终的决战——滑铁卢之战——之前,拿破仑让他去追击普鲁士军队,而拿破仑在滑铁卢与英军对垒。格鲁希并没有追击到普鲁士军,同时他得到了拿破仑进攻英军的消息。这时,他应当违抗命令去增援拿破仑军队,而非固守命令,但他在一瞬间做出了错误的决定,贻误了战机,普鲁士军增援了英军,拿破仑战败。

格鲁希在命运垂青他让他立功时丢失了机会,但他之后的事情是一个好汉所为:他立刻纠正了自己的错误,率军冲出了重围。他没有建立不世之功,但他堂堂正正。

	\item 歌德在74岁高龄遇见了让她心动的19岁的姑娘,他想向她未婚,即使是女孩的父亲也支持他,但女孩拒绝了他。在这样剧烈的爱情的指引下,歌德写出了浓烈的爱情诗歌。但经过这样的痛苦,歌德并没有消沉,他重新拾起了生活的激情,写下了宏大的《浮士德》,而那时他已经将近八十高龄了。
	\item 祖特尔是瑞士的普通公民,为了逃脱法律的惩罚,他抛弃了自己的妻子和孩子,独自一人来到美国。他发了点财之后,主动集合了一批人到现在的加州探险并“跑马圈地”得到了旧金山,生活更加富贵。

但是很“不幸”,旧金山发现了金矿,而且是非常容易开采的金矿,只要将沙子粗粗地筛选一下就有得到一粒粒的金子,这引发了著名的“淘金热”,大量的人涌进旧金山——祖特尔的“领地”——来淘金,甚至包括祖特尔雇佣的人,祖特尔的农场和设施遭到毁坏。

于是祖特尔开始了漫长的“上访”的道路。他向州里打官司,证明了自己对旧金山的“主权”,法院支持了他的请求。但是从四面八方涌来的淘金的人无视他的“权利”,他被逼得家破人亡,无奈之下他继续到华盛顿上访,这一下子就是二十多年,他不断被各种人哄骗利用,被政府官员推诿,最终死在台阶上。

这篇故事没有前面那么激情四射,稍显平淡。而且就我来看,本身美国抢夺大西部就是一种掠夺行为,没有什么正义性可言,祖特尔的悲惨遭遇,只是那个没有正义的粗话的西进运动的牺牲品而已。
	\item 陀思妥耶夫斯基在1849年因宣传法国社会主义理论而被行刑,但最后一刻被沙皇赦免而流放西伯利亚,本文即是对这最后一刻的描述与渲染。全文使用诗歌的形式,尤其是描绘了老陀听到赦免后那奔涌的文思与激情。
	\item 居鲁士·弗·菲尔德是美国的巨富,年纪轻轻,但野心很大,他承接了大西洋海底电缆的工程,筹募资金去实现这个奇迹。1857年开始铺电缆,找两艘大船从英国爱尔兰和美国纽芬兰之间的中点同时向两端铺设。第一次试验饱受人们关注,但在铺设过程中下缆机突然坏掉,电缆掉进海里,失败。第二次试验,人们不再关注这件事情,暴风雨使船体受损,电缆损坏。第三次终于成功,电缆铺设完成。但在人们热烈庆祝一天后,发现信号很差。六年后,菲尔德卷土重来又铺设了一次。\emph{一个人对奇迹的信念永远是一个奇迹或一件美妙的事情所以能够产生的首要前提。}菲尔德在遭受失败后一次次的坚持和激情让人震撼。这件事情一波三折,茨威格写得也是十分精彩。
	\item 托尔斯泰在生命最后的日子里(1910年10月末),眼睁睁看到国内局势变得火热和冲动,人们倾向于暴力手段的革命,他反对这样的做法。他是博爱的,对于反对自己的敌人,也充满了怜悯和同情。他说:\emph{作恶的人的灵魂是不幸的,要比遭受恶行的人更为不幸,我怜悯他,但我不仇恨他。}这样的心态是不被崇尚复仇和暴力的人所能理解的。

同时,他的晚年一直想逃离自己的贵族生活,逃离自己的妻子和家庭。他的妻子索菲娅歇斯底里,情绪不稳定,但他13年前尝试过而放弃了。然而现在,她的妻子在承诺不跟踪他后仍然偷偷监视他,这让他下定了决心出走。

他果然出走了。在决定放弃自己著作的著作权后,他逃向边境,并到了车站。在车站,他的崇拜者们热烈欢迎他,他发烧了,在车长的狭小简陋的房间里独处,然后去世了。当局怀疑他,认为他危险,但他的行为表明他彻底是一个和平主义者,一个温和而坚定的殉道者。

这个故事是茨威格仿照托尔斯泰未完成的剧作《在黑暗中发光》而写的,充满了作者对这位伟人的敬仰和钦佩。

关于索菲娅,波伏娃在《第二性》里根据她的日记分析了她在与托尔斯泰的婚姻生活里的压抑与不幸,这算是一个另外的角度吧。
	\item 1910年,英国海军上校斯科特开始组织考察队远征南极,希望成为人类历史上第一个发现南极的人。然而,他们经过长途跋涉后抵达南极附近,依靠极大的毅力离目标越来越近,却发现来自挪威的阿蒙德森走在了他们的前面,他们成为第二个把国旗插在南极点上的考察队。在返程途中,他们遭遇了更加恶劣的天气,同时他们的身体和精神状态也更差。他们一个接一个地倒下了,以至于最后斯科特上校在寒冷中写下了自己的遗书。在遗书里,他没有表现现沮丧,而是对自己的妻子讲述自己的心境,对祖国表示热爱。他们的尸体和笔记在春天被发现。虽然只争得了第二,但他们的勇敢和不屈不挠是可贵的。
	\item 1917年,列宁住在瑞士,他听闻俄国暴发了宫廷政变,于是决定返回俄国领导革命。然而,当时正处于第一次世界大战期间,他很容易被敌对国德国通缉抓捕。于是,他以莫大的责任感和舍我其谁的勇气,“代表”革命成功后的俄国与德国统帅谈判停战,并租用一个车厢穿越德国回到俄国。	
\end{itemize*}

\subsubsection{一些标注}

(马霍梅特)既虔诚又残暴,既热情又阴险,既有教养、酷爱艺术、能阅读用拉丁文书写的凯撒和其他古罗马人物的传记,同时又是个杀人不眨眼的野蛮人……他证明自己一身而三任:不知疲倦的工人,凶悍勇猛的战士,厚颜无耻的外交家。

\subsubsection{书评}
茨威格用充满激情和力量的语言,为我们描述了人类历史上20个精彩有瞬间。这些故事,大部分都不是平铺直叙的,而是充满了挫折;然而,无论挫折有多大,茨威格对人类都充满了自信和希望。每个事件,都像是一首慷慨激昂的交响曲,它有低沉,有转折,同时也有突然的激越与昂扬。真不愧是奥地利人啊!

评分:10/10。