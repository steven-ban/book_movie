\subsection{《记忆记忆》}

标签: 俄罗斯文学 \ 犹太文学 \ 俄罗斯历史 \ 家庭史

作者:【俄】玛丽亚·斯捷潘诺娃

编辑在推荐里写赫然写道这本小说是“宣告俄罗斯文学重返世界文坛之书”,这不禁会让看到这句话的人大大拔高对这部小说的期待。曾经的沙俄时代,俄罗斯文学简单是古典时代以来文学的顶峰,像托尔斯泰、陀斯妥耶夫斯基、屠格涅夫这样的作用完全可以在人类最优秀的作家榜里排上前二十。然而,进入苏联时代以后,俄罗斯的文学传统断裂了,百年来再也没有一位可以与前人比肩的人物出现。更有甚者,西方整体对苏联之后的俄罗斯有着一种敌视和仇恨的心态,无论是在传播领域还是在纯粹的学术领域,对苏联及解体之后的俄罗斯都缺少正面报道。这样一部作品打着这样的旗号,不由会让人陡然起敬。

然而,抱有上面这种心态的人肯定会失望了。作者是个犹太人,“记忆”的是自己的家族史,而非国家史。犹太人对俄罗斯这样一个民族国家有没有认同感?从这本书里来看是没有的。对于百年来的共产主义制度,作者采取了一种漠视甚至仇恨的态度,这对于本书的主旨而言是一种莫大的讽刺——一个试图去回忆自己百多年来家庭史的书,怎么能跳开二十世纪那些激动人心的历史呢?然而充斥这部小说大量的议论中的,是作者对记忆本身的碎碎念,她对于家族前辈所生活的时代,采取了一种极为模糊的、冷落的做法,那就是从不正面回忆那个时代,不去直面那个国家机器。作者甚至大段回忆纳粹德国时期的犹太人,也不去哪怕多说一句关于红色苏联的话。这个犹太家族,在国家危难时置身事外,而他们的后代,本书的作者,又刻意去淡化人们对这段历史的回忆(哪怕是批判也好)。这让我对犹太人的好感,一下子大打折扣,毕竟把纳粹德国打败的,主要还是那个红色苏联。这算哪门子俄罗斯文学?这跟俄罗斯本身又有什么关系?这能代表俄罗斯文字回到世界文坛吗?这仅仅是“犹太文学”,是“身份文学”啊!而犹太文学在世界文学史上又有什么位置呢?没有!所以这个编辑的评语,实在是搞笑地离谱。

而作者的所谓家族史,究竟有多么“特别”吗?没有。反而,她的家族普通得不能再普通,最起码在作者看来,她的家族是被苏联亏欠的。作者洋洋洒洒写了30多万字,然而所谓的家族历史被译者几句话概括得十分直观:
\begin{quotation}
太姥姥萨拉积极投身1905年俄国革命;祖父尼古拉先后参加了苏联别动队、苏联红军,后来险遭党内清洗;外祖父年仅二十岁的姨弟廖吉克在旷日持久的列宁格勒保卫战中丢掉性命;太姥姥和姥姥差点被卷进“犹太医生案”;父亲曾参与1965年拜科努尔秘密航天器的研发;而作家本人则亲历了苏联解体。
\end{quotation}

如果作者原原本本把上面几个人的经历写出来,不用这么多添油加醋那么多,也绝对够得上波澜壮阔和惊心动魄了。正如作者自己所言,“朴素是人类最重要的优点之一”,可惜作者自己却没有做到这一点。作者说,“他们所有人都没能成为自己”,然而作者只是给了这些祖先整页整页的感伤,和对那个红色年代的决绝,却没有去发掘,究竟是什么让他们“没能成为自己”?是宽泛的命运的捉弄吗?还是仅仅是红色苏联?如果他们在当时的德国或者美国,就能成为自己了吗?(显然,也不一定)但是没有,作者对于这几个人的追述,也是以一种主观的、情绪化的语言来“追忆”,并且借以回忆的东西无非是家族里的某个玩具、某张照片或者某个笔记本,中间夹杂了巨量的她对世界、对人生的书卷气的看法。这种写法,让我想起中国的蒋方舟,她们都喜欢掉书袋,都喜欢堆砌无关痛痒的议论,都和叙述对象隔着一层现实的距离。这样的文风,真不知道有什么意思。

其实真的那个时代的这些祖先,对当时的苏联未必是批判的(并不仅仅是因为言论不自由),而是他们知道当时的历史条件下,苏联制度的存在有一定的必然性:
> 激励金兹堡围困札记的主导思想正是关于益处的思考。她说,西方世界无力抵抗希特勒,唯一能够与之抗衡的是苏联利维坦,这个体系以恐吓和教唆使个体失去个性,学会忘我牺牲。当个体被恐惧攫住、丧失理智与人性的,意义便会重新回归——以集体对抗绝对邪恶的名义。

确实以当时的历史现实,在帝国主义国家的围攻下,苏联制度有必然性,而且它确实为世界反法西斯斗争做出了莫大贡献,这种贡献远大于美国和英国这种眼里只有利益的国家。今天的以作者为代表的俄国犹太知识分子(可能还不止犹太人),看到解体后满目疮痍、资产被变卖的穷困的国家,不知道真的以为是那个制度十恶不赦,还是没有自己的民族和国家立场?作者是真的信了世上唯有西方式的完美制度,还是仅仅是向西方谄媚?如果是后者,那么这仅仅说明她是个机会主义者,这种人遍地都是,各国都有;如果是前者,那么就真的算是可悲了,这说明30年来,前苏联国家的知识分子,对自己历史的反思,依然是处在一个非常肤浅的程度上,他们的肉体还在俄罗斯,在波兰,思想却离开了地球。这些人,但凡有一丁点要复兴国家和民族的想法,就不会像她批判历史那样决绝——有国族认同的人,起码会有一些迷茫:苏联失败了不假,可它做对了什么?

作者式的知识分子,在中国也绝对不少,像前面说的蒋方舟,还有文革后一批彻底与民族和国家对立的那些人(批判前三十年,和彻底与民族国家对立不是一回事,这里指的是后者),他们都是失去了根基的人,认为个人乃至家族的苦难,全然是某一项制度造成的,隐含了一个假设:存在一个不给个人和家庭苦难的制度(这个制度是之前制度所对立的那个)。这种简单的二元对立、非此即彼的思想,是作者这样的小资产阶级知识分子的很大的缺陷(如果不是犹太知识分子的缺陷的话,毕竟他们的国家认同、民族认同是什么样的,我是没搞明白)。

评分:2/5。