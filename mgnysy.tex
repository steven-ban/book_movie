\subsection{《美国纽约摄影学院摄影教材》}
\subsubsection{书评}
专业的摄影教材,涉及摄影的方方面面:曝光、快门、胶片、滤光镜、构图……相比其他浅尝辄止的科普文章或者水平较低的书,这本书的内容无疑更加专业和系统。

但是本书的缺点在于内容较老。本书只讲了胶片摄影和黑白摄影,事实上现在基本上已经不再使用。关于胶片摄影中胶片的选取、冲印以及黑白摄影与彩色摄影的区别,本书花费了大量的笔墨,这些已经严重过时了,毕竟现在柯达都倒闭了。

另外,本书也没有涉及后期处理。事实上现在任何一本摄影读物,都会花费大量笔墨来教怎么使用Photoshop,因此后期已经成为摄影必不可少的一部分了,甚至在拍摄前,就要考虑后期怎么做。

本着开卷有益的原则,这本书还是值得一看的,特别是对于摄影的原理(曝光、快门、景深三大件)以及滤光镜、构图等知识,本书的介绍应该属于最好的那一批。即使是胶片和黑白摄影这些内容用不上,看看也没有坏处,毕竟数码摄影里参考了很多胶片时代的做法和术语,有利于理解摄影的历史和来源。

评分:3/5。

\subsubsection{标注}
1. 大多数专业人士都故意使彩色反转片稍曝光不足大约0.5挡光圈,以增加颜色饱和度。

2. 曝光不足会增加颜色的饱和度。

3. 但是如果我们使用彩色反转片--幻灯片时该怎么办呢?如果是这样,情形正好相反,胶片曝光不足的宽容度要大于曝光过度的宽容度。这恰恰与黑白底片或彩色负片的宽容度情况相反。

4. 面对高反差景时,我们建议运用分界曝光法进行拍摄。即对阴影区测光,然后分别缩2挡光圈和1挡光圈进行拍摄。

5. 绝对多数黑白胶片在曝光不足方面的宽容度差不多都一样即都是2挡光圈

6. 对所有的负像胶片(包括黑白底片和彩色负片)来说,曝光过度的宽容度大于曝光不足的宽容度。

7. 绝大多数彩色胶片具有更窄的宽容度。这也就是为什么彩色胶片更难获得完美曝光的原因。

8. 所有胶片具有的光强范围均比自然界中的光强范围小。

9. 测光表从灰板上测到的光线与落到被摄体上的光线是完全相同的。

10. 推荐使用的测光表是反射光类型的。

11. 想制做幻灯片,就用彩色反转片;如果想印制照片,就用彩色负片拍摄。

12. (摄影胶片)负像上黑暗(厚的)部分就是曝光较多部分;明亮(薄的)部分就是曝光较少;部分;全透明部分就是没有受到光照射的部分。

14. 每种胶片(包括彩色胶片)都包括两个基本组成部分:一个单层的或多层的感光乳剂层。 一个感光乳剂层的支持体----片基。

15. 两种增加景深的方法了,即:  1. 使用较小的孔径。 2. 向更远的点聚焦或者使照相机距离被摄体更远些。

16. 焦点越远,景深越大。

17. 85mm到135mm之间的任何一只镜头都可以用来作为"肖像"镜头

18. 为了使肖像充满画面,对于广角镜头必须极为接近被摄对象。对于任何一种镜头,当非常接近被摄体到一定程度时,就会产生这种失真。越接近被摄体,失真越严重。正是由于希望被摄体充满画面,而恰恰进入了广角镜头的失真距离范围。

19. 既然已经了解到照相机镜头的前后表面或许都是镀膜的,那么在清洁镜头的任何一端时都要格外小心。粗糙的擦拭会将镀膜除去。

20. 变焦镜头带来了以下三个基本问题:  1. 价格昂贵; 2. 体积大; 3. 在任何确定的焦距下,其成像往往都不如最好的定焦镜头成像清晰.

21. 如果使用通过镜头式测光表,不必进行曝光补偿,它会进行自动调整.

22. 对于2×远摄增距镜,开大两挡光圈,对于3×远摄增距镜,开大三挡光圈.依此类推.

