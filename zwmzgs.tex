\subsection{《植物名字的故事》}
如果把中国的植物仔细甄别鉴定一遍,最后的总数可能只有20000种左右,和同纬度的、面积相仿的另一大国美国(19473种)相当。

柚和宽皮橘的杂交产生了橙,所以橙子既有像柚子那样难剥的皮,又有像宽皮橘那样的甜酸味而没有柚子的苦味;宽皮橘和橙的杂交又产生了柑,所以柑皮的难剥程度介于橘和橙之间;枸橼和酸橙或柑的杂交产生了各种柠檬,它们的果汁青出于蓝而胜于蓝,在酸度上达到了极致;柚和甜橙杂交则产生了西方的上层人士酷爱的甜酸苦香齐备的葡萄柚……

除了枳和金橘,剩下的所有柑橘类也许都只是3个野生种的后代! 这3个野生种是枸(jǔ)橼(yuán)、野生柚和野生宽皮橘,

橘的最大特点是“宽皮”,也就是果实成熟时橘子皮与橘子瓤脱离,极易剥去;橙的特点则是橙皮和橙瓤紧密结合,很难剥离(所以一般是切着吃);柑皮的难剥程度则介于二者之间。不过,在日常用语中,橘、柑、橙常常相互混淆,比如市场上的“广柑”实际上是橙,“芦柑”实际上是橘,“温州蜜橘”实际上是柑。

他在南美洲采到一种尖吻蟾,又采到一种小型的美洲鸵的标本(其实是达尔文吃剩的鸟头、骨架和皮肤)(我的天啊!!!)

林奈的植物分类系统以植物的性器官(花朵)为主要分类依据,所以被称为“性系统”。

他对中国西部这些秀美景色和其中的神秘“原始”人类文化的描述,能够满足美国城市里那些住大房、开汽车的中产阶级的精神需求。鲁迅曾经说过,有的西方人“愿世间人各不相同以增自己旅行的兴趣,到中国看辫子,到日本看木屐,到高丽看笠子,倘若服饰一样,便索然无味了”,虽然尖刻,却是实话。

敦,大也;煌,盛也。

又从武威郡析置张掖郡,从酒泉郡析置敦煌郡,这就是著名的“河西四郡”。

林奈对大多数植物都采用了“双名法”的命名方式。所谓“双名法”,就是用两个词来为植物命名,第一个词叫做“属名”,首字母要大写,可以视为植物的“姓”;第二个词叫做“种加词”,首字母通常都小写,可以视为植物的“名”。

所有动物智力中最高级的技能——用语言表达概念。

夕者,冥也。冥不相见,故从口自名。”