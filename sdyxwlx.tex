\subsection{《上帝与新物理学》}

标签: 科普 \  物理学 \  神学

《上帝与新物理学》2012年版封面:\url{https://img1.doubanio.com/lpic/s2912179.jpg}

英文名:God and the New Physics

作者:【英】Paul Davies
译者:徐培

\subsubsection{科学与上帝}
物理学的发展,究竟给上帝留下了多大的空间?

曾经上帝主宰着人类对于神秘世界的认知,人们认为那些并非显而易见的、无法解释的现象时,都留给了未知的神秘力量来解释。在东方,人们信奉阴阳五行;在西方,希腊时代的人们相信众神,相信各种各样的类似五行的说法。当有了宗教之后,人们把这些神奇的神秘的现实留给了上帝,上帝也得以依靠这些无法解释的现象大行其道。

然而,自从物理学兴起,人们发现了天体运行的秘密,发现了力和运动的关系,继而发现了电磁场与电磁波。人类得以发明火车、轮船、无线电。人们认识到,那些似乎毫无关系的现象背后,有一条清晰的、简洁的物理学的逻辑在主宰。因此,即使人类还相信这些规律只是人类更接近上帝的方式,但上帝似乎不再需要事事操心,不再需要把每件事都安排好,只要设计好那些规律和“初始条件”,就可以袖手旁观。

进化论的提出,击破了神学最核心的观点:人是上帝造的,人与动物截然不同。之后,上帝越来越退居幕后,人类的科学越昌明,越不需要上帝,人类本身就成为自己各种领域的上帝。对于一个接受现代科学教育的人来说,即使他有或明或暗的神学信仰,但这更多是一种形而上的、精神层面的、抽象的信仰和慰藉。上帝开始更关于道德,而非关于创造与规律。

\subsubsection{本书讲了什么?}

作者开宗明义,在“前言”中就给出了本书的\emph{四大存在问题}:
\begin{itemize*}
	\item 为什么大自然的规律是现在这样的?
	\item 为什么宇宙是由现在组成它的各种东西所组成的?
	\item 这些东西是如何起始的?
	\item 宇宙如何获得了组织?
\end{itemize*}

这些内容涉及宇宙大爆炸、四种基本力、黑洞、时间与空间、量子力学、相对论、热力学、复杂系统等物理学基本问题,对于普通的物理爱好者而言属于比较常见的问题。事实上,本书写于20世纪80年代,很多内容可以在随后的《宇宙的琴弦》《黑洞与大爆炸物理学》等科普读物中出现。对 于一个熟知现代物理学发展历史的人而言,这些内容都是人类在物理学的认知框架内对上述*四大存在问题*的实际探索,在这些探索领域内,上帝并不需要存在。

\subsubsection{物理学的发展会带来哪些更深的思想和信仰变革?}

物理学是一种研究运动与变化的科学,是人类对科学最基本问题的最基本理解,代表了人类科学认知的最高水准。自从启蒙运动以来,物理学的发展日新月异,冲击了宗教观念,大大提供了人类利用自然、改造自然的能力。

作为一个无神论者和唯物主义者,我认为上帝根本就不存在,因此实际上并不存在“上帝与新物理学”的问题。这个世界上只有物理学,没有上帝。经过马克思主义的教育,中国大多数知识分子以及受过高中教育的人,都不会信仰什么基督教和上帝;即使存在一定程度上的泛灵论与不可知论的观点,中国大多数受过教育的人,也不会太过于看重宗教。虽然从历史的长河来看,否定上帝似乎并不一定安全(很难想象人类在多年以后会不会重新返回到单纯的信仰中),但从现在来看,上帝,真是已经被物理学和其他科学挤到了一边。在人类可以认识的边界以内,不需要一个上帝。
