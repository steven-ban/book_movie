\subsection{《为女权辩护:关于政治及道德问题的批判》}
玛丽身为18世纪后半叶的女人,依据启蒙主义的精神,从神学出发(上帝造男人和女人)来反驳卢梭等人认为的女人应当取悦男人、应当美丽温柔、应当仅学习如何使自己更”女性化“的当时观点,认为女人应当具备理性,因此应当学习理性,需要和男人一样接受相同的教育。玛丽认为,女人可以选择男人,因而如果不接受理性教育便不能充好女人之职。
 
当时的风俗似乎鼓励女人关注感性,而认为理性令女人不再可爱。但玛丽认为,只关注感性而缺乏理性训练的女人并档真的可爱,并且有理性的男人不会也不应该只关注女人的感性。 
很难想象这是18世纪的女人的观点,这些古典的女权主义论述其实当代都可以拿来直接”反传统“了。目前社会上反智主义盛行,很多人对“女博士”指指点点,觉得“学历太高的女人嫁不出去”,这样想人的不仅男人多,很多女人受此影响也存在这样的想法,根据我的推测,一些女人可能只是眼红学历高的女人比自己聪明,试图增强自己的“竞争力”,因此大肆贬损高学历者,以显示自己的“单纯”“可爱”。这些女人只不过是男权思想的受害者而忆,她们只是利己而自私,殊不知同样的条件下,那些有脑子的男人,会倾向于选择高学历的女人作为妻子,因为她对于子女教育和家庭关系更加理性和擅长。总之,这些反智主义的想法都只是无智者或少智者的愚蠢见解而已。女人受教育不仅必要,而且是多多益善。 

这本书不仅语言流畅,逻辑性强,而且引经据典(主要是圣经和一些著名人物的论述),而且语言浪漫华丽,大量引用圣经中的比喻,反问句比比皆是,令人无可辩驳。

关于中国社会上目前的女权主义,还需要多说几句。按理讲,女权主义(或者“女性主义”“平权主义”等等,它们的内涵应当相同)是社会进步的体现,旧时代女人与男人地位明显不平等 ,抹煞了她们的聪明才智,因此社会需要把从她们那里收回来的权利还给她们。但如何还?如何平权?不同的人有不同的见解。很多人(应该都是些女人)没读过《第二性》,没看到李银河的《女性主义》,更没有就女权主义的历史演进进行研究进而结合中国实际进行理性思考,而是简单粗暴地按着字面意思来理解或被人带节奏,认为应当是给女人的利益越多越好,进而要求明显失之公平的利益。这样的言论,在微博和豆瓣上有很多,根本不需要列举。我所理解的女权主义,应当是“拉女人一把,让她们像男人那样奋斗并争取利益”,而非“坑屌丝男一把,让他们无偿并且无差别给女人更多利益”。很多女人借自己生孩子有功,向男方索要天价彩礼,在家里颐使气指,对男人百般挑剔,全然没有启蒙主义以来的理性和平等精神。更有甚者,将女权主义和消费主义捏到一起,认为女人就应当享受,借此大手大脚,不管钱是怎么挣到手的,非要把自己打扮得精致无比,或花钱享受,或寻求“诗和远方”。这些女人,其实是被商人们洗了脑,认为享乐就是平权。还有一些女人,可能是被父亲歧视,也可能是在感情里受了伤,于是产生了严重的厌恶男性的心态,把一些恶毒的词语无差别加诸男人头上,什么“拔屌无情”啊,什么“你国男人”啊,什么“宅男屌丝”啊,一个比一个难听,甚至将这上升到民族层面,认为中国男人自古如此,因而比白人男性低等,宁肯为白人男性做牛做马,也不愿看中国男人一眼。这种逆向民族主义的心态,只是感性的诉诸语言暴力,可以说和玛丽都差了很远,更不要说和波伏娃这样的准哲学家了。这些女人,把男人推向了对立面,使很多人对女权主义产生了厌恶的看法,其实无助于社会问题的解决。那些号称女权主义的女人们需要(当然男人们也需要)多读书,多独立思考,而不是撒泼打滚呼天抢地。任何权利,都是自己争取来的。想想看一百年前工人阶级争取自己的利益,流了多少血,做了多大牺牲,才换来更好的福利保障,今天的女权主义运动,不应当也如此吗?

以上这都是大错特错的想法。所谓平权,应当是像男人那样与世界战斗,而不是借舆论甚至国家力量再搞一些收入再分配。因此,从女权主义的起源,到发展,其实都是自由主义框架下的演进,都应当是个人依据自己独立的精神向世界开战。在共产主义框架内,女人根本就是和男人一样的人,能顶半边天,都是同样的劳动者。这也是为什么中国在在旧社会男女不平等十分严重,而经过共产主义革命后女人的地位上升极快,超过了作为发达国家的日本和韩国,也超过了台湾。其实在《第二性》里,波伏娃也认为工人阶级才能实现男女平等,而资产阶级(自然也包括小资们)女性难以平等,因为后者脱离了劳动,因而放弃了争取自己平等的机会。因此,根据中国的现实,努力保障女人的受教育和劳动的权利,提高生育保障,才能有效地促进男女平等。如果从宏观层面,政府的做法不宜激进,而应当努力为将来的更加平等创造条件,不能插手干涉过于微观的问题,这些都应当交给具体的个人去实现。