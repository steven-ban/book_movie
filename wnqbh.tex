\subsection{《为女权辩护:关于政治及道德问题的批判》}
玛丽身为18世纪后半叶的女人,依据启蒙主义的精神,从神学出发(上帝造男人和女人)来反驳卢梭等人认为的女人应当取悦男人、应当美丽温柔、应当仅学习如何使自己更”女性化“的当时观点,认为女人应当具备理性,因此应当学习理性,需要和男人一样接受相同的教育。玛丽认为,女人可以选择男人,因而如果不接受理性教育便不能充好女人之职。
 
当时的风俗似乎鼓励女人关注感性,而认为理性令女人不再可爱。但玛丽认为,只关注感性而缺乏理性训练的女人并档真的可爱,并且有理性的男人不会也不应该只关注女人的感性。 
很难想象这是18世纪的女人的观点,这些古典的女权主义论述其实当代都可以拿来直接”反传统“了。 

这本书不仅语言流畅,逻辑性强,而且引经据典(主要是圣经和一些著名人物的论述),而且语言浪漫华丽,大量引用圣经中的比喻,反问句比比皆是,令人无可辩驳。