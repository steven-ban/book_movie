\subsection{《为女权辩护:关于政治及道德问题的批判》}

\subsubsection{标注}
1. 我聚焦于人物作为女性、而非作为个体的发展历程……但我意图展现的是不公之法律与社会之习俗带给女性的苦难与压迫。

3. 原文考究的语言也反应了沃斯通克拉夫特本人在当时的英国有着很高的社会地位。她的丈夫,同时也是《女人之罪》遗稿的编者威廉·戈德温,是当时有名的政治评论家和小说家。他们的女儿玛丽·雪莱是著名小说《科学怪人》(又名《弗兰肯斯坦》)的作者,而女婿则是鼎鼎大名的诗人帕西·雪莱。

4. 女性大多数的愚蠢行为都来自于男性的专制。

5. 道德必须要在任何情况之下都建立在不可改变的原则之上,服从于理性之外的任何权威的人,都算不上是有理性的或者是有德行的人。

6. 道德学家们一致认为,除非是在拥有自由的条件之下去培养美德,否则美德永远不会得到它应有的力量——他们这样说是针对于男性,而我将这个论点扩展到所有人类。

7. 在社会没能变得更加平等之前,在阶级消失、女性获得自由之前,我们无法看到那种高尚的家庭幸福,无知而受损的心灵无法体会这种质朴伟大的乐趣。

8. 作者反对的当时上流女性三件事: 占星术催眠术, 专于情感而非理想的庸俗小说, 迷恋服饰。

9. 即使是那些富有德行的女性在与人交往的过程中也从来无法忘记自己的女性身份,因为她们总是在试图讨人喜欢。

10. 男性在想要追求权力和名声的时候,有许多偶然的机会以及可供选择的道路。同一行业的男性彼此竞争,很少成为朋友,可是他们和绝大多数其他男性不会发生冲突。但是女性之间相处的情形与此截然不同,她们彼此全都是竞争关系。

11. 上帝的神性存在于他所创造的万物以及我们的理性之中,

12. (这是偏见)文明阻断了瓦舍茅檐之下常有的那种与动物交流的机会,而正是这些交流使人们对家畜产生了感情。文明国家的人们则沉迷于社会上流行的种种礼数规矩,未经教化的人们被富人们踩在脚下,只能通过欺侮动物来发泄他们在上位者那里受到的羞辱。

14. (不一定)低阶层的人民温和地对待驯顺而不会讲话的家畜,这在未开化的国家里比在文明国家中更为常见。

16. “严于律己,宽以待人”是成年人最高尚的美德之一。

17. 邪恶或者懒惰的人总是渴望能够通过推行专制的特权获益,通常他们也同样忽略了去履行那些能够使特权变得合理的责任。

18. 努力塑造孩子心灵和拓展孩子智识的父母,给他们所履行的身为父母的责任增加了尊严,虽然所有动物都会履行作为父母的责任,但是唯有理性可以为其增添尊严。这就是富有人性的父母之爱,它远远超越了本能的亲情。

19. 女性由于在任何境况下都是偏见的奴隶,因而很少会能够付出明智的母爱,

20. (胡扯)我在前面曾经说过,男性应该抚养被他们诱惑的女性。这是矫正女性行为的一种方法,可以消除那种对人口和道德有着同样破坏力的恶习。

22. 我知道很多女性,当她们不再爱自己的丈夫之后,也就不爱其他任何人了。她们忘掉一切家庭责任,让自己完全沉浸于虚荣与消遣之中。

23. 端庄是心灵纯净,是敞开内心感受一切人性的特质,而非将它局限在自私的欲望之中,是常常思索需要调动理性的问题,而非热衷于想象,淳朴的端庄自会为思考增色。

24. 那些最大程度地发展了自己的理性的女性,一定也是最端庄的——虽然她们变得仪态安娴,举止上不再呈现出年少时那有趣而迷人的羞涩。

25. 谦逊的人坚定,自卑的人怯懦,浮夸的人专横,

26. 如果那做丈夫的既聪明又正直,他就不会长久地表现得像个情人一样。

27. 女性,从童年起就被塑造为成年妇女,却在应该与学步车永别时又被当成孩子,她们没有足够的心智力量去消除那些节外生枝的联想,只能任它们遮蔽了自己的天性。

28. 宗教中确实包含着诗意的部分,能够温暖人的感情,激发人的想象,但是这个部分只能让人享受愉悦,却无法帮助人们成为更有道德的生灵。宗教或许可以成为世俗生活的一个替代品,但却不能扩展人的心灵,反而让它更加狭隘。

29. 年轻人一定要行动,因为如果他们有老人家的经验,就更适合走向死亡而非继续活着。

30. 过早地了解人性的弱点,或所谓的通达人情世故,会使人变得心胸狭窄,并且压抑天然的青春热情,而这热情正是伟大才能与美德的源泉。硬要在心灵的小树苗抽枝展叶之前,就逼它长出经验的果实,不仅徒劳,还会枉耗其精力,阻断其自然成长,

31. 美貌、温柔等等等等,或许可以赢得男性的爱情,但是唯一长久的感情——尊重,却只能靠理性所带来的美德赢取。要想使得他人对她的柔情始终鲜活,唯有赢得对方对她理性的尊重。

32. 芦苇随风摇摆,活不过一年的时间,而橡树则坚定地站立,勇敢地面对着经年累月的雨打风吹。

33. 有些作家的作品,一方面表示臣服于女性的个人魅力,另一方面却有暗中贬低女性的恶劣倾向。无论怎么去揭露他们,都不算多,也不算过分。

34. 一对夫妻在共同生活六个月以后,美貌就不会再有价值,甚至不会再引起注意。

35. 女孩儿几乎从一出生就被当作是成年女性,

36. 人没有知识,就不会有道德。

37. 我并不希望她们有权力支配男人,只希望她们有力量支配自己。

38. 女性的愚行和恶行在很大程度上是由心智的狭隘所造成的。

39. 女性也许需要履行一些与男性不同的责任,但是两性的责任都应当是人类的责任,我坚持认为,用于规范这些行为的原则应该是一致的。

40. 我期望女性会对她们的丈夫抱有爱慕之情,这种感情应该与对信仰的爱建立在相同的基础之上。这是家庭幸福唯一的基石。她们应该注意不要被所谓的“爱情”迷惑,那通常不过是肉欲享乐的粉饰之语。

41. 以牺牲另一种同样高贵并且不可或缺的属性为代价,去拔高某一种属性,是善变的人类一时头脑发热才会做出的事情。

42. 身体上的依赖性必然会导致精神上的依赖性。

43. 智能高超的人常常也有超人强健的体魄。

44. 我是在为了全体女性而非少数杰出的女士争取地位。

45. (错误)不幸的婚姻对家庭有利,而被冷落的妻子更有可能成为最好的母亲。

47. 专制统治者与肉欲主义者都致力于让女性蒙昧不开,只不过专制统治者需要的是奴隶,而肉欲主义者想要的是玩物。事实上,肉欲主义者是最危险的专制者,女性被她们的情人欺骗了,就好像王子被弄臣所惑,却还以为自己才是统治的一方。

48. 我心目中最完善的教育,是通过锻炼理性来强健体魄和塑造心灵,或者换句话说,能够将美德养成为一个人发自内心的习惯。

49. 我坚信任何一个靠着森严的等级来建立权威的职业,都有损德行。

50. 我坚信上帝是万能的,所以我认为世间的任何罪恶都是因为上帝让它发生才会存在的。

51. 理性、美德与知识的多少,决定了人们本性的完善程度与谋求幸福的能力,也区分开了每个人、指引着规范社会的法律。知识与美德是在一个人践行理性的过程中自然而然地产生的。

52. 如果女性不被允许享有合法的权利,她们就只能寻求非法的特权,从而使男性和她们自己都陷入邪恶的境地。

53. 言语缠绵、多愁善感、趣味高雅几乎是软弱的同义词,而一个仅仅是被人怜悯的对象,她所享有的因怜悯而生的所谓爱慕,很快就会变成轻蔑。

54. 贵族女性是人类之中最值得怜悯的一个群体!贵族教育让她们变得空虚无助,成长中的心灵由于缺乏对那些赋予人类尊严的职责的践行,而无法变得坚强。她们活着只为了享乐,而自然的法则是——“种瓜得瓜,种豆得豆”,于是她们很快就只能享受空洞乏味的乐趣了。

\subsubsection{书评}

玛丽身为18世纪后半叶的女人,依据启蒙主义的精神,从神学出发(上帝造男人和女人)来反驳卢梭等人认为的女人应当取悦男人、应当美丽温柔、应当仅学习如何使自己更”女性化“的当时观点,认为女人应当具备理性,因此应当学习理性,需要和男人一样接受相同的教育。玛丽认为,女人可以选择男人,因而如果不接受理性教育便不能充好女人之职。
 
当时的风俗似乎鼓励女人关注感性,而认为理性令女人不再可爱。但玛丽认为,只关注感性而缺乏理性训练的女人并档真的可爱,并且有理性的男人不会也不应该只关注女人的感性。 
很难想象这是18世纪的女人的观点,这些古典的女权主义论述其实当代都可以拿来直接”反传统“了。目前社会上反智主义盛行,很多人对“女博士”指指点点,觉得“学历太高的女人嫁不出去”,这样想人的不仅男人多,很多女人受此影响也存在这样的想法,根据我的推测,一些女人可能只是眼红学历高的女人比自己聪明,试图增强自己的“竞争力”,因此大肆贬损高学历者,以显示自己的“单纯”“可爱”。这些女人只不过是男权思想的受害者而忆,她们只是利己而自私,殊不知同样的条件下,那些有脑子的男人,会倾向于选择高学历的女人作为妻子,因为她对于子女教育和家庭关系更加理性和擅长。总之,这些反智主义的想法都只是无智者或少智者的愚蠢见解而已。女人受教育不仅必要,而且是多多益善。 

这本书不仅语言流畅,逻辑性强,而且引经据典(主要是圣经和一些著名人物的论述),而且语言浪漫华丽,大量引用圣经中的比喻,反问句比比皆是,令人无可辩驳。

关于中国社会上目前的女权主义,还需要多说几句。按理讲,女权主义(或者“女性主义”“平权主义”等等,它们的内涵应当相同)是社会进步的体现,旧时代女人与男人地位明显不平等 ,抹煞了她们的聪明才智,因此社会需要把从她们那里收回来的权利还给她们。但如何还?如何平权?不同的人有不同的见解。很多人(应该都是些女人)没读过《第二性》,没看到李银河的《女性主义》,更没有就女权主义的历史演进进行研究进而结合中国实际进行理性思考,而是简单粗暴地按着字面意思来理解或被人带节奏,认为应当是给女人的利益越多越好,进而要求明显失之公平的利益。这样的言论,在微博和豆瓣上有很多,根本不需要列举。我所理解的女权主义,应当是“拉女人一把,让她们像男人那样奋斗并争取利益”,而非“坑屌丝男一把,让他们无偿并且无差别给女人更多利益”。很多女人借自己生孩子有功,向男方索要天价彩礼,在家里颐使气指,对男人百般挑剔,全然没有启蒙主义以来的理性和平等精神。更有甚者,将女权主义和消费主义捏到一起,认为女人就应当享受,借此大手大脚,不管钱是怎么挣到手的,非要把自己打扮得精致无比,或花钱享受,或寻求“诗和远方”。这些女人,其实是被商人们洗了脑,认为享乐就是平权。还有一些女人,可能是被父亲歧视,也可能是在感情里受了伤,于是产生了严重的厌恶男性的心态,把一些恶毒的词语无差别加诸男人头上,什么“拔屌无情”啊,什么“你国男人”啊,什么“宅男屌丝”啊,一个比一个难听,甚至将这上升到民族层面,认为中国男人自古如此,因而比白人男性低等,宁肯为白人男性做牛做马,也不愿看中国男人一眼。这种逆向民族主义的心态,只是感性的诉诸语言暴力,可以说和玛丽都差了很远,更不要说和波伏娃这样的准哲学家了。这些女人,把男人推向了对立面,使很多人对女权主义产生了厌恶的看法,其实无助于社会问题的解决。那些号称女权主义的女人们需要(当然男人们也需要)多读书,多独立思考,而不是撒泼打滚呼天抢地。任何权利,都是自己争取来的。想想看一百年前工人阶级争取自己的利益,流了多少血,做了多大牺牲,才换来更好的福利保障,今天的女权主义运动,不应当也如此吗?

以上这都是大错特错的想法。所谓平权,应当是像男人那样与世界战斗,而不是借舆论甚至国家力量再搞一些收入再分配。因此,从女权主义的起源,到发展,其实都是自由主义框架下的演进,都应当是个人依据自己独立的精神向世界开战。在共产主义框架内,女人根本就是和男人一样的人,能顶半边天,都是同样的劳动者。这也是为什么中国在在旧社会男女不平等十分严重,而经过共产主义革命后女人的地位上升极快,超过了作为发达国家的日本和韩国,也超过了台湾。其实在《第二性》里,波伏娃也认为工人阶级才能实现男女平等,而资产阶级(自然也包括小资们)女性难以平等,因为后者脱离了劳动,因而放弃了争取自己平等的机会。因此,根据中国的现实,努力保障女人的受教育和劳动的权利,提高生育保障,才能有效地促进男女平等。如果从宏观层面,政府的做法不宜激进,而应当努力为将来的更加平等创造条件,不能插手干涉过于微观的问题,这些都应当交给具体的个人去实现。