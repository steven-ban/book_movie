\subsection{《西游记》}

\subsubsection{标注}
常言道,长安虽好,不是久恋之家。待我们有缘拜了佛祖,取得真经,那时回转大唐,奏过主公,将那御厨里饭,凭你吃上几年,胀死你这孽畜,教你做个饱鬼!

老沙原系凡夫,因怕轮回访道。云游海角,浪荡天涯。常得衣钵随身,每炼心神在舍。因此虔诚,得逢仙侣。养就孩儿,配缘姹女。工满三千,合和四相。超天界,拜玄穹,官授卷帘大将,侍御凤辇龙车,封号将军。也为蟠桃会上,失手打破玻璃盏,贬在流沙河,改头换面,造孽伤生。幸喜菩萨远游东土,劝我皈依,等候唐朝佛子,往西天求经果正。从立自新,复修大觉,指河为姓。法讳悟净,称名沙僧。

猪先世为人,贪欢爱懒。一生混沌,乱性迷心。未识天高地厚,难明海阔山遥。正在幽闲之际,忽然遇一真人。半句话,解开业网;两三言,劈破灾门。当时省悟,立地投师,谨修二八之工夫,敬炼三三之前后。行满飞升,得超天府。荷蒙玉帝厚恩,官赐天蓬元帅,管押河兵,逍遥汉阙。只因蟠桃酒醉,戏弄嫦娥,谪官衔,遭贬临凡;错投胎,托生猪象。住福陵山,造恶无边。遇观音,指明善道。皈依佛教,保护唐僧。径往西天,拜求妙典。法讳悟能,称为八戒

悟空解得是无言语文字,乃是真解。”

杨木性格甚软,巧匠取来,或雕圣象,或刻如来,装金立粉,嵌玉装花,万人烧香礼拜,受了多少无量之福。那檀木性格刚硬,油房里取了去,做柞撒,使铁箍箍了头,又使铁锤往下打,只因刚强,所以受此苦楚。

自家思虑道:“我若没本事化顿斋饭,也惹那徒弟笑我,敢道为师的化不出斋来,为徒的怎能去拜佛。”

周天之内有五仙,乃天地神人鬼;有五虫,乃蠃鳞毛羽昆。这厮非天非地非神非人非鬼,亦非蠃非鳞非毛非羽非昆。又有四猴混世,不入十类之种。”菩萨道:“敢问是那四猴?”如来道:“第一是灵明石猴,通变化,识天时,知地利,移星换斗。第二是赤尻马猴,晓阴阳,会人事,善出入,避死延生。第三是通臂猿猴,拿日月,缩千山,辨休咎,乾坤摩弄。第四是六耳猕猴,善聆音,能察理,知前后,万物皆明。此四猴者,不入十类之种,不达两间之名。

如来笑道:“汝等法力广大,只能普阅周天之事,不能遍识周天之物,亦不能广会周天之种类也。”

心高不认天家眷,性傲归神住灌江。

又有四个大天师来奏上:“太上道祖来了。”玉帝即同王母出迎。老君朝礼毕,道:“老道宫中,炼了些九转金丹,伺候陛下做丹

凡诸仙腾云,皆跌足而起,你却不是这般。我才见你去,连扯方才跳上,我今只就你这个势,传你个筋斗云罢。

一日,祖师登坛高坐,唤集诸仙,开讲大道,真个是天花乱坠,地涌金莲。妙演三乘教,精微万法全。慢摇麈尾喷珠玉,响振雷霆动九天。说一会道,讲一会禅,三家配合本如然。开明一字皈诚理,指引无生了性玄。