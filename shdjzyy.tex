\subsection{《上海最后的金枝玉叶》}

标签: 上海名媛 \ 永安公司 \ 郭婉莹(戴西) \ 文化大革命 \ 陈丹燕

本书是作者陈丹燕为解放前的上海名媛郭婉莹(戴西)所写的类似人物传记的东西,以她保存的各个年龄段的照片为线索,从她出生一直到90岁高龄去世(1998年),详尽记录了她跌宕起伏而从容淡定的一生。这是作者“上海三部曲”的一部,另外两部是《上海的风花雪月》《上海的红颜遗事》,作者认为这三部“其实是一本书,这本书就是上海”。

回溯她的一生,可以说是一小部红楼梦了:
\begin{itemize*}
    \item 郭婉莹出生于澳大利亚,是家里的老四,父亲郭标是在澳华侨,在澳大利亚发家,然后回上海开创了永安公司。
    \item 郭婉莹的儿童时期是在澳大利亚度过的,她在那里上了幼儿园,回国后还不会说中文。她的少女时期是优渥而尊贵的,出门有防弹汽车有保安。
    \item 她上中西女塾,和同期的男生扮夫妻,养成了“优雅”、倔强而尊贵的心态,\emph{“我不能嫁给一个会和说谈丝袜结实不结实的男人,no fun”},她追求的显然是浪漫而有趣的生活。
    \item 她早期的一个追求者艾尔伯德威胁她说不和她结婚就杀了她或饮弹自尽,她说的一句话可以说是自信而恰当的典范:
    \begin{quotation}
你不杀我,我不愿意和你结婚,你要是杀了我,我也不会和你结婚,因为我再也不能和你结婚了……现在你好好地回家去,只是不和我这样一个人结婚,要是你杀了你自己,你就永远不能结婚,就连整个生活也没有了。
    \end{quotation}

    \item 23岁的时候,她父亲去世。24岁时,她对心理学感兴趣,考去了燕京大学去学习心理,25岁时她嫁给了清华大学、MIT毕业的吴毓骧(她叫他YH),两人门当户对,并且郭婉莹也觉得吴毓骧有趣。但是丈夫在她难产时打牌、平时也喜欢四处游荡,并且在四十年代里与寡妇出轨,她开车去把他“接回去”。两人在心理上似乎并不是十分亲密。
    \item 1936年她和闺蜜开了一家服装设计公司,后在1937年关张。
    \item 1945年吴毓骧奉政府之命接收德国资产,将一些资产据为己有,并成立自己的进出口公司兴华科学仪器行,与德国人做生意,解放后改为公私合营。
    \item 解放后最初的几年里,她一家人还经常去香港探亲,看到亲友过得并不好,于是决定留在大陆,之后就再也不能出境了。她再次工作,利用自己的英文技能成为丈夫公司的英文秘书。她随后不能再穿旗袍,只穿长裤。
    \item 她丈夫因为私藏她哥的枪支(两人有点天真,对政治缺少敏感性,把一枝枪埋在自己花园里,后丈夫在狱中交待,被警察到家中经郭婉莹指认而取出)、与德国人做生意时报低价从中渔利、收犹太人钱倒卖外汇等被捕,她儿子中正因此上不了大学(后来为了避免麻烦而改为“忠政”),她女儿因此而不能随芭蕾舞团出国,她自己也需要去农场劳动。一次夜晚回家的路上,她因犯困睡觉而坐过了站,不得不走回家去。面对警察的询问时,她常常利用自己的心理学知识和“不会中文”作为遮掩,与警察打马虎眼。
    \item 丈夫死在了监狱里,认尸时他已经十分瘦弱,郭婉莹只能通过尸体上的手而认出他。她失去了吴家花园,失去了家里的车,甚至还需要为丈夫的过失而付十几万元钱。
    \item 在困难的环境下,她依然保持着生活的勇气与曾经的智慧和优雅。她在煤炉上烤吐司。四清时在业余大学里被妒忌她英文才能的英文老师举报(都是一些不构成任何过失的问题)而成靶子,被批斗,写材料。文革期间她需要去劳动,卖农场里的鸡蛋水果(她依靠经验可以把好的商品留给客人,客人喜欢让她卖东西),刷马桶,倒马桶,清理河底的淤泥等脏活累活,甚至爬上杆子这样的危险的活,她都去做了。文革期间她穷得只能吃饭馆里最便宜的不带肉不带咸菜的阳春面,和儿子住在3-2.4米的小阁间里,与邻居共用厨房和卫生间。她父亲的墓被红卫兵破坏。她还需要在毛主席像前站十五分钟来思过,因此戴上了手表来计时。她也要学习语录。
    \item 文革后在面对西方记者(包括华莱士)的有目的性的追问时,她没有去抨击那个“红色恐怖”的时代,而是回避自己吃苦的回答,维护着自己的尊严(与之对比,当年的红卫兵小将们出了国,却拼命抹黑自己的国家)。她说,**在你没有经历的时候,会把事情想得很可怕,可是你经历了,就会什么都不怕了,真的不怕了,然后你就知道,一个人是可以非常坚强的,比你想象得要坚强得多**。
    \item 1971年,62岁的时候,她“光荣退休”回家了,得到一张奖状,她很喜欢。回家后她给儿子带孙子(她自己的孩子是仆人带的)。1976年,67岁她和一个业余大学的同事、英国牛津毕业生汪孟立(戴维·汪)结婚,两年后教授死去,她就再也没有结婚。
    \item 文革后,她的一些家产被政府退回,丈夫被平反,她利用自己的英语技能开始教英文,她提倡学英文是一种教育而非仅仅一种技能,拒绝给某些急功近利的人教学。之后她安心度过晚年,与曾经的仆人互相照料,90岁高龄离世。
\end{itemize*}

从出身上来说,郭婉莹是名副其实的“金枝玉叶”,是衔着金钥匙出生的人,相当尊贵。她接受的教育是西式的,近代的,因此与普通的中国人有着比较大的差别,主要是她性格上比较独立,有自己的想法,特别是她对自己的追求者说的话,展现出她的勇气与独立。她追求生活中的fun,自己想和有趣的人过一生,这也和当时绝大多数中国女人不同,明显是西化的。另外,她受过高等教育,有一定的文化素养,这和当时大多数中国人显然也不在一个层次上。她设计女装,十分新潮。她对孩子的教育是开放的,宽容的,显然这和她优渥知礼的生活分不开。她的自尊自爱,在她文革后面对西方人记者不怀好意的采访时暴露无遗:她拒绝渲染自己的坏生活,明白西方人不了解中国,会拿着她的话大加报道,甚至歪曲,这一点是确实无愧于她的尊严,比一些红卫兵出国后抹黑中国形成鲜明对比,这一点尤其让我佩服。她不媚西方,不媚国内权贵,这种独立精神我认为是她“高贵”的内核。

但是,郭婉莹身上,也有旧式中国大户小姐和少奶奶的影子。解放前,她工作时间很短,大部分在吃自己家和丈夫的收入,养着仆人照顾自己的生活,甚至带自己的儿女也不是亲力亲为。她穿着打扮吃穿都很讲究,出门坐自己的汽车,与当时中国穷困的基本事实形成很大的冲突。她私德还算可以,无过无失,但从大的家国天下的角度讲,她对中国的进步没有帮助。从阶级的角度上讲,她就是上海典型的买办阶层,无论是娘家还是夫家都是做商业生意,不是技术,不是实业,靠的是洋人和中国的商业往来。解放后她被打成资本家,这本身是没有毛病的,虽然当时中国对前资本家太过苛刻了。她虽然上了大学,但学术上没有成就,估计课业本身也很一般。她丈夫解放后确实有了违法事实,这一点无可否认(当然罪罚太严重了)。他们解放后不学习法律,做了明显违法的事情,却有意无意地否认推脱,这一点是当然是错误的。她对丈夫的依赖,使得她在丈夫出轨并将丈夫从寡妇情人那里带走时都没有大吵大闹,也没有离婚(放在一些西方人身上可能已经离婚了),可见她并不是那种真的性情暴烈外方之人。

剥离掉郭婉莹的出身,我觉得她本质上或者说她的精神内核是一种淡然的甚至是有些凉薄的女人,她虽然追求fun,但对任何事情没有很热衷或沉迷,因此她事实上是一事无成的(她工作上没有任何成就可以流传下来)。因此,她事实上是一个安于平庸的女人,虽然她本人可以说是温润。这种女人,其实中国也有挺多的,任何一个经历过那个时代的人里,都有这种人。

因此,郭婉莹虽然是金枝玉叶,虽然有着独立的人格,但对于国家和社会来说,并不是好事情。她的这些素质建立在她买办阶层的地位上,这个阶层整体上对国家是弊大于利的。她的同时代里,对国家民族有更大贡献的人多得很,因此她的高贵,并不能超脱于中国社会整体之上。

相对于郭婉莹本人的精彩经历,陈丹燕却简单地把郭婉莹的遭遇写成“一个富家女在红色城市里如何受难”的故事,不时把自己的想法显现在字里行间。陈丹燕对上海这种老贵族,是仰望的,是羡慕的(甚至是跪舔的),而对平民则是轻蔑的和讽刺的,似乎在她眼里,世界就应该围绕着这些金枝玉叶而转,似乎只有这些人才足够高贵。她满足于对这种华贵生活的想象,认为郭婉莹的高贵情操与这种富足有超过事实上更加紧密的联系。然而,并不是所有的金枝玉叶都是这样的,毕竟任何一个阶层的人都有好人坏人,买办里也有倒卖国家资产、炒卖物价喝人血的人。郭婉莹的精神,与她本人的出身和教育当然有关,但很大部分也是天性使然,而陈丹燕过于执著于她的社会地位了。陈丹燕过多地慨叹于解放后政府如何从郭婉莹手里剥夺了财产和地位,对新中国是一种敌视的态度。事实上郭婉莹经历过那个时代,她本人已经看开了,虽然吃了不少苦,但对于自己上了岁数仍然劳动却有苦有甜,但陈丹燕抓住这一点不放,一直强调曾经的大少姐姐如何做脏活累活,殊不知同时代的其他人受过的苦比这要多多了。谁能保证一个富家小姐会永远富足呢?文革已经在郭婉莹这样的人心里过去了,然而在陈丹燕这样的后来者身上却没有过去。从这里可以看出,陈丹燕本人是一个媚富媚权的人,是一个典型的小资。这本书里,陈丹燕对郭婉莹和家人犯过的错一笔带过或者轻描淡写,例如郭婉莹丈夫当国民党对德资产的接收人员,“或多或少像他的同事一样”有中饱私囊的行为,殊不知这种行为是罪大恶极的,是极为不“高贵”的,但作者有“为尊者讳”的心理,为他们隐去了。因此,陈丹燕本质上,并没有把郭婉莹他们当成一个与自己、与他人平等的人来看待,而是有着明确的尊卑秩序的:她当然把这些“金枝玉叶”们摆在最高位,把最好的溢美之词都给了他们,给他们美化,并强调这些品德更多是由于他们的出身;她自己、郭家的那些仆人们,当然是第二档的,他们更接近那些金枝玉叶,沾染了他们的贵气,感激于他们能够理解自己;再下一层的,当然是那些平民,不懂吃和穿,反对郭婉莹的阶层;在鄙视链的底层,自然是新中国的那些当政者、红卫兵,他们破坏了那些金枝玉叶赖以生存的社会基础,让他们成为普通的劳动者,夺去他们的资产和尊严,自然是罪不可赦。在这一点上,陈丹燕夹带私货过多,我觉得她挺让人恶心的。

评分:2/5。