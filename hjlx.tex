\subsection{《环界2:螺旋》}

标签: 推理 \ 恐怖 \ 悬疑 \ 科幻

作者:【日】铃木光司

这是环界的第二部,其中已经把之前的章节简要介绍了出来。如果说环界很少人听说,那么说它是日本国民级恐怖电影贞子系列,马上就知道它要讲什么了。然而,这部小说讲的可不仅仅是鬼片或者怪力乱神——虽然它有超自然的情节,但是却自觉地用科学或者现有理论去解释,这种写法类似于《鬼吹灯》(当然,环界的写法更早)。

大概情节:
\begin{itemize*}
    \item 主角是个法医,一年以前和妻子儿子去海边游泳,儿子游远了坠海,妻子很痛苦,和他离了婚,他每天就活在失去儿子的悔恨中;
    \item 他解剖了自己大学同学兼能力极高的人的尸体,发现了一些诡异的现象——尸体缝合时露出奇怪的数字(他解密后认为那是英文单词ring),体内出现了地球上已经消失了的天花病毒;
    \item 解剖时法医遇见了死去的同学的女友高野舞,后者也在整理死者的论文,发现了一盘录影带,里面有一些现实和超现实夹杂的意义不明的影像;
    \item 法医自己也展开了调查,发现了一共有六个人因为看了录影带而死亡,四个是一起去山木屋中看的,两个是记者的妻子和女儿,只要是看的,就是同时死亡,记者自己也看了,但是却在车祸后意识昏迷(在后面死掉了),而这个记者,就应该是第一部里的主角了,他调查了录像带的内容和来源,和那个死去的法医的“大学同学”(记者和他是高中同学)一起获知了真相,并写在了报告里,在报告里详细描写了录像带的内容,报告即将出版,即《铃》;
    \item 原来录像带是一个名叫贞子的几十年前的超能力者产生的,她在疗养院照顾爸爸,被“日本最后一个天花患者”强奸并抛入井中,于是贞子得到天花,获得了天花的DNA信息;她运用强大的意念力形成了电视影像,侵入到小木屋的电视里并被转录成录像带,并且受此诅咒(指影像通过病毒去感染人类),看了录像带的人必须在7天内让另外的人也去看,否则就会死去,这形成了一种病毒式的传播;
    \item 高野舞看了录像带时恰好在排卵期,贞子的意念化成遗传物质进入了她的子宫内,仅一个星期就发育成了个体,她操控了高野舞的行为,让她跑到无人去的屋顶的水沟里生下了自己,高野舞死去,她又用一个星期时间在高野舞房间里发育成了死前的样子,这中间法医来过,感受到了房间里诡异的生命气息,但是没有看到她;之后法医再来她的房间,包括去高野舞死去的大楼,都和她正面相遇;
    \item 法医根据从大学同学体内的特殊的天花病毒,分析出一段重复多次的DNA序列是他给自己传递的信息mutation(突变);
    \item 复活的贞子跟随法医,劝说他不要阻止《铃》的出版——只要人们看到《铃》这本书,就相当于看了录像带,那么病毒就会侵入阅读者体内,然后被贞子控制;贞子的条件是,她可以复活法医的儿子;法医最后选择了服从,背叛了人类,他儿子和大学同学都复活了,他则和儿子一起隐居起来。
\end{itemize*}

贞子的故事,就是一个“咒怨”的传统日本式的“讲谈”故事,即人被害后变成鬼报复社会;而法医为了自己的儿子复活而和恶魔做交易,也很日本(“就算我不做,也有其他人做”“反正人类一旦灭绝,历史也就不存在了”)。灵异与科幻纠缠,咒怨、复仇,反社会人格的阵阵惊悚,尸体解剖、高野舞孤独地生下贞子,在人的心理上都十分贴合。

这本小说的节奏很好,作者对人恐惧心理的描写十分细腻,而且情节推进环环相扣,十分紧凑。作者并不急于把情节往前推进,而是在每一个节点上,都十分严谨细致,把那个时间点上主人公的心态描写得十分逼真,特别不在于他把主人公自己的小心思如妒忌同学的成就、对儿子溺亡的悔恨、对高野舞萌生的性欲、对贞子的恐惧同情等等。推理的过程也是把所有可能的推理可能性都一一推理出来。

评分:4/5。