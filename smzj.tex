\subsection{《神们自己》}

标签: 科幻 \  阿西莫夫 \  平行宇宙

作者:阿西莫夫(Isaac Asimov)
英文名:The Gods Themselves

\emph{面对愚昧,神们自己,也缄口不言。}

\subsubsection{人物}

\begin{table}[htpb]
\centering
\caption{《神们自己》人物表。}
\begin{tabular}{p{0.15\textwidth}|p{0.3\textwidth}|p{0.5\textwidth}}
人物 & 身份 & 事件 \\
\hline

彼得·拉蒙特 & 一号电子通道实验室的科学家 & 反对哈兰姆的理论,认为平行人类造成了电子通道\\
弗里德里克·哈兰姆博士 & \emph{电子通道之父},放射化学家,但资质平庸 & 30年前,一个偶然的机会发现了电子通道,并提出平行宇宙的假说  \\
迈隆·布罗诺斯基 & 考古学家,古语言学家,思维敏捷,研究已经灭绝的伊特鲁里亚语(古罗马时代的少数种群的语言) & 他的语言被校长误读,因此为了让自己的工作被人记住而选择与拉蒙特合作,破译平行人类的语言 \\
本杰明·阿兰·狄尼森 & 哈兰姆曾经的同事 & 在他的刺激下哈兰姆才发现了衰变现象,与哈兰姆不合,后去男性化妆品公司工作,做到副总 \\
约翰·FX·麦克法兰 & 高层大气动力学家,研究太阳风 & \\
亨特·加里森 &  & 哈兰姆为了对付拉蒙特,让他做了学术顾问 \\
巴特参议员 & 技术环境委员会的负责人 & 与哈兰姆曾经交恶,讨厌他。消极地支持拉蒙特的想法 \\
约书亚·陈 & 有3/4的中国血统 & 提倡用电子通道获取能源 \\
杜阿 & 女性,\emph{情者},喜欢奥登,与奥登、崔特组成“三者家庭”,喜爱孤独和自由 & \\
奥登 & \emph{理者} & 喜欢杜阿的特立独行和独立思考精神  \\
崔特 & 杜阿的右伴,与杜阿交媾,孩子的\emph{抚育者},男性 & 热衷于交媾和抚育,对理性思考不感兴趣 \\
 & 杜阿的哥哥,理者 & \\ 
 &  杜阿的另一个哥哥,抚育者 & \\ 
 罗斯腾长老 & 奥登的导师 & 由杜阿的抚育者父亲融合而成,他了解杜阿,为杜阿找的伴侣 \\
伊斯特伍德 & 一个新长老,由杜阿、奥登和崔特最后交媾而形成 & 创造了一种区别于太阳的人工食物 \\
赛琳娜·琳德斯托姆 & 月球人,假装是个导游 & 前代人有地球人基因工程留下的“预言基因”,有很强的预言能力 \\
巴伦·内维尔 & 赛琳娜的男友,物理学家 & 试图在月球上建立像地球上那样的电子通道 \\
路易斯·蒙特兹 & 50岁,地球人 & 在月球上工作了较长时间,做专员工作,马上回地球,之前警告哥特斯坦要留意月球人的动向 \\
科纳德·哥特斯坦 & 30-40岁,地球人,驻月球专员 & 巴特议员曾经的下属,负责电子通道的调查
\end{tabular}
\end{table}

\subsubsection{事件}
\begin{itemize*}
	\item 2070年10月3日(30年前),哈兰姆发现他办公桌上的钨消失了,在狄尼森的刺激下,他持续研究,发现它变成了钚186,并且还在不断衰变中,释放出正电子,并且释放的正电子越来越强。随后他和团队发现,这是平行宇宙与我们宇宙的法则不同,两者之间相互传送电子,每发生一次核反应,我们世界就少了20个电子。我们宇宙多出来的正电子与电子中和,会产生能量。
	\item 拉蒙特与哈兰姆会面,后者的傲慢令他不悦,他找到布罗诺斯基,两者合作破译平行人类的语言
	\item 平行人类向人类发送一个信息:FEER。随后他们发来信息,希望地球人停掉通道。
	\item 平行宇宙中的凡人在家庭中分为三种角色,这三种角色是固定的:情者,负责生孩子,类似地球上的女性;理者,负责理性地思考,地位较高,类似地球上的男性;抚育者,负责抚养儿女,类似地球上的男性。理者和抚育者可以交媾,但必须有情者,三人一起才能更“快乐”,这时情者成了一种类似气体的东西,而理者和抚育者的身体相互交叠在一起,花费几天时间这种交媾才能完成。这些生物似乎是依靠吸收太阳的能量来生存的。
	\item 平行宇宙中还存在着“长老”。他们的数量更少,而且没有年轻的长老。他们负责教育理者,传授他们以知识。
	\item 平行宇宙里的太阳的强度逐渐降低,导致像杜阿这样的情者无法在地表吸收到足够的能量(食物),因此无法生出小情者。同时,生物的数量也在不断减少。
	\item 平行宇宙间的凡人们密度很小,这是由于他们世界的核力大于我们的世界,因此微粒之间更为分散(而非像我们世界一样中子、质子结合成为原子),可以进入对方身体内而交媾(粒子之间的相互作用,会引发凡人们的“快感”)。岩石的密度则大一些,情者可以渗入岩石中进行“石慰”(类似地球上生物的自慰)。长老们的密度比岩石还大,他们不与凡人接触。
	\item 平行宇宙的核力更大,导致这个世界拥有更多的恒星,但恒星的体积更小。随着电子通道的运行,越来越多的电子被传送到平行宇宙,会导致核力减小,越来越像地球所在的那个宇宙,这边的生命都会灭绝。
	\item 崔特为了生出小情者,去偷了放射性强的“食物”并假装给杜阿吃,然后顺利怀孕。知情后杜阿离家出走,并决心关掉电子通道。她认为:这个世界里只有长老才是真正的生命,他们不会生育,寿命极长,只能依靠培养小理者并传授给他们以知识才能解闷;如果电子通道一直运行,他们就不再需要凡人,因此他们要灭绝凡人。杜阿向人类的宇宙发送信息,想让人类首先关闭电子通道。
	\item 然而杜阿是错的。奥登发现:凡人们在生完孩子后,会马上老去,他们需要最后的交媾,从而形成了长老。在他们每一次交媾中,短暂的时间内会形成密度较大的长老,在这个过程里,理者的回忆会中断,直到最后的交媾时,理者会记起成为长老的一些时刻。
	\item 杜阿向人类发送信息,最后衰竭而濒临死亡。奥登和崔特找到了她,三个交媾,形成了长老,就是伊斯特伍德。
	\item 地球经历过20世纪末期的一次大战,人口减少到20亿,之后就对科技十分谨慎,科学发展停步不前,高精尖产品都来自月球。相反,月球人积极进取,十分先进。月球上的质子加速器是地球人的,同时电子通道也仅在地球上才有。
	\item 狄尼森来到月球,与哥特斯坦和内维尔分别会面,并在月球上进行介子研究。两方都想争取他,试图从对方那里打探情报。
	\item 狄尼森和塞琳娜合作,发现人类不仅可以建立电子通道从杜阿的世界里传送电子,还可以建立另外的通道从核力比人类的宇宙更大的平行宇宙里传送能量和动量,从而平衡掉宇宙内核力的变化,从而稳定下来。
	\item 月球准备建立那样的通道。狄尼森和塞琳娜合作的文章在地球上发表,战胜了哈兰姆的科学独裁,为拉塞特恢复了名誉。月球上的人也基本上不支持内维尔的独立主张。
\end{itemize*}

\subsubsection{书评}
本部小说包括三个部分,分别对应三个故事:拉塞特决心打倒哈兰姆的故事、平行世界里奥登一家的故事、狄尼森在月球上发现其他通道的故事。三个故事有着强烈的关联:拉塞特和平行世界里的杜阿通信,狄尼森也反对拉塞特。三个故事都围绕着地球与平行世界之间进行能量交换的故事展开。与普通的”爽文“不同,这里除了明显的善恶划分,并没有那么强烈的”正义战胜邪恶“的故事,而是人类和平行宇宙间的智能生命都能够凭借自己的智慧,解决自己面临的问题,饱含着作者的乐观主义精神。

一部科幻小说的核心是一个idea,这个idea撑起了故事,是作者思考的精华所在。本书的这个核心就是”平行世界“:如果平行世界和我们的世界之间打通了,可以传送物质和能量,都会带来哪些改变?人类和平行世界的生物会如何应对?会导致哪些矛盾?如何解决这些矛盾?这些问题反映了阿西莫夫在50年前的想法,今天看来还不算过时。

阿西莫夫对人类社会的细节描写不够好,特别是对”反对学术权威“哈兰姆的独裁的描写,根本不符合人类现实。无论是哪个社会,科学家(特别是自然科学家)都难以长期地依靠某一个理论来垄断学术界(好吧,前苏联的李森科除外,但那是有意识形态方面的支持),科学家能好好地对抗民众的愚昧和无知,能被广大群众不看作怪胎,就谢天谢地了,还想什么独揽大权?更何况,有着那么多科学圈里的反对者,根本不用议员动用政治力量,很快就会被理为严谨的论证所推翻。因此,作者事实上树立了一个错误的不切实际的靶子,使得整体的故事走向不很让人信服。

另外,对科学家偏爱名利、易怒好斗、爱结小圈子的描写,可能真的有这回事,但让它单独撑起一个故事,就显得很单薄了。另外,对两性的描写也显得草率,这反映出科幻圈一个长期的现象:文学性差,文笔差。这种现象在刘慈欣的《三体》中也有所表现,而看了阿西莫夫的《银河帝国》系列以及《机器人》系列后,发现阿西莫夫的文笔还不如大刘呢!

本书让人惊喜的地方在于第二部分对平行宇宙的描写,可谓妙笔生花。一旦脱离了具体的人类社会的描写,进入想象力的领域,阿西莫夫的优势完全显现出来。他的想象大胆有趣,而又浪漫,把杜阿的独孤感、奥登的理性、崔特的本能也描写得相当逼真可信。三性组成一个家庭,依靠微粒间的间隙进行交配、性别的分工、长老的形成都很具备想象力,让人惊叹!可以说,这一部分是本书的亮点,是本书优于其他同类小说的根本。

本书的叙事方式类似于POV写法,和《冰与火之歌》有点像,只是规模和复杂程度远不如后者。这种写法给人一种猜谜的乐趣,读起来兴趣盎然。

评分:8/10。