\subsection{《新石头记》}

作者:【清】吴研人

\subsubsection{标注与点评}

1. 那篇奇文是预备丈夫读,不预备奴隶读;预备君子读,不预备小人读。所以,那吃粪媚外的奴隶、小人,到了那里,那石面上便幻出几行蟹行斜上的字,写的是: All Foreigners thou shalt worship; Be always in sincere friendship. This the way to get bread to eat and money tosp end. And upon this thy family's living will depend; There's one thing nobody can guess: Thy countrymen thou canst oppress.

2. 必要热心血诚,爱种爱国之君子,萃精会神,保全国粹之吏夫,方能走得到,看得见。若是吃粪媚外的人,纵使让他走到了灵台方寸山斜月三星洞,也全然看不见那篇奇文。
\emph{以上是全书的总纲,作者似乎是现在的小粉红附体,粗暴地把人分为爱国的大丈夫和崇洋媚外的小人,和义和团如出一辙。}

3. 从此女娲氏用剩的那一块石就从大荒山青埂峰,搬到文明境界灵台方寸山斜月三星洞去了。
\emph{这是作者想象力飞奔的一例,基本上是把中国魔幻拼凑到了一起,怎么吸引眼球怎么来,倒也挺好看。}

5. 看万国和平会。此刻和平会被各国公议到中国来办,举中国皇帝做会长。北京永定门外,已经盖了一所极大极大的会场。这里博览会开过之后,便是和平会第一次开会。\emph{这是当时绝对是意淫。中国饱受外国欺负多年,由此生发出一种叛逆和“报仇雪恨”的心理:老子将来一定会翻身让你们刮目相看的!充满了血性和不服输,这大概是近代以来中国能不断抗争的源泉吧。欣慰的是,2010年中国真办了“万国博览会”(世博会),作者地下有知,应该会很欣慰了吧!}

6. 这根棍尖的铁,是吸铁做的,这种电气放出去,跟着他的吸力走,那准线便斜了,电都被吸去了。

7. 春夏天的闪电,也是电气在真空界上发火。

8. 到了没有空气的地方,便是真空,电气到了真空的地方便要发火,制造空气,只能把窗门关紧了,人在里面自制自吸,断不能放到外面来。那车的机轮,一切都是用电的,岂不要全车发火?因此不敢轻举妄动,不然,早就有人上去了。”
\emph{作者对于现代科学,只是有粗浅的了解,而对其背后的微观的原因不加探索,更缺乏科学实证和思辨精神。上面对于电的认识,半真半假。}

9. 不要说是没有这种人,没有这种事,就是字典上‘娼、妓、嫖’三个字都是没有的。

12. 我们重的是德育,就德育而论,只有公德是男女一样的。至于私德,女子与男子就有点不同了。所以读的书,男女都不同,何况将来的专门学,又与男子迥别的呢?”宝玉道:“请教女子专门学些甚么?”老少年道:“门类多得狠!女红之外,大约轻巧的工艺,都是女子学得多。近来,医学之中,也拨了儿科、妇科两种,归入女学专门。”宝玉道:“据这男女没有界限说来,那<礼经>上‘七年男女不同席’,与及男女‘不亲授’的礼法,都可以废了?”老少年道:“这里面,另是一个道理,大约文明未进化之时,淫乱之风,在所不免。所以圣人定礼以为防闲。不信,但看<国风>那淫奔之诗,十居七八,这就可想了。至于文明进化的时候,人人都有‘道德’两个字充满了心腹,那里还用得着这些呢?可笑那食古不化崇拜古人的,动不动就说唐虞三代之风不可及,他不过因为当日有了个尧舜文武罢了。须知尧只一个,尧舜只一个舜,文王也是一个,武王也是一个,未必当时百姓个个都是尧舜文武呀。莫说是淫风,譬如百姓,个个都是击壤老人,有了这些无识无知的百姓,

13. 这里没有男女界限,固然没有那接手、搂抱、接吻的恶习,也没有那一定回避男子的形迹。男女相见,亦犹如男与男相见,女与女相见一般。”
\emph{中国的正人君子们,总是喜欢拿道德说事,似乎一切的事情,都可以用道德来解决:大至宇内国家制度,小至人心欲望,都可以用四书五经来约束提升。这是礼教的滥觞。性欲乃是人的基本欲望,一夫一妻(多妾)根本无法真正约束,因此娼妓和嫖客永远不会消失。至于对男女平等的理解,当时还处于男女生理结构不一样因此道德要求和社会角色不一样的认知上,因此认为读的书不同。其时当时女权主义已有流传,欧洲再有十几年便可以妇女投票了,而作者竟作此梦呓,让人失望透顶。作者人私人感情上看,认为当时西方流行的“接手、搂抱、接吻”是恶习,可见思想之顽固。}

15. 这种机器只第一次用时,要烧一回火,蒸来气出,连动了机器,生出了电火,从此就借电火蒸气。蒸出气来。仍是连动器,机器仍能发出电火。所就周复始,生生不已,取之不尽,用之不竭了。
\emph{这他妈就是永动机,欧洲在19世纪搞出来好多,但自从热力学第一定律和第二定律流传并被人接受,就少多了,没想到作者还抱着半个世纪前的东西在意淫中华之强大。更可悲的是,一个多世纪后的现在,很多民科还对这种东西研究个不停,如同中了邪一样。}

16. 处处要有人和他比较,才肯用心。没有人和他比较,就不肯用心。所以要靠竞争,才有进步。不知就是没有竞争,只要时时存了个不自足的心,何尝没有进步呢!
\emph{当时的科学思想虽然流传的少,进化论却流传得多,特别是由生物进化论类比出来的社会进化论深入人心,这大概是因为中国人受多了欺负,不得不从落后中吸取教训的原因吧!}

17. 这聪明是一件无影无踪的东西,如何制造得出来?大抵大的智愚,关乎脑筋的多少。他研究了脑的原质,就把这原质合起来,研了细末,加入药料与及轻清之气,叫人拿来,当闻鼻烟去闻。鼻窍通脑,借着脑中的热气,便成了脑筋,添补在上面,自然思想就富足了。”
\emph{这是典型的民科思想。中国缺少严密的解剖学,因此对人体的认识落后西方一个世代,大概是此时还没有认识到西方医学之先进,不太接收西医的思想。}

18. 戊戍四月之后,那一个不说要进京去伏阙上书,那一个不说就条陈呈请督抚代奏。及至政变了,这一班人吓的连名字都改了,翻过脸来,极力的骂新党。推他前后的用心,那一回不是为的升官发财!

19. 每一出去,便看见那些百姓,奴颜婢膝的跪着迎接洋兵,大有“箪食壶浆以迎王师”之概。遇了洋兵欢喜的时候,便一直过了,不去理会他;碰了他们生气时,反嫌他跪着路,不是一拳,就是一脚,那被打的倒反笑脸相迎。

20. 只见几十个兵排队而来,路旁另有十来个人,在地下跪着,衣领背后都插着一面小旗子,也有写“大英顺民的”,也有写“大法顺民”的“大美”、“大德”、“大日本”都有,底下无非着顺民两个字。各人手里也有奉着一盘馒头的,也有奉着热腾腾肥鸡、肥肉的。


22. 只见家家门首,都插着些“大英顺民”、“大德顺民”等小旗子。沿路巡察的的洋兵不少,偶然站定了看看东西,那洋兵便要来盘问。

23. 我恨洋货,不过是恨他做了那没用的东西来,换我们有用的钱!也恨我们中国人,何以不肯上心,自己学着做?至于洋人,我又何必恨他呢?
\emph{戊戍变法和八国联军进北京是作者描述较多的情节,对朝廷上官员见风使舵、百姓奴颜婢膝的丑态进行了揭露,也特别描述了义和团的愚昧和迷信,批判是很有力度的。}


25. 说起今日在制造局所看的机器,自然都是外国买来的了,不知中国自己做不会。伯惠道:“会只怕是会的,就怕的是器具不齐,做不起来。
\emph{这是因为当时中国的产业极度不全,产业链严重缺失,很多工业中间品依赖进口。}

\subsubsection{书评}

这是吴研人写于1905年的“穿越小说”:贾宝玉睡了一觉就穿越到了19世纪末20世纪初的中国,先是看到了《红楼梦》,后来是自学了英文读了报纸和外国小说,之后见到了薛蟠参观了近代化的工厂,坐了轮船火车,到了北京亲眼看见了义和团闹事、两宫逃往西安和八国联军进北京,后来更是进入了一个虚幻的“真·文明国”,上天猎鲲下洋捉蟮,见识了未来社会的发达和文明的“进步”。因为进文明国的原因,这部小说又被称为中国第一部“科幻小说”,深得凡尔纳真传。

文笔肯定是比不过曹雪芹的,人物形象也是扁平化严重,吴研人也是借贾宝玉之行来抒发自己对时局的看法。话说这里涉及的时事也就四五年,作者竟然毫不避讳,可见创造自由不比今天差。今天如果有人写小说来记叙2012年重庆的大事,恐怕早进去了。

印象比较深的不是贾宝玉参观现代化工厂,而是进北京后看到的底层“愚民”对待外国人的态度:闹事时是神经兮兮神棍满天飞,嘴里跑火车却不敢上去冲锋(围教党时),洋鬼子进了城则点头哈腰彻底丧失自尊。说实话,吴研人对义和团们的鄙弃肯定也代表了当时知识分子和社会精英对本国人民的怒其不争,这种批判姿态延续到民国时的鲁迅和后来的柏杨对“国民性”的反思,贯穿了中国的独立和现代化过程。以此观之,吴研了算是开一代风气了。

贾宝玉进入文明国的一系列科技奇遇:飞车、海底历险、医学发达……基本代表了那个时代(不仅是中国)对未来科技发展的憧憬,一百多年后再看,很多都已经实现。但是吴研人虽然熟悉先进技术,似乎却不懂科学,特别是医学方面对人体解剖、免疫等涉猎太浅,像吃了药不生病这种基本上是不可能的。又如文明国里研究人的智慧可以直接“灌注”,这基本上已经是神棍了。

更让我错愕的是吴研人对社会制度和政治体制的思考。借文明国的人之口,吴研人认为,当时的中国是文明而衰弱,但洋人是野蛮而先进,这都不算真的文明。中国文明是因为有儒家的教化,这算是一种“文化自信”了吧?而洋人虽然器物先进,但自身是野蛮的,总是侵略他国(这是事实),真正的文明应当是既有中国的教化又有洋人的科技。其实这种思想,从“中学为体,西学为用“到现在的”中国特色社会主义“和”制度自信“,不绝于中国整个的现代化进程。但是以吴研人对科技的粗浅理解,恐怕认识不到”科学“的本质是什么,”科学“根本不是靠着皮毛的粗浅理解就可以学到的,需要花很大力气。经济产业的发展,也不是靠一句”中学为体“就可以轻易学到的,中国百年来的经济建设史证明了,破除传统的小农经济以及基于小农经济的儒家经济观和思想观是前提,必须转换这个”体“,实行市场经济。吴研人的这种思想,其实还没有真正开眼看世界,没有老老实实地向西方人学习,还带着”天朝上国“和”中华为尊“的傲慢。其实想想,中国当时已经被洋鬼子欺负了半个多世纪,中国人的自尊已经掉到了深渊,知识分子还是抛不开四书五经的尊贵地位,恐怕是自卑处向先人那里寻找精神慰藉。因此,中国的现代化之曲折,就可以完全想见了。

比如对西医的鄙弃:
\begin{quotation}
	可笑世人鼠目寸光,见了西医便称奇道怪,又复见异思。不佑西医的呆笨,还不及中国古医。此种新发明(指用“先进科技”来实施中医理论),他更是不曾梦见。中国向来没有解剖的,而十二经终分别得多少明白。西人必要解剖看过,便诩诩然,自以为实事求是。不知一个人死了之后,血也凝了,气也绝了,纵使解剖验视,不过得了他的部位罢了。莫说不能见连动,就连他颜色也变了,如何考验得出来?
\end{quotation}

可见吴研人对西医根本就不了解,缺乏基本的医学常识。当时的西医虽然没有现在这么发达,但很多手术都能做了,已经发现了感染原因。可悲的是现在中国依然有很多人借此反对西医狂捧中医,恐怕连吴研人的见识都没有。

另让我遗憾的是吴研人的政治观。文明国是一种“开明专制”:君主是个开放贤明的君主,专权管理着世界,是可以禅让的;大臣是忠诚高尚的士大夫,能干也不贪婪;群众是听话而文明的群众,各司其职。这几乎是把儒家理想的“三代”换到了二十世纪,只不过披上了科技昌明的外衣。对于民主制度,吴研人认为,西洋人和东洋人本来就是野蛮人,无论是民主制还是立宪制(他竟然把这两个词并列,我们姑且按他的理解来理解好了)都是因为人性的野蛮才实施的,多党之间有党争,耗费了社会资源,并且社会还治理不好,因此民主是虚伪的。这种认识,到今天依然还很有市场。吴研人没有看到,代议制的民主虽然有党争(例如目前的台湾),但带到了政府操作的更加透明化和各利益团体的透明博弈,这正是现代化政治的核心。开明专制只存在于虚幻中,谁能保证存在一个完美的君主呢?中国的道路,还是需要民主\footnote{这是2017年读的,到了2021年,经历了新冠肺炎疫情下中西政治制度的比较,我对西方民主制的信心大打折扣,但并没有完全否定这种制度。}。文明国的德治现状是:
\begin{quotation}
	敝境的捕役,非没有神奇的手段,便连捕役也没有一个。……敝境近五十年来,民康物阜,夜不闭户,路不拾遗,早就裁免了两件事:一件是取文明字典,把“盗贼”“奸宄”“偷窃”等删去; 一修的是从占中刑部衙门起,及各区的刑政官、警察官,一齐删除了,衙门都改了仓库。
\end{quotation}
这基本上就是在意淫和扯淡了。

总之,《新石头记》是一个披着科幻和穿越外衣的政论小说,而作者的政论,放到今天来看,是不值得一驳的。

评分:3/5.