\subsection{《民国老教材》}

\subsubsection{书评}

Kindle商店里的套装书很便宜,往往十块钱不到就能买到好几本书,但是有一点不好:这几本书不是单独发送到Kindle和APP里的,而是统一地发送,打开来文件很大,需要很长时间才能读完。如果仅仅是小说类,这么做还好,一口气读完一整个故事还是挺吸引人的;但如果是散文,这体验就差远了,完全没有手捧一本小书的轻盈感,往往一套书要读一年或更久。读书的乐趣之一,是把一本书合上之后那种“啊,终于读完了”并把这本书删除的满足感,但套装书直到读完最后那一该才将这本书删除,对于阅读中是有点沉重烦闷的。

似乎是2004年左右,突然流行起民国时代的语文课本来,这是那时候的“民国热”的一部分,包括林徽因啊才子们啊老知识分子们啊等等,原因大概是大家生活水平提高,知识面扩展,就对官方历史的叙述有了反叛心理,对官方一直反对描黑的民国的那些“正统”有了好奇心和渴望。看看那时候大教授们的收入、对知识分子的尊敬、传统国学的流行、审查政策的相对开放等等,都有大众(主要是知识分子)有了一定的冲击。在此风气之下,一些民国的课本也进入人们的视野,甚至有学校专门找来让学生做课外读物。

这套书是民国时的语文教材,时间大概是1947年左右,包括以下五本:
\begin{itemize*}
	\item 商务国语教书(初级学生用),大概是现在小学1-4年级的水平,循序渐进,初期的课文里都有图画,主要是一些礼节、历史小故事、生活风物等,最长不过三百字,精练,朗朗上口,我觉得其实挺适合现在的小朋友读的,这些东西没有什么历史界限,是历久弥新的。
	\item 世界书局国语课本,大概也是小学1-4年级水平,主要是一些儿歌、生活中常见的动物植物和历史上的小动物,内容颇有童心(冰心是编者之一),比如第88课“猜谜儿”:
\begin{quotation}
		哥哥说,我有一个谜儿,给你们猜。一粒小红枣,头戴玻璃帽。不要说他身体小,一间屋子装不了。姐姐想了一想,笑着说,不是桌子上的灯吗?
\end{quotation}
	\item 开明新编国文读本(甲种本) 上,分第一、二、三册,这内容已经比较像现在的小学高年级和初中生的水平了,主要是一些当时国内外作家的散文或长篇中的一段,已经和现在的一些课文有重叠,比如萧红《呼兰河传》中的《火烧云》、朱自清的《背影》等,其选题和目前的课文其实差不多,只是题材范围不一样。课后会有一些对本文的赏析和类似思考题,一般只有两道。
	\item 开明新编国文读本(甲种本) 下,也分第一、二、三册,大概相当于现在的高中语文,选题的思想性更深了,出现许多鲁迅等人的对中外社会现实的描绘和思考的文章,也有对读书、文学等的思考,每篇后面也是两道总结、提示或简单的问题。
	\item 开明新编国文读本(乙种本) ,分第一、二、三册,都是文言文,但除了古代的一些名篇(如孟子)或诗词等,主要是近代或同时代人的一些以文言写就的文章,甚至一些外国的译著。将文言文单独编成一本,对我而言还是有点惊讶的。我曾经读过90年代的语言课本,文言文的比重很大,将有1/3,等我上了高中就只剩下少于1/4了。这些文言文没有注释,只有每篇后面的两三段话,提示一下重要的语法。相比之下,现在的高中文言文的难度,应该是明显低于这些的,可能是当时离白话文流行也比较近,社会上有人需要,也觉得应当学习的原因。
\end{itemize*}
	
这些文章,每天读上一两篇还是很惬意的,选题质量不低,但说实话也不算高,没有有些人吹得那么厉害。比现在的语文课本好的是,没有刻意地去体现某些“微言大义”甚至是“概括作者的思想”,更没有政治上的说教,比较生活化,读起来很舒服。

因此我觉得现在的学生,不一定非要拿它当课外读物(不如直接去读名著),但作为有闲人士,偶尔去翻翻,还是有价值的。

\subsubsection{一些标注}

“姑苏”“长洲”“吴”三个名称,实际是指一个地方。“姑苏”是明朝称苏州府的一块地方的别名,“长洲”和“吴”都是苏州府管辖的县,县治同在府城里。

北平人是以他们的大白枣、小白梨与牛乳葡萄傲人的。

河水既容易出事情,个人想减轻责任。便以为凡事都俨然有天作主,由天处理,照书行事,比较心安,也少纠纷。(注:书指的是历书)

	

