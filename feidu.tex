\subsection{《废都》}

标签: 中国当代文学


作者:贾平凹

这本小说算是贾平凹的代表作了,写于1993年,出版后洛阳纸贵,正版盗版一起卖得很火,之后被禁,2009年才解禁。市场火爆的原因,我想并不是文学性多么吸引人,而是书中大胆而直白的性描写实在是露骨,并且情节本身也不算晦涩难懂,因此成为大众追捧的样板。

小说主要写了西京(西安)城中的所谓四大文化名人的故事,主角是庄之蝶。他出身普通甚至是贫苦,靠写作成为“四大名人”之首。他生于1952年(与作者贾平凹一年生,这里可以看出作者的自省之意),大学毕业后留在西京,“才华横溢”(人设如此,至于有什么才华我觉得倒也未必),成为人们特别是女人们敬仰的对象,不少人巴结他往上爬。他的妻子牛月清却不是什么文化人,结婚后相夫教子(两人不孕不育,没有孩子),上得厅堂下得厨房,却没有自己的事业,可以说把自己的生命都献给了丈夫。然而他们两人生活中已经没有了激情,庄之蝶也把妻子的付出当成理所当然。两人几乎没有性生活,也没有孩子,甚至庄之蝶在牛月清面前无法勃起。

家庭之外,庄之蝶的生活就丰富多了。他受各种女人景仰,像从县城里私奔出来的唐宛儿、家里的保姆柳月、住在贫穷民房嫁给粗人丈夫的阿灿、好友妻子等人都对他青睐有加,这种青睐一方面是对“才华”(然而书里并没有展现庄之蝶有多少惊人的才华,从他的表现来看顶多算是一个中学语文教师的水平)以及才华带来的名气和权力的景仰,这种景仰是一种对权力的膜拜。自古以来,性与权力就是一种双生关系,让女人仰望的神秘爱之力量,似乎都和权力有关。这些女人对他有着或多或少的纯爱,但似乎都是因为她们生活的平淡里,缺少激情,因此需要这种缥缈的、带有文人气的想象来作为背景。

在这些人里,着墨最多的就是唐宛儿和柳月。柳月相对比较简单,她出身农村,高中毕业,借机会来到庄之蝶家当保姆(之前她似乎偷过东西,行为不检点)。庄之蝶视她为玩物,她也借着庄之蝶最终成为了市长儿子的老婆,一时间成为“人上人”。她对庄之蝶和牛月清,有着一种家人的关心,特别是和牛月清关系不错。但是,她一旦和庄之蝶有染,便有点自高身份起来,有种不识抬举的轻浮。

最让庄之蝶牵挂的,以及最热切地爱着庄之蝶的,是唐宛儿。她与周敏私奔到西京,与周敏成婚,但是周敏毫无根基,她内心也看不上周敏,两人几乎没有性生活。她背着周敏,通过欺骗的手段,与庄之蝶私通。她是真的爱庄之蝶,不仅仅是因为他的名气,两个人在身体上是融洽的,是性与情、肉与欲的统一。她对庄之蝶的感情,可谓炽热,这大概是因为她是逃出原来的丈夫家的,对庄之蝶带来的光明如同飞蛾般热切,便更加奋不顾身起来。然而,舍此之外,这个女人并没有什么值得让人记住的地方。她最终最原来的丈夫掳走,并虐待至死。

阿灿这种人,和唐宛儿倒是有点像。她也是有着一定的文化,有文学、浪漫有着追求,但现实却十分骨感。于是她拼命并心甘情愿地把身体献给庄之蝶,觉得和他睡一觉这辈子就“值了”,不可谓不“下贱”。

众女子都对庄之蝶投怀送抱的事,在我看来有点匪夷所思。诚然很多女人热恋权力、财富、名气,但庄之蝶本人似乎并不帅气,和《金瓶梅》里“潘驴邓小闲”相比还是差了些。虽说八十年代作家是全民偶像,但会不会让这么多女人甘心投怀送抱还是有点疑惑。作者似乎是把自己(至少是一部分)投射到庄之蝶身上,他如此书写女性,让人觉得是看低了女性,甚至于是侮辱了女性,以及拔高了自己。这样看来,作者是相当猥琐的。书里的性描写,都是在庄之蝶与这些女人之间展开,虽然现在的版本里都删除了,但是从前后文来看,这种描写是赤裸裸的,毫无美感可言的,如果说这不算是对女性的丑化,我真不知道什么叫丑化了。

情欲之外,小说还描写了九十年代初西安的众生相。这种众生相,是卑微的,丑陋的,罪恶的。四大名人各自不干好事,不是赌博(儿子吸毒),就是迷信气功算命,走穴赚钱。他们手眼通天,一旦有事就去找厅长市长,送礼讲关系,毫无斯文可言。甚至于尼姑庵里的法师,也是长袖善舞,甚至于乱搞关系去打胎。更不消说街道办的主任非礼女大学生把人逼疯这事了。四个名人之间,虽然平时以兄弟相称,但画家死去,庄之蝶就趁危去低价卖他的画,给他儿子毒品;庄之蝶还与好友的妻子通奸,当面却不著一词。当然,庄之蝶本身也是人性沉沦的受害者:他的书店被手下人转卖资产,赔了不少。

最终东窗事发,牛月清知情后离家出走,庄之蝶死在车站。

小说的语言是细密的,全是《金瓶梅》那样的白描。除此之外,小说还学习了马尔克斯的那种魔幻现实主义,当然这种超现实在中国古典小说的也常见,比如天气的变化,算命,一个捡破烂的对世道人心的洞察,胡言乱语中透露着智慧。城墙上吹埙,庄之蝶喜欢听哀乐,喝牛乳房下的奶等等。以至于,牛都有了思想,去对比城市的污浊和自然的清明。牛月清母亲晚上睡在棺材上,自称看到西京“百鬼夜行”,能与鬼交流,这本身就是一种超现实意义的上的讽刺。

小说的格调、语言都是模仿《金瓶梅》,特别是一男三女的设计,再明显不过。小说的本意是揭露现实,批判现实,这也和《金瓶梅》相一致。

然而这本小说却无法与《金瓶梅》比肩。与《金瓶梅》相比,作者对人物的描摹缺少功力,不够准确,作者笔下人物虽多,但人物之间的交互却很简单,都是是庄之蝶为中心,人物的语言也极为单调,如果遮去名字,则无法分辨是哪个人说的话,比如女性的语言都是一种风格语气。作者在对女人们臣服庄之蝶这一行为上的描写相当一致,和《金瓶梅》那种冷静和真菩萨心相比差了不是一点半点。

评分:2/5。
