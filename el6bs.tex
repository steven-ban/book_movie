\subsection{《月亮和六便士》}
\subsubsection{一些标注}
“女人可以原谅男人对她的伤害,”他说,“但是永远不能原谅他对她做出的牺牲。”

人们动不动就谈美,实际上对这个词并不理解;这个词已经使用得太滥,失去了原有的力量;因为成千上万的琐屑事物都分享了“美”的称号,这个词已经被剥夺掉它的崇高的含义了。一件衣服,一只狗,一篇布道词,什么东西人们都用“美”来形容,当他们面对面地遇到真正的美时,反而认不出它来了。他们用以遮饰自己毫无价值的思想的虚假夸大使他们的感受力变得迟钝不堪。正如一个假内行有时也会感觉到自己是在无中生有地伪造某件器物的精神价值一样,人们已经失掉了他们用之过滥的赏识能力。

她只差一点儿就称得起是个美人,但是正因为差这一点儿,却连漂亮也算不上了。

女人们总是喜欢在她们所爱的人临终前表现得宽宏大量,她们的这种偏好叫我实在难以忍受。有时候我甚至觉得她们不愿意男人寿命太长,就是怕把演出这幕好戏的机会拖得太晚。

在我同他打交道的时候,正是这一点使我狼狈不堪。有人也说他不在乎别人对他的看法,但这多半是自欺欺人。一般说来,他们能够自行其是是因为相信别人都看不出来他们的怪异的想法;最甚者也是因为有几个近邻知交表示支持,才敢违背大多数人的意见行事。如果一个人违反传统实际上是他这一阶层人的常规,那他在世人面前作出违反传统的事倒也不困难。相反地,他还会为此洋洋自得。他既可以标榜自己的勇气又不致冒什么风险。但是我总觉得事事要邀获别人批准,或许是文明人类最根深蒂固的一种天性。一个标新立异的女人一旦冒犯了礼规,招致了唇枪舌剑的物议,再没有谁会象她那样飞快地跑去寻找尊严体面的庇护了。那些告诉我他们毫不在乎别人对他们的看法的人,我是绝不相信的。这只不过是一种无知的虚张声势。他们的意思是:他们相信别人根本不会发现自己的微疵小瑕,因此更不怕别人对这些小过失加以谴责了。

“女人的脑子太可怜了!爱情。她们就知道爱情。她们认为如果男人离开了她们就是因为又有了新宠。

那时候我还不懂女人的一种无法摆脱的恶习——热衷于同任何一个愿意倾听的人讨论自己的私事。
\subsubsection{读后感}
作者犹如一个热爱八卦而又极度腹黑的……嗯,八婆,整天不干事却往来于社交场合,打听人的隐私八卦,面露和色却内心翻腾,让人讨厌。