\subsection{《西文字体-字体的背景知识和使用方法》}

作者:【日本】小林章

这是一种介绍西文字体类型、结构、使用的通俗读物,在专业人士和字体爱好者圈子里评价很高。

喜欢读书的人,肯定会连带对书籍的装帧、排版和字体有所涉猎。我读书很多,大学以前很多时候看的是盗版书,根本谈不上多么好看的排版。之后买了一些排版精良的正版书,但数量较少。研究生以后读书少,再后来看Kindle。Kindle上的书,排版都比较一般,采用的字体也比较单一,无法显现出书籍排版的美感来。

以前接触过\LaTeX 的排版,里面会涉及一些排版和字体知识,但对字体的认识也仅限于衬线、无衬线、美术(书法)字体的分类,以及字重等简单的名词,但对字体、特别是西文字体缺少系统的认识。

这本书就填补了这个空白。本书里的行文很通俗,具备大量的图片实例,图文并茂的排版很适合听着音乐慢慢地一个字符一个字符地欣赏。各种字体中有些差异比较小,但其间微妙的差别值得细细品味。

以前认为,英文字母加上常用的字符数量远少于中文字型的数量,因此其设计应该也是简单的。但是通过阅读本书发现,本文字体也有悠久的书法历史,不同历史阶段的字体也是姿态各异。另外,西文字体除了手写体外,还有大量的活字字体。想来活字虽然是中国人的发明,但在中国的应用并没有那么广泛,反而在西方发展壮大。形态各异的西文字体具有不同的美感,无论是在活字时代,还是在激光照排时代,还是在计算机排版时代,西文字体的发展可谓百花齐放。当一页页纸翻过去,慢慢欣赏每种字体的结构、黑度、架构、衬线,不得不惊吧他们这个行业的传承。

对于字体的知识,特别是不同字体何时用于标题(此时字符间距应该小)何时用于正文(此时字符间距应该大一些以利于阅读),不同国家的典型字体(如德国的哥特体),不同年代的代表性字体等等,这本书都会给出详尽的介绍。以往对于西文字体不熟悉,以为一种字体就可以一用到底,没想到正文、标题、斜体还可以混用,并且不同字体间的搭配还有很多规则或规律可循。

我觉得,相比于西文字体,中文字体真是匮乏得可怜。翻开随便一本中文书籍和英文书籍(或者各种科技期刊),就能看出来中文的排版水平实在是太落后。中文的排版字体选择并不是很多,而且设计得也不够悦目。而另外一些书法痕迹过重的字体,又不太适合于长文排版。再加上很多出版社和期刊的行间、段落、和篇目的排版实在太差劲,让人根本没有阅读的欲望。反观西文排版就专业得多,比如很多Kindle中的英文书,字体美观大方,易于阅读,书形多样,简直就是一个个的艺术品。

但是,这本书我不喜欢的地方:正文字体采用汉仪旗黑,其实我觉得采用宋体是更好的选择。黑体抹除了汉字笔划间的变化,虽然易于识别,但在书籍印刷中显得呆板而没有生机。

总之,读完本书并不会让外行的爱好者设计出一款字体,但会大大提高一个人的字体审美。