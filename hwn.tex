\subsection{《我们在哈瓦那的人》}

标签: 间谍 \ 古巴 \ 英国

作者:(英)格雷厄姆-格林

\begin{quotation}
浪漫的人总是心存恐惧,害怕事情不如预期。他们的期望总是太高了。
\end{quotation}

小说名叫“我们在哈瓦那的人”,是英国情报机关的口吻:我们自然指的是英国政府,哈瓦那是古巴一个地方,“人”是指间谍。

这是一个构思精巧的故事。六十年代的巴西,英国人伍尔摩的妻子已经死去多年了,他带着自己的女儿生活在哈瓦那,他开一家卖吸尘器的店,生意不好不坏,挣的钱不多,而女儿则已经是青春期,虽然在教会学校上课但是有着不同于他呆板的各种离经叛道的想法,比如和当地的警长一起兜风、和他去骑马等等。他有一个好朋友——海斯巴契医生,他是一个德国人,曾经参加过一战和二战。为了给女儿筹钱,伍尔摩阴差阳错被一个英国间谍发展成了下线,学会了码文。他并没有线索源头,于是依靠自己的想象力编造了好几个下线,甚至按照吸尘器的样式画了一个军事基地,竟然瞒过了英国总部。为了支持他的工作,总部派下来一位女助手贝翠丝。伍尔摩担心贝翠丝发现自己的破绽,于是尽力圆谎,讽刺的是他编造的信息被对手截获,他们杀死了这些“下线”(仅仅是因为名字相同),现实竟然像谎言那样巧合。后来,有人来暗杀他,他依靠自己的智慧躲过了暗杀(下毒,伍尔摩把毒酒喂给了狗,亲眼见到狗死去),虽然他的朋友海斯巴契医生被那些敌对的间谍杀死。他为了复仇,借口与警长下棋喝酒,灌醉了警长,拿走了他的手枪,找到对方间谍(他们在下毒的宴会上见过)杀死了他。在这个过程里,他和贝翠丝暗生情愫,拿到钱和女儿一起回到了英国,送女儿去读大学,和贝翠丝幸福美满地生活在一起。

这本书的行文是那种典型的英国二流小说的调性,流露着与社会隔开一段距离,远远地看穿社会和生活的狡黠和小聪明,不断地抖机灵、嘲讽。人物呢则面具感很强,没有一种栩栩如生的临场感和真实感,情节上没有那种压迫感,而是一种“我知道你在讲故事,并且这是一个大团圆的故事,主要的人物都没事的,但肯定有一个对作者很重要的人会死掉”的感觉,读者不会有捏一把汗念念不忘的感受。本书的情节过于巧合,设计感或者说匠气过浓,这不像是俄国或者法国那种厚重的现实感与写作者的深度参与感(作者与故事深度捆绑在一起,对人物倾注着感情,甚至这种感情是不理性的,有褒贬的),而是英国的轻佻感。伍尔摩这个人物形象仍然是模糊的:他普通得有点平庸,但仍然有着狡猾心思,会编故事,会在宴会中金蝉脱壳;他看重朋友,讨厌当地的政治和警察,也讨厌英国间谍总部的愚蠢。当然,这一点也是作者的看法。但是,我仍然不喜欢这种文风,英国的一些作家,像是毛姆、《银河系搭车客指南》这种,就像是听相声一样,读着有趣味,但掩卷不会让人记住太多东西。我读书,是想从文字里品味出伟大的,这类作品,因此并不让我尊敬。

评分:2/5。