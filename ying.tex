\subsection{影}

标签: 张艺谋 \  电影 \  水墨 \ 邓超 \  孙俪

十足的张艺谋的风格,即使是水墨两色(加上少许的红色)也表明老谋子对画面效果的极致追求。每一帧的画面都极美,武术设计不花哨,可以看到《英雄》的影子。不过,叙事上扎实了不少,情节上没有太大的硬伤,对白也算合理不出戏。演员的演技都至少在线,邓超和孙俪的表演很不错,应该是邓超迄今为止最好的一次表演了(上次是《烈日灼心》)。\footnote{《影》电影海报:\url{https://gss2.bdstatic.com/9fo3dSag_xI4khGkpoWK1HF6hhy/baike/c0\%3Dbaike272\%2C5\%2C5\%2C272\%2C90/sign=6cce6a6ebf1c8701c2bbbab44616f54a/63d0f703918fa0ec4a47b0872a9759ee3c6ddbbc.jpg}}

张艺谋的电影,每一次都会带来很大的争议,观众还是太过于苛刻了。其实每部电影再过几年,都可以看出是当年的经典,特别是《英雄》《十面埋伏》这种。观众如果能心平气和地把中外的电影放在一起比较,去除掉对国产电影不合理的期待以及对外国电影的朦胧感受,老谋子的电影每一部都是在线的。

张艺谋的电影每一部都在拓宽自己的边界。如果他安于以前的风格,拍一辈子《红高粱》《大红灯笼高高挂》《一个都不能少》,自然可以保住名声,但他一直在尝试电影工业的国际化、工业化、流程化。熟悉他早期电影的人,都知道他在90年代早期就开始尝试美国好莱坞式的大片(《代号美洲豹》),一直到中外合拍制作的《英雄》《十面埋伏》《满城尽带黄金甲》《长城》,他对中国电影的贡献可不仅仅是作品本身,还在于对于电影制作的试水。虽然张艺谋的编剧功底不深厚,但他对演员的调教,对场面的调度,这依然是中国顶尖的。\footnote{《影》电影剧照:\url{http://img5.mtime.cn/pi/2018/03/01/095641.74784421_1000X1000.jpg}}

本片的原声音乐十足的中国内,古筝、箫、笛绵绵不绝,与故事节奏的控制相得益彰。电影结束几分钟,我还坐在座位上听音乐放完。

评分:9/10。


