\subsection{《新卖桔者言》}

标签: 张五常\  经济学\ 改革开放 \ 交易成本 \ 自由市场

作者:张五常

\subsubsection{标注}
理论永远不可以被事实证实(cannot be porved by facts),只可以被认可(can be confirmed by facts)。找不到事实推翻就是认可,这是科学方法的一个重点。

构思《新卖桔者言》的目的,是希望可以训练同学们的观察力,训练同学们的想象力,训练同学们用简单的经济理论与概念来解释表面看是复杂无比的世界。

跟今天后来者的经济学者不同,我这一辈的喜欢用浅观察来先找答案的大概,再用其他日常观察加以引证,然后再考虑要不要多走几步。

经济学是一门验证科学(empirical science)。这种科学是要有实验室的工作训练的。……经济学的实验室只有一个,那就是真实的世界。

民主的问题不少,而关于决策出错这个话题,其困难在于投选票而不是投钞票。投钞票,投者入肉伤身,不能不慎重考虑切身代价与切身利益;投选票不需要从袋中拿钱,只模糊地希望他人的钱可以投到自己的袋中去,或模糊地期望某些利益。没有明确的代价与肯定的回报,不会慎重考虑,容易受到煽动与误导。

我是四十年前在洛杉矶加大与芝加哥大学接受经济学训练的。这两处当时被认为是市场经济的圣殿。于今回顾,当年的师友差不多包括二十世纪的信奉市场的所有大师了。离开这两所圣殿一两年内,我在想,市场不能办到的,政府不要干。后来改变了主意,同意郭伯伟、夏鼎基等人的看法:市场可以办到的,政府不干。再后来的想法——今天的想法——是市场不能办到的,政府要考虑干不干,甚至考虑大干特干。这观点的转变,是经过多年在公司理论上的苦思而得到的结果。公司与政府的性质相同。既然市场不能办到的公司可能办到,政府也有类同的职责了。只要记着郭伯伟还是对:市场可以办到的,政府不干。

正在盛行的,是员工炒老板,不顾而去另谋高就,老板跪下来也留不住。(这简直是扯淡,老板没有员工只需要撤掉投资就行了,什么时候给员工下跪过?)

\subsubsection{书评}
张五常出生于香港,但是在广西长大,日子十分苦,当时正是抗日战争时期,兵荒马乱,他一个小孩子曾一个人坐火车去另外一个城市里找母亲,并和母亲坐船往外逃。他后来能去美国读经济,做出《佃农理论》这样的博士论文,后来又研究公司理论。他与科斯这样的经济学巨匠过从甚密,对后者提出交易成本理论有着巨大的启发作用,并代替科斯去领诺贝尔经济学奖。他在中国改革开放初期对中国的经济改革有一定的指导作用。这一系列巨大的名头,会让读者对此书有巨大的期待,希望它是一部既有理论又有实际,并且读起来不会太费劲的科普读物。但是,这样的读者(比如我)肯定会失望了。

本书是一系列小文章,并不是严格的学术论文,语言十分潦草,东一句西一句,前后之间也缺少逻辑上的联系,特别是一些登于报纸上的类似专栏的千字左右的文章,简单是胡写胡说(话未必不对,但态度极其敷衍),文风恶劣,这种文字拿出来自娱可以,但是发表到报纸上让读者付钱阅读,就无法起到传递信息的作用,可以说是挥霍自己的名声。既去《新卖桔者言》,则必是有《卖桔者言》在先,后者是1984年《信报》出版的,这是在后者基础上增加了一些新的文章面成的。

对于张五常正经的经济学理论,这本书里没有集中的介绍,而是散见于各篇文章中。这些内容,我无法做出评价,也不知道正牌的经济学家们,对于他的理论会做出怎样的盖棺论定。不过,这本书里的其他内容,我倒是有些意见。比如,他提倡用实际的街头经济学实践(如卖桔子、卖玉、去苹果园调查、观察真实的店铺运营)来替代抽象的说教,这应该是不错的,也是符合经济学这一社会科学的定位的,而不仅仅是数学上的游戏。他自己就卖桔子,还鉴别文物,在大陆置房。但是,这种街头实践,是不是可以替代计量经济学那样的普遍的统计呢?我在心里打了个问号,毕竟香港一地的集市,是不是有其特殊性,而在其他地方会遵从不同的行为准则?他鉴宝应该是不专业的,而且一些文物流失到他那里,是国家的损失,而不仅仅是市场的流通那么简单。他鼓吹开挖秦始皇陵,这是置文物保护工作于不顾,只顾一时的旅游业绩,放在任何国家和时代都是愧对祖宗的行为。

他认为政府不应当去管菜市场里螃蟹绑了咸水草,认为真实的价格与绑没绑感水草无关(他总结为“欺骗定律”,在竞争下,卖家一律欺骗与一律不欺骗会有同样的效果),这种规范到底有没有必要,我也说不准。螃蟹有没有绑咸水草,绑了多大比重,可能容易看清,但是一些不熟悉的则未必。再说如果卖家在做广告的时候不骗,卖的时候骗,顾客认知理念上有差异,不就吃亏了吗?类似的问题还有玉石是不是剖白,这属于比较专业和小众的学问了,我也说不清楚。同样的逻辑在于,是不是要打击假货?张五常认为可以不打,但事实上各国名面上都在打,特别是都在杜绝这种行为,这在中国特别明显:十多年前山寨产品很多,但现在比较少了。这时面有知识产权的考量,有鼓励创新的考虑,并不仅仅是张五常所说的顾客可以分辨、真货鼓励假货的单一维度考量。

对于他提倡的产权要清晰的理论,我基本上是支持的,相应地应当鼓励私产、保护私产,这也是经过各国实践证明是正确的,特别是中国改革开放,很重要的一条就是要保护合法合理的私有产权,他的理论也可能就此被中国的改革者所接纳实现。

他最大的争议,是对新劳动法的连篇累牍的否定,现在看来这种否定是错误的。新劳动法大概是2008年生效,他就写了一系列文章来反驳,但理由都不是很充分。新劳动法更大程度地保护的劳动者的权益,在实施中提高了各地的最低工资标准,并强化了政府对劳动者的偏向,他则表示全盘反对。他的理由无非是,劳动市场是“公平”的,工人怕老板,老板也受工人的限制,即使没有最低工资,老板和工人也会商量出一个合适的价格,此处不留爷,自有留爷处,工人可以自由择业(“让市场有合约选择的自由”),提高最低工资,会让工人失业,最终让劳动群体受损。这当然是书生意气,他不知道工人在与老板的谈判中,是处于弱势的,如果没有国家兜底,老板会想尽办法盘剥工人的工资,最终让工人在一个极低的工资收入下为自己劳动,工人则会受尽剥削而不自知。老板没了工人,顶多收回投资去炒股炒房,而工人没了工作,则很容易进入赤贫状态。对于中国而言,很多农民工背井离乡就是为了得到比农业劳动更高的收入,而农业劳动的收入是极低的,老板可以利用这一点把农民工的工资定在仅比农业收入高出有限的价位,此时工人还是会去劳动,但利润几乎被老板独吞。不仅是中国,各个国家都会去规定最低收入,尽量向工人这一方倾斜。自由择业似乎是一个稀松平常的东西,但是对于八十年代的中国而言则是新事物,这对于经济活力的增加有很大的推动作用,在广州的港商盖的酒店的那篇分析中,张五常的观点是正确的。但是,所谓的择业的“自由”并不是无限的,国家职工的择业自由和农民工、补鞋少女的择业自由,有很大的不同。另外,他还提倡要放弃农产品自给自足的保护主义,置国家粮食安全与农民失业于不顾,可以说是十分错误了。张五常的这些分析,是严重脱离实际的,这不符合他提倡的去一线调查的学术原则,事实上他所谓的一线调查,除了上面的卖桔子,卖文物,去街市观察,真的少之又少,比如他就不曾去考察中国内地的农村的实际生产生活,对于农村的认识仍然停留在小时候在广西的经历,家里的仆人告诉他信阳有大片的农田他才知道,以为中国的农田都像广西那样在山上一小片一小片的。他自称佃农专家,但是却没有去做佃农的调查分析,这些还不如写出《寻乌调查》的毛泽东。他也没有去一线观察工人与老板是如何进行工资谈判的,仅仅是道听途说,这样的治学态度,只能去做一些架空的理论分析,怎么能够去指导社会实践呢?他虽然自称“永远站在劳苦大众一边”,“结交的穷朋友无数”,甚至因担心新劳动法损害劳动权益而“脾气顿发”,但恐怕对于穷人(特别是不同地方的不同的穷人)的所思所想缺少直观认识。以至于他发出“农民工比我儿子还要娇贵”这样的感叹,真是可笑之极。现在看来,劳动法并没有造成工人失业,而是导致用工成本降低,一些企业把厂子迁到内地或者东南业,而更多的企业则是把成本转移到了产品上,最终的成本是全社会承担,这在效果上来看是相当于变相对工人发补贴,提高了工人的工资水平(最终生活水平是否提升则难以量化,我的观察是提升了),这在道义上是正确的。

张五常反对“格雷欣定律”即劣币驱逐良币,认为买卖双方都知道何为劣币何为良币,这是钻了牛角尖,因为这并非其本意,货币使用者会刻意保留良币而另一方不一定知道。他也反对博弈论,认为其没有用处。他对于这些理论,有种全盘否定的架势,但是这些理论都流传下来了,可见他的认识,也不一定正确,有口出狂言的可能。

总的来说,张五常2000年前的文章(主要还是老文章),或可一看,2000年后的文章则没什么价值,自吹自擂的成分太高。中国的读者可能被他的名头唬住,相信外来的和尚好念经,但还是要辨析他的分析是不是有道理。所谓盖棺论定,希望这个时代能给这个顽童一个恰当的评价。

评分:2/5。