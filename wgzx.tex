\subsection{《无辜之血》}

作者:【英】P. D. James

英文句:The innocent blood

这并不是一部推理小说,虽然作者本人的地位是“阿加莎去世后,只有P. D. 詹姆斯配得上推理女王的桂冠”(出版社在书封上写的溢美之辞),而是一部反映亲情、仇恨、复仇、救赎的故事。

十年前,玛丽·达克顿和她丈夫一起,杀死了一个小女孩,起因是她那懦弱的丈夫是个恋童癖,他对准那个没有走寻常回家路的女童子军下手,利用她的单纯和善良对她进行性侵犯,玛丽回家后,为了“保护”那个女孩的名声而杀死了她(“不管他脑袋里想的是什么,那都不会是强奸。他是个温和、羞怯、软弱的家伙。我猜那就是他会被孩子们吸引的原因。我以为我能帮助他,因为我很坚强。但是,那不是他想要的。他应付不了。孩子气、脆弱才是他想要的。他没有伤害她,你懂的,肉体上。那是法律意义上的强奸,但是他没有使用暴力。我想如果我不杀了她的话,她和她的父母以后会控诉他毁了她的人生,她再也不能拥有美满的婚姻了。或许,他们的担忧有道理。心理学家们声称孩子们永远无法克服早期遭受性侵犯的阴影。于是,我剥夺了她破碎的生命。我不是在为他辩解,只是你不需要把它想象得比事实更糟糕。”),并冷静地实施了一个抛尸计划,利用看电影、图书馆借书作为掩护来试图把尸体抛到树林中,然而事情很快败露,两个被捕,丈夫被绞死,玛丽被判了十年。在这之前,他们已经有了一个女儿,然而玛丽脾气暴躁,经常对女儿进行家庭暴力,并且希望她被人收养,这个女儿也是多次被不同的人收养,最后被大学教授莫里斯看上。莫里斯喜欢她对知识的渴求,于是收养了她,给了她良好的教育和“菲莉帕”的名字。莫里斯不孕不育,他的前妻却给他戴绿帽生了个儿子,但这个儿子和前妻因交通事故早夭,之后因为大学里的一个助教希尔达对此表现出了同情,他对希尔达表现出好感,两人结婚。希尔达不是一个聪明的女人,生活也比较乏味,心思单纯。

十年后,菲莉帕成为了一个骄傲乃至于自负的女孩,她获得了剑桥大学的录取通知书,三个月后就去上学。此时她对自己的身世有了兴趣,于是找到当时的收养令,甚至拜访了亲生母亲当年的故居,得知自己的母亲是个杀人犯。她去监狱探望母亲,并且约定出狱后在伦敦租房子。她自己找好了房子,打扮一新,玛丽出狱后两人生活在一起,菲莉帕也获知了当年的案情,两个的亲情开始复燃。

与此同时,当年被杀死的小女孩的父母,则对玛丽恨之入骨,他们无法水解女儿被杀的痛苦,于是策划在玛丽出狱后复仇杀死她。然而妻子几年后染病,将死之时把这个任务交给了丈夫斯凯思,丈夫于是独自承担了复仇的重任。他很快找到了监狱,在出狱时跟踪两人,又通过希尔达得到了母女的住址。他买了刀子、雨衣等凶器,偷了钥匙,潜入租房的内部,获知了两人的行动规律。

玛丽在回到莫里斯住处时看到莫里斯与女学生上床,原来他做这事已经多年,但从莫里斯口中获知自己是在凶杀案发生前被亲生母亲“抛弃”的,于是对母亲的态度发生了变化,在当天晚上出门,又被流氓骚扰,很晚才回到租房的住处,却发现凶手用刀子割喉已经自杀的母亲。她知道了凶手的情况,让他走掉,然后叫来了莫里斯,报警并伪造了事情经过。她跟莫里斯回去,两人发生了肉体关系。莫里斯其实在领养时就“爱”上了菲莉帕,一段时间后她再次遇见凶手,凶手已经和他租房住的盲人女孩结婚。

这个故事本身并没有十分稀奇的地方,是一个典型的三流小说的梗概。如果读者因为作者的名气而抱着刺激的情节、缜密的推理来读这本书,那肯定会失败,这恐怕也是豆瓣上大量低分的原因。不过,我觉得这里面的几个人物倒是很有意思。

首先是菲莉帕,她身在一个充满知识、理性的收养家庭,内心却一直在追寻自己的过去,对真正的“亲情”有着执念,她与玛丽的关系也由疏远而走向亲密,对收养家庭却渐渐疏远。不过,长期的养育还是在她心里打上了烙印,希尔达虽然无趣却一直很关心她,莫里斯虽然有着冷漠的理性却给了她父亲的关怀、长远的规划(如坚持让她上大学,认为上大学对她而言十分重要),并且在玛丽自杀后也第一时间赶过来帮助她。当然,莫里斯对她,不是亲生父亲的那种单纯的亲情,而是夹杂了情欲、自己儿子死亡的补偿心理、赏识等多种情感的混合。他有着英国知识分子那种冷静、冷漠甚至无情的态度。
\begin{quotation}
她所受的道德教育是一种语义学,一种针对享乐和伦理的纯理性探究,莫里斯称之为道德灌输,并得意地归结为自身正直的影响。面对他人时得体的举止是基于某种的抽象的概念:良好的公共秩序,愉快的生活,自然正义——不管它意味着什么——总之是绝大多数人最优秀的品质。对于大多数人而言,善意地对待他人是为了对方也善意地对待你。这意味着那些聪明、诙谐、漂亮或者富有的人不需要这种权宜之计;树立榜样似乎更适合他们。

……南伦敦学院名义上遵循基督教义,然而在菲莉帕看来每天清晨为时十五分钟的集体礼拜只不过是一种传统仪式,确保女校长宣布当天的通知时全校师生都在场。有些女孩信奉宗教。圣公会,特别是高派圣公会,因其圆满地调和了理性和神话为人所接受,因其优美的祷告文为人所公认;然而从本质上讲,它不过是一种自由人文主义的普遍宗教,通过仪式迎合每个个体的喜好。至于自尔是高派圣公会教徒的加布里埃尔,在菲莉帕看来也不过如此。那些为数不多的罗马天主教徒、基督教科学派信徒和不信奉国教的教徒被视为受家庭传统支配的怪人。
\end{quotation}
而像菲莉帕这样的女孩,本身就是社会的接班人,就是高人一等的人,是被鲜花和掌声欢迎的人,她们(当然也包括同样阶层的男生)出身优雅高贵,学业出色,有着光明的未来。但是,从上面的描述中,我们不难看出这一阶层的虚伪和自大。

另一个重要人物是斯凯思。他小时候常常受母亲打击,自尊心比较低,母亲和家里的长辈甚至嘲笑他“杀人?就你?”。他养成了小偷小摸的习惯,偷来的钱和东西让他在同龄的小孩里成为让人羡慕的人,这成为他自信心的来源。后来他爱上了下棋,于是才金盆洗手。成年后他的生活单调而乏味,生命中只剩下了复仇,而妻子的去世更加加重了他的这种心理负担。他看到女凶手玛丽与女儿在一起时那种“更像朋友,克制、和睦、冷静”时生发出嫉妒夹杂仇恨的感情,这种缺少爱和亲密的感受成为他跟踪她们时的主要情感。但是,经过长时候的准备,最终他杀死的,却是玛丽的一具尸体,这是一种无情的黑色幽默,是对他整个生命的无情嘲讽。最终他和盲女在一起,意味着他终于从仇恨的阴影中走出来。

整本小说的色调是阴冷而潮湿的,就如同英国的天气一样。故事里的这些人,过得都不幸福,都有着灰暗的过去,而眼下的生活也是灰暗的,只有结尾处斯凯思与盲女结婚之间有一抹亮色。

这些不幸的人,都有一个悲惨的、不快乐的童年。不知道英国人是不是都是这样。

整体来说,这是一本介于三流和二流之间的小说。繁复的心理描写为人物增加了立体感,这是这本小说价值最大的地方。但是,情节上(特别是菲莉帕与养父莫里斯上床)却又十分狗血,对情节走向没有必要的帮助。阴冷的色调、冷漠的人际关系、疏远的亲情、冷酷的“理性”共同构成了这本小说的基调。基调还是不错的,我很喜欢,但情节上确实弱了一些。

评分:4/5。