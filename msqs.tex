\subsection{《谋杀启事》}

作者:阿加莎-克里斯蒂(Agatha Christie)

英文原名:A Murder Is Announced

行文反复出现“残菊”

\subsubsection{人物}
\begin{longtable}{p{0.15\textwidth}|p{0.4\textwidth}|p{0.4\textwidth}}
    \caption{《谋杀启事》主要人物。}\\
\hline
人物 &	特征 &	事件 \\
\hline
\endhead

\hline
\endfoot

乔尼-巴特 & 邮递员	 & \\
斯韦特纳姆太太	& & \\	
埃德蒙 &	斯韦特纳姆太太的儿子 & 喜欢菲莉帕,想和她结婚 \\
劳拉-伊斯特布鲁克太太 &	比丈夫年轻30岁	& \\
伊斯特布鲁克上校	& &有一把左轮手枪,但没有持枪证 \\
莱蒂西亚-布莱克洛克小姐	 & 百万富翁兰德尔的秘书 & 可能几周后继承他的遗产,后去世 \\
夏洛特-布莱克洛克小姐 & 60多岁,小围场主人,莱蒂西亚的妹妹 & 小时曾经得甲状腺肥大症,后来依靠手术治愈,但留下手术的痕迹,使用珍珠项链遮掩,战时曾在鲁迪家里的位于蒙特罗的旅店住了一年,姐姐死后使用姐姐的名字 \\
艾米-穆加特罗伊德小姐 & 住砾石山庄,身姿丰腴,神色可亲 & 枪杀案发时在门后,手电筒没有照到她脸上,她能够回忆起谁不在场 \\
欣奇克利夫小姐 & & 住砾石山庄,养鸡	\\
戴安娜-哈蒙太太 & 身材圆胖,昵称圆圆 & \\
朱利安-哈蒙牧师 & 60岁,但看起来像35岁 & \\	
“朱莉娅-西蒙斯” & 	真实姓名是艾玛-乔斯林-斯坦福蒂斯,索妮亚的孩子,兰德尔遗产的可能受益人,大约25-26岁,假装是朱莉娅 & 父母分手后跟着父亲斯坦福蒂斯,父亲穷困时就被扔进修道院 \\
帕特里克-西蒙斯	& & \\	
多拉-邦纳小姐 & 与布莱克洛克小姐是同学,年轻时很漂亮,忠诚,乐于助人,但愚蠢 & 六个月前穷困,向布莱克洛克小姐求助,住到小围场\\
菲莉帕-海默斯太太 & 身材修长,相貌标致,面相憔悴,实际上的“皮普”,索妮亚的孩子,兰德尔遗产的可能受益人,大约25-26岁	& \\
米琪 & 小围场的女仆,中欧政治难民,家人都被杀死	& \\
鲁迪-谢尔兹 & 瑞士国籍,饭店接待员,有面具 & 曾向布莱克洛克小姐要钱回瑞士,被拒绝,有前科,偷过珠宝,伪造证件入境,骗子挣一些小钱花,行凶的那把左轮并不是他的 \\		
乔治-赖德斯代尔 & 米德尔郡警局局长,沉默寡言,中等身材,浓眉,眼神犀利	& \\
德尔蒙-克拉多克警督 & 案件负责人,善用头脑,富于想象,严于律己,办事稳健 & \\	
亨利-克莱瑟林爵士 & 苏格兰场前警察厅长,老人,身量高挑,仪表堂堂,马普尔小姐的侄子 & \\	
罗兰森 & 死者供职处的经理 & \\
莫娜-哈里斯 & 亮丽红发,鼻梁高挺,死者女友 & 死者在她面前总是说大话,体贴女人 \\
兰德尔-戈德勒 & 百万富翁,1938年左右死亡 & 落魄时曾向布莱克洛克小姐借钱,后发家 \\
贝拉 & 兰德尔的妻子,卧病在床,几周后可能死去	& \\
索妮亚 & 兰德尔唯一的妹妹	& \\
迪米特里-斯坦福蒂斯 & 索妮亚的丈夫,是个无赖,希腊人或罗马尼亚人,后来称自己为德-古西 & 两个孩子出生三年后与索妮亚分开\\
罗纳德-海默斯 & 菲莉帕的丈夫 & 在印度曾经是逃兵,十天前来到本地,救下一个即将车祸死亡的男孩,自己因伤不治而死 \\
\hline

\end{longtable}

\subsubsection{事件}
背景:奇平克莱格霍恩,英国小镇,二战后从欧洲大陆、印度、中国涌来大批外籍人口。他们身份复杂,人际关系全无,难以查证。

\begin{enumerate}
    \item 死者去刊登杀人启事
    \item 报纸上登出启事:谋杀将于10月29日周五晚上于小围场发生,邀请人们前去,大家都以为只是“杀人游戏”。
    \item 布莱克洛克小姐认为很多人来家里,准备好招待客人
    \item 6点20前,布莱克洛克小姐把鸭子关起来
    \item 帕特里克拿来新雪莉酒和开瓶器
    \item 6点30分,客厅灯熄灭,黑影男子开了两枪,又开一枪自杀
    \item 多拉-邦纳小姐在生日宴会后吃了布莱克洛克小姐的阿斯匹林,中毒在睡梦中死去
    \item 艾米-穆加特罗伊德小姐回忆起了枪杀案时谁不在场(女性),随后被勒死
    \item 马普尔小姐做的推理笔记:台灯。紫罗兰。装阿司匹林的瓶子在哪儿?美味之死。咨询。勇敢地承受起痛苦的折磨。碘。珍珠。莱蒂。伯尔尼。养老金。
    \item 埃德蒙和菲莉帕结婚
\end{enumerate}

真相:借名莱蒂西来的夏洛特-布亚克里克是枪杀案的策划者,也是后两起谋杀案的凶手;她是一个懦弱且心地善良的人,这样的人更容易背信弃义。一旦软弱的人变得害怕,她就会变得更加残忍,更加没有自制力。

\subsubsection{书评}

阿加莎创造了两个经典的侦探形象:一个是全能的、爱惜胡子的波洛,一个是弱不禁风的、风烛残年的老太婆马普尔小姐。在前者的故事里,我们能感觉到波洛的无所不能,会把精力放在案件的进展和推理上,而在后者的故事里(比如本篇),侦探与案件的关系更加紧密,甚至会担心马普尔小姐遭遇不测。本篇故事里,侦探的工作(如采访、调查、大部分推理)主要是由克拉多克警督完成的,马普尔小姐仅担任了关键几步的推理。

与波洛的善辨、自恋(对自己的胡子)不同,马普尔小姐更家常,时常缅怀旧时光,对时下的风俗变迁做出尖锐的、幽默的评价。阿婆本人,以及她笔下的很多人,常有着英国人惯有的一针风血和恶毒,这种英伦风情是阿婆小说的阅读乐趣之一。

本书的推理性较弱,但叙述技巧高超,重读才会发现阿婆在故事和语言安排方面的机锋。

评分:8/10。