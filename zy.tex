\subsection{《昼颜》}

\subsubsection{日剧}
纱和与丈夫有着无性婚姻,丈夫娘气不想生孩子。在贫乏的生活里,纱和的生活也是萧索无味,但潜藏的被呵护和渴望激情的内心一直在涌动。作为全职主妇,纱和在超市做兼职,一时心起偷了口红。口红鲜艳的颜色是她内心渴望被爱的写照,偷口红一事也是她不满于婚姻和渴望激情的写照。偷口红的事情被利佳子看见。利佳子家庭优渥,有着三个孩子,但喜欢一夜情,利佳子帮纱和打掩护,两人相识。初识利佳子的纱和虽然帮忙外遇对象联系利佳子,但对利佳子这种背叛家庭和婚姻的行为充满不屑。

命运果然给纱和开了个玩笑。纱和遇见了生物老师裕一郎,裕一郎温柔帅气,纱和虽然初次约会后心里突突跳,但既然婚姻无法填满她的内心,就只能在欲望下一次次沉沦。两人约会的次数越来越多,直至发生肉体关系。另一方面,裕一郎的妻子乃里子是他的同学,在研究所做研究人员,工作太忙,两者虽然有着五花八门的约定,但很少面对面交流,因此婚姻里除了忠贞外缺少感情上的羁绊。乃里子发现了丈夫出轨,但极力挽留婚姻,迫使裕一郎和纱和签署协定,以后不准见面。

同时,纱和的丈夫在公司里勾搭上了女同事,丑事败露,同时纱和的出轨事件也被他所知,两者离婚,纱和离开。

与纱和的逐渐“堕落”相映成趣的是,利佳子和丈夫下属的签约画家发生了关系,两人不可救药干柴烈火,利佳子甚至愿意抛弃家庭和他私奔。但丈夫发现后极力维持家庭,迫使画家不再与利佳子见面,利佳子也在孩子们的联合挽留下重新回到了家庭。

贯穿本片的表面上是出轨与婚姻的矛盾,其实探讨的是一夫一妻制的危机。三个家庭情况不同,但均是在一夫一妻制下用忠贞和家庭来约束人的内心。可悲的是,这三个家庭除了这种道德和法律上的约束,对夫妻双方感情本身却缺少关注。失败的一夫一妻制,当无法满足人的爱和被爱的需求时,还极力排斥和摧毁其他爱的可能。我在看完本片后,写下意义不是那么明确的一句话:世俗婚姻即使不幸福,也会排斥其他的爱的方式。当然,本片仅仅指出这种问题,并没有反对一夫一妻制,更不是在鼓励出轨。悲剧本身可能会为人带来更大的反思:我们面对这样的婚姻善,应当怎么办?

我在很长的一段时间内,对婚姻和爱情的要求也充满了道德感:感情应当忠贞不渝,约束自己的行为。记得老赵在2007年就说过,一夫一妻制是反人类的,反天性的。当时的我不以为然,天性就是该被规范和约束的。可是越来越多的离婚事件使我想到,人的天性是多么难以压制,它总会在你脆弱时将你击垮,后天的道德和制度表面看起来光鲜和政治正确,但在金玉在外,其内的败絮往往不为人道亦不为人知,人在这种约束下活得够辛苦够悲惨。如果人的天性是世间最大的正义,那么制度和道德为什么不可以做出妥协呢?更何况,这种制度或道德往往实现的是经济和社会上的目的,像宋明的理学,把人作为工具,发展出礼教这种吃人的制度,如今奉为圭臬的一夫一妻制,只是比鼓励女性守贞严防女人逾礼的礼教好了一点而已吧?

我工作的三年里,分别见证了部门领导和科室领导的婚内出轨事件,女方都是自己的女下属。虽然男方已经有了家室,也有了孩子,但谁人知道婚姻是否幸福?谁人又能知道,在看似是权力交易之下,是否真的没有一点的天性散发和真爱在里面呢?

\subsubsection{电影}
如果说电视剧版关注的是婚姻中的挣扎的话,电影版完全就是纯爱系的凄美了。纱和离婚后到海边居住,本以为可以安安静静地按照不与裕一郎接触的条约逃避掉过往,但偶然的一次演讲中又邂逅了命中注定的那个人,两个接触后一发不可收,重新燃起爱情的火苗。但是,裕一郎与纱和的接触被妻子乃里子发现,三人相见后裕一郎决定和乃里子离婚。在离婚的前夜,裕一郎为纱和买了戒指,准备离婚后立即与纱和结婚,但乃里子在车上突然情绪失控,车速过快冲下山道,裕一郎当场死亡,乃里子摔伤,纱和痛苦欲绝,想要卧轨自杀,想到已经怀了裕一郎的孩子,终于悔悟放弃自杀。
电影版的光影处理更加细腻,以至于无可挑剔,配上男女主角日式的衣服,看起来很养眼。