\subsection{《复活》} 

标签: 俄国文学 \ 托尔斯泰

作者:托尔斯泰

译者:草婴

\subsubsection{人物}

\begin{longtable}{p{0.2\textwidth} | p{0.2\textwidth} | p{0.5\textwidth}}

    \caption{《复活》人物表} \\
    \hline
姓名 & 特点 & 事件 \\
\hline
\endfirsthead

(接上表) \\
姓名 & 特点 & 事件 \\
\hline
\endhead

\hline
\endfoot

卡秋莎·玛丝洛娃 & 女犯 & 是未婚女农奴的私生子 \\
索菲娅 伊凡诺夫娃 & 玛丝洛娃的教母 & 想把玛丝洛娃培养成自己的养女 \\
玛丽雅 伊凡诺夫娃 & 索菲娅的姐姐 & 对玛丝洛娃要求严格,想把她培养成侍女,经常打骂她 \\
德米特里·聂赫留朵夫 & 索菲娅的侄子 & 引诱玛丝诺娃,让她怀孕,后去远征 \\
薇拉·谢基尼娜 & 革命党人,政治犯 & 流放途中爱上政治犯克里雷卓夫 \\
? & 贵族女 & 想嫁给聂赫留朵夫,势利,俗气 \\
娜塔丽雅 & 聂赫留朵夫的姐姐,大他十岁 & 婚前与聂赫留朵夫十分谈得来,只在肉体上爱自己的丈夫 \\
拉戈任斯基 & 聂赫留朵夫的姐丈 & 精神自私 \\
西蒙松 & 政治犯,贵族出身 & 在监狱中喜欢上玛丝洛娃,并一起流放 \\ 
诺伏德伏罗夫 & 政治家,受人尊敬 & 自视甚高,迷恋人们对他的尊敬,信奉暴力推翻政权 \\
\end{longtable}

\subsubsection{事件}
\begin{itemize*}
    \item 聂赫留朵夫爱上了姑妈家的女仆玛丝洛娃,两人上床,后聂赫留朵夫去当兵,放浪形骸,忘掉此事
    \item 玛丝洛娃因此恨上聂赫留朵夫,精神堕落,后成为妓女,并被卷入一场谋杀案。
    \item 审判时聂赫留朵夫已成为一方名流,作为陪审团人员,看到玛丝洛娃想起之前的事情,十分羞愧
    \item 玛丝洛娃被错判流放做苦役,聂赫留朵夫找到她想要帮助她,与她结婚,获得心灵上的救赎,但玛丝洛娃不领情。
    \item 聂赫留朵夫为此案(还有一些其他冤假错案)奔走。他回到姑妈的庄园,了解了玛丝洛娃被自己抛弃后的事情。他准备把家产留一部分后分给贫农,虽然那些人将信将疑。
    \item 聂赫留朵夫奔走莫斯科和彼得堡,利用自己的贵族身份和私人交往关系为那些冤假错案求情,但玛丝洛娃依然没有被除罪,还是要流放。
    \item 聂赫留朵夫回到监狱,告知玛丝洛娃结果,准备与她一起流放。玛丝洛娃在之前会见聂赫留朵夫后已经醒悟,精神上不再堕落。
    \item 聂赫留朵夫跟着流放队伍,对玛丝洛娃的感情变由责任和虚荣变为同情和怜悯,并扩散到一切人类。
    \item 流放地中途的监狱里,西蒙松告诉聂赫留朵夫他要和玛丝洛娃结婚,聂赫留朵夫接受了。
    \item 玛丝洛娃的案子被皇帝复审,决定由苦役改为流放。为了不牵连聂赫留凩夫,她决定和西蒙松结婚生活。克里雷卓夫在监狱中病死。聂赫留朵夫在《马太福音》中找到了自己的精神共鸣。
\end{itemize*}

\subsubsection{书评}

沙俄时代的贵族,由东正教的严苛戒律引发,对民众有着巨大的同情。与新教的重视私人财产不同,这些贵族有种巨大的冲动,就是把自己的土地和财产分给贫苦的农奴。虽然聂赫留朵夫最终没有这么做,但是他确实已经在尝试了。1812年左右,同样是托尔斯泰笔下,那些贵族们已经开始认为地主不应当集中那么多的土地和财富了;过了半个世幻,沙俄已经有了更多的革命者、进步者了,这种趋势更加明显,我们不难预测1917年的革命了。

这部《复活》是托尔斯泰较为后期的作品,写于1889-1899年,与《战争与和平》中的饱含民族与仁爱的昂扬激情不同,这本书中作者更加纠结,也更加内敛。这个故事源于朋友给他讲的一个真实的案例,不过那个堕落的妓女很快就死掉了。当然,他基本的现实主义的技法依然炉火纯青——他的笔触犹如放大镜一般真实、细致地展现了社会和人物的容貌,逼真而细腻(“艺术家之所以是艺术家,全在于他不是照他所希望看到的样子来看事物。”),不是靠巧合这种二流技法去推动情节。他笔下的聂赫留朵夫是一个旧贵族,年纪轻轻时与姑妈家的女仆玛丝洛娃通奸,随后就把这事忘掉,去当兵了,并过上了声色犬马的虚浮生活;然而,几年过后,他在法庭上作为陪审团看到了那个因为自己一时的性欲而堕落为妓女的女人(“她”因为被聂赫留朵夫抛弃而对世间的美失去信心,开始作贱自己,以“报复”人世和聂赫留朵夫,她看到聂赫留朵夫时甚至想用职业的妓女式的媚笔来从他身上搞到钱),她被卷入一桩凶杀案中并被冤枉,他动了恻隐之心,开始为当年自己的错误而到羞愧,特别是之前自己还为了那种浮华的生活而洋洋自得,这更加重了他的羞耻之心(这种自我检讨的勇气真是让人敬佩!),并决定赎罪。当然,其实,玛丝洛娃被错判并非因为那些法官和陪审团非要治她的罪,而是由于程序的一些错误,而这种错误在后面的程序中没有被矫正过来,并且那些司法人员内心是想给她轻判的。他开始为这桩冤假错案奔走,甚至为政治犯和其他一些被不公正的司法体系陷害的人去奔走,他利用自己的贵族身份和交际圈,去各处关节做活动,甚至要与玛丝洛娃结婚、把自己的土地分给农奴来赎罪。这个过程里,他不仅接触了底层的农民、手工业者,也接触了司法系统中法院、监狱、枢密院,更接触了一些革命者,十九世纪后期的沙俄的社会在托尔斯泰笔下为我们展开。最终,玛丝洛娃受他帮助与一个革命者结婚,也受到了减刑,他自己也在宗教中得到救赎。两个人,都“复活”了。

当然,聂赫留朵夫在年青时遇见玛丝洛娃时是爱她的,虽然更多是情欲;而在多年以后再与她接触,他依然爱他,不过这种爱,掺杂了很多宗教的普世的爱(博爱),他想与她结婚也是为了能够使自己赎罪,并让她脱离那种妓女的堕落生活,而没有过多考虑两个人在婚姻生活中是否真的合适。婚姻对于聂赫留朵夫而言,确实更多是一种途径和工具,它是作为一种牺牲的方式来实现的,也即只有通过这种牺牲,聂赫留朵夫才觉得自己是在赎罪。玛丝洛娃与西蒙松相识相爱,对于他而言,妒忌之心是有的(他已习惯了自己对玛丝洛娃付出),但更多是她选择了西蒙松,同时就阻止了他的牺牲。

基于神爱世人的博爱和平等思想,聂赫留朵夫对世间的人充满了慈爱之情,司法不公和官僚主义让他认识到当时的刑罚制度不仅造成了冤假错案(一些罪犯仅仅是因为阻止了上层对下层的掠夺和搜刮),而且让那些人不得不在监狱中脱离生产劳动,与小偷、杀人犯这样的人为伍,人格也随之堕落。他于是反对这样简单粗暴的司法制度。同时,他在下层人民和革命者中看到了人性之美,而在上层的腐化、虚荣和体制化中看到了道德的沦丧。他对上层的腐化和虚伪生活是憎恶的,也看到了上层对下层的压迫是造成下层痛苦的原因(“腐化堕落的人想去纠正腐化堕落的人,并想用生硬的方法达到目的,结果是缺钱而贪财的人就以这种无理惩罚人和纠正人作为职业,自己却极度腐化堕落,同时又不断腐蚀受尽折磨的人。”)。对于司法制度,聂赫留朵夫(也即是作者)的思考是它无法纠正人们的错误行为,于是应当取消(“要永远饶恕一切人,要无数次地饶恕人,因为世界上没有一个无罪的人,可以惩罚或者纠正别人。”),并让那些有罪的人“在上帝面前承认自己总是有罪的”,“人人只要履行这些戒律就行,不必再做别的,人生唯一合理的意义就在于此。凡是违背这些戒律的就是错误,立刻会招来惩罚。这是从全部教义归纳出来的道理”,这近于一种无政府主义的行为显示出作者对社会问题的思考并非尽善尽美。他最终的救赎是在宗教中完成的,这恐怕也是作者给当时的社会开的药方,当然这种药方能否真的能医治社会呢?后来沙俄社会的发展给我们带来了答案。作者同情劳动人民,也同情革命者,在他们身上都倾注了爱和尊敬,但对革命者也有批判,例如有些革命者只是为了哗众取宠,并不是真的为人民着想。

总之,这是一部饱含深情的现实主义巨作,不仅有巨大的文学价值,还对当时的俄国社会做了全面的描写和反映。虽然作者有深厚的人文主义精神,但是给社会开出的药方并非良策。托尔斯泰是伟大的文学家,现实主义技法十分高超,但是给出的历史的机械主义的解释(《战争与和平》)和本篇体现的司法制度的改良,都显示出他并非一个好的社会学家。本书的翻译是大师级的,用语典雅而流畅。

评分:5/5。