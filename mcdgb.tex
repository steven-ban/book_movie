\subsection{《漫长的告别》}

标签: 小说 \ 推理 \  侦探

书名:The Long Goodbye

作者:【美】雷蒙德·钱德勒

\subsubsection{标注}

1. (哈兰·波特)我们生活在所谓的民主社会,多数人统治少数人。一个美好的理念,要是行得通就好了。人们选举,但提名候选人的是党派机器,党派机器的有效运转依赖于花费大量金钱。钱必须有某人出,这个某人无论,都会期待一定的回报。我和我这种人期待的是允许我们享受体面的隐私生活。我拥有几家报纸,但我不喜欢他们,在我看来,它们永远在威胁我们所剩无几的隐私。它们总是在叫嚣的新闻自由,除了极少娄可敬的例外,只量兜售丑闻、犯罪、性爱、感官刺激、仇恨和含沙射影的自由,只是政治和金钱利用宣传工具的自由。报纸这门生意挣钱靠的是广告收入。广告收入靠的是发行量,你知道发行量取决于什么。……金钱有个特别之处……数量大了,它就会拥有自己的生命,甚至自己的道德准则。金钱的力量会变得难以控制。人类向来是贪婪的动物。人口的增长,战争的海量消耗,抢夺性重税的无止境压力——这些东西让人变得越来越贪婪。普通人活得疲惫而惶恐,一个疲惫而惶恐的人负担不了理想。他必须养家糊口。我们这个时代见识了公德和私德的令人震惊的退步。人们的生活遭受品质缺乏的	戕害,你不可能期待他们拥有品质。大规模生产没有品质可言。你不希望货物的品质太好,因为品质好就会太耐用。于是你用式样替代品质,这是一种商业欺诈,旨在人工营造过时的感觉。大规模生产必然让今年的货物到明年看上去不够时髦,否则明年的货物就卖不出去了。我们拥有全世界最洁白的厨房和最闪亮的卫生间。然而可爱的洁白厨房没法让普通的美国家庭主妇做出能下嘴的食物,而可爱的闪亮卫生间基本上只是个容器,用来存放除臭剂、通便药、安眠药和所谓化妆品业这个欺诈行当生产的各色产品。我们制造全世界最精美的包装,马洛先生,里面的东西以垃圾为主。

(马洛的回答)你不喜欢世界运行的方式,于是运用权力制造出一个私密的角落,尽可能像你记忆中五十年前大规模生产时代没有到来时的人们那样生活。你有上亿家产,给你带来的却只有头疼。

2. 大钱就是大权,大权永远被滥用。这就是制度。也许是我们能得到的最好的制度,但离无懈可击还差得远呢。

3. 世上存在叫法律的那种东西。咱们被它都淹到脖子了。它的功能就是为律师拉生意。要是律师不教黑帮老大如何动作,你觉得他们能混多久?

4. 生命的悲剧,霍华德,不在于美丽的事物过早衰亡,而在于它们变得苍老和鄙俗。

5. 黑帮、犯罪辛迪加和打手团体会存在不是因为奸猾的政客和他们在市政厅及立法机构的走狗。犯罪不是疾病,还是症状。警察就像给脑瘤患者开阿司匹林的医生,除了警察更喜欢用警棍给病人治病。我们是博大、粗鲁、富裕、狂野的一群人,犯罪是我们为之付出的代价,有组织犯罪是我们为组织付出的代价。犯罪会陪伴我们很长时间。有组织犯罪仅仅是万能金钱的肮脏一面。

\subsubsection{人物}


\begin{longtable}{p{0.18\textwidth} | p{0.3\textwidth} | p{0.45\textwidth}}
\caption{《漫长的告别》人物表。}\\
姓名 & 特点 & 情节 \\
\hline
\endhead

\hline
\endfoot
“我”(菲利普·马洛) & 私家侦探 &  \\
特里·莱诺克斯 &右脸有疤痕 &1942年曾经在挪威战场服役,被炸伤 \\
西尔维娅·莱诺克斯 & &  与特里多次结婚复婚 \\
哈兰·波特 & 西尔维娅的父亲,富豪 &强势,用金钱和权势来维持自己的隐私 \\
塞沃尔·恩迪科特 & 哈兰雇佣的律师 & 把马洛从拘留所里捞出来,去墨西哥处理特里的后事 \\
门南德斯(门迪) & 黑道人物 & 受雇于哈兰,平时挣钱很多,警告马洛不要插手,与特里是战友 \\
斯塔尔 &赌场人士 & 特里的战友 \\
罗杰·韦德 & 作家,成功,精神不正常&与西尔维娅有奸情 \\
琳达·洛林 & 西尔维娅的妹妹 & \\
洛林医生 & 琳达的丈夫 & \\
艾琳 &韦德的夫人,漂亮 & 和特里十年前有情史 \\
坎迪 & 韦德家的仆人,智利人 & \\
莱斯特·乌坎尼奇医生 &低等医生  & \\
阿莫斯·瓦利医生 &有大房子,开一家临终老人疗养院 & \\
霍华德·斯宾塞 & 韦德的出版人 & \\
乔治·彼得斯 & 调查机构的人 & \\

\end{longtable}

\subsubsection{主要情节}

\begin{itemize*}
    \item 二战时保罗·马斯顿与艾琳相爱,后被炸伤。战后回美国,改名特里。
    \item 特里与“我”结识
    \item 艾琳杀死西尔维娅和韦德,后自杀。
    \item 特里找“我”讲明妻子西尔维娅被击碎头部杀害,“我”帮他离开洛杉矶。在墨西哥,经过门南德斯和斯塔尔的协助,特里瞒过当地警察,成功逃脱并整容。
    \item 韦德失踪,艾琳找马洛,找到韦德。
\end{itemize*}

\subsubsection{书评}

钱德勒这本书属于侦探小说,与阿婆那种重推理、情节离奇、步步为营不同,这本书里的主人公马洛是所谓“硬汉侦探”。他通晓人情事故,混得不好不坏,有一点名气但并非名满天下人人皆知。他重义气,重个人感觉,爱憎分明。他极度入世又和世界的污浊保持着距离。他吃得苦,为了朋友可以蹲监狱,可以开枪,可以为了朋友而替他隐瞒实情。总之,这是一个有血性有感情的侦探,一个值得托付的朋友。“钱德勒的主人公菲利普·马洛并不是个老套的硬汉形象,他是一个复杂的、情感丰富的人物,会结交新朋友,会上大学,会说一些西班牙语,有时候会羡慕墨西哥人,喜欢国际象棋和古典乐。”(卷尾编辑总结)书名《漫长的告别》也即是对朋友离开的那种不舍,有“后会有期”的味道,这个“漫长”也充实着小说超过五分之四的篇幅,帮助朋友逃离之后,马洛通过自己的查访,很早就知道了朋友没有死,一直在为下次相见做着准备。

小说的故事发生在50年代的洛杉矶,围绕着一个退伍军人的“杀妻案”展开。随着情节步步推进,我们知道钱德勒是想讲一个战争摧毁了相爱的两个人的故事。保罗·马斯顿与艾琳相爱,但欧洲战场使两个天隔一方,艾琳以为保罗已经死去,但战争过后保罗回来,两个遇见,但已经物是人非,分别属于各自另外的枕边人。艾琳为了曾经的爱人,接连杀死了保罗放荡的妻子和自己的丈夫,最后自杀。马洛在这个过程里,又是替保罗遮掩逃脱,又是因这事坐牢,又是为艾琳照顾她的丈夫。当他得知真相后,又替艾琳遮掩。

这是一个相对离奇的故事,钱德勒对叙事节奏的把握相当好,一波一波的事件接踵而至,丝毫不会让读者乏味,也使得这本书可以吸引人一口气读下去。钱德勒的语言风格风趣而冷峻,尤如一把尖刀,割开了美国社会的浮华与凉薄。他的叙述不是平铺直叙的,而是显露着对人世的洞察。

与主线的爱情故事相比,我更喜欢钱德勒对五十年代美国金钱和资本主导下的社会的讽刺。作者借马洛和警察、富豪之口,对美国社会金钱万能和主宰一切的现实进行了揭露,这一点可能是本书最为深刻的一点。但是,作者显然浅尝辄止,没有在这样一部通俗读物中深挖下去(情有可原)。

当然,这究竟是一本通俗小说,情节本身并没有逃出离奇和悬疑的套路,对人世的认知仍然是通俗的而不是超脱的,与真正的文豪相比仍有一定的差距。至于村上春树对钱德勒的追捧,我觉得正表现出他的平庸。求乎上者得乎中,求乎中者得乎下,怪不得村上春树不是一流的作家,这和他的眼光不无关系。

作为一本美国小说,这个小说的翻译还得过得去的,虽然没有到达大家水准,但已属合格了。

评分:6/10。