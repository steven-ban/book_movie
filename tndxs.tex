\subsection{《童年的消逝》}

标签: 美国文化 \  传播学 \  童年 \  文化史

作者:【美】尼尔·波兹曼\footnote{参考\url{https://baike.baidu.com/item/尼尔·波兹曼/10369934?fr=aladdin}}

\subsubsection{童年如何产生?}
童年的产生和印刷术密切相关。希腊和罗马文明是有儿童的位置的,他们对待儿童与对待成人已经有了不同。但是,随着欧洲中世纪的到来,蛮族入侵取代了罗马文明,文化被摧毁,知识被把持在教会等少数人手中,婴儿一旦过了七岁,就和成人无异,他们需要做成人同样的劳动,无需学习(因为没有东西可以学习),知识的传播依靠口口相传和行会。这时,成人和儿童之间除了年龄的差距,其认识水平并没有差别,成人之间的谈话毫无顾忌地出现在儿童前面,这包括性、暴力和各种粗口。

然而,随着印刷机的发明,事情起了变化。印刷机使得知识的传播大大加快,同时人们需要经过学习才能掌握阅读的技巧,才能掌握由书本承载的知识和文明。印刷机促进了个人主义,也产生了知识差距。据此,儿童和成人之间出现了一个鸿沟——需要通过学习才能掌握的阅读技能。因此,**与其说是“童年的产生”,不如说这是“成人的产生”更为合适**。当然,印刷机的影响是深远的,它还打破了教会对知识的垄断,产生了宗教改革运动,促进了民族文字的产生和发展,促进了科学和思想的传播。印刷机带来了启蒙运动。

随着成人概念的兴起,相应的儿童概念成应运而生。为了掌握阅读的技能,儿童需要进入学校,学校对于不同阶段的划分了细化了儿童本身的阶段。中产阶级的兴起,为儿童的产生做好了铺垫。在家庭内部,儿童不再被称为“小大人”,而是“未掌握成人技能的人”。成人世界和儿童世界开始分野,儿童被加入了各种纪律的学习、自制力、性禁忌等内容。这一切,就发生在15-17世纪里。虽然遭受了使用童工(发生在低阶层的人里)、教会的阻挠等等,但儿童与成人这种差异最终发展了起来,成为不可避免的趋势。

\subsubsection{童年理论的构建}
18世纪里对儿童理论的构建,存在着\emph{洛克}和\emph{卢梭}两种认识。前者认为:儿童是不完成的成人,需要成人进行引导,按照成人的样子(读写能力)进行教育;而后者认为:儿童是一株天然的植物,应避免对他们天性的剥夺。19世纪,美国主要继承了洛克的学说。

在19世纪末,\emph{弗洛伊德}和\emph{杜威}给出了两种不同的童年的解释。前者认为:儿童存在着自然状态,但需要成人给予正确的教育和引导;后者认为:应当关注儿童当下的状态,以儿童\emph{现在}的需求来进行塑造。

\subsubsection{童年如何消逝?}
电报的发明开始了童年的终结。电报以及之后的电话、电影和电视等发明,取消了信息传递的现场性,使得信息更多以“不相关的新闻”的方式出现,而非印刷时代重视逻辑性、参与性和主动性的方式。电视的产生彻底终结了成人和儿童之间的认知和能力上的鸿沟:电视以影像和声音的方式来传递信息,儿童不再需要严格的训练就能看懂电视,因此它取消了成人的独特性,借此也取消了成人的对立面——儿童。

电视带来的认知方式是直观的,同时它强调“现在的信息是这样,但我们下一个话题是……”,这和阅读完全不同。电视的影像塑造了一种“现场感”,永远都是“现在”。它用直观的感官刺激和“快乐”的心理暗示来诱惑观众,用支离破碎的事实来塞满人们的时间。相比于“文字时代”,这是一个“叙事时代”。作者说:
\begin{quotation}
在电视广告寓言里,邪恶的根源是“技术无知”(technological innocence),即对工业进步所带来的种种益处一无所知。
\end{quotation}

作者可能只是想表达表面上的意义(这样的表述在整个逻辑链里并没有那么重要),但在我看来,抽离语境,这其实是整个电视时代观众素质的暗喻。由于仅仅享受电视带来的现在感,人们忘记了过去,陶醉于这个工业时代,而对这个时代得以来源和运行的原因一无所知。在中国,电视时代是伴随着义务教育和工业化同时来到的,但在现在的社会里,仍然存在着大量识字的文盲。

另外,由于电视长驱直入,进入了儿童的世界,它也消除了成人世界里的神秘和禁忌,儿童可以收看到成人相同的内容,他们的道德观念由教育替换为电视的熏陶。童年消失了。因此,\emph{“童年的消逝”同时(概念和时间上)也是成年的消逝}。

在第8章里,作者论述了童年消失的几个证据(在我看来它们作为证据的因果性并不强):儿童发育提前、电视上儿童的形象越来越和成人相似、电视节目越来越模糊儿童与成年人的差别、儿童体育活动越来越成人化、青少年犯罪越来越严重……整体来看,美国的儿童正在越来越像成年人;相应地,成人也越来越像儿童,他们不再以“成人”自居,而是喜欢抛弃自己的责任,甚至像儿童那样任性。在我看来,这或许能表明作者构建出来的“印刷时代”的儿童概念的消失,但从社会整体角度上来看,其实是青少年文化的兴起,这个社会里的儿童越来越不耐烦漫长的不被重视的儿童时光,他们急于长大,急于被当成成人,于是他们主动或被诱惑地模仿成人,参与到成人的活动中去。电视媒体只是被动地去迎合这个市场而已。至于青少年犯罪问题,其实我觉得中世纪和印刷时代未必比现在少,只是那时候没有被记录而已。作者在对上述文化现象进行剖析时,有点杞人忧天了。

\subsubsection{关于童年的六个问题}
作者在第9章提出了关于儿章的6个问题:
\begin{itemize*}
	\item \emph{童年是被发现的,还是被发明的?}这其实是问:\emph{童年是生物学上的,还是文化意义上的?}显然,根据上面的论述,作者认同后一个观点。
	\item \emph{童年的衰落预示着美国文化的普遍衰落吗?}20世纪电视文化的兴起,代表着科技力量的兴起,作者对科技保持着敬畏和怀疑。当然,作者明显沉醉于发源于印刷文化的美国近代历史,他的思想是偏保守的,而且更多是从人文的角度上来看科技的进步。作者举了奥威尔、赫黎胥、阿西莫夫等人的例子,我觉得这个语境不太正确。无论是反乌托邦还是科幻小说,对于科技的审视角度都不全面:前者保持着怀疑,后者保持着过分乐观。就我而言,科技的进步会创造出更强大的人文,然而这个过程并非自动进行的,在这个过程里,需要人文学者抛弃成见,积极拥抱技术进步,并从人文角度对科技进行反思和一定程度的限制。几十年过去了,世界并没有因为技术进步而更差。
	\item \emph{道德多数组织和其他宗教激进组织在保存童年方面究竟出了多少力?}这其实是美国社会里保守派和进步派的争论,是右和左的争论。面对新生事物,“道德多数组织”往往采用过去的道德观点来看待,过于指责,虽然这也有巩固社会共识和平衡社会力量的作用,但道德多数组织应当更充分地去接触和认识他们所指责的东西,而非乱指一气。世界的变化是永恒的,老办法往往解决不了新问题。人类需要冒险,社会也需要一定程度的冒险。
	\item \emph{有没有一种传播技术具备某种潜能,足以保持童年存在的需要?}根据成人与儿童能力上的差异,作者认为电脑可以类似阅读那样需要儿童的大量学习才能掌握,因此可以成为一种“反电视”的媒介。考虑到本书写于80年代,电脑还没有成为家庭必备的产品,图形用户界面还没有普及(因此人们需要学习大量的命令和操作才能使用电脑),甚至互联网还没有被发明。20年代,作者恐怕会收回自己的话。电脑的普及比电视更加使成人与儿童的差别减小,互联网带来的传媒革命比电视更甚。看到现在的新新人类,作者肯定要感叹电脑其实是更电视的电视,更加速了“童年”的消逝。
	\item \emph{有没有任何社会机构足够强大,并全心全意地抑制童年消亡的现象?}作者认为家庭和学校应当去做这件事情。现在看来,答案是没有。现代社会家庭越来越小,不婚主义和丁克现象越来越多,童年的消亡无法阻挡。作者下面的话如果在2018年说出来,估计要被女权主义者喷成筛子了,虽然我不认为“喷”是正确的:
	\begin{quotation}
正是妇女,也只有妇女,才是童年的监护人,她们始终在塑造童年和保护童年。让男人抚养孩子的说法无论多么有道理,男人不可能在抚养孩子方面承担任何妇女所扮演的并依然在扮演的角色。
\end{quotation}
	\item \emph{在抵制下所发生的一切时,个人完全无能为力吗?}作者倡导父母为孩子的教育倾注更大的心血。
\end{itemize*}

\subsubsection{中国的文化里有童年吗?}
按照作者的标准——儿童与成人之间存在认知能力上的差异使童年得以产生——,中国古代也是不存在统一的儿童观念的,虽然人们常常使用儿童一词。最为贫苦的农民占据了人口的大多数,他们不可能识字,同时他们的后代也难以有读书的机会。考虑到古代极低的认字率,他们的后代需要在十几岁就跟着他们务农,十五六岁结婚(甚至到我初中时,十四岁的同学已经在父亲的要求下结婚了,并且很快有了孩子)生子,因而对于这些孩子来说,是不存在童年的。有些农民会有要求儿子读书的想法,因而这些孩子需要进行学堂的学习,并开始着手准备科举考试,这样一个学习的过程也造就了童年。至于上等人和一些书香门弟,必然会促使自己的后代读书认字,因而也存在童年的概念。

但是,这和欧洲十六世纪后的情况相比还是有所不同。欧洲当时也是中产阶级最热衷于孩子的教育,而无产阶级没有能力让孩子读书。但是,到了十七世纪,英法这些国家的识字率已经高到50\%左右了,这比清末的中国的识字率还高。并且,由于国家层面上推行一体化的教育,因而对儿童进行教育的习惯很快由中产阶级推广到下层。相比之下,虽然中国也有很自觉的读书传统,但大部分人仍然没有机会进行教育。因而,作为整体而言,依照作者的标准,恐怕中国古代也是没有童年这个概念的。它只存在能够读书的少部分人身上,例如《红楼梦》里宝钗小时候看杂书而被父亲斥责烧毁的行为,明显就是“童年”的特征。


\subsubsection{作者对东方文化的无知}
在论及印刷机时,作者拿远东的情况进行了类比:
\begin{quotation}
中国人和朝鲜人(他们在古登堡之前就发明了活字印刷术)当时可能有人,或者没人有天分看出活字印刷术的潜在价值,但他们肯定没有字母,即一个字母体系的书写方式。因此,他们的“魔鬼”又去睡觉了,没有发挥出应有的作用。
\end{quotation}

活字印刷术是中国发明的,但在中国出版行业中没有太多应用,这是由中文作用象形文字的特点所决定的。在活字印刷术以前,中国就已经成熟应用了雕版印刷术。但是,由于中文有着大量的单字(常用的好几千,加上不常用的可能上万),采用活字印刷极易出错,因此在刻版完成后需要进行校对,不如由单独的雕版(只需要一个工匠就能完成)便易,同时常用的汉字在排版时需要使用多个,字形不统一,不美观,字模之间不整齐也会千万排版出来的汉字不能整齐划一,因此活字印刷只用来进行简单的排版,大量的印刷工作是由雕版来完成的。对于大规模印刷的经、史、子、集这样的经典典籍来讲,一旦雕版完成,则可方便地进行复印,不需要重复排版,更适合雕版印刷。

另外,雕版在刻版时只需要识字的人先在模具上写上字,再由工人依样雕刻,工人不需要识字就能完成,降低了人力成本。而活字印刷不仅需要预先排好,还需要工人认字,这在古代中国是很难的,会增加成本。另外,字模的材料、吸墨效果(古代均为水性墨)及成本,均是考量因素。

作者认为中文没有字母体系的书写方式,认为中国和朝鲜是“野蛮”的,这简直是大错特错了。中国能将从南到北十几个省紧密联系成一个大帝国,所靠的正是文字的统一和印刷术的实施。虽然印刷与抄写同时存在,但这不影响中国文明的先进。

因此,活字印刷在不同的文化背景里,并没有任何优劣之分。字母文字的活字少,因此适合活字印刷,而象形文字更适合雕板印刷,技术需求不同,因而技术路径也有差别。

西方学者对中国文化的无知和漠视,常常会导致他们得出只适合西方而不适合东方的结论,借此来概括世界的情况,也会偏颇甚至错误。相比中国普通研究者对西方的熟稔,他们对于世界整体的看法,很可能弱于中国学者,虽然目前中国学者的观点很难进入他们的圈子。

\subsubsection{作者的观点值得被赞同吗?}
我对作者的论述持怀疑态度。诚然,在欧洲和美洲,印刷术的流行加剧了儿童与成人的差别,儿童需要接受漫长的教育才能成为真正的“成人”。然而,这恐怕只是儿童与成年人众多差别中的一个。如果作者能更多地关注世界整体历史(特别是中国的历史),把童年看成是“成人化”的一个阶段,那么其实童年一直都在,也一直没有消失。电视的流行,以及随后互联网的流行,虽然使得人们接受娱乐的方式同质化,但儿童还是需要经历那些成长的伤痛才能觉得自己是一个大人。因此,我觉得作者的眼光太狭隘了,同时论述也不是很有逻辑。作者所抱持的那种精英化的和保守的立场充满了整个的论述,限制了他对这个问题的理解。

与他的成名作《娱乐至死》相比,本书的价值要低上很多。本书也就电视文化进行了详细论述,这和《娱乐至死》很多观点和价值观是雷同的。

评分:6/10。
