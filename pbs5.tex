\subsection{越狱第5季}

越狱是我第一部看的美剧,大学时在宿舍用组装机和大头显示器追了前四季,之后打开了我观看美剧的大门,比如《24小时》《权力的游戏》等等。
前四季一口气越了四季的狱,几个人亡命天涯与国家机器斗智斗勇。第四季时,麦克死了,大概福克斯也想终结了这个系列。之后的七年,麦克的扮演者Miller出柜了,谁也不会想到这个剧集能重启。

但第五季真的来了,从情节上比前四季更大气更刺激,不过,硬伤也更多。第五季里,越狱的场所更换到了也门,CIA、ISIS、腐败的也门当局成了几方角力的战场,麦克七年前诈死骗过了亲人,成了大反派波塞冬的白手套,潜入狱中是为了把ISIS的头目拉马尔救出来。萨拉已经改嫁给了大反派,和小麦克一起过着没有过去的生活,直到林肯把冰山一角揭开。林肯只身前往也门解救麦克,花钱打通政府关系。而狱中的麦克策划的越狱行动四年前就失败,这一次也失败了。困到狱中的麦克在ISIS攻进城内、监狱工作人员仓皇出逃、罪犯试图杀死拉马尔的情况下和拉马尔协力逃出,途中拉马尔被麦克的小弟杀死,几人与林肯汇合后开车逃往北部,麦克只身引开追兵,受伤生还。几人回到国内,和波塞冬展开较量,最终救回儿子。

T-Bag回归了,小波多黎各人也回归了。让人印象深刻的是T-Bag,被植了机械手,遇见了儿子(就是麦克的小弟)。可惜结局太仓促,儿子被枪杀。我觉得这一点太可惜,也不太符合常理,以麦克的智商,不应该犯如此错误。

萨拉相较前四季更成熟了,也机智和勇敢了不少。只身前往希腊为麦克疗伤也是真厉害。

林肯还是那么猛,兄弟情更加醇厚。

麦克多了苦逼相,计划往往赶不上变化:越狱一再失败,出狱后赶往火车站也是赌徒心态居多,与波塞冬的较量也是智商不够天外情节来凑。

波塞冬起点很高,但后期太弱,明显不合逻辑。