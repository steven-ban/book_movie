\subsection{《人间失格》}
\subsubsection{书评}

这本书包括了《人间失格》和太宰治的其他几个短篇,内容无非是沉郁消极的男人与深情的女人在日本二战前琐碎悲惨的生活。

不知道什么原因,太宰治这几年挺流行,言洞悉人性云云。但看了此书,对这个人很失望,对大众也失望。此人不过毫无斗志、阴沉消极的渣男而已,算哪门子的洞悉人性啊!

太宰治自称对人性看得很透彻:
\begin{quotation}
大抵说来,我对文学一无所知。可正是得益于此,我能将人性看得更加真切。所谓人类,不过是菜市场的苍蝇。于我而言,作者其人才是全部,作品则毫无意义。
\end{quotation}

这简直就是胡扯,即使可能断章取义,这句话这任何时候也经不起推敲。首先就是逻辑上不通:为何正因为“对文学一无所知”,就更看清人性了呢?文学难道不是人性的表达吗?你连媒介都不懂,何谈看得清传达出来的人性?其次,作者作为讲述者,并非作品的的全部。一部作品一旦创作出来,就属于大众,大众可以对作品进行解读,对于高明的作品而言,这种解读会历久弥新,并不会仅仅在乎作者的诠释。马东的一句话对其有好解释“被误会是表达者的宿命”,因此作者对于作品而言,有时也可以忽略不计。比如《金瓶梅》,现在也不知道作者身世,但不影响这是一部最伟大的作品。

《人间失格》大约是作者的自传,作者声称自己”我的不幸,恰恰在于我缺乏拒绝的能力。我害怕一旦拒绝别人,便会在彼此心里留下永远无法愈合的裂痕“,这种事事”为他人着想“的能力,其实只是一种乡愿而已。我觉得做人,能做到道德完美固然好,但这种人本来就少,而大多数人,只是平凡的有道德缺陷的人,但同样可以活得豁达,不必为一点道德瑕疵耿耿于怀。

作者小时被家人和仆人嘲笑,于是觉得了搞怪的本领,在引起他人发笑的过程中隐藏了自己的虚弱和无助。成年后进入东京,同样继承了这种做法,于是深得女人喜爱。但他做事缺少抱负,无可无不可,主动丧失了自己对事件的控制力和责任,因此对女人随波逐流,一有困难就退缩,因此慢慢被社会隔离,不堪忍受世间的冷眼而自杀。因此,作者只是从偏颇视角中看清了某些世人的一些行为,并没有对人性有什么洞察。人性不仅有阴暗,还有光明,但作者只看到了阴暗,只顺从阴暗,因此越来越居于下流,自杀也算是一种解脱。

多说一句,从太宰治的行为和在日本文学中的地位,也大约能看出日本文学虽然有深刻的一面,但过于剑走偏锋,没有中华文化的中庸和正道,因此人多诡诈猥琐,也就不奇怪了。

因此,在我这里,太宰治的作品不值得推崇。满分五分的话,这部作品只能给两分,这两分,也是给太宰治的。
\subsubsection{标注}
1. 少年们啊,无论你们今后度过多少岁月,都请不要介意自己的容貌,不要吸食香烟,若非节日,也别喝酒。长大后,请多加爱惜那性格内向、不爱浓妆的姑娘。

3. 你是贤妻良母,而我是不良少年、人之渣滓。

7. 至少在都市男女中,女人比男人更具有仁厚的侠义心肠。男人们做事大都畏首畏尾,只重门面,还很吝啬),

8. 那碗年糕红豆汤和开心地喝着汤的堀木,让我看到了都市人节俭的本性,看到了东京百姓清楚区分内外关系的真实面目。城里人的生活将我这个不分内外、只会不断逃避人生的肤浅的笨蛋彻底拒之门外,甚至于堀木也弃我于不顾。

10. 如果他当时直截了当地说清,我应该也会照他说的去做。可是,由于比目鱼过分谨慎、拐弯抹角,令这次谈话很不顺利,甚至完全改变了我的人生轨迹。

11. 我总是尽可能地避免介入人世间的纠纷。被卷入是非纷争的旋涡,让我感到恐惧。

14. 自己对绘画的理解一直存在偏差。一直以来,我捕捉美好的事物,努力展现它原有的美好。这种做法太过稚嫩、太过愚蠢了。真正的大师,能以主观力量,在平淡无奇的事物中创造出美,或许丑陋的事物令他们隐隐作呕,但仍无法阻挡他们的兴趣,大师们沉浸在表现事物的喜悦中。换言之,他们不被他人的想法所左右。

15. 对人类极度恐惧的人,反而会比任何人都渴望亲眼见识妖怪的可怕。愈是敏感、愈是胆怯,愈会企盼暴风雨降临得更加猛烈。

16. 女人却不懂得适度,永远不断索求,我为满足她们毫无节制的要求,时常筋疲力尽。她们着实能笑。女人似乎比男人更能享受快乐。

17. 与男人们的鞭笞不同,女人带来的伤痛犹如内伤,经久不愈。

19. 我隐忍不言的孤独气息,总会被大多数女性本能地捕捉到。这也成为多年之后,自己频频被女人乘虚而入的诱因之一。

20. 年幼时我受到家中用人的侵犯,是他们让我体会到了世上的悲哀之事。我至今依然认为,对幼小孩童做出此等行径,是人类所犯罪行中最为丑陋、低级且残酷的。但我却忍气吞声,只觉得又发现了人类的一种特质,对此,我唯有无力地苦笑。

21. 但我是那种即使饿了,也无法自察的人。

