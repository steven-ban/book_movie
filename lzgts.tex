\subsection{吕思勉《中国通史》}
汉代人的议论,我们要是肯细看,便可觉得他和后世的议论,绝不相同。后世的议论,都是把社会组织的缺陷,认为无可如何的事,至多只能去其太甚。汉代人的议论,则总是想彻底改革的。

民族愈开化,则其自觉心愈显著,其斗争即愈尖锐。处于现在生存竞争的世界,一失足成千古恨,再回头是百年身,诚不可以不凛然了

奴隶的源出于异族。

国人的性质较文,野人的性质较质。

一个国家,其初立国的基本,实在是靠国人的,即征服部族的本族。

民主的制度,可以废坠,民主的原理,则终无灭绝之理。

原始的制度,总是民主的。到后来,各方面的利害、冲突既深;政治的性质,亦益复杂,才变而由少数人专断。这是普遍的现象,

封之义为累土。两个部族交界之处,把土堆高些,以为标识,则谓之封。引申起来,任用何种方法,以表示疆界,都可以谓之封

社是土神,稷是谷神,是住居于同一地方的人,所共同崇奉的。故说社稷沦亡,即有整个团体覆灭之意。

政权的决定,在名义上最后属于一人的,是为君主政体。属于较少数人的,是为贵族政体。属于较多数人的,是为民主政体。

氏是所以表一姓之中的支派的。

姓的起源,是氏族的称号,由女系易而为男系,

《清律》:分析家财、田产,不问妻、妾、婢生,但以子数均分。奸生之子,依子量与半分。无子立继者,与私生子均分。

在国家兴起以后,此项权力,实与国权相冲突。所以国家在伦理上,对于此等大家族,虽或加以褒扬,而在政治上,又不得不加以摧折。

聚居之风,古代北盛于南,近世南盛于北,

分工使个性显著。

大宗宗子无后,族人都当绝后以后大宗。

凡是法律和习惯限制男女性交之处,即有卖淫之事,随之出现。”

一切社会制度,皆以经济状况为其根本原因。

血族结婚,有害遗传,本是俗说,科学上并无证据。\footnote{作者这种说法当然是错误的。}
当大局阽危之际,只要能保护国家、抗御外族、拯救人民的,就是有功的政治家。

公元3世纪中叶,中国商船开始西向,从广州槟榔屿,4世纪至锡兰,5世纪至亚丁,终至在波斯及美索不达米亚独占商权。

政治上的纲纪若要挽回,最紧要的是以严明之法行督责之术。

在古代,亚洲东方的民族,似可分为三系,而其处置头发的方法,恰可为其代表,这是一件极有趣味的事,即北族辫发、南族断发、中原冠带。

研究的方法,要合之而见其大,必先分之而致其精。