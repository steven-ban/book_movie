\subsection{《人类砍头小史》}

\subsubsection{感想}
这本书是和《人类酷刑史》一块买的,看书名两者类似,但其实内容相差极大。《人类酷刑史》充满着作者的幽默,充满了小故事,是餐后零食;《人类砍头小史》是砍头与头颅的深沉思索,是大菜。我看《人类酷刑史》用了一个星期,而《人类砍头小史》五个月才看完。

人类对同类的头颅有着极大的兴趣,曾经为了获取宗教、少数种族或战场上的人头而无所不用其极。人头的获取可能是刽子手充满技艺下的杰作,可能是越来越精密的断头台的一瞬间切断,也有可能是花费金钱与土著的讨价还价,也有可能是战场上九死一生下的战利品。人头是恐怖的,也是刺激的,因此可以是有价格的,尽管人命向来认为不能用金钱衡量。人头离开了某个人,既是这个人生命的延续,又不再是生命本身,因此人头所代表的dignity是复杂的。

人头的保存需要良好的解剖、腐蚀和福尔马林浸泡,最终成为骷髅。人头也可能经过干燥处理而将肌肉和毛发保存,从而可以瞻仰死者的音容笑貌,但那种表情肯定是恐怖的,因此它太像生者而不再是生者,类似人造玩偶的恐怖谷效应。

现代文明不再认为人头可以买卖,但人头成为科研和医学下的资源。医学院的学生会解剖人头,初期会不适应,会怀疑死者与生前的关联性。在伦理方面,脑死亡是否真的是死亡还存在争议,因为即使大脑停止了运转,身体本身还仍在维持一定的生命活动。将来如果人体冷冻技术足够发达,死亡后的人脑可以保存下去,那么砍头行为还能称为生命的终结吗?或者只能成为残疾。

总之,本书的信息密度特别大,作者(是个大美女)的思辨使得阅读本身变得极其缓慢,但仍旧是值得读完的。

不喜欢本书的书封:百分之九十九喜爱《晓松奇谈》的人都喜欢本书。其实喜欢《晓松奇谈》的人,应该会喜欢《人类酷刑史》,不太会喜欢看这种沉闷的书。不理解为什么出版社去碰瓷一个毫无关联的网络脱口秀。

\subsubsection{标注与评论}
1. 在人类历史上的大部分时间里,活人一直转向死人寻求一点点魔力,因为死人的尸体是令人兴奋的东西。

2. 19世纪初,恐怖是大买卖。廉价恐怖小说都是畅销书,与此同时,凶险的戏剧表演也吸引了密密麻麻的观众,在蜡像博物馆里可以找到“恐怖屋”,还有幻影魔法灯会吸引着围观的人群,上演的是起死回生的骷髅和死尸,更不用说巴黎市停尸房里展出的真家伙,或者断头台本身的定期演出,它始终能确保庞大的观众群。

3. 在19世纪,停尸房是巴黎最吸引公众关注的地方之一。

恐怖电影就是寻求刺激的好东西,这和恐怖的生理刺激紧密相联。

4. 更常见的情况是,男性艺术家把自己看作是女性勾引者的受害人。

5. 那些搬动人头、切断男人脖子的女人,几乎是必然会被神话化为妩媚迷人的女人的。她们几乎不可能单纯凭借身体的力量制服一个男人,但她们可以用自己的美貌解除男人的武装。

6. 斩首常常被看作是一种色情行为。

砍头与性。

7. 断头台让时间停止在最重要的舞台上,停止在热心观众的面前;它制造出了一幅“终极肖像”,用真正的肌肉组织和表皮做成,解除了艺术诠释的约束。

8. 手起刀落便砍掉一个活人的头的困难超乎想象,即使这个人被五花大绑并蒙上了眼睛,这还没有考虑到一群闹哄哄的观众乱扔东西、嘲笑辱骂所造成的注意力分散。

砍头是个艺术,也是技术。

9. 恐怖主义的要害是制造恐惧、引发混乱——但是,除非有媒体支持这种行为,并尽可能让更多的人看到,否则这种情况就不会发生。”

其实恐怖本身也是这样,恐怖的心理需要传播,需要人被人记住,需要进入人的意识,因此任何以恐怖或恐惧来震摄人的统治者,也都会毫不犹豫地把恐怖传播到治下的每一个角落,从古代的屠城到现代的极权监控,概莫能外。

10. 一位海军陆战队员评论道:“我希望我是在跟德国人战斗。他们是人,像我们一样……但日本鬼子就像动物一样……他们习惯于丛林,仿佛他们就是在那里长大的,像有些野兽一样,在他们死去之前你绝对见不到他们。”

看来美国人还是不太能理解他们之外的文化。德国再凶狠,也是文化类似,他们能理解;日本人生存环境如何残酷,人性的标准自然低。

11. 尤金·斯莱奇写道:“激烈的生存之争……侵蚀了文明的虚假外表,让我们全都变成了野蛮人。”

12. 要想得到一杆枪,唯一的方式就是卖掉一颗人头,于是,“人头换枪”的生意在南美得以确立。

15. 欧洲人想要舒阿尔人的干缩人头,舒阿尔人想要欧洲人的刀枪大炮。

人头贸易,这也算是殖民者少为人知的黑历史了。

16. 你可能一直在试图保住你的头(译者注:keep one's head,意思是保持冷静)或者不丢掉你的头(译者注:lose one's head,意思是失去理智);在口头上咬掉某人的头(译者注:bite someone's head off,意思是严厉斥责某人)或者在身体上敲掉他们的脑袋(译者注:knock someone's block off,意思是痛打某人);笑掉了脑袋(译者注:laugh one's head off,意思是狂笑)或者继续把脑袋安上(译者注:keep one's head screwed on,意思是保持头脑清醒);悬赏要别人的头或者为了别人而把自己的头放在砧板上(译者注:put one's head on the block,意思是遭受指责,联系起来看,这句话的意思就是代人受过);想让某人的头放在盘子里,要不然就只能眼睁睁地看着由于你犯下的错误而人头滚滚。

关于砍头的一些俚语,从侧面可以看出这一恐怖行为有多么融入人的潜意识,尽管没人特别去反思。
