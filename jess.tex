\subsection{《饥饿的盛世》}

\subsubsection{感想:}
1.乾隆对发妻的深厚感情令人动容。一个男人心里最爱的,肯定是他的第一个女人。

2.作为一代雄主,乾隆的驭臣之术可谓高明。对三代老臣张廷玉,他虽然表面优抚,但在道理上不让分毫,总是自己有理,恩威并施,张廷玉生前郁郁而终,死后被褒奖,可谓活着就折腾人,死了再封号,这是统御下属的法宝。

2.不管怎么说,乾隆时代中国人口翻了一番,税负降低,乾隆本人也极度重视低层民众生活,兴修水利赈灾舍得花钱,这一点还是值得赞扬的。 

3.平准战争虽然雄才大略有长远眼光,但对准格尔部的种族灭绝政策让人不齿。满族人大概血管里还有嘉定三屠江南十日的野蛮血液。 

4.乾隆极重名声,但对大臣的名声却毫不在乎,同时大加羞辱。在这方面,明清两朝的皇帝如出一辙,不过清代更甚。明代的杖廷只是身体上的羞辱,但清代对士人名声的忌惮却达到了顶峰。乾隆每每下诏对大臣的心机进行揣测并加以羞辱,使读书人丧尽了尊严,同时培养出了大批的官场老油条。 

5.中国的国力在乾隆中期达到顶峰,但随后同样陷入马尔萨斯陷阱(何况即使在盛世的顶点,人均粮食占有量低于其他几个盛世)。但是,根据中国历来统治者的,乾隆已经做到了一个专制者的最好,手段也做到了最高明。近代世界的通行规则,乾隆皇帝不可能知道也不可能遵守。中国的近代化,还是在列强的坚船利炮下被迫进行的,而这个“三千年未有之大变局”,中国走得非常艰难,历经三世(清末、民国、共和国)才算基本完成,而对现代化的深层次消化和吸收,恐怕还有很长的路要走。 

6.这是乾隆皇帝的一本相对详细的传记,记述了其生平和简要的社会现状,尤其从他的经历性格出发,追迹其生平事迹,介绍了盛世的由来和走向衰败的原因,剖析了盛世来临的必然性和衰败的必然性,对了解乾隆时代的情况不可或缺。

\subsubsection{一些标注:}
关于羞辱士大夫:
\begin{quotation}
被扒掉裤子当众打屁股,对英国绅士来讲,是无法想象也无法容忍的耻辱。
\end{quotation}
看过李银河《虐恋亚文化》,表示呵呵。

财产权与个人的自由有着直接的关系。财产权不是一种物的关系,而是一种道德的关系,一种与因果关系相联系的涉及预期的稳定性的社会关系。没有它们,人们在社会生活中的预期是不可能的。

乾隆减免的农业税占乾隆朝财政总收入的7.57%。

乾隆一朝所减免的农业税总数为2.0275亿两白银,是中国历朝之冠。

“乾隆十三、十四年间,为高宗生平的第一变,由寅畏小心,一切务从宽大而一变为生杀予夺,逞情而为。”(高阳《清朝的皇帝》)

从来知臣者莫如君,知子者莫如父,则知妻者莫如夫。