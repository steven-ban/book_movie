\subsection{《人类简史》}

\subsubsection{有趣}

这是一本很有阅读趣味的书,特别是农业革命之前的人类史部分,因此我在三天之内读了一半。

阅读趣味这个东西,其实对于阅读而言是很重要的。生活本身就是乏味的单调的,而读书是人寻找心中桃花源的方式,对我这种不喜欢旅游的人是诗和远方,是逃避生活嘈杂和单调的方式。如果读书本来就很没有意思,不能在阅读期间给人以正向的快乐激励,那么阅读将变成一种负担。

很多书行文晦涩,甚至题材上就很呆板无趣,那么读起来就很吃力。比如《两宋文化史》,这是我2014年初在Kindle上买的第一本正版书,买的时候才九毛九,几乎是白捡的,但这本书几乎没有鲜活的史事,只有史料档案的堆叠,章节按照两宋时文化的方方面面来编排,只能当字典用,阅读时很得不到乐趣。因此三年多来,我只读了不到1/3,2015年后几乎没有再打开过。唯一的用处,恐怕就是学了一些宋代的基本文化常识,但这种学习也是零碎的、抽象的,恐怕第二天就会忘记。

而像东野圭吾的小说就很有意思,叙事节奏好,格调高,翻译也流畅(日本文学的翻译似乎都比英文好),通常一本小说两三天就能读完。

社科类的小说,如历史类,恐怕就没有小说那么好读。但这本《人类简史》不同,虽然也没有的动人故事性情节,但胜在作者旁征博引,不泛泛而谈,更兼文笔流畅,知识点(特别是人类学方面)也足够新鲜(毕竟中国的通识教育里历史多而人类学少),因此很有让人一直读下去的欲望。

另外,四百页的书,想讲清“人类”的历史,并不是一件容易的事,要求作者能够放弃过小的细节和偶然,提取规律来形成历史的动力学。这方面,作者的功底恐怕称得上评价中的“青年才俊”的称号,将人类三次革命的缘由、过程和思考都深深融入文字里,有始有终,也能引人共鸣和思考。

\subsubsection{史观}
作者的史观是人本主义的,从他对殖民主义、资本主义的反思中能强烈看出来。我也看过一些纯“西方”学者的文章和著作(如《人类砍头史》《人类酷刑史》),但那些西方人,其实对几百年来的罪恶讳莫如深或顾左右而言他,看起来很虚伪。本书作者是以色列人,站在了西方视角之外,抨击起来和东方学者的立场很相似。反对西方中心主义喊了很多年,但你去看BBC的记录片或英美人的著作,发现这更多是一种政治正确而非发自内心的认同,西方人总会有意无意将自己几百年来形成的偏见加诸东方世界。但近几十年来东方政治力量崛起,西方从顶峰跌落甚至相对衰落下去,他们必须越来越多地听到东方人的声音,这些声音自然也包括来自近东的以色列。

\subsubsection{史实}
依照作者的说法,人类(智人)变成今天地球主宰的过程,有三个重要阶段:认知革命、农业革命和工业革命。后两者其实之前有耳闻,但认知革命所知尚少。我们反溯人类历史,往往看到一万年前农业革命时,就觉得那些的人类科技落后,便认为文明源于那时(固定村落和城市的形成)。但对于那之前的状态,便一直以为是茹毛饮血蒙昧落后。但随着考古、人类学、基因学研究的深入,其实那时的人类,与今天的人类的生理及心理已经没有区别了。大约七万年前,人类已经“进化”完成。区别于人类与其他动物、以及其他智人的,是人类有虚构故事的能力,从而突破了实际交流对社群数量的局限(仅靠直接的交流,社群数量不会超过150),从而形成了更集体化的社会,远远超过了其他物种和人种,直接步入“逆天改命”的新时代。

在《枪炮、病菌与钢铁》中,对人类步入农业社会也有论述,本书的观点基本和它相同。人类从捕猎时代进入农业时代,有赖于几种作物和家畜的驯化,这或许是一种幸运,我们觉得现在的作物很多,却无视这样一个现实:绝大多数生物,根本无法驯化。小麦、水稻、小米,以及狗、猪、牛、羊的驯化,可以使人类的食物可以大量积累,从而使一部分分脱离了直接的食物生产,使人类社群的数量又进一步增加,为大规模国家奠定的基础。但在本书中,作者指了, 这对人类整体而言虽然是一个好事,但对于个体而言,却是倒退。渔猎时代的人,营养均衡,身材高大,而农业时代的农民营养缺乏,农业耕种对骨骼系统是一种摧残,人均寿命有所降低。这或许是一个陷阱。