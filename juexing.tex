\subsection{《觉醒》}

作者:凯特·肖邦

\subsubsection{人物}

\begin{longtable}{p{0.2\textwidth}|p{0.5\textwidth}|p{0.25\textwidth}}
    \caption{《觉醒》人物表} \\
    人物 & 特点 & 事件 \\
    \hline
\endhead

\hline
\endfoot
    
莱昂斯-蓬特利尔先生 & 40多岁,戴着眼镜,中等个子,体态修长,稍微有点驼背,棕色的直发齐齐地梳向一边,胡子剃得短短的,干净整洁 & 经常出门,出门经常给妻子寄礼物 \\
勒布伦太太 & 精神,漂亮	& \\
罗伯特-勒布伦 & & 喜欢奉承阿蒂诺尔夫人,喜欢叫上阿蒂诺尔夫人去游泳\\
埃德娜-蓬特利尔太太 & 双眼灵活明亮,呈金棕色,长相端丽,并不是一个母性十足的人,28岁 & 夫妻关系压抑,不习惯感情外露,也不喜欢别人那样做,喜欢音乐,把音乐想象成画面 \\
玛格丽特 & 埃德娜的姐姐 & 也不喜欢感情外露,母亲早亡,挑起家庭重担 \\
珍妮特 & 埃德娜的妹妹 & 由于感情外露而与姐姐关系不好 \\
拉乌尔 & 蓬特利尔的大儿子,5岁	& \\
阿黛尔-阿蒂诺尔 & 十分美丽,很喜欢蓬特利尔太太,结婚7年,有3个孩子,已怀上第4个 & 十分依赖丈夫的照顾 \\
艾蒂安 & 蓬特里尔的小儿子,四岁	& \\
法瑞尔 & 罗伯特的弟弟 & \\
赖斯小姐 & 小提琴演奏家	 & \\
芒代勒医生 & 蓬特利尔的朋友,洞察人性 & \\
? & 埃德娜的父亲,上校 & 性格和埃德娜不和,但能较好共处 \\
阿尔塞·阿罗宾 & & \\
\hline
\end{longtable}	

\subsubsection{事件}
\begin{itemize*}
    \item 在吉尔曼岛,埃德娜学会了游泳,这让她惊奇万分。她遇见了罗伯特,喜欢上他的陪伴。
    \item 罗伯特去了墨西哥,埃德娜很伤心
    \item 蓬特利尔家回到新奥尔良的家里,相思和长久以来的个性压抑让她发狂
    \item 罗伯特给赖斯小姐写信,信里只说埃德娜的事情。埃德娜看到了信。
    \item 蓬特利尔出了远门,埃德娜很开心,她去赌马,遇见阿罗宾,阿罗宾喜欢她,她放纵了一下但马上后悔。埃德娜内心的兽性被激发,两人逐渐亲密。
    \item 埃德娜准备搬出大房子。罗伯特继续写信给赖斯小姐,倾诉对埃德娜的爱情。埃德娜看了信。
    \item 埃德娜开始和阿罗宾约会。
    \item 罗伯特回来了,两人互诉衷肠,罗伯特离去。
\end{itemize*}

\subsubsection{书评}

埃德娜有一个爱她的丈夫(虽然可能不是那么理解她,也不算精神伴侣,但对她也算体贴和温柔),有了孩子,是一个典型的资产阶级妇女。她和丈夫之间,已经没有了激情,甚至没有感动,更多是一种婚姻下的责任和安于现状的心态。同那个时代大多数(几乎可以说是全部)的已婚女性一样,她没有工作,只是在家里带孩子,照看着家里。大多数女性,对这种生活应该是向往的,即使不向往,也会安于现状。

但凯特-肖邦写出了这种女性的情感。在千篇一律和枯燥的生活里,埃德娜渴望着激情,渴望着改变。小时候的她就是压制着自己的情感,不会轻易表达情感。她没有那么爱照料孩子,经常任孩子生病而不知道;她常和朋友一起出去,学会了游泳,学会的一瞬间让好惊奇万分。另外,她的生活里还有一个乐于对女性献殷勤的罗伯特,他无微不至的关怀触动了她的内心,她的生活慢慢地开始改变。

她开始在被占据的生活里寻找自我。爱情(严格的说是婚外情)是一个契机,点燃了她内心的欲望,使她开始关注自己的内心。她还继续在绘画的道路上前进,甚至想过离婚去巴黎学习画画。考虑到那个时代,女性的工作很少,她这样已经是很勇敢的举动了。她在这种矛盾中重新发现了自己,她不会再让其他人来定义自己,她首先是她自己,其次才是其他角色。

罗伯特在互诉衷肠后离开了她,她和阿罗宾的感情似乎也无疾而终。婚外情并非这部小说的主题,甚至对婚姻的反思也不是。凯特·肖邦要表达的,是女性在社会角色里发现自己本身的故事,是从第二性到第一性超越的故事。

从伦理道德的角度上讲,埃德娜的婚内出轨(主要是精神出轨)是错误的,是对丈夫的背叛。凯特·肖邦对埃德娜是爱护的,没有真的让她做出出格的事情。

这篇小说的情节简单,人物更少,相比于《咎》显得更平直。但肖邦惯常的心理描写在这篇小说里大放异彩,更为协调。单一主人公及单线叙事避开了肖邦不擅长的讲故事技巧,也避开了她很差劲的多人物情节处理。只有一个主人公,使得作者把更多笔墨放在她心理的微妙变化上,使得故事更为可信,也更为精巧。小说整体比《咎》好太多,人物内心逻辑也更加一致。由于采用单主角单线叙事,集中在埃德娜的心里感触和转变,避开了作者贫乏的写作技巧缺陷,小说也显得更丰满和立体,没有太多多余的话。

评分:4/5。