\subsection{《长恨歌》}

\subsubsection{王安忆}

王安忆的文字在这篇小说里琐碎絮叨,写一个上海女人如何从少女时代到死亡的生活细节和心态演变,“说不尽的愁肠百转”。王安忆笔下有烟火气,是过日子把所有的激情和棱角都磨平再泡软的心态。

第一部里李主任程先生的情节已经写得很好,第二部里邬桥里轻轻一转,到了严家师母和毛毛娘舅,又是打牌又是算命,又是吃鸭又是比美,恍若隔世的感觉,翻出新情节,赞叹作者的笔力绵长沉稳。然而,王琦瑶自从死了金主,便如同死了自己,再也提不起生活的乐趣,也无法再爱上哪个男人。她的后半生(其实是从18岁以后),便如同寡妇一般,失去了所有的鲜美颜色和激情,只剩下过去的回忆和狼藉的内心。一个女人的一生,基本被年轻的小三生活给毁掉了(警醒啊女同志们!)。

这篇小说写女人,在我看来比绝大多数男作家要好得多。刘慈欣阿西莫夫这样的就不说了,他们写女人和写男人完全没有分别,性别在他们笔下根本无从展现;即使是金庸这样的专业作家,笔下的女人虽说“千种风情”,但给人感觉还是如出一端,都是作家自己的意淫和捏抟;曹雪芹写女人当然基本上已经是最好的文笔了,但还是有男作家的影子。只有女性作家自己体会自己观察,才能把女人写好。在这一点上,最起码,王安忆做到了。

王琦瑶和蒋丽莉、吴佩珍间的友情或“闺蜜”情,在我这种直男癌或外人看来似乎十分亲密并且掏心掏肺,但在作者笔下却是一种暗暗战争暗暗较劲甚至“阳奉阴违”的过程。女人心海底针,三个女人一台戏,不是没有道理的。女人心里的那种小九九,让男人去写完全不行,非得王安忆这种女作用才能写出来。其实这种东西,虽然称不上深刻高尚,但写下来仍旧很耐看。特别是王安忆擅长写人物的感觉,擅长用一些简短的比喻来修饰人心的描画,这种琐碎的不厌其烦的描写往往也比较耐看。

女人柔情,女人琐碎,女人“没见识”,女人爱时尚爱衣服,女人“小心眼”。闺蜜之间是这样的:
\begin{quotation}
她们的做伴,其实是寂寞加寂寞,无奈加无奈,彼此谁也帮不上谁的忙,因此,倒也抽去了功利心,变得很纯粹了。
\end{quotation}
她们见多了繁华,因此要“过好日子”,苦日子她们应该是受不来的,她们见过爱情,也见过婚姻,但似乎都没有那些流动的繁华和万人瞩目更让人倾心,
\begin{quotation}
两人又默默地走了一段,王琦瑶缓缓地劝慰说:其实再怎么样,也还是结发夫妻最恩深义长。严家师母笑了,点着头道:是啊,有恩有义是不错,可你知道恩和义是什么吗?恩和义就是受苦受罪,情和爱才是快活,恩和义是共患难的,情和爱是同享福的,你说你要哪样?
\end{quotation}
女人和女人混在一起,彼此相距很近,她们的内心却未必近,
\begin{quotation}
好在女人和女人是不怕种下芥蒂的,女人间的友谊其实是用芥蒂结成的,越是有芥蒂,友情越是深。她们两人有时是不欢而散,可下一日又聚在了一处,比上一日更知心。
\end{quotation}
王琦瑶和严家师母间“比美”的心态,很值得玩味,很细腻,男性作家写不出:
\begin{quotation}
自从烫了头发,王琦瑶又有了些做人的兴趣了,从箱底翻出旧日的好衣服,稍做修改便是新。她也开始化妆,修眉毛的钳子、眉笔、粉扑都还在,一件件找出来摆开。她在镜子前流连的时间多了些,镜子里的人是老朋友,也是新认识,能与她说话的。严家师母看见她的变化,暗中加了把劲追赶。王琦瑶显见得比她懂打扮,也是仗着年轻有自信,样样方面都是往里收,留有余地,不像严家师母是向外扩张,非做到十二分不可。一个是含而不露,一个是虚张声势;一个是从容不迫,一个是剑拔弩张。严家师母不使劲还好,越使劲越失分寸,总是过火。王琦瑶当然觉察出严家师母的用力,更上了几分心。像她这样的聪敏,不上心就是合适,再要上心便是格外好了,由不得严家师母不服气。有几次,她甚至是忍了泪的,回到家中无由地向娘姨发脾气,还把新做的头梳乱,自己报复自己的。但脾气发过了,还是重整旗鼓,再与王琦瑶较量。这几日,严家师母到王琦瑶家,不是为别的,专是挑战而来的。她越这样,王琦瑶越不让她,每天都给她个出奇制胜,并且轻而易举,不留痕迹。严家师母话里面就有几分酸意了,说王琦瑶真是可惜了,这般的浓妆淡抹也相宜却无人赏识。王琦瑶知道她是发急,嘴里说的未必是心里想的,听了也当没听见,只是下一回再用些心,更上一层楼,叫她望尘莫及。这两个人勾心斗角的,其实不必硬往一起凑,不合则散罢了。可越是不合却越要聚,就像是把敌人当朋友,一天都不能不见。
\end{quotation}
相比之下,似乎男人之间的距离就大多了。所以,家里需要有一个女人,没有女人,似乎就不是过日子,
\begin{quotation}
家里有两个女人,再没个男人来解围,事情是真难办。倘要以为这个没有父亲的家庭会受到种种压力,那也大错特错了。
\end{quotation}

\subsubsection{上海}

王安忆写上海几十年的变迁,不仅从人物身上,从事件身上,也从潮流时尚(穿衣、跳舞、电影)上,从人情和生活琐碎上。上海的弄堂里有鸽子,它们眼看着上海几十年的恩怨,看王琦瑶沉沉浮浮。上海的弄堂里有爬山虎,有洗衣服和晒衣服的气味,这一点上王安忆把握得很细致(虽然未必称得上是精准,不是上海人我无法判断)。

上海在王安忆笔下,繁华里带着落寞,气派里带着小家子气,温柔里带着勾心斗角。上海人是势利的,是小市民的(所谓“小家碧玉”无非就是小市民);上海的繁华是权力和财富带来的,是外人(外国人和外地有钱或有权的人)带来的,因此上海人是倚势的。他们称呼人是“瘪三”,称呼人是“乡下人”,似乎自己高人一等。王安忆一针见血(却温柔)地说:
\begin{quotation}
这城市里的真心,却惟有到流言里去找的。\emph{无论这城市的外表有多华美,心却是一颗粗鄙的心},那心是寄在流言里的,流言是寄在上海的弄堂里的。这东方巴黎遍布远东的神奇传说,剥开壳看,其实就是流言的心子。
\end{quotation}
上海是如何变成上海的呢?
\begin{quotation}
由于人口繁多,变化也繁多,这城市一百年里积累的隐私比其他地方一千年的还多。
\end{quotation}
与其说是繁华,不如说是浮华,这种繁华只是场面,见惯了风月场,见多了金钱,但生活还是生活,这种“见识”离生活有点远,于是人就不“踏实”了,就眼看着远方却不看脚下了,
\begin{quotation}
街上走的人,都是戴了假面具的人,开露天派对的人,笑是应酬的笑,言语是应酬的言语,连俗套都称不上,是俗套外面的壳子。
\end{quotation}
上海的女人美丽,有见地(看看王琦瑶、张永红、康明逊等人对生活精细和穿衣时尚的重视和追求有多强烈!),但这种见地仍然只是小家子气的,是要通过和人比较的,不是自我的信心,而是来自他人的羡慕的眼光,
\begin{quotation}
上海的小姐们就是与众不同,她们和她们的父兄一样,渴望出人头地,有着名利心,而且行动积极,不是光说不做的。她们甚至还更勇敢,更坚忍,不怕失败和打击。上海这城市的繁华起码有一半是靠了她们的名利心,倘若没有这名利心,这城市有一半以上的店铺是要倒闭的。上海的繁华其实是女性风采的,风里传来的是女用的香水味,橱窗里的陈列,女装比男装多。那法国梧桐的树影是女性化的,院子里夹竹桃丁香花,也是女性的象征。梅雨季节潮黏的风,是女人在撒小性子,即唧哝哝的沪语,也是专供女人说体己话的。这城市本身就像是个大女人似的,羽衣霓裳,天空撒金撒银,五彩云是飞上天的女人的衣袂。 
淮海路的女孩还是有些野心的,她们目睹这城市的最豪华,却身居中流人家,自然是有些不服,无疑要做争取的。
\end{quotation}
上海并非孤悬海外。上海人也是南方人,特别是苏州人(上海与苏州的关系,有待好好探求),
\begin{quotation}
苏州是上海的回忆,上海要就是不忆,一忆就忆到苏州。上海人要是梦回,就是回苏州。甜糯的苏州话,是给上海诉说爱的,连恨都能说成爱,点石成金似的。上海的园子,是从苏州搬过来的,藏一点闲情逸致。苏州是上海的旧情难忘。
\end{quotation}
上海的小市民们,似乎对政治不上心,这可能是远离政治中心的缘故,也可能是北京离他们太远的缘故,也可能北京带来的政治文化和手段(共产党是泥腿子进城,要打倒官僚资本主义和小资产阶级,不就是把旧上海的人和生活连根拔起嘛?)和他们的生活太遥远的缘故,
\begin{quotation}
(王琦瑶和程先生)和所有的上海市民一样,共产党在他们眼中,是有着高不可攀的印象。像他们这样亲受历史转变的人,不免会有前朝遗民的心情,自认是落后时代的人。他们又都是生活在社会的心子里的人,埋头于各自的柴米生计,对自己都谈不上什么看法,何况是对国家,对政权。也难怪他们眼界小,这城市像一架大机器,按机械的原理结构运转,只在它的细部,是有血有肉的质地,抓住它们人才有依傍,不致陷入抽象的虚空。所以,上海的市民,都是把人生往小处做的。对于政治,都是边缘人。你再对他们说,共产党是人民的政府,他们也还是敬而远之,是自卑自谦,也是有些妄自尊大,觉得他们才是城市的真正主人。
\end{quotation}
这一点和北京人几乎完全相反。北京人见多了权力,见多了兴亡,对这种天子威严很有爱好,以此作为骄傲。

看看程先生,再看看李主任,他们对待女人的做法完全不同,得到的结果也完全不同。李主任有权力,经历的女人多,知道女人的胃口喜好,所以游刃有余如烹小鲜。这是实力,没这实力就表现不出这种自信,只能强撑场面。

\subsubsection{爱情}

对待女人,不应该周旋,应该强势侵入。周旋是女人的做派,女人生活在女人世界,唯一缺乏的就不是周旋。程先生不停周旋,只会让蒋丽莉和王琦瑶都不满意。女人需要的侵入,如同男根的侵入一样,是饱满的自信的,是雄伟的,是斩金截鉄的。女人只需要说 yes 或 no,不需要费尽心机去选择去做决定(美其名曰“你要对我负责”)。这样的侵入不是贫穷和委顿能带来的,只能从充沛的体力、雄厚的财富和志得意满的权力得来。

王琦瑶对程先生不上心,对李主任可谓情深义尽。从个人的角度上看,可以会觉得程先生是一个“好男人”,而李主任不过是个有钱有权有能力的大叔。但王琦瑶似乎缺乏一种父爱和稳固的安全感,对李主任的这种角色深深迷醉不能自拔,她的小聪明和“小”深情都献给了这位“宋思明”。

读到康逊明那段还好,再往下就是三观齐毁道德败坏自作自受了。自己的孩子却给别人戴绿帽,自己不负责任,婆婆妈妈不当机立断,害人害己,真想来一句活该。不过,王琦瑶和康逊明挑明感情那段写得真好。

\subsubsection{王琦瑶}

不喜欢这种女人:殊无才华,身无长物,追逐时尚潮流,依靠男人,却又目下无尘,过不得日子。

不过首先来说,王琦瑶是个美女,
\begin{quotation}
王琦瑶的美不是那种文艺性的美,她的美是有些家常的,是在客堂间里供自己人欣赏的,是过日子的情调。她不是兴风作浪的美,是拘泥不开的美。她的美里缺少点诗意,却是忠诚老实的。她的美不是戏剧性的,而是生活化,是走在马路上有人注目,照相馆橱窗里的美。从开麦拉里看起来,便过于平淡了。
\end{quotation}
出道前的王琦瑶有着少女的羞涩与拘谨,甚至是呆板,不灵动,但选美让她潜在的美都绽放开来,
\begin{quotation}
一笑,表情舒展了,脂粉的颜色里有了活气,便生动起来。再看那镜子里的美人,也不那么生分和隔膜了。
\end{quotation}
其实这种美,是深得男人喜爱的。男人自然各种口味都有,也常常换口味,但对这种美女,大概绝大部分男人的绝大部分时间里,是不会拒绝的。这或许能解释,为什么王琦瑶这个多年在男人心中长盛不衰的原因,为什么她身边总不缺少爱慕者的原因。

王琦瑶的一生都被突如其来的恩宠荣耀给毁了,被当小三的轰轰烈烈给毁了,被顾影自怜和自作自受给毁了,被绿茶婊的内心和眼高手低给毁了。年轻之后的种种悲惨遭遇都和当时的选择不无关系。其实她的生活,本来可以平衡如湖水,但非要渴望大风大浪,因为虚荣心,拿自己的美貌玩了一次野心,
\begin{quotation}
(为了选美)王琦瑶其实是真正地起了奢望。她的心本来是高的,只是受了现实的限制,她不得不时时泼自己的冷水。她知道这世界上的东西真是太多了,越想要越不得,不如握牢自己手中的那一点,有一点是一点,说不定反会有意外的获得,所以是越不想越能得。
\end{quotation}
大体上讲,王琦瑶是小家碧玉,不是大家闺秀,有小家碧玉的野心,有小家碧玉的精明和择木而栖的圆滑,见过大阵仗,有上流社会的经验却不能时时过上上流生活,有知识文化,其实本质上也算独立,但就是会自视太高给自己找罪受。

看看王琦瑶被李主任包养前的这段心理,好好的干吗干这行?可是,在她们眼里,女人只有依附权力,依附财富,依附男人,才能光鲜亮丽,才能让她们内心有希望:
\begin{quotation}
望了一溜烟而去的汽车,王琦瑶是有点怅惘的。李主任说来就来,说去就去,来去都不由己,只由他的。明知这样,还要去期待什么,且又是没有信心的期待,彻底的被动。以后的几天里,李主任都没有消息,此人就像没有过似的,可那枚嵌宝石戒指却是千真万确,天天在手上的。王琦瑶不是想他,他也不是由人想的,王琦瑶却是被他攫住了,他说怎么怎么,他说不怎么就不怎么。这些日子里,王琦瑶成天地不出门,程先生也拒绝见的。倒不是有心回避,只是想一个人清净。清净的时候,是有李主任的面影浮起,是模糊的面影,低着头用眼里的余光看过去的。王琦瑶也不是爱他,李主任本不是接受人的爱,他接受人的命运。他将人的命运拿过去,一给予不同的负责。王琦瑶要的就是这个负责。这几日,家里人待王琦瑶都是有几分小心的,想问又不好问。李主任的汽车牌号在上海滩都是有名的,几次进出弄堂,早已引起议论纷纷。王琦瑶的闭门不出也是为了这个。
\end{quotation}

\subsubsection{程先生}

婊子自有备胎磨。程先生大体算是文艺青年,偏偏看上小家碧玉的王琦瑶。推想一下,程先生条件不差,跟王琦瑶算是一个档次的,郎才女貌,品性眼光都很配。程先生当然看不上蒋丽莉那样莽撞的没有涵养的女生,但真正的豪门千金大家闺秀他又吃不起。王琦瑶是程先生层次上最好的选择,爱上王琦瑶对他而言一点不意外。

程先生相对于毛毛娘舅,相对于萨沙,还是比较有担当的,但这种所谓担当无非是做一个接盘侠来暂时照料怀孕和生产中的女神,他并没有“非分之想”,没有勇气跨出这一步。这真是屌丝的悲哀!在屌丝眼里,女神圣洁不可侵犯,不把自己的占有欲望当回事,错失良机。

值得玩味的是,正如大多数屌丝一样,程先生付出,接近,自己被自己感动得稀里哗啦,
\begin{quotation}
他站在她的身后,嗫嚅了一会儿,说道:伯母,请你放心,我会对她照顾的。说完这话,他觉着自己也要流泪,赶紧拎起热水瓶回房间去了。
\end{quotation}
你说你有啥好哭的?直接推倒过日子不就得了?

程先生敏感、纤弱,不敢得到却常常怅然于失去(其实什么也没有失去),
\begin{quotation}
程先生的两次恋爱都是折磨人的,付出去的全是真心,真心和真心是有不同,有的是爱,有的是情义,可用心都是良苦,然而收回的是什么呢?因此,他开始从根本上怀疑有没有什么两情相悦。他想男女之情真是种瓜不得瓜,种豆不得豆。不得是磨人,得也是磨人。
\end{quotation}
最终程先生在文革开始里自杀了。程先生这种人,根本就缺乏生存的活力和欲望,不能在这种残酷的条件下生活。即使不是因为王琦瑶,他这种人也活不长久。

\subsubsection{康明逊(毛毛娘舅)}

此君生母是个小,把大母当亲妈,见过新生母亲无人可言的凄凉,就产生自怜和自暴自弃的情绪。另外,他也缺少担当(除了李主任,这部小说哪个男人有担当?这些男人啊,细腻倒是细腻,知心倒是知心,但就是缺少一股子男性气概),自己有了女儿却不敢认。王琦瑶也是“活该”碰到这种人。