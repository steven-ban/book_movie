\subsection{《红楼梦》}
这是第十来次看《红楼梦》了吧,之前没有完整地看过脂批,这个版本里脂批比较全,对各个版本的整理也算精当。脂砚斋对曹雪芹创造的“捧”占了脂批大部分,另外还有对于创作的一些看法和对过往生活的感叹,这对缺失后三四十回的《石头记》来说是不可多得的资料,红学家也大多在这个地方下功夫去考证佚失的底本。

1.王夫人这个人,极愚蠢极专制(愚蠢的人更容易专制,而且是愚蠢的专制,非开明的专制)。对待晴雯这种漂亮风流的女子极为严苛。我最讨厌这种人。

2.宝钗是儒家最欣赏的女子。我不是很喜欢儒家,但也不讨厌。儒家在社会管理上也有称道之处的,讲得清权利与义务,并用权责把人固定在一个固定的位置。宝钗学识极厚,会戏会画,以淑女自许,“珍重芳姿昼掩门”。但宝钗虽有学养但无灵气,风姿远不及黛玉,风流远不及湘云。同时宝钗有管理才能,察人心之细远过探春,阴柔手段远超凤姐。

3.黛玉之灵,甚得我心。《红楼梦》读得粗疏的,往往认为黛玉刁钻小性,这是受宝黛爱情里的儿女之态所误,试问哪个女孩在男友面前不是刁钻小性的呢?事实上在“金兰契互剖金兰语”前,因为有宝钗的存在,黛玉以为她心里藏奸,喜欢抓住金玉良缘这个问题来试探宝玉;在宝钗袒露心胸后,黛玉便视宝钗为亲姐姐,甚至薛姨妈后来住在潇湘馆照顾黛玉,形同母女,十分亲密,钗黛毫无嫌隙可言。另外,“识分定情语梨香院”后,宝玉为龄官所感,心中已有黛玉为唯一感情托付,两人不会为金玉良缘所惑,在此后的感情虽然不再浓墨重彩,但愈加深笃,如历生死。黛玉为人,大概娇柔有余而强力不足,但不至于小性。另外,下人虽更爱宝钗,但以林黛玉之聪慧,又加贾母深爱,也不至于有愤恨之意。紫鹃对黛玉之深情,比之亲姐妹更加亲密,试忙玉一节可证。这种深情,在宝钗和莺儿之间肯定看不到。黛玉之灵秀,乃是极深情极敏感,有诗人气质,作者曹雪芹行文之间,难掩偏爱之情。

4.晴为黛影,但这次看书,对晴雯实在爱不起来,她太直白太尖利,个人对袭人麝月的平稳温柔更加喜爱。

5.《红楼梦》是柔弱的,是追悔的,甚至是消极的。

