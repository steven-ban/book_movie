\subsection{《天朝的崩溃》}

作者:茅海建

这可能是我2017年以来,读到的最好的历史学专著。这是专著,是严肃的研究著作,而非泛泛而谈的随笔。这是一部断代史,研究的对象是几年之内的历史,而非横贯几千年的通史著作;因此,它包含着足够的细节,有着足够多的和精彩的历史瞬间,这是通史、甚至断代史所不具备的。我读史重视“历史的逻辑”,但历史只有包含足够多的细节,才能有符合逻辑的结论,泛泛的所谓逻辑只会落入机械论的不幸窠臼里。

鸦片战争是中国近代史的开端,这个开端带着屈辱和自卑,一再地被我们这个民族回忆和反省。然而,民众在反省时,缺少我上面提到的太多的细节,似乎只把它当成一个转折点,好像命运那么不公,残忍地把这个不幸加诸中华民族。但是民众不知道,这些所谓的转折并不是那么明显和干脆的:千万鸦片战争失败的原因在几十年、几百年前就已经种下了。从事后来看,鸦片战争中国必定失败。但是,仅仅有这个认识还不够;阅读历史的乐趣在于这里面足够多的细节,此时思路才可能更清晰,认识才可能更深刻。

作者的研究集中在了一个拥有绝对制海权的近代化国家与一个中古帝国间的对抗,这里面更多的是对后者运行方式的揭示。作者反复强调,清朝做出一系列举动,最关键的是体制问题——这是一个成熟又落后的体制如何应对从而从未应对过的近代战争的故事。这个体制里,既有高高在上、试图守成的皇帝,也有各怀鬼胎、老成圆滑的政客,每个在都在体制里扮演着自己的角色,按照自己长久以来的逻辑行事。分开来说:
\begin{itemize*}
	\item 道光皇帝资质平平,只想守成,行事上主要是因循守旧,不敢创新。对待下属,他是苛责有之又喜怒无常,少有恩典。他本人对西方和近代情况没有了解,也没有了解的欲望,一心只想在旧的“天朝”的进贡体系下行事,维持天朝的脸面。对待敌国,他被各级官员蒙蔽,自己也缺少整体谋划,先抚后剿而后又抚,态度一直在摇摆不定。他看不穿下级官员的欺骗行为,被老谋深算的各级官吏牵着鼻子走,无形中成为这场战争结束后签订不平等条约的“冤大头”。
	\item 各级官吏对敌国认识不清,不知道英国在哪儿,更不知道英国通商和战争的根本目的。如林则徐这样的人,了解得还算多一些,仍然错会了英国的意图,以为仅仅是想要恢复通商,而非实际上的打开中国大门。武将如颜伯焘、林英各有守御动作,迷信思想盛行,军事思想落后,事前志得意满十分轻敌,战时目睹英国海军和陆军的神威后几乎都是一触即溃,缺少顽抗的意志。而对于奕山、裕谦这样的钦差大臣,由于凡事都由道光皇帝做主,他们为保官运,没有把真实的情况报告给皇帝,只拣利于自己的话说,隐瞒不报甚至歪曲事实,既蒙蔽了上听,又损害了平民和国家的利益。
\end{itemize*}

清朝一方对英国的认识基本上就是无知。他们可能见识到了英国”船坚炮利“,但对此缺乏准确的认识,认为只靠自己的落伍的土炮和优势的地形,就能退敌于陆上。更为可笑的是,他们认为英国人不会下跪,所以膝盖有总是,因此运动不灵活,不善陆战,因此骄傲而无知地认为在陆上不是中国军队的对手。清朝的军事思想和战术思想,还停留在两百年前,而不知道历经多次近代化战争(如最近一次的拿破仑战争)洗礼后的英国,经历了工业革命和大航海的英国,和清朝根本不在一个次元上,中英战争更像是英国对中国实施的一次”降维打击“。

\subsubsection{主要人物}
\begin{longtable}{p{0.1\textwidth}|p{0.3\textwidth}|p{0.5\textwidth}}
\caption{《天朝的崩溃》主要历史人物}\\
\hline
人物 & 职责角色 & 事件 \\
\hline
\endhead

道光皇帝	& 中方最高权力者,最后决定者	 & 先力主剿,后力主抚,而后再力主剿,受下臣蒙蔽,一心保留大清国威 \\
林则徐	& 两广总督	& 销烟,误判英军意图,以为是为了求通商,不认为是侵略,文官,于军事不懂行 \\
琦善	& 钦差大臣	& 定海战役失败后,受道光皇帝委派,安抚英军,谈判,出价远低于义律要求 \\
林芳	& 琦善继任钦差 & 	虚报战功,但未受严惩 \\
义律	 &	英国谈判代表,了解中国,不按英政府的命令办事 \\
懿律	 & 第一任海军司令和英方全权代表	& \\
奕山	& 靖逆将军 & \\	
伯麦 & 	英国远征军第二任海军司令和全权代表 & \\
璞鼎查(Henry Pottinger)& 接替义律的英方全权代表	 & \\
巴加(William Parker)&	与璞鼎查一起的英方海军司令 & \\
颜伯焘 & 厦门之战中国守军 & \\	
裕谦	 & & \\
\hline
\end{longtable}

英军的进攻战术:从海上进攻时,以优势的海军兵力正面相抗,陆军从清军炮台两侧包抄合围。

\subsubsection{主要事件}
\begin{longtable}{p{0.1\textwidth}|p{0.2\textwidth}|p{0.25\textwidth}|p{0.45\textwidth}}
\caption{《天朝的崩溃》主要历史事件}\\
\hline
战争的阶段 & 时间 & 主要人物 & 主要经过 \\
\hline
\endhead

战前中国 &	1839年6月3日	& 林则徐	& 林则徐收缴行商鸦片并销毁;道光皇帝令林则徐杜绝鸦片贸易同时不引起边衅,而英国发动战争的主要原因就是维持鸦片贸易,因此这是一种悖论,林则徐没有意识到问题的核心是英国政府,而非行商 \\
战前英国	& 1839年10-11月 &	英国外相巴麦尊	 & 下院通过决议决定对华发动战争,林则徐以为开来的船是运送鸦片的,没有意识到战争的到来 \\
九龙之战 &	1839年9月4日	& 清军为大鹏营参将赖恩爵所率3艘战船,英方为小船	 & 英军开炮,清军还击,英军撤退 \\
穿鼻之战	& 1839年11月3日 & 	清军为虎门提督关天培,英军为两艘军舰	& 炮击,关天培英勇力战,清方撤退 \\
英军开到	& 1840年6月28日	& 英国远征军总司令兼全权代表懿律	& 英军全部集结完毕,有海军战舰16艘,武装轮船4艘,地面部队4000人,总兵力6000-7000人 \\
舟山之战	&	&	& 1840年6月30日,英军开到;1840年7月5日下午2点半,9分钟摧毁炮台,陆军侧翼占领东岳山并向县城开炮,守军溃逃,守兵张朝发落水后死亡,知县姚怀祥投水自尽;随后英军封锁沿海重要出海口 \\
厦门炮战	&	&	& 1840年7月3日英军炮艇逼近厦门岛,清军开炮;英军此举是为了找人谈判,但送信目标未达成 \\
虎门大战	&	&	& 沙角、大角之战,1841年1月7日,清军战死277人,英军仅受伤38人;横挡之战,1841年1月8日协调等待谈判,2月23日重新打响,2月26日下午5点结束 \\
广州内河战斗	&	& &	1841年2月27日乌涌之战;3月18日英军重新占领商馆 \\
广州之战		& &	& 1841年5月17日义律下令进攻广州,次日开始行动,5月23日抵达广州附近,5月24日下午2时攻城;5月27日从城北进攻广州时,停战;5月29-31日,英军受三元里民众抵抗,余保纯劝散民众 \\
厦门之战	&	& 英军璞鼎查,清军颜伯焘	& 颜伯焘在1841年4月底完成厦门石壁军事堡垒;8月26日下午1时45分,英军发起进攻,下午4点后南岸失守,8月27日逼至厦门城城外,1841年9月5日留下3艘军舰和550人驻守,其他撤离北上 \\
第二次定海之战		& & 清军裕谦,英军海军司令巴加,陆军司令郭富 &	裕谦设置土城;	英军于1841年9月25日侦察,9月26日英军开始北上,10月1日战斗,下午2时攻进南门,战斗结束 \\
镇海之战		& & &	1841年10月9日英军驶往镇海,下午2时战斗结束,裕谦欲自杀不成,逃离。 \\
\hline
\end{longtable}

\subsubsection{林则徐}
鸦片战争开始于中英贸易,开始于鸦片。鸦片的故事,以及林则徐禁烟的故事,大家都已经很熟悉了。林则徐本人,也一直有着“睁眼看世界第一人“的称号。面对从前没有出现过的西方式贸易、外交和政治事件,林则徐相比于同时代的人更开放,更开明,了解得更多,认识得更深。但是,林则徐在这个体制里,依然放不开手脚,依然有着自己的局限。作者认为,林则徐最大的错误,在于没有认识到英国派来的军舰不是商人,而是赤裸裸的侵略,以致于他在战争之初就错估了形势。但是,这怎么能怪林则徐呢?任何一个人,放在他的位置上,都不可能做得更好。殖民者的无耻,毕竟任何一个清朝的官员,都没有见识过。

\subsubsection{杨芳和奕山}
两人都是官场的老油条,一面欺骗道光皇帝,一面在战场上节节败退,丢尽了中国的人。

\subsubsection{历史事件与人物的道德评判}
长久以来,人们对鸦片战争的评价充满了道德上的褒贬,简单地把人物分为善的和恶的,能力强的和能力弱的,一心为国的和贪生怕死自私自利的。这种非黑即白的论断失之于简单粗暴,但正由于简单明确而更容易传播和形成思维定势。一旦去了解这段历史的真实经历和诸多细节,理解各种历史人物在那个关点处的选择,肯定会有不同的结论。这本书就提供了充足的细节和作为历史学专著的严谨思辨。

喜爱接受道德判断的另一个原因是人们思想里认为当时的中英军事差距没有那么大,合国之力可以取得胜利。这本书用充分的细节和史实告诉我们:以清朝当时的经济财政水平、动员水平、军事实力、军事战术等诸多因素来看,对英作战根本没有取胜的可能。即使如颜伯焘这种力战之人,也无法抵抗英军的猛烈攻击。毕竟,当时的中国所有人,都没有近代视野,对于这种降维打击,只有被动挨打或主动挨打的份,根本没有还手之力,更不可能有取胜的机会。

军事之外,是外交。英国人发动战争,是为了经济利益;英国人在谈判桌上的诸多条约,也是为了经济利益。贸易(特别是鸦片贸易)可以为英国赚取大量外汇,大量白银流入英国,可以用来在殖民地和本国发展工业,从而发家致富。久居上国的清朝上下诸人,生活在自给自足的小农经济几千年,不需要也不依赖贸易,不了解工业和科学。中国人在天朝体系,和英国那种贸易商业立国的方式完全不同,这时谁拳头硬,谁就有制定规则的权力。可惜,上帝这时站到了工业化一方,因此清朝这一课,迟早会上。外交规则,由战胜的英国来制订,中国人不懂,只能在不熟悉的规则下枉受制裁。因此,在任何道德上的标准来要求清朝的这些涉外官员不“卖国”都是无用的。

可是,中国传统的人物评判标准上,失败了就要由人来背锅,而这种评判往往带上道德的色彩,如同忠奸的分别、爱国与卖国的分别、力战与临阵胆怯的分别……为了补偿中国人的“历史自豪感”“民族自豪感”,由那些不光彩的“妥协”“投降”“撤退”“媾和”实施者混淆为卖国者或投降派是不同时期历史叙事下的惯用的便宜的伎俩。马东说过,“被误解是表达者的宿命”,而历史人物本身就是一种“表达”,这是作为历史人物必然的代价。而历史学的精彩之处,就是对这些细节进行辨析,还原出一个主观上可能合理的历史。在这一点上,本书做到了,也成功了。

\subsubsection{外交}
中英双方在这场战争里的外交或许比战争本身更发人沉思。英军在战前就确定了交战的目的,远征军司令具有明确的作战计划,虽然是客场作战,但仍然维持着近代化战争的水准。虽然英方换了几次全权谈判代表,但一切为了通商的目标从来没有变。反观清方,具有最高决策权的道光皇帝没有亲临一线,但凡事都必须经由他作主,南方的战事最新情报需要十天上下才能到他手上,再加上他的指示也需要十天到达前线,因此中间的时间差非常大。他受到前方谈判代表和将领的蒙蔽很深,根本得不到一手的真实情报,因此做到的决策都是严重滞后和偏离事实的。他派出的钦差大臣也因为官场的生态而不得不对他隐瞒一些实情,同时谈判时只接受了道光皇帝不可辱没天朝威严和尽量少开边衅的最高指示,在具体的事务上要么自己做主先斩后奏,要么是不顾国家利益(要么看不到认识不清,要么直接违背指示)而随意自我降价。很多国家权益都是这样丧失的,比如领事裁判权、片面最惠国待遇等。

被英国打怕了以后,又以相同的心态和法国、美国签订了类似的条约,这些条约使中国丧失了比中英条约更大的利益。从法美的角度上来看,这几乎是趁火打劫和捡漏,从中国的角度上来看,这只是对其他夷国一体而论和“平等对待”,是对他们同样的恩惠。这种完全不明白现代外交规则的行为是战争失败的被迫行为,也只能是一声叹息以对了。

\subsubsection{本书}
作者是历史专业学者,考证丰富,治学严谨,因此本书的学术性很高,值得好好阅读。当然,作为外行,对于这种内行的书籍,只有学习的份,在专业性上没有什么发言权。

作者在本书中对鸦片战争的分析,流露出他对清朝当时种种行为的无奈、嘲讽和痛恨。作者当然是爱国的,这种对本国历史中自己方面的基于事实的分析和反思才是真正的爱国主义和民族主义,这和前些年流行的“逆向民族主义”有着根本的不同。鸦片战争成为中国民族主义叙事中的重要一环,虽然在清朝之后的时期内反思不够,但民国以后对这场战争的反思不可谓不深刻,起到的激励不可谓不显著。这种激励虽然有偏差,但对于中国民族主义的形成有着重要的作用。它开启了中国悲情化近代历史的叙述基调,使得每一个接受义务教育的中国人都从反而认识到国家强盛的重要,有时这种激励真的比正向的“民族自豪感”来得猛烈和有效。

这一篇书评比较全面深刻,值得一读:\url{https://www.zhihu.com/question/21016607/answer/16903023?gw=1&utm_source=com.ideashower.readitlater.pro&utm_medium=social}

概言之,本书的一个缺点,是以今度古的成分。

评分:10/10。
