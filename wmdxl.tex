\subsection{《文明的溪流》}

作者:(英)H. G. 威尔斯

英文书名:A Short History of the World

一些摘抄:
\begin{itemize*}
    \item 世界上的一切拼音字母,都是由苏美尔楔形文字和埃及的象形文字的混合体转变而来的。
    \item 青铜,特别是炼铁术,大概都是游牧民族发现的。
    \item 最初的航海者一定是能抢劫时就抢劫,不得已时才经商的。
\end{itemize*}

这本书的内容十分老旧,已远远落后于时代,因此不建议在这个时代去读。本书从宇宙空间和起源一直讲到一战后的世界(大概成书于二战前),涉及各个主要文明的主要历史,对世界的认识带有那个时代的印记。本书对于每段历史都是一笔带过,特别是中国历史(我只读了这一部分,以此来鉴定本书对历史描摹的深度)仅简单地叙述了春秋、秦、唐等阶段,都是很简略的,也没有抓住中国历史和文化的特点。

评分:0/5。