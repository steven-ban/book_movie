\subsection{芳华}

又名《致我们已经逝去的青春》。

上个世纪七十年代,部队文工团,一群不到二十岁的文艺兵们排练红色歌舞和慰问演出,这里有出身高贵、盛气凌人的郝淑雯、陈丁丁,有热爱写作的萧穗子,有出身贫苦、喜爱助人为乐学做雷锋屡受表彰的刘锋,有同样出身高贵(父亲是昆明军区高层)但毫不显山露水的陈灿,还有母亲改嫁、亲生父亲劳改的何小萍。

这些出身迥异的人,如果放在二十一世纪的中国,可能根本走不到一起。但在那个封闭和集体主义的年代,他们因为没有高考和同一个文艺兵的身份,生活在了一起。但是,看似集体主义的外表下,出身的不同仍旧造就了三六九等,人与人之间并非兄弟姐妹情谊。高干子女永远可以获知一身信息,在即将变革的前夜随时抽身离去,而出身贫苦如刘锋的人,即使热心为他人并常常受表彰,但“学雷锋”的刻板印象让大家挥之不去,大家把他的付出当成了理所应当。还有何小萍,因为爱出汗,身上总有一种味道,受到所有女生的排挤。她第一天没有军装,偷了别人的军装去照相,完成生父的心愿,但被发现后受到责骂和孤立。

在这个封闭保守的时代里,爱情是无法言说的秘密。刘锋爱上了陈丁丁,忍着多年没有说,甚至主动放弃了进修的机会。当邓丽君的音乐吹来时,受到鼓舞的他向陈丁丁表白,陈丁丁或许是因为害怕,或许是因为与集体主义的合拍,检举了刘锋,于是他被开除,入了伍。而何小萍,因为受这个集体的排斥和刘锋的关心,也对文工团产生怨恨,不再跳舞,最后加入了战场的军医行列。越战爆发,刘峰作为副连长,在战场上右小臂以下死亡,何小萍见证了死亡与残疾,因为勇敢被这个集体“接纳”而成为英雄,但因为刺激太大而精神失常。一年后的慰问演出上,随着音乐,她的心灵打开,翩翩起舞。
之后的岁月在这个变革的年代里变得残忍:刘锋复员,妻子出轨,自己受尽地方联防队的欺负;何小萍终身不嫁;郝淑雯与陈灿结婚,做生意发家;陈丁丁出国,身材变胖;萧穗子成为知名记者。被联防队扣掉车子的人遇见了萧穗子和郝淑雯,两人的阶级差距对比鲜明,命运截然不同。萧穗子的旁白里一直念叨学雷锋的刘锋脱离了时代,殊不知命运并非因为他爱学雷锋,而是因为他们的出身本来就不同,注定会走向不同的结局。那个看似平等的年代,实质上早就给不同出身的人写下了不同的剧本,不论你是热心助人,还是盛气凌人,还是蝇营狗苟,在命运的大潮中都难以翻身。

多年以后,刘锋生命无法处理,何小萍照顾他的生活,电影在这种感叹时光流逝中结束。

影片的情节跨度很大,情节太多并太密,主线比较模糊。冯小刚和严歌岺试图在150分钟的故事里放进太多岁月的感慨,于是无论是无孔不入的旁白、应接不暇的一个又一个时代符号(毛泽东像、红色歌舞、邓丽君、越战……)、复杂却缺少主线的人物关系、冗长而被动的推动情节的台词、一次又一次的感慨……这些元素交织在一起,使得电影显得很满很啰嗦。甚至战争场面的残酷,也被导演不加修饰地剪进去。其实,这部电影野心太大,冯小刚和严歌岺对于过去那个集体主义和文工团的记忆太过偏爱,舍不得去掉任何一个元素。经历过那个时代的人,可能会有很多共鸣,每一帧都会让他们落泪(特别是音乐和随着音乐起舞的青春肉体),但对于没有经历过那个时代的年轻人,恐怕除了冗长的情节外再也感受不到故事线的推进。全片似乎只是营造了一种怀旧的气氛,像一个老人在絮絮叨叨年轻时的点点滴滴,没有一条清晰的故事线让观众期待。所以你在电影里会看的是:故事向前推进,似乎下一秒就是宏大和高潮,但音乐声响起,并没有期待的壮烈和升华,而是下一段故事。说句不好听的,就像是一个中年人做足了前戏,但迟迟不射,那种温吞和迟疑让人着急。

在具体的镜头运用上,影片的镜头离人脸太近,太飘太晃动,看着很吃力。

如果满分十分的话,我只愿意给六分,勉强及格。