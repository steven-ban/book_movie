\subsection{The Big Bang Theory}
2019年结束了时长八季九年的Game of Thrones,也结束了长达十二季的The Big Bang Theory。作为这两部剧的粉丝,对前者是五味杂陈,对后者则充满了感谢。

无论是哪个国家的剧集,基本都不会把收视目标放在nerd和死宅身上,也不会把这些人作为主角,然而TBBT做到了,而且如此成功。对于普通观众来说,他们了解到了这些人原因并不“奇怪”,而是有着自己的喜好和浪漫;对于nerd和死宅本身而言,他们会在这些人身上看到自己,会感到同样的感觉。

作为一部分情景喜剧,几位主角的性格不同,境遇不同,甚至还有作为反差的学渣的Penny。剧集的台词和编剧充满了物理学、生物学、美漫、科幻的梗,从专业度是来说是挺难的。科学的部分虽然无法深究(例如一个神经学家肯定是看不懂超对称性的论文的,别说是神经学家了,就算是研究方向不同的理论物理学家,相互之前也看不懂对方的理论),但很“像”,演员的演技也很到位。

十二年的长度下来,主创人员见好就收,适时结束了剧集,造就了这样一部分后来人肯定为“神作”的作品。感谢剧组!