\subsection{《寻乌调查》}

这是毛泽东于1930年左右在江西省寻乌县做的社会调查,全文七万余字,展示了寻乌社会生活的方方面面。内容包括寻乌的地理、经济、人口、阶级、思想文化等方方面面,内容既全且细,细致之处令人“发指”:本文竟然包括寻乌有几个杂货铺、店主何人、家庭状况、店里的营业额、主营产业等等。各个方面都很细密。拿到这个报告,可以说已经对这个县了如指掌。我怀疑,当时的县长和县里的居民,都未必比经过调查后的毛泽东更了解这个县。

当时的寻乌已经被苏维埃政权夺取,进行了土地改革,给农民分了田地。以此观之,中国共产党对社会了解之深入、之深刻、之全面,比那个不成器的国民党要强多了。共产党以空前的热情对社会进行改造,这种敢做敢为的风格不仅明显有别于早期的中共,也明显区别于当时本应该做这些事的执政党。因此,共产党如果不能夺取政权,简直是没有天理的。

至于毛泽东其人,从这篇调查报告里也能看出来他对社会的果敢、热情、以天下为己任的胸怀,不由让人生出敬佩之情。