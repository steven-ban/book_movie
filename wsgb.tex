\subsection{《无声告白》}

\subsubsection{一些标注}
在这个夏天剩下的日子里,以及以后的很多年,詹姆斯和玛丽琳说话时会选择真正能表达自己的意思的措辞,无论是对内斯,对汉娜,还是互相之间。他们需要说的太多太多。

对逝去的心爱之人的记忆,会自动变得平顺和简单,它会把各种复杂纠结的成分当成丑陋的鳞片一样甩掉。

\subsubsection{读后感想}
玛丽琳无疑是这本小说的主人公。她的母亲无时无刻不希望她能成为一个典型的家庭主妇,每天为丈夫及孩子们烹饪出可口的饭菜。但是,母亲的强烈愿望受到了玛丽琳同样强烈的反判。她想“与众不同”,摆脱家庭主妇的命运。于是她在女子大学里选修了理科科目,并发誓就读医学院,成为一名医生,彻底走出母亲的阴影。然而,天真的她在没有任何计划的情况下和华人后裔詹姆斯坠入了爱河并很快结婚,虽然母亲在婚礼当日反对他们的结合。母亲的理由很简单:你们不一样。母亲没有说出口的是,他们是如此不一样,以致于在以后的生活里需要互相迁就的地方太多。然而,年轻的玛丽琳当然看不到这些,热恋中的她只看到“独一无二”的詹姆斯,看到他与众不同的那些闪光点,而这一直是她热切想得到的。但是,由于抚养孩子的压力,以及更重要的,社会对女性的偏见,使她不得不中断学业,在家相夫教子,成为自己最不想成为的家庭主妇。果然,对母亲的反叛,最终绕了一个圈,却过起了母亲那样的生活。但是这样的生活她是忍受不了的,那个”成为医生“的愿望是如此强烈,驱使她在已经有了两个孩子的情况下逃避到远方的社区大学去进行未完成的学业。然而,家庭的负担、对儿女的思念还是压过了她的个人理想,她不得不彻底向现实屈服,回到家里。但是,此时的她再也不会去用心烹饪,同时她把自己全部的未实现的愿望都毫无保留地转移给很像自己的女儿莉迪亚,以爱的名义”强迫“她好好学习,将来成为一名医生。她把莉迪亚的课程排得满满的,直接向那个最后目标冲刺,不给女儿丝毫的空间和自我意志。同时,对于其他孩子,比如大儿子内斯,以及小女儿汉娜(她和詹姆斯更忽视汉娜),他们几乎没有给予任何主动的关心,即使有,也是伴随着莉迪亚的。玛丽琳的这种失败母亲的做法,为家里的悲剧埋下了巨大的隐患,然而,在”实现自己理想“和”督促莉迪亚实现自己的理想“的道路上,她根本没有任何反省,对于母亲的生活方法,完全是一种玩命逃脱的姿态。

作为一个华人后裔,詹姆斯的父母只不过是飘洋过海的底层工人而已,种族的鸿沟使他敏感、自卑,渴望融入美国的”主流社会“,渴望获得白人的承认,渴望与人的交流。玛丽琳作为一个标准的”美国人“和”白人“,使他误以为通过爱情和婚姻就能将鸿沟填满,于是在玛丽琳天真的初吻后,他立刻接纳了她。但是,作者虽然没有明说,婚后的家庭里,凡事都是玛丽琳说了算的,他作为一个父亲却唯玛丽琳马首是瞻,在家庭里没有占据主导地位(美国家庭或许不如中国那样”传统“,但像詹姆斯这样把家事完全托付给玛丽琳的行为还是太”前卫“了一点)。她对儿女唯一的的教导就是拿自身缺乏的那些东西进行说教,希望他们能够”多交朋友“”融入社会“。当家庭里的重心渐渐被玛丽琳占据,他被排斥在儿女考育之外,只能三点一线,专注工作。在高校里,他遇见同样是华人后裔的路易莎,由于具有相同的文化背景,属于相同的种族,尤其是都能体会到少数族裔被主流社会有意无意排斥,他们越走越近,不过并未发生关系。当莉迪亚溺亡,玛丽琳歇斯底里的态度令他彻底对家庭失望,他无奈投入路易莎的怀抱。整篇小说里,詹姆斯很重要,但并非缺点,他似乎只是玛丽琳得以发现自己的一面镜子。

受到母亲”故意走失“的巨大刺激,莉迪亚对母爱的渴求非常强烈,她担心母亲再度”走失“离开自己,于是以答应母亲的所有要求来与母亲”交换“。在这种可悲的交换中,玛丽琳毫无所知,所有的伤害都被几个孩子默默承受。整个家庭由于母亲的偏执而无法进行坦成布公的交流,各个人的情感处于严重的孤立和阻塞状态。但是,孩子会长大,孩子的承受能力也有个限度。长期的为了单一目标的付出,以换取与同龄人的交流为代价。莉迪亚在学校里没有朋友,被人忽视,吃饭时只能找哥哥,上学放学也和哥哥一起做校车,节假日让母亲陪着去参观图书馆禁物馆,或者去勉强学习高年级的课程。在孤独的学习中,母亲竟然偏执到只让莉迪亚学习只与将来读医学院相关的课程,而在其他课程上莉迪亚越来越差。同样,莉迪亚又要讨好希望自己多交往的父亲,于是她学会假装和同学朋友打电话而故意让父亲听见,同时假装自己成绩很好。母亲严密的关注使她窒息,在孤独的境况下只有找到同样孤独的邻家男孩、"坏学生”杰克。与杰克的交流使她暂时逃脱了家庭对自己的窒息,她一步步接近杰克,甚至想和杰克发生关系。然而,在最后关头,杰克告诉她自己的秘密,使她与杰克进一步的交流不再可能实施,于是她只能重新逃离回家庭。内斯作为她的哥哥,很早就和她建立起一种默契,母亲的关注和爱给她的多,给他的少,前者却被爱压得透不过气,后者却时时想获取这种关注和爱。只有有了内斯,才能分担起父母的这些关注,使她稍微有一些自我空间。然而,内斯即使步入哈佛大学所示自己的宇航员梦想,他一旦离开,自己就要独立承受父母的“关心”,无路可逃的莉迪亚选择了偷偷藏起内斯的录取邮件。但内斯终于步入大学,受伤的莉迪亚给他打电话,醉酒的他接了。对于内斯而言,这个家庭是最终要逃离的,他渴望这一天,甚至不惜与莉迪亚分开(他似乎不那么依赖莉迪亚的存在)。内斯潦草的回答成了压死莉迪亚的最后一根稻草,她独自进入湖面,试图撇开内斯的帮助,通过自己的力量游到岸边。然而,不会游泳的她,如同不会独自承担父母“关爱”的她一样,一个人根本不行。试图游过去却最终溺入湖底身亡。

莉迪亚作为两代人的核心,她的死亡宣告了这个家庭的分崩离析。玛丽琳失去了满怀希望的女儿,也失去了自己十几年的梦想;詹姆斯将自己交给非理性的出轨,在路易莎的怀里寻求慰藉;内斯失去了妹妹,将仇恨转移到杰克身上;汉娜一直被父母忽视,在这个时候依然不被人想起。事情的转机来自于玛丽琳和詹姆斯在调查女儿死亡过程里对过去的回溯,他们发现女儿竟然没有一个女朋友,竟然成绩这么差。玛丽琳发现了女儿藏起来的、她自己母亲留下来的、而玛丽琳自己拼命要逃离的烹饪书。在这本书里,莉迪亚发现了母亲,又通过这本书,玛丽琳发现了自己疯狂逼女儿实现自己理想的原因。一家人最终和解了,然而,我觉得,在和解之前,这个故事就已经结束了。和解不和解不重要,作者已经把几个人的遭遇剖析出来,这个故事的目的也就达到了。美满的结局,恐怕只是给读者一个温暖的交代而已。女儿的一些秘密,玛丽琳和詹姆斯可能永远也不知道,但他们已经拿到了如何正确对待自己、对待家人的钥匙,这就足够了。

不得不赞赏一下作者伍绮诗的叙事功底。这本是一个普通的故事,但作者将过去和现在平行穿插,把这个故事妇道来。叙述的张力在于,读者通过阅读而获知事件的真相,詹姆斯和玛丽琳通过回忆找寻过去。到故事的最后,读者对所有细节一览无余,而詹姆斯和玛丽琳还活在过多的未知里。故事的推进不紧不慢,情节扣人心弦(好老套),我一个下午和一个上午的时候一口气读完了。另外,作者对细节的把握也相当细致,人物的心理描写很到位很真实。

我们终其一生都活在别人的期待里,如何摆脱这种期待并找寻自己,这是这本小说的主旨。然而,通过这本小说,可以看到更多:五六十年代少数族裔融入美国主流社会的不易、美国传统家庭里女人的地位、女性权利的上升、家庭教育的漏洞、婚姻的保持。感触最深的,是玛丽琳对自我价值的偏执,这种偏执来自于母女关系中的羁绊、控制、逃避、反叛……失败的母亲造就失败的女儿,而这种失败不会因为女儿脱离家庭而消失,它会”遗传“给下一代。作者并没有为婚姻花费太多笔墨,但婚姻对双方性格和教养的依赖也是其能够成功的重要因素。成年人可以通过婚姻而脱离自己原来的家庭,但青少年不会,他们逃无可逃,便会被家庭的弊病所压垮。成人可以逃离,但青少年逃无可逃。

这并非一本完美的小说,但我给她打五分。