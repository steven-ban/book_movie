\subsection{《小夜曲》}

作者:石黑一雄

这是诺贝尔文学奖获得者石黑一雄的五个短篇,偏早期的创作,文笔比较稚嫩青涩,抒情显得刻意。“很难知道哪里可以安身,何以安身。”五个短篇,反映的是作者对现代生活的反思——荒谬、挫败、失意,作者本身是抽离于故事的,虽然有“我”在故意中观察记录,但并没有参与太多事件的推动。

老实说,我不喜欢石黑一雄。他的这种观察,在我看来是没有意义的,显得无病呻吟。虽然它们确实是现实的一部分,是很多现代人的真实心理体验,但并非十分重要。看来,诺贝尔文学奖的名头虽大,但水平是存疑的。