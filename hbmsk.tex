\subsection{《华北民食考》}

\subsubsection{评论}
1.华北穷人的选择好少,没钱买花样繁多的食材,没工夫做复杂的烹饪,只能整天守着不多的几种原材料:面,小米,玉米,红薯……然而,在仅有的集中食材中,却又开发出种类多的惊人的吃法,蒸炸煮烙煎烤好吃的东西却又很多。

2.我小的时候,农村商品经济还不发达,很多东西没得买,在吃上十分匮乏,曾经买排骨熬完菜啃一次后又在下一顿下锅煮,当时却觉得很香,现在还想着那味道是幸福的。

3.我妈做的擀面条又软又香,但有十年没做了。

4.我爸做的烩饼和炒面茄都很好吃,可惜再也没有机会吃到了。他熬的菜炖的排骨炖的鸡也很好吃。

\subsubsection{标注}
山东发面所制的食品最多,河北省吃馅子的种类最多,山西省吃面条的种类最多

千余年来,北方的民食,可以说是没什么大变动,总是在杂粮米面中想法子,菜蔬次之,肉类则极少见。