\subsection{《李鸿章传》}

1.梁启超对李鸿章的态度:敬李之才,异李之识,而悲李之遇也。

2.“曾国藩虽治兵十余年,然所任者仅上游之事,固由国藩深算慎重,不求急效,取踏实地步节节进取之策;亦由朝廷委任不专,事权不一,未能尽行其志也。故以客军转战两湖、江皖等省,其间为地方大吏掣肘失机者,不一而足,是以功久无成。”清廷向来不信任汉人。

3.对于太平天国,“欧美识者,审其举动,乃知其所谓太平天国,所谓四海兄弟,所谓平和博爱,所谓平等自由,皆不过外面之假名,至其真相,实于中国古来历代之流寇,毫无所异。因确断其不可以定大业。”外国人看得清楚,当时的人也看得清楚,可惜我们的历史课本却把太平天国捧上了天。当然,还有一个崇拜洪秀全的,那是香山县的孙文博士。

4.“曾文正曰:凡军最忌暮气。”太平军乱了十四年,曾国藩李鸿章左宗棠这样的名臣剿了十四年,真的挺辛苦。

5.”本朝之绝而复续,盖英法人大有功焉。彼等之意,欲籍以永保东亚和平之局,而为商务之一乐园也。“欧洲人来中国就是为了做生意,没有近代以来国人认为的那样悲怆。
6.”至其所以失败之故,由于群议之掣肘者半,由于鸿章之自取者亦半;其自取也,由于用人失当者半,由于见识不明者亦半。彼其当大功既立,功名鼎盛之时,自视甚高,觉天下事易易耳。又其裨将故吏,昔共患难,今共功名,徇其私情,转相汲引,布满要津,委以重任,不暇问其才之可用与否,以故临机偾事,贻误大局,此其一因也。又惟知练兵,而不知有兵之本原,惟知筹晌,而不知有饷之本原,故支支节节,终无所成,此又其一因也。“

7.”李之受病,在不学无术。“

8.”李鸿章之外交术,在中国诚为第一流矣,而置之世界,则瞠乎其后也“,其对外国,”盖有一战国策之思想,横于胸中焉。“

9.李鸿章”不学无术,不敢破格,是其所短也;不避劳苦,不畏谤言,是其所长也。“
10.剿太平军期间,李鸿章给友人信中说”都中群议,无能谋及远大,但以内轻外重为患,日鳃鳃然欲收将帅疆吏之权。又仅挑剔细故,专采谬悠无根之浮言”。

11.剿捻期间,李鸿章说“兵与贼皆取资于民,千里无寨,所过已如梳蓖,故民仇兵甚于仇敌,久必不堪设想”。

12.光绪五年,“七月初六日李夫人病,中医束手,延英医诊治,旋愈。以是倾服西医,聘为家庭医师”。

13.清廷之不信任汉员:
\begin{quotation}
故二百年来,惟满员有权臣,而汉员无权臣。若鳌拜,若和坤,若肃顺、端华之徒,差足与前代权臣比迹者,皆满人也。计历次军兴除定鼎之始不俟论外,若平三藩,平准噶尔,平青海,平回部,平哈萨克布鲁特敖罕巴达克爱鸟罕,平西藏、廓尔喀,出师十数,皆用旗营,以亲王贝勒或满大臣督军。若夫平时,内而枢府,外而封疆,汉人备员而已,于政事无有所问。如顺治康熙间之洪承畴,雍正乾隆间之张廷玉,虽位尊望重,然实一弄臣耳。其余百僚,更不足道。故自咸丰以来,将相要职,汉人无从居之者。及洪杨之发难也,赛尚阿、琦善皆以大学士为钦差大臣,率八旗精兵以远征,迁延时机,令敌坐大,至是始知旗兵之不可用,而委任汉人之机,乃发于是矣。故金田一役,实满汉权力消长之最初关头也。及曾、胡诸公,起于湘鄂,为乎江南之中坚,然犹命官司文以大学士领钦差大臣。当时朝廷虽不得不倚重汉人,然岂能遽推心于汉人哉?曾、胡以全力交欢官文,每有军议奏事,必推为首署;遇事归功,报捷之疏,待官乃发,其伪谦固可敬,其苦心亦可怜矣。试一读曾文正集,自金陵克捷以后,战战兢兢,若芒在背。以曾之学养深到,犹且如是,况李鸿章之自信力犹不及曾者乎?吾故曰:李鸿章之地位,比诸汉之霍光、曹操、明之张居正,与夫近世欧洲日本所谓立宪君主国之大臣,有迥不相侔者,势使然也。
\end{quotation}


14.对洋务运动的评价与分析:
\begin{quotation}
李鸿章所办商务,无一成效可睹者,无他,官督商办一语,累之而已。中国人最长于商,若大授焉。但使国家为之制订商法,广通道路,保护利权,自能使地无弃财,人无弃力,国之富可立而待也。今每举一商务,辄为之奏请焉,为之派大臣督办焉,即使所用得人,而代大臣(?)者,固未有不伤其手矣。况乃奸吏舞文,视为利薮,任挟狐威,把持局务,其已入股者安得不寒心,其未来者安得不裏足耶?故中国商务之不兴,虽谓李鸿章官督商办主义,为之利阶可也。

吾敢以一言武断之曰:李鸿章实不知国务之人也。不知国家之为何物,不知国家与政府有若何之关系,不知政府与人民有若何之权限,不知大臣当尽之责任。基于西国所以富强之原,茫乎未有闻焉,以为吾中国之政教文物风俗,无一不优于他国,所不及者惟枪耳、炮耳、船耳、铁路耳、机器耳,吾但学此,而洋务之事毕矣。此近日举国谈时务者所异口同声,而李鸿章实此一派中三十年前之先辈也。是所谓无颜效西子之颦,邯郸学武陵之步,其适形其丑,终无所得也,固宜。
\end{quotation}

15.对于甲午战争失败后国人对李鸿章毁谤的看法:
\begin{quotation}
当中日战事之际,李鸿章以一身为万矢之的,几于身无完肤,人皆欲杀。平心论之,李鸿章诚有不能辞其咎者,其始误劝朝鲜与外国立约,昧于公法,咎一;既许立约,默自认其主,而复以兵干涉其内乱,授人口实,咎二;日本既调兵,势固有进无退,而不察先机,辄欲倚赖他国调停,致误时日,咎三;聂士成请乘日军未消集之时,以兵直捣韩城以制敌而不能用,咎四;高升事未起之前,丁汝昌请以北洋海军先鏖敌舰,而不能用,遂令反客为主,敌坐大而我逾危,综其原因,皆由不欲衅自我开,以为外交之道应尔,而不知当甲午五六月间,中日早成敌国,而非友邦矣,误以交邻之道施诸兵机,咎五;鸿章将自解曰:量我兵力不足以敌日本,故惮于发难也。虽然,身任北洋整军经武二十年,何以不能一战?咎六;彼又将自解曰:政府掣肘,经费不足也。虽然,此不过不能扩充已耳,何以其所现有者,如叶志超、卫汝贵,素以久练著名,亦脆弱乃尔,且克减口粮盗掠民妇,时有所闻,用并纪律而无之也,咎七;枪若苦窳,弹若赝物,弹不对枪,药不随械,谓从前管军械局之人皆廉明,谁能信之,咎八;平壤之役,军无统帅,此兵家所忌,李乃蹈之,咎九;始终坐待敌攻,致于人而不能致人,畏敌如虎,咎十;海军不知用快船快炮,咎十一;旅顺天险,西人谓以数百兵守之,粮食苟足,三年不能破,乃委之于所阘冗恇怯之人,闻风先遁,咎十二。此皆可以为李鸿章罪者若夫甲午九十月之后,则群盲狂吠,筑室道谋,号令不出自一人,则责备自不得归于一点若尽以为李鸿章咎,李固任受也。

又岂惟不任受而已,吾见彼责李罪李者,其可责可罪,更倍蓰于李而未有已也。是役将帅无一人不辱国,不待言矣。然比较于百步五十步之间,则海军优于陆军,李鸿章部下之陆军,又较优于他军也。海军大东沟一役,彼此鏖战五点余钟,西人观战者咸啧啧称赞焉。虽其中有如方伯谦之败类,然余船之力斗者固可以相偿,即敌军亦起敬也。故日本是役,惟海军有敌手,而陆军无敌手。及刘公岛一役,食尽援绝,降敌以全生灵,殉身以全大节,盖前后死难者,邓世昌、林泰曾、丁汝昌、刘步蟾、张文宣,虽其死所不同,而咸有男儿之概,君子(?)之。诸人皆北洋海军最要之人物也,以视陆军之全无心肝者何如也,陆军不忍道矣。然平壤之役,犹有左宝贵、马玉昆等一二日之剧战,是李鸿章部下之人也,敌军死伤相当。云其后欲恢复金州、海城、凤凰城等处,及防御盖平,前后几度,皆曾有与日本苦战之事,虽不能就,然固已尽力矣,主之者实宋庆,亦李鸿章旧部也。是固不足以偿叶志超、卫汝贵、黄仕林、赵怀业、龚照(左王右与)等之罪乎?虽然,以比诸吴大澂之出劝降告示,未交锋而全军崩溃者何如?以视刘坤一之奉命专征,逗留数月不发者何如?是故谓中国全国军族皆腐败可也,徒归罪于李鸿章之淮军之不可也。而当时盈廷虚㤭之气,若以为一杀李鸿章,则万事皆了,而彼峨冠博带,指天画地者,逐可以气吞东海,舌撼三山,盖湘人之气焰犹咻咻焉。此用湘军之议所由起也。乃观其结局,岂惟无以过淮军而已,又更甚焉,嘻!可以愧矣。吾之为此言,非欲为淮军与李鸿章作冤词也。吾于中日之役,固一毫不能为李淮恕也,然特患夫虚㤭嚣张之徒,毫无责任,而立于他人之背后,摭其短长以为快谈,而迄未尝思所以易彼之道,盖此非实亡国之利器也。李固可责,而彼辈又岂能责李之人哉?

是役也,李鸿章之失机者固多,即不失机而亦必无可以幸胜之理。盖19世纪下半纪以来,各国之战争,其胜负皆可于未战前决之,何也?世运愈进于文明,则优胜劣败之公例愈确定。实力之所在,即用于之所在,有丝毫不能假借者焉。无论政治学术商务,莫不皆然,而兵事其一端也。日本三十年来,刻意经营,上下一心,以成此节制敢死之劲旅,孤注一掷以向于我,岂无所自信而敢乃尔耶?故及其败然后知其所以败之由,是愚人也,乃或其败其犹不知其致败之由,是死人也。然则徒罪李鸿章一人,呜呼可哉?

西报有论者曰:日本非与中国战,实与李鸿章一人战耳。其言虽稍过,然亦近之。不见乎各省大吏,徒知画疆自守,视此事若专为直隶满洲之私事者然,其有筹一饷出一旅以相急难者乎?即有这,亦空言而已。乃至最可笑者,刘公岛降舰之役,当事者致书日军,求放还广丙一舰,书中谓此舰系属广东,此次战役,与广东无涉云云。各国闻者,莫不笑之,而不知此语实代表各省疆臣之思想者也。若是乎,日本果真与李鸿章一人战也。以一人而战一国,合肥合肥,虽败亦豪也。
\end{quotation}



16.对李鸿章外交失利的分析:
\begin{quotation}
夫天下未有徒恃人而可以自存者。泰西外交家亦尝汲汲焉与他国结盟,然必有我可以自立之道,然后,可致人而不致于人。若今日之中国,而言联某国联某国,无论人未必联我,即使联我,亦不啻为其国之奴隶而已矣,鱼肉而已矣。李鸿章岂其未知此耶?吾意其亦知之而无他道以易之也。要之,内政不修,则外交实无可办之理。以中国今日之国势,虽才十倍于李鸿章者,其对外之策,固不得不隐忍迁就于一时也。此吾所以深为李鸿章怜也。虽然,李鸿章与他役吾未见其能用手段焉,独中俄密约,则其对日本用手段之结果也。以此手段,而造出后此种种之困难,自作之而自受之,吾又何怜哉?
\end{quotation}
