\subsection{《局外人》}

作者:【法】加缪

译者:柳鸣九

故事很短,仅仅五六万字,也很简单。“我”是一个普通职员,平时就是一个“老好人”,平凡甚至于平庸。“我”由于经济原因无法照料母亲,把她送到养老院。母亲去世了,“我”去奔丧,表现得表面上看上去冷漠。回来后第二天,“我”和情人约会游泳。“我”住处的邻居西蒙与女友发生矛盾,“我”为他作证,虽然不诚实但并不算什么罪恶。西蒙想报复女友的哥哥,一个阿拉伯人,“我”应约前往。双方打了一架,之后回到海滩,发现阿拉伯人还在,于是“我”等西蒙走远,对那个阿拉伯人开了一枪,之后又开了四枪。“我”被捕后,没有急于为自己开脱,而是佛系地让律师代办,律师让“我”怎么做怎么说就怎么做怎么说。无论是预审法官,还是正式法庭上的检察官,都紧紧抓住我将母亲送进养老院和丧事时的似乎无动于衷,认为我因此而有罪,并非因为打死阿拉伯人有罪。他们义愤填膺地对我控诉,认为我道德败坏,十恶不赦。“我”在法庭上没有机会为自己辩解,律师只是尽责为“我”解脱。最终,“我”放弃了上诉,被关了一年多,并决定上断头台。

“我”确实杀了人,杀人偿命,在法律上来讲,“我”的死亡并不算冤枉。即使是“稀里糊涂”地杀人,也是杀人,也是夺去他人生命,就要付出法律上的代价。在杀人现场,“我”拿着枪,“天气酷热,刺眼的阳光像大雨一样从空中洒落而下,即使站在那里,我也感到很难受”。“这太阳和我安葬妈妈那天的太阳一样”

整个叙述中,“我”似乎流离在事件的发生之外。在丧事期间,我浑浑噩噩,身体的触觉大于事件本身对“我”的影响。“我”会注意到太阳的毒烈,灵堂的气氛冷淡,案发现场海滩上炙热,庭审时的“热”。整个司法程序里,审视“我”做过的事情才是最重要的,“我”反而是不重要的,被排斥在司法之外。“我”成为整个事件的“局外人”。

法庭抓住“我”对母亲的“冷酷无情”不放,认为在这件事情上“我”有罪,而显然这并不构成罪过;他们的理由,无非就是“我”应当做一个道德模范,成为一个母亲的孝顺儿子,与母亲生活在一起,其乐融融。他们在这个角度上“审判”,认为完全无关的这件事与杀人有关,“究竟是在控告他埋了母亲,还是在控告他杀了一个人?”这构成一种荒谬,法庭代替了上帝对人的罪过进行审判,而非就事论事,显然有违法治精神。

“我”放弃了上诉,也放弃了临终前神甫的安慰。神甫代表上帝,对“我”的罪恶表示宽恕,并希望“我”获得上帝的正义审判和拯救。“我”表示不信上帝,罪恶是由人类来审判的,而非上帝,而且事件的意义就在于事件本身,不需要获得上帝的安慰。这种无神论的坚持,显现出“我”对宽恕的拒绝。“我”想要的,无非是事件本身所表现出的意义,而非宏大的、联系的、信仰上的帮助。“我”乐于让市民去看自己行刑,这种悲凉与绝望,让人不寒而栗。

小说的意义不仅仅是对司法制度的反思。在更深刻的层次上,加缪描述了一种主人公与事件本身之间的抽离。故事是简单的,其周身却散发出一种感觉的、触觉的迷雾,而“我”就迷失在这种迷雾中。“我”似乎对自己的命运缺少直接地、有力度地控制,而是任由事情发生,有种深深地虚无与绝望,而事实上“我”并没有什么苦大仇深的遭遇,也没有戏剧性地、“经典”的挫折与失败,“夸张与过分是喜剧所需要的成分,而蕴藏、敛聚、深刻才是悲剧的风格”