\subsection{《战争与和平》}

标签: 俄国文学 \  托尔斯泰

作者:列夫·托尔斯泰

\subsubsection{人物}

\begin{longtable}{p{0.15\textwidth}|p{0.4\textwidth}|p{0.4\textwidth}}
\caption{《战争与和平》人物表}\\
\hline
人物 & 身份特征 & 事件 \\
\hline
\endhead

华西里公爵 & 虚伪做作,以儿女的婚姻来攀亲附贵 & \\
文森特盖罗 & & \\ 
安娜·舍勒 & & \\
莫特玛子爵 & 通过罗亨家的关系同蒙莫朗西家沾亲,是法国的一个望族,真正的高等侨民,有点自命不凡,但有教养,对谁都彬彬有礼 & \\ 
莫里奥神父 & & \\ 
冯克男爵 & & 太后想任命他当维也纳使馆一等秘书 \\
阿纳托里 & 华西里公爵的小儿子,长相俊美 & 轻浮放荡,曾结过婚又追求娜塔莎 \\
伊波利特 & 华西里公爵的大儿子, 安分守己,傻 & \\ 
保尔康斯基老公爵 & 严肃正派,古板 & 想把玛丽雅留在自己身边,对她严肃刻板 \\
玛丽雅·保尔康斯基公爵小姐 & 具有虔诚奉献的宗教精神,长得不漂亮,对父亲十分敬畏 & 每天弹钢琴,做几何习题。对阿纳托里一见钟情,但“让”给了布莉恩小姐 \\
布莉恩小姐 & 玛丽雅小姐的女伴,法国人,没有地位 & 对阿纳托里一见钟情 \\
宋尼雅 & & 深爱尼古拉 \\
娜塔莎 & 尼古拉的妹妹 & \\ 
彼嘉 & 尼古拉的弟弟 & 渴望建功立业 \\ 
安德烈·保尔康斯基公爵 & 英俊,很有钱,知识渊博,但很吝啬,很聪明,脾气怪 & 住在乡下,先帝在世时就退役,绰号“普鲁士王”。对夫人有点倦怠 \\
安娜 & 安德烈公爵夫人,天真轻佻 & 生完小尼古拉后死去 \\ 
皮埃尔 & 安德烈公爵的好友,住在华西里公爵家,和阿纳托里混在一起。笨头笨脑,身体肥胖,比普通人高,不善交际 & 十岁时被带到国外,待到20岁。崇拜认同拿破仑 \\
陶洛霍夫 & 25岁左右,步兵军官,无权无钱,喜欢赌博,逢赌必赢,酒量好 & \\ 
库图佐夫 & 俄国老将军,俄法战争的总司令 & 沉着冷静,经验丰富 \\
杰尼索夫 & 尼古拉的战友 & 与娜塔莎恋爱,但很快结束 \\
海伦 & 华西里公爵的女儿 & 放荡自私,爱财 \\
拉斯托普庆伯爵 & 莫斯科战时的卫戍司令 & 典型的官僚 \\
魏列夏金 & & 莫斯科围城期间被拉斯托普庆下令处死,后被人群打死 \\
\hline
\end{longtable}

\subsubsection{情节}

\begin{itemize*}
	\item 娜塔莎与保里斯相恋,后者答应四年后战争结束向她求婚。
	\item 皮埃尔继承了遗产,成为身份尊贵、财富众多的公爵,并与华西里公爵的女儿海伦结婚。
	\item 奥斯特里茨战役中,拿破仑击破俄奥联军,安德烈负伤。
	\item 安德烈夫人在童山保尔康斯基老公爵的庄园里难产而死,安德烈负伤回来后见到她时死去
	\item 尼古拉在奥斯特里茨战役中见到了皇帝,回家后与宋尼雅相见,但拒绝了她的求爱。他沉迷赌博被陶洛霍夫骗了几万卢布,之后参军。
	\item 海伦喜欢交际,与皮埃尔不和,皮埃尔容忍了她。
	\item 陶洛霍夫与海伦偷情,皮埃尔与之决斗受伤。之后皮埃尔离婚,加入共济会
	\item 安德列公爵遇见娜塔莎,爱上了她,并约定一年后正式结婚,但父亲和妹妹玛丽雅公爵小姐不喜欢娜塔莎。
	\item 保尔康斯基公爵为反对安德烈公爵与娜塔莎的婚事,与布里恩小姐十分亲密,威胁如果安德烈公爵与娜塔莎成婚,则自己就与布里恩小姐成婚。
	\item 保里斯放弃了宋尼雅,贪恋裘丽的财富,与后者结婚。
	\item 娜塔莎受阿纳托里诱惑爱上了他,阿纳托里在陶洛霍夫的帮助下准备和娜塔莎私奔,被娜塔莎的家人拦下,阿纳托里逃走。
	\item 拿破仑发动法俄战争,亚历山大持迎战。
	\item 皮埃尔爱上娜塔莎。
	\item 保尔康斯基公爵去世,玛丽雅公爵小姐的父亲的压迫感消失,渴望自由的她情感复杂。尼古拉遇见被法军包围和被农奴欺压的玛丽雅公爵小姐,救她出去,并爱上了她。
	\item 皮埃尔去战场上“参观”,后在莫斯科城陷落时准备刺杀拿破仑,假装农民后被捕,行刑前保住一命。海伦因堕胎而死。
	\item 安德烈公爵负伤,在战地医院见到负伤的阿纳托里,对他生出怜悯原谅之情。
	\item 法军攻向莫斯科,罗斯托夫一家准备逃走,此时负伤的安德烈公爵住到罗斯托夫家并随车队撤离。娜塔莎发现了安德烈公爵,开始照顾他。但是安德烈公爵在玛丽雅公爵小姐到达前死去。
	\item 法军在莫斯科放火烧城,后弃城逃走。逃走时带走了皮埃尔作为俘虏。
	\item 彼嘉渴望建功立业,在战场上冲锋陷阵并被子弹打死。战斗胜利后,杰尼索夫和陶洛霍夫救出了作为俘虏的皮埃尔。
	\item 法军向边境撤退,减员很多。俄军追赶,减员也很多,最终的战役里没能歼灭法军。
	\item 皮埃尔回到家乡,遇见娜塔莎,两人重燃爱火并结婚。
	\item 尼古拉回到家,家道中落,父亲去世,母亲要这要那。他重遇玛丽雅公爵小姐,两人结婚,后家境好转。
\end{itemize*}

\subsubsection{名言警句及一些标注}

权势在社会上是一种资本,不应随便动用。华西里公爵深谙这个道理。他知道,他要是有求必应,以后自己有事就不能去求别人了,因此难得使用自己的权势。

1. 想用它们来说明依我看是支配历史的那条宿命的规律,以及那条心理学规律,它促使做出最不自由行为的人从回顾往事中虚构出一系列结论,以证明他本身的自由。

2. 理性表达必然规律,意识表达自由的实质。

3. 我们总以为一旦离开走惯的道路,一切就都完了;其实美好的新东西才刚刚开始呢。有生活,就有幸福。来日方长。

4. 其实所有的战役,包括塔鲁季诺战役、鲍罗金诺战役、奥斯特里茨战役,都不是按照部署进行的。

5. 财富也罢,权力也罢,生命也罢,也就是人们努力争取和保护的一切,这一切如果有什么价值的话,也只在于有可以放弃它们的乐趣。

6. 在耍弄诡计上,愚人往往胜过聪明人,

7. (托尔斯泰的历史观)只有承认无穷小的单位——历史的微分(就是人的个人倾向)而进行观察,并掌握求积分的方法,我们才有希望认识历史的规律。

9. (拿破仑)这个人应负的责任比谁都多。他的理智和良心不仅在这一天、这一小时变得暗淡无光,直到生命的末日,他永远无法理解真、善、美,无法理解自己倒行逆施、灭绝人性的行为的意义。他不能放弃自己受半个世界歌功颂德的行为,因此他也就不得不放弃真和善,放弃一切合乎人性的东西。

11. 每逢大祸临头,人的心里总会响起两个同等强烈的声音:一个声音非常理智地说,人应该考虑自己处境的危险和避免危险的方法;另一个声音更加理智地说,要预见一切和逃避大势是非人力所能及的,因此面临危险时还是别去想它,否则太痛苦,还是多想想快乐的事为好。单身独处,人往往听从第一种声音;众人群处,人往往听从第二种声音。

12. (安德烈公爵)他很可怜这个受惊的好看的女孩子。他不敢看她,但又忍不住想看看她。他望着这两个女孩子,想到世界上还有一些和他截然不同的人,他们也有他们的生活乐趣,他的心里不禁涌起一股从未有过的快乐暖流。

14. 其实他们全是一些身不由己的历史工具,做着他们自己并不明白而我们却是了解的事。这就是一切实际活动家的必然命运:他们官做得越大,就越不自由。

17. 自古以来对权势人物早就炮制了一套天才论,因为权力就在他们手里。其实战争的成败不取决于他们,而取决于那个在队伍中高喊‘完蛋了’或者‘乌拉’的人。只有在这种队伍里服务,你才能满怀‘自己有用’的信心!

18. 法国人之所以自信,是因为他们认为他们的智力和肉体,不论对男人还是对女人,都具有魅力。英国人之所以自信,是因为他们是世界上组织最完善的国家的公民,英国人永远知道他们应该做什么,而且所做的一切绝对正确。意大利人之所以自信,是因为他们情绪激动,容易忘乎所以,旁若无人。俄国人之所以自信,是因为他们一无所知,也没有求知欲,因为他们根本不相信人能知道什么。德国人的自信最糟糕,最顽固,最可憎,因为他们自以为懂得真理,懂得科学,其实这种科学是他们臆造的,但他们却认为是绝对真理。

19. 玛丽雅公爵小姐依旧是个胆怯、丑陋的老姑娘,永远生活在恐惧和苦恼中,毫无意义毫无欢乐地虚度着青春年华。布莉恩还是一个春风得意卖弄风情的姑娘,快乐地享受着生命的每一瞬间,并且满怀最美好的希望。安德烈公爵觉得,她只是变得更加自负。

20. (皮埃尔)他具有许多人特别是俄罗斯人所具有的可悲能力;看到并相信善和真是存在的,但同时对生活中的邪恶和虚伪又看得太清楚,因此无法认真地参与生活。在他看来,任何活动都和邪恶与欺骗联系在一起。不论他要做个怎样的人,不论他从事什么活动,邪恶和虚伪总是与他为敌,堵塞他的一切道路。然而他总得活下去,总得做点事情。这些无法解决的人生问题使他太痛苦,因此他一有机会就寻欢作乐,以便忘记这些问题。他出入交际场所,纵酒狂饮,收购图画,大兴土木,而更多的是读书。

22. (尼古拉)他离家越近就越想家,仿佛人的情绪也服从引力和距离成反比的定律。

24. 凡是智力不发达的人总喜欢提到时代,认为他们懂得并且重视时代的特点,而且人的本性是随时代而改变的。

25. 人要幸福,必须相信能获得幸福。

26. 一个真正的共济会会员,在国家需要他担任公职时,应该是个勤恳的公务员;在他没有被聘用时,应该是个冷静的旁观者。

27. 真共济会的主要责任在于自我完善。我们常常以为摆脱生活中的困难,就能更快地达到这一目的;其实相反,先生,只有在尘世的骚乱中我们才能达到以下三项宗旨:第一,自知,因为人只有通过比较才能认识自己;第二,自我完善,只有经过奋斗才能自我完善;第三,达到主要的德行——视死如归。生活的变化无常最能显示它的空虚,增强我们天生对死亡或重生的爱。”

28. 我们所能知道的就是我们的无知。这也就是人类的最高智慧。”

29. 有些人表面上似乎软弱,遇到不幸的事却不愿向人倾诉,宁肯独自默默地忍受痛苦。皮埃尔就是这一类人。

30. 在这群上流社会矫揉造作、琐碎无聊的趣味中,融入了一对漂亮健康的青年男女相互倾慕的真挚感情。这种感情压倒一切,远比那些装腔作势的闲谈高尚。

31. 有些人只在自以为纯洁无瑕的时候才显得坚强,而皮埃尔就是这种人。

32. 我们没谈恋爱的时候,就等于在睡觉。我们是尘世的女儿……一旦恋爱,我们就成了神,就同创世第一天一样纯洁……

\subsubsection{书评}
托尔斯泰写这部小说的目的之一是探讨历史的运行逻辑,思索个人自由意志和历史发展的关系。在小说里,作者不厌其烦连篇累牍地讨论历史事件如何运行,批评神学史观和英雄史观。只看这些文字,读者可能会认为自己在阅读哲学或史学著作。

作为十九世纪的人,托尔斯泰认为历史是由一个个微小事件或人物来推动的,如同物理学那样符合因果律。这种机械论思想没有超出那个年代哲学和科学认识的局限性,对当代人而言不算什么真知灼见。

托尔斯泰毕竟不是历史学家,史学修养和观点不够专业,甚至有点民科的味道。

作为一部文学作品,《战争与和平》在艺术上简直是登峰造极的:

首先就是气势恢宏格局宏大,俄国卫国战争背景下上至皇帝下至农奴都有展现,时间脉络多而不乱,人物错综复杂交织缠绕,一般的作家根本驾驭不了,写着写着就写崩了。

其次是笔法直接而精准,不事修饰,没有虚词,在美学上浑然天成,人物事件寥寥数笔就活灵活现。要做到这点很难,因为大部分作家心中有挂碍有认知障,下笔褒贬很多,甚至人物事件沦为自己见解的工具,成为自己的嘴巴,于是人物本身就扁平而缺少生命力。高明作家下的人物都是有自己的行事逻辑,不以作者意志为转移,这样的作品(个人认为)是一流作家的“金线”,越过去就能在文学史留下位置。

然后就是托尔斯泰对笔下人物的感情饱满而富有激情,能体会到他对这些人(即使有缺点)心存博爱之心,比如对库图佐夫的敬佩,对拿破仑的鄙视(甚至有点过了)。

最后是俄国作家对祖国对人民那种宗教般深沉的爱,以及苦修式的行事风格。皮埃尔、玛莉雅公爵小姐、安德烈公爵这些人对真理执着而深入的追求和思索,代表着俄国人的民族精神。尼古拉那种渴望建功立业、对沙皇的狂热崇拜追随的精神是俄罗斯强大的根基。

总之,这部小说不愧是世界十大名著之一,其水准之高在现实主义小说里应该罕有其匹了。俄罗斯文学是世界文学的天花板!