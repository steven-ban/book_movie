\subsection{《大教堂与集市》}
\subsubsection{标注}

一般而言,任何语言,若是不能得到至少Linux或某种BSD的支持,以及/或者不能得到至少三家厂商的操作系统的支持,都不值得想当黑客的你学习。

创造性头脑是无比珍贵的有限资源,它们不应浪费在重新发明轮子这种事上,尤其是还有这么多迷人的新问题在那里等着的时候。

黑客搞建设,骇客搞破坏。

如果开发成员数目为N,工作量会呈N倍增长,但复杂性和bug率会以N 2增长。N 2体现着各开发者代码之间的通信路径(以及可能的代码接口)。

有时候,要想成为一只更大的青蛙,最佳办法就是让水池更快变大,这就是技术公司参与公开标准(完全可以将开源软件看成是可执行标准)的经济原因。

软件很大程度上是一个服务行业,虽然长期以来都毫无根据地被错认为是制造行业。

如果你想获得最有效率的产品,你必须放弃促进程序员生产力。做好他们的后勤,让他们自己做主,并忘掉最后期限。

只有当程序员非常积极以至于没有奖励他(她)也愿意工作时,才是唯一应该给予绩效奖励的时候。

固定工资不会降低人们的积极性,但计件工资和奖金会,

活动越复杂,就越容易被外部的报酬损害。”

与完全出于兴趣的工作相比,被委托的工作通常表现出较少的创造性。”

自由市场经济是全世界范围内通过合作获得经济效能的最佳方法,这一点看来已成为历史定论,同样,基于声誉竞争的礼物文化可能是通过合作产生(和检验)高质量创造性工作的全世界范围内的最佳

很多关系型数据库也采用了同样的启发性方法解决死锁问题,当两个线程在资源上造成死锁时,在当前事务中投入最少的那个线程将会成为死锁受害者并被终止掉,而运行事务时间最长的或级别更高的,则会成为胜利者。)

移交准则:所有者/维护者不再维护项目时,应公开地将权利移交给某人。

相对于修补bug而言,给软件增加功能特性有可能得到更多回报——除非这个bug异常令人厌恶或者难以寻找,因为将这种bug找出来本身就证明了非凡的技术和才能。

开源软件倾向于长期停留在beta版,开发者只有在确信软件不会有很多问题时,才会发布1.0版。

即“自称是黑客不代表你就是黑客,只有其他黑客认为你是黑客,你才是黑客”

在第三个千年的开始,我们大可预言开源会转向最后一块处女地——写给非技术人员的程序。

欧裔美国人对“自我”通常所持的否定态度。

开源黑客社会,可以很清楚地看出它就是一种礼物文化。

开源的所有权理论在实质上等同于英裔美国人关于土地所有制的习惯法(common law)理论。

我在“魔法锅”(The Magic Cauldron)一文(本书后面的一章)中提出了这样的观点:未来软件产业的经济关键是服务价值。

按“命令与纪律原则”行事和按“共识原则”行事之间的重要区别。前者在军队检阅时的作用令人钦佩,但在真实生活中却一文不值,想要达到目标,必须要靠众人的齐心协力。”

Brooks定律已经被广泛地视为真理,但在本文中我们已经通过多种方式论证了开源软件的开发过程不满足这个定律背后的一些假设——并且从实践上看,如果Brooks定律普适于所有开发项目,Linux是不可能完成的。

在《人月神话》中,Fred Brooks发现程序员的时间是不可替代的,增加开发者进入一个已经延迟的软件项目,只会让项目更加延迟。

当你发现自己在开发中碰壁时,当你发现自己苦思冥想也很难做出下一个补丁时,通常你不该问自己是否找到了正确答案,而是该问你是否提出了正确的问题,因为也许问题本身需要被重新定义。

12.通常,那些最有突破性和最有创新力的解决方案来自于你认识到你对问题的基本观念是错的。

9.聪明的数据结构配上愚笨的代码,远比反过来要好得多。

bug很容易集中在不同人写的代码的交互接口上,沟通/协调的开销会随开发者间接口数的增加而增多,也就是说,问题规模和开发人员间的沟通路径数相关,即和人数的平方相关(更精确地讲,应该是N(N-1)/2,N代表开发者数目)。

Brooks定律指出,随着开发人员数目的增长,项目复杂度和沟通成本按照人数的平方增加,而工作成果只会呈线性增长。

在你第一次把问题解决的时候,你往往并不了解这个问题,第二次你才可能知道怎么把事情做好。所以,如果你想做对事情,至少要再做一次。

卓越程序员们有个很重要的特征是“建设性懒惰”,他们知道人们要的是结果而不是勤奋,而从一个部分可行的方案开始,明显要比从零开始容易得多。

优秀的程序员知道写什么,卓越的程序员知道改写(和重用)什么。

\subsubsection{感想}

Raymond是开源运动的吹鼓手。开源的开发模式相对于传统软件公司工厂式的开发模式,确实有一系列的优点:开发者参与度更高、给予开发者更多自由和控制权、会有更多的人反馈以解决bug问题、会有更快的功能迭代等等。Raymond介绍的黑客的“礼物文化”也确实存在于很多开发者心里。

本书写成于2000年左右,距离现在已经有16年了,很多想法和现在都有了差别。

对于“礼物文化”,中文互联网上很多用户太得寸进尺,稍微不满意就冷言相向甚至辱骂搞人身攻击,使得很多开发者失去了热情。
中国互联网才发展了二十年,程序员的社会保障并不好,很多开发者在公司里业务缠身,三十岁以后就想脱离开发岗位,想着转岗或者转行,更别说去做无报酬的开源工作了。
移动互联网兴起后,特别是IOS开发兴起后,开源虽然也越来越多,github的存在使开源有了更大发展,但基于客户端最终产品的开发还是闭源的,比如IOS上大量的游戏。很多游戏引擎如cocos2d、unity等都开源,但基于其的游戏本身并不开源。
开源许可证也更多样,像MIT许可证更好地满足了商业需求,在开源许可上没有那么激进。
像微软这样的反开源大户现在竟然也成了开源大户,恐怕是Raymond当时怎么也想不到的。
开源运动现在不仅有软件,还有硬件,像树莓派这样的小玩意越来越多,可玩性比软件高多了。
开源进入中国应该比较晚,我估计是在2000年以后,对于普通人来说现在还是个新鲜概念,对于geek来说恐怕也是通过linux的推广。王垠当年的那篇讨伐微软檄文写得声情并茂(现在王垠已经加入了微软),大概是不少人的开源启蒙。开源的产品大多还是linux相关的,但大多数人并不会使用linux。中国的盗版软件泛滥相当严重,很多人没有版权意识,软件拿来就用。如果中国的知识产权的保护更得力,恐怕逼得很多人不得不去用免费的开源软件了。中国的软件厂商里大概很多都用了开源的库,但有没有遵守开源协议就很难说了,我估计绝大多数也是拿来就用,根本不会把自己的产品开源出去。

对于普通用户来说,其实无论开源也好闭源也好,只要产品好用价格便宜,不用在乎是怎么开发出来的。现在移动互联网的开发倾向于小团队甚至个人,开发模式和二十年前的开源是类似的,和大公司的闭源工厂模式差别反而更大一些。用户花钱买开发者的劳动,这个行业才会发展得更好,否则总是希望开发者出卖免费劳动力无法持续。Android上免费但广告多得要死的软件,用户体验远远比不上IOS上花几块钱买的小软件。