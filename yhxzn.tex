\subsection{《银河系搭车客指南》}

标签: 科幻 \  星际 \  宇宙

英文名:Don't Panic

作者:【英】道格拉斯·亚当斯

译者:姚向辉

\subsubsection{人物}

\begin{longtable}{p{0.15\textwidth}|p{0.3\textwidth}|p{0.5\textwidth}}
\caption{《银河系搭车客指南》人物表}\\
\hline


人物 & 特征 & 事件 \\
\hline
\endhead

亚瑟·邓特 & 三十多岁 & 房子被强拆 \\
福特·大老爷 & 宇宙流动调查员 & 为《银河系搭车客指南》撰写内容,在地球上生活了15年,认识亚瑟·邓特,带他离开被摧毁的地球 \\
赞法德·毕博布鲁克斯 & 银河帝国政府总统,福特·大老爷的半表哥 & 以当总统为幌子,在前总统的指示下偷走了飞船“黄金之心”,它拥有“不可能性引擎”。在地球上曾以菲尔的名字参加派动,夺走亚瑟·邓特当时着迷的翠西亚 \\
普洛斯泰特尼克·活贡·杰尔茨 &  & 他的沃贡施工飞船的旗舰摧毁了地球 \\
马文 & 机器人,拥有自主意识 & \\
翠西亚·麦克米兰 & 地球人,拥有数学和天体物理学学位 & \\
沉思 & 超级计算机 & 设计出地球这一比它更伟大的计算机 \\
银辟法思特 & 玛格里西亚人 & 设计了地球上的海岸线,特别是挪威 \\
\hline
\end{longtable}

\subsubsection{主要情节}

\begin{itemize*}

	\item 坂裘战争是银河系一场古老的战争。有一个坂裘星(谐音板球)。他们被隐藏在一片星云里,这片星云是机器人的dyt
	\item 深思决定建造一个比自己更伟大的计算机,取名为“地球”,这里人类是实验品,老鼠才是主人,用以计算生命及世界的本质,得到了结果“42”
	\item 赞法德以当总统为名,偷走“黄金之心”飞船
	\item 因为阻碍宇宙通道的施工,地球被普洛斯泰特尼克·活贡·杰尔茨的舰队摧毁,并抓住了亚瑟·邓特和福特·大老爷
	\item 普洛斯泰特尼克·活贡·杰尔茨将亚瑟·邓特和福特·大老爷扔出太空舱,两人被赞法德·毕博布鲁克斯的太空船救起
	\item 三人旅行到玛格里西亚,遇见老鼠,老鼠想取出亚瑟的大脑来研究,后来他们逃走,被杰尔茨追杀,在赞法德曾祖父的鬼魂的帮助下逃脱,祖父让他去找银河系的造物主。
	\item 赞法德和马文游落到小熊星座贝塔星,这里有《银河系搭车客指南》编辑大楼,这里他们受到了蛙星人的围剿,赞法德和雄鸡被蛙星人抓走。
	\item 在蛙星赞法德找回了口袋里的“黄金之心”,和翠西亚、福特、亚瑟一起到了多年之后的蛙星当地的餐馆,宇宙即将毁灭,餐饮里正在准备迎接这个时刻。马文留在了多年以前,打电话让赞法德回来。
	\item 亚瑟他们目睹英国一个板球场上的奖杯灰烬杯被板裘星人抢走,引发了银河系的战争。
	\item 亚瑟被炸到一个星球上,长途跋涉,住在山洞里。
	\item 亚瑟-邓特曾经一次次地杀死一个生物,这个生物不断地转生,并且一次次被亚瑟杀死:它有一世是苍蝇,有一世是牵牛花,然而这次变成了妖怪,依然被亚瑟杀死。
	\item 亚瑟和翠西亚一同说服了将板裘星隐藏在一片星云里的机器人并将板球带回地球的球场上,投球拯救了银河系的毁灭。
\end{itemize*}

之后的情节就很怪异了(其实一直很怪异),没有稳定的人物性格,只有一些抖机灵,这些情节没有深究的必要。

\subsubsection{妙言妙语}
这类语句俯拾即是,这里做点不完全的摘抄:
\begin{itemize*}
	\item 据说,赫林-佛洛特米格创始了,把诚实和理想主义设为基础准则,然后就破产了。
\end{itemize*}

\subsubsection{书评}
一个普通的英国人亚瑟,他的房子被强拆了,但随后他遇见了外星人,外星人告诉他地球马上就要毁灭了,两人一起离开了地球。果然,几分钟后,地球被毁灭,原因很简单:地球阻碍了银河糸修路,也被强拆了。两个人于是展开了漫漫旅行路……

这样的轻松的夸张的幽默贯穿了全书。内容虽然是“银河系漫游”,但很多情节和逻辑往往和平时生活里的幽默和尴尬联系在一起。比如,卷二浓墨重彩地描绘了人们拼命时间旅行到宇宙灭亡前的一瞬间去酒馆喝酒的故事。

本书主要的形式即是一地一事的直线型叙述,情节靠”奇遇“来推动,亮点全在细节里,曾在BBC制作成为广播剧。然而作为一部小说,它缺少贯穿全书的主线结构,也缺少宏观的结构。因此,这部小说可以轻松地阅读,可以随时中断,但不会吸引你一口气读到最后,这可能也是它作为文学作品缺少内涵和艺术性的原因。就像大多数广播剧、电视剧看完之后不想再听或再看第二遍一样,本书也是快餐式的消遣,读者恐怕很难有兴趣去看第二遍。

本书的人物也是充满着夸张和不正经,人物性格缺少特点(或者说完全没有传统意义上的性格),不会给人留下不可磨灭的记忆。这或许从英国人”亚瑟“的名字就能看出来(类似的名字就像是中国的”张伟“”赵阳“一样),这个名字太普通了,不会给人留下什么印象。

评分:5/10。