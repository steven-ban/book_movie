\subsection{《使女的故事》}

标签: 英美文学 \  反乌托邦 \  女性文学  \ 平权 \  宗教

---

作者:【加拿大】玛格里特·阿特伍德\footnote{参考\url{https://baike.baidu.com/item/阿特伍德/4610058?fr=aladdin}}

\subsubsection{人物}

\begin{table}
\centering
\caption{《使女的故事》人物}
\begin{tabular}{p{0.18\textwidth}|p{0.4\textwidth}|p{0.4\textwidth}}

 人物 & 特征 & 事件 \\
 \hline
琼·奥芙弗蕾得(June Offred)& 小说主人公,“我” & 使女,用来给高阶层人物生育孩子 \\
丽迪亚嬷嬷 & “我”的教引嬷嬷 & 与珍妮搞女同,性格严厉 \\
奥芙格伦(Ofglen)& 另一个使女,和“我”一起出去好相互监视 & 和“我”关系逐渐熟稔,地下组织成员,后自杀。 \\
赛丽娜·乔伊夫人 & 原名帕姆,Fred家的夫人,不喜欢“我” & 以前喜欢演讲,宣传官方宗教思想,曾躲过暗杀,曾经是个小歌唱演员。现在歇斯底里,善于流泪。找“我”密谈,建议和尼克“借精生子”,作为回报,她答应给“我”弄来女儿的照片,她确实办到了。\\
尼可 & Fred家的司机 & 干活时偷偷抽烟,帮助大主教和“我”幽会,同时是大主教去“荡妇俱乐部”的司机\\
丽塔 & Fred家的”马大“(女仆),六十多岁 &  \\
卡拉 & Fred家的”马大“ & “我”孕吐时帮助“我”撒谎,似乎喜欢撒谎 \\
卢克 & “我”在列基共和国成立之前的男友,喜欢咬文嚼字 & 有妻子时曾经经常和“我”在旅馆幽会,他因此离婚,前妻打电话咒骂,后被消失 \\
珍妮 & ”我“之前在感化中心认识的人,现在也是使女 & 在感化中心时会在祷告时因为祷词而哭泣,已怀孕,肚子很大,仍在街上行走,与医生偷情,后因生下畸形儿而被送入劳动中心 \\
莫伊位 & 大学时代住在“我”隔壁,古灵精怪,无忧无虑,健壮敏捷 & 女同性恋,常骑一辆自行车,背一个远足用的背包。在感化中心逃走,辗转多处到了“荡妇俱乐部”并与“我”会面,后因触犯圣经而被割去双脚。 \\
? & “我”的妇科医生 & 诱惑让使女(包括“我”)生孩子,以避免因为不育而被杀 \\
? & “我”和卢克曾经的女儿,五岁时死亡,如果不死现在八岁 & 曾经在超市被偷,但小偷立刻被抓 \\
? & “我”的母亲 & 去了隔离营,清理各种东西 \\
? & “我”房间的前任使女 & 大主教和她幽会,被赛丽娜·乔伊夫人事发后上吊死亡 \\
大主教Fred & 给“我”书看,帮“我”掩饰,带“我”去“荡妇俱乐部” & 曾经有多个使女   \\
\hline
\end{tabular}
\end{table}

\subsubsection{故事背景}

《使女的故事》(2017版)封面: \url{https://gss0.bdstatic.com/-4o3dSag_xI4khGkpoWK1HF6hhy/baike/c0\%3Dbaike116\%2C5\%2C5\%2C116\%2C38/sign=7ff6a5bc4036acaf4ded9eae1db0e675/fcfaaf51f3deb48f384d475bfb1f3a292df5780b.jpg}

本书写于1984年,是基于空想和”担忧“而写就的小说。在不远的将来,美洲被一个称为“基列共和国”的国家代替,它基于基督教的原教旨和禁欲,社会里的阶级分野十分明显:普通男人没有性交和生育的权利,只有“大主教”和其他高阶级的人才可以生育;女性失去了工作和拥有财产的权利;低阶级女性要么去做“马大”(女仆),要么去做“使女”,后者实际上是专供高阶级男人的生育机器,她们被“感化中心”里的人洗脑,让她们放弃了任何自由,乖乖做一个无法改变命运的机器。使女如果被发现不育,则会被送入隔离营。法律规定没有不育的男人,只有不育的女人,不育是女人的错。使女并不属于一个特定的大主教,而是在成功生育后可以辗转多家去生育。自由市场被取消,像香烟这样的嗜好品被禁止,只有在黑市上才能买到(但很多人通过这样的渠道去购买)。

这个社会对女性进行了身体和思想上的压迫。在思想上,感化中心教育未来的使女们\emph{“知即诱惑,不知者免受诱惑”},剥夺了她们求知和工作的能力。在身体上,她们成为社会的螺丝钉和工具。

本书的主人公“我”即是一个使女。故事发生的时候,“我”从上一家里出来,去了Fred家里做使女。使女需要定时同时与大主教和夫人一起同房,大主教与使女性交,夫人的身体也压在使女身上,与大主教抽插的过程相配合(很搞笑)。而旁边还有观众,有大主教的仆人和司机,这些人称为他的“家人”。

怀上孩子对于使女来说意味着安全(否则便会被处死),但只是部分安全——她们有四分之一的概率生下畸形儿,这同样会导致她们的厄运。她们的命运与生育紧紧相连,她们是生育机器。

\subsubsection{情节}
“我”在Fred家的房间的橱柜底部有一行字\emph{Nolite te bastardes carborundorum(别让那些杂种骑在你的头上)}。这可能是之前使女留下的,她“没能熬出来”。

大主教私自找“我”去跟玩拼字游戏,并让“我”亲吻他。之后一周秘密会面两三次,他送给“我”私藏的杂志以及从其他高等人士那里取得的护肤品。他与主教夫人已经同床异梦,几乎没有什么交流。之后的行房里两人之间似乎有了一种感情,“我”也开始“嫉妒”正牌夫人。

大主教暗示“我”目前的社会制度应当被推翻。他给“我”穿暴露的(其实是以前正常的娱乐场所装束)衣服,尼克开车载我们去俱乐部(他们私下里称为“荡妇俱乐部”)。这个俱乐部类似于以前的舞会。在俱乐部里“我”遇见莫伊拉。可见,大主教对这个荒唐的社会从内心深处是不屑的,但他并不明确地反对它,只是利用自己的地位来获取性方面的新鲜感。他厌倦千篇一律的受到束缚的女人,追求刺激和情色。他和“我”在“荡妇俱乐部”做爱,享受权力带来的特权感和新鲜感。

“我”和尼克“借精生子”,并有了孩子,但在通奸时被发现,此后尼克被处死。在处死他的仪式上,奥芙格伦知道他是地下组织的成员,因此帮助他更快地死亡以获取解脱。

\subsubsection{书评}
这本书在描述一个原教旨宗教化的社会,有意无意,这样的小说都会拿来和《1984》相比。然而,与后者极力渲染的那种压抑和恐怖气氛相比,本书在描述这种类似的压抑性气氛则是失败的。试想,一个靠着宗教管制的社会里,人们是不是要反抗?如果反抗,那么人们就会产生作者那样或多或少的反叛思想,这样这个社会就不会稳定存在,一定存在着导致它灭亡或瓦解的明面上的势力,甚至它治下的顺民也会时不时去反抗它,从而冲淡这种思想上的钳制。然而,本书的矛盾便在于此:这种治理方式竟然是稳定的。这显现出作者阿特伍德对真正的极权社会和政教合一社会缺少了解,只是把这种社会的一部分特征与北美目前的社会相衔接,创造出一种想象的社会形态。真正的极权社会(如苏联、北朝鲜)和政教合一社会(如穆斯林国家)的社会比本书的社会要丰富多了,它们存在着更大程度的合理性,以至于这种思想被任何一个人不同程度地接受;可能会存在反对,但这种反对也是在既定的框架下与官方思想形态同构。即使是《1984》和《美丽新世界》那样的社会里,大家也是可以回忆“革命以前”的美好生活的。可惜作者没有书写出这种丰富性,使得本书的社会描写过于单薄无力,缺少共鸣。

本书的故事是多线穿插进行的:现在的、“我”在感化中心时的、“我”与卢克和女儿的、“我”与母亲的故事……

作为主角的“我”具有大量的心理描写,事无巨细。“我”讨厌和憎恨这样的社会,但竟然在这个社会里生存良好。“我”观察力敏锐,别人的一举一动都逃不过法眼,同时对他人的心态也是精准地捕捉,这削弱了第一人称叙事带来的主观性视角下的可信性。作为比较,乔治·马丁的《冰与火之歌》主观视角下的人物则更加可信和生动。作者的笔力明显不明驾驭这种复杂的社会形态。另外,作者视野狭窄,“我”的性格似乎过于凉薄,让人爱不起来。

反乌托邦的setup,明显借鉴了《1984》。普通欧美人对极权社会的想象,总带有那么些自由主义者的想当然。基列共和国以严格的基督教教义来运行,但下至马大和使女,上至大主教,都对这种思想钳制虚与委蛇,表面热烈拥护,暗地里挖墙脚。我无法想象这样的政权能够稳定存在,在人人反对的情况下,借助军政府得以进行的教团革命根本不可能发生。

不是我搞歧视,目前看来女性作家对社会宏观运作的把握相较男性作家还是差了很多,光有政治观点并不足以写就真实感优秀的架空小说。奥威尔参加过左翼革命,所以能写出震撼人心的共产主义极权;乔治·马丁做了多年编剧,熟读世界历史(特别是英国历史),对蒙古和欧洲的历史也很熟悉,才能写就残酷诡谲的政治斗争。而作者阿特伍德的水平,基本就是全职家庭妇女转行作家,对于小说写作比较业余。

在自由主义盛行的北美,除非发生巨大生存危机(如同名剧集里的核污染),否则极权主义者不可能上台。

作者叙事功底太差,对长篇故事的统摄力不足,多线交叉不算新颖,也没发挥出多线并进的威力。

碎碎念式的意识流削弱了试图表现残酷现实的努力。观点盖过情节,没什么阅读乐趣。

评分:6/10。
