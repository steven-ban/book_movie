\subsection{《宿命》}

标签: 推理 \ 日本 \ 辛亥革命

作者:【日】陈舜臣

\begin{quotation}
所有的事情最终要么被埋进土里,要么被封印在皱纹里。
\end{quotation}

这是一本常见的“探寻”式的推理小说,就像挖土里的文物一样:文物就在那里,但是挖的过程是逐步的。这本书的独特之处,或者说与其他日本推理小说不一样的地方,在于这个“文物”与发生在辛亥革命前后革命党在日本的革命活动资金贪污挪用事件相关。

1910年革命党人吴练海在日本的花隈爱上了一个妓女,于是拿日本国内筹集的革命资金为她赎身并结婚去了中国,这中间受到了一个叫叶村康风(妓女的哥哥,两人之前没见过面,叶村康风就是为了找妹妹才去的)的帮助(叶村康风承认是自己贪污了钱,算是帮他们揽了下来),叶村康风去了南洋并死在那里,一个叫叶村鼎造的人知晓了这件事,多年以后他的儿子叶村一郎、儿媳甚至他们的女儿顺子也知道了这件事,认为自己有权利继承吴练海的遗产(这三观啊!),三人便有了一个计划——冒充是叶村康风的后人去接近吴练海及其遗孀的身边,骗取他们的信任并利用他们要报恩的心态得到大笔钱财;叶村康风还有一个小儿子叶村省吾(主人公),他不知道这件事,但在哥哥和嫂子的欺骗下一直以为自己是叶村康风的后人;多年以后吴练海又回到日本并死亡,期间嫂子查到他的资料;哥哥一郎病危,省吾受他们欺骗去调查“父亲”贪污的事件,要为父亲“平反”,于是到了京都,遇见了以后是自己女朋友的三绘子,并一起调查;这期间他受公司委托保存了一种化学原浆,但受人陷害,险些被钢管砸中、被毒死;他查到了吴练海的遗孀,那个艺伎,说自己是叶村康风的后人,得到她的信任,并被确立为继承人,期间还粉碎了管家要杀死自己的阴谋,管家和省吾公司的会计一起要试图被立成继承人,并知道整件事,前面设计害死省吉的事就是他们一起干的;整个过程中,省吾慢慢知晓了整件事,最后的他不仅获得了遗产并用于侄女顺子的成长,还获得了公司老总的器重并将女儿三绘子许给了他。

从整个故事的构思来看,可以说是非常俗套。从推理和叙事节奏上来看,作者的文笔很一般,节奏把握有点差,详略不当。从故事内核来看,无论是哥嫂一家,还是省吾自己,都把继承吴练海的钱当作心安理得,可以说是三观毁坏,作者竟然还把他当作单纯正义的大好青年,不知道心里想什么!正常的道德下,这笔钱根本不是吴练海的,而是与中国革命前途有关,是属于国民党或者中国人民的,应当还给中国人。这一点上,我与这位\url{https://book.douban.com/review/12646230/}的看法一样:叶村康风和吴练海是革命队伍里的败类,是中国人里的败类,即使真相大白,这两人也一点都不冤枉。整个故事内核,都是幽暗的人性之恶的故事被掀开一角再盖上的事情,日本的一些国民性,在这本书里展现得淋漓尽致。总之,这类故事,仅仅是一种休闲与谈资,上升不到严肃文学的高度。

评分:2/5。