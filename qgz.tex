\subsection{《潜规则》}

\subsubsection{标注}
帝国制度轮回十余次而基本结构不改,根本的原因,是不能形成冲出农业文明的力量。

官僚代理集团对代理人私利的不懈追求。最高统治者无力约束这种庞大的私下追求,弱小分散的小农阶级又无力抵抗各级权势集团整体或个体的巧取豪夺,于是就有了潜规则体系对儒家宣扬的均衡体系的替代,就有了王朝更替和治乱循环。王朝更替是帝国制度对过度失衡的自我校正机制。

帝国制度则不然。它是复杂形式的单一暴力-财政实体,各种资源集中在顶端,中层则由官僚代理人构成的支架代替了贵族领地的巨石,基层是一盘散沙般的小农。这种结构可以比喻为金属管材建构的井架,动力在顶端,资源在基层,两端之间的钢管架构就是负责上传下达的各级官僚代理人。由于破除了世袭的等级制贵族政体,对各级行政官员的选择范围从贵族扩展到平民,选择标准也从血统转向称职。

在潜规则的生成过程中,当事人实际并不是两方,而是三方:交易双方再加上更高层次的正式制度代表。

潜规则的定义:
\begin{enumerate*}
	\item 潜规则是人们私下认可的行为约束; 
	\item 这种行为约束,依据当事各方的造福或损害能力,在社会行为主体的互动中自发生成,可以使互动各方的冲突减少,交易成本降低; 
	\item 所谓约束,就是行为越界必将招致报复,对这种利害后果的共识,强化了互动各方对彼此行为的预期的稳定性; 
	\item 这种在实际上得到遵从的规矩,背离了正义观念或正式制度的规定,侵犯了主流意识形态或正式制度所维护的利益,因此不得不以隐蔽的形式存在,当事人对隐蔽形式本身也有明确的认可; 
	\item 通过这种隐蔽,当事人将正式规则的代表屏蔽于局部互动之外,或者,将代表拉入私下交易之中,凭借这种私下的规则替换,获取正式规则所不能提供的利益。
\end{enumerate*}

左良玉的兵一半要算群盗,甚是淫污狠毒。

私派比正赋要多。[73]

红包书记说的逢年过节送红包,还有利用生日送礼,在清朝的术语叫“三节两寿”。这个词通行全国。“三节”是指春节、端午和中秋,“两寿”是指官员自己和官员夫人的生日。现在领导干部出差收授的红包,在清朝叫“程仪。”请官吏办事送的红包,在清朝叫“使费”。请中央各部批准什么东西,递上去的红包叫“部费。”还有几十种名目,譬如告别送别敬,冬天送炭敬,夏天送冰敬或瓜敬,向领导的秘书跟班送门敬或跟敬,等等。

红包书记说的逢年过节送红包,还有利用生日送礼,在清朝的术语叫“三节两寿”。这个词通行全国。“三节”是指春节、端午和中秋,“两寿”是指官员自己和官员夫人的生日。

讲官吏与老百姓的关系:《身怀利器》、《老百姓是个冤大头》、《第二等公平》。 讲官吏与上级领导包括皇上的关系:《当贪官的理由》、《恶政是一面筛子》、《皇上也是冤大头》。 讲官场内部的关系:《摆平违规者》、《论资排辈也是好东西》。 把几种关系混在一起讲:《新官堕落定律》、《晏氏转型》。 总结:《崇祯死弯》。 目录就是按照这个结构排的。

\subsubsection{书评}
以我的理解,潜规则的存在,是因为社会本身是一个丛林社会,是一个利己主义横行的社会,公平契约精神无人遵守,所有人只能主动扩大自己的利益才能防止自己的权益受到损害,而作为国家制度的整套社会制度设计并不能从微观上防止这种丛林社会的运行,只能以一种低成本的儒家式的精神规范和相对弱小的中央朝廷来进行社会治理。因此,从整体来看,国家权威和制度对于这种丛林社会是无力和低效的,特别是作者吴思反复引用的明清社会的各种欺诈百姓和官场陋规,这是农业社会的动员能力低下导致的。

如何破解“潜规则”?我认为主要在于让承担法治、宪政和民主精神的国家权威内化到个人,落实到各级行政上去,特别是各级民主一定要开展,自下而上严格监督行政者的行为,防止越界。