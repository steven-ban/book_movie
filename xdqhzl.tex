\subsection{《X的奇幻之旅》}

标签: 数学 \  科普

作者:【美】Steven Strogatz

\subsubsection{一些标注}

\begin{itemize*}
 \item \emph{代数基本定理:任何多项式的根一定是复数。}
 \item 排座问题:一个足够大的电影院,每两个人相邻,其他人不能与这两个人相邻,如果只允许两个人同时坐,并且不允许换座,最后的空座率是多少?答案:$\frac{1}{e^2} = 0.135\ldots$。
\end{itemize*}

\subsubsection{书评}
本书是一本关于数学的科普读物,分为不同的章节,分别介绍初等到高等数学中不同的概念和应用,内容浅尝辄止,很容易读懂。

数学的科普相对于物理学的科普而言显得很不足道。普通的受过高中教育的公众,能够说出相对论、量子力学和宇宙大爆炸,然而却对概率论、贝叶斯统计、拓扑、场论闻所未闻。大批量的民科都在”钻研“狭义相对论和量子力学,却对最简单的微积分都不懂。这种情况,在中美都相当普遍。美国的Discovery以及英国的BBC都对医学、物理学、宇宙学制作了大量的科普短片,但却很少见到关于数学的片子。

中国人善于计算,算术都学得好,美国人在这方面就差得很多。然而除此之外,中美普通民众对现代数学的发展甚少了解,认为数学与自己无关。

在这种情况下,本书能够向公众介绍粗浅的现代数学,应该还有很有积极意义的。然而,我很怀疑本书的销量。虽然身处21世纪,但普通人视科学为某种巫术甚至是杂技,自己只是观众,宁肯听信大力丸、长生不老药、保健品,也不相信正规医院的诊查结果。普通人对科学的敬而远之的态度,可能很大程度上来自于中学时代对抽象的逻辑思维的抵触,那些抽象的内容带来的成绩差以及随之而来的家长和老师的斥责、冷漠、白眼等等,内化为一处对逻辑的排斥感,成为长久的记忆。其实,人类的大多数思维方式,还是处于一万年前丛林时代的本能水平上,想要突破这种本能,需要长时间的训练,这对于大部分人而言是困难的。

然而,对我而言,这本书的内容还是太粗浅了。相对于吴军博士的《数学之美》,这本书对大多数内容都是浅尝辄止,只起到一种导言的作用。如果你是理工科的本科生或拥有这以上的教育经历,后一本书可能更让人觉得酣畅淋漓。

评分:2/5。
