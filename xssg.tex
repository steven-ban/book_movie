\subsection{黎东方《细说三国》}
关羽就引了汉水的水,灌在樊城城墙之外(方法是:(一)把汉水下游堵住;(二)绕着城墙,再造一圈土墙;(三)引水进入这两墙之间)。

几十年前,笔者曾经在巴黎请教过袁世斌(冠新)先生:“什么样的人,才可以办大事”’袁先生说:“脑筋清楚,就可以办大事。”我又问:“怎么样的脑筋,才算得是清楚?”袁先生脱:“清楚,就是有条理:懂得提纲挈领,把事情分出一个大小先后。” 诸葛亮读书“观其大略”,可能便是如袁先生所说:研究出事情的大小先后,及其处理的方法。

孙权一生,在早年之时英明,在晚年十分糊涂。他早年之所以有英明的表现,我们不能不归功于张昭、顾雍二人。

古语说:“得师者王,得友者霸”,倘若无师无友,或目空一切,自以为天下无人可及,而不屑以任何人为师为友,那就不仅不能王,不能霸,可能会亡。

这延津从宋朝起,代替酸枣成为县的名称,位于今日黄河之北,在开封的西北,偏北,中牟的正北,偏东。今日中牟与延津之间的黄河,在当时是济水。济水发源于济源县,流向山东利津

(刘备)他之所以获得这许多人才的爱戴,是由于他秉性真诚,习惯于对朋友推心置腹,无话不谈,先向朋友表露了无保留的信任,于是就换得了朋友们对他的信任。

天下的事,应与天下人共谋之,至少应访求天下之头等人才而共谋之,

9.(黄巾军) 它之所以失败是由于领导人物之不学无术,既没有对于当前客观环境的正确了解,又没有对于未来的理想社会与理想政府的构想,更不曾聚集或培育军事的与政治的干部人才。

所谓“三权分工”,是丞相与太尉分治文武之事,御史大夫专管监察之事。