\subsection{《银河帝国》}
\subsubsection{主要情节}
\paragraph{基地三部曲}

\paragraph{基地前传}

\paragraph{机器人五部曲}

\paragraph{外篇:我,机器人}

\paragraph{钢穴}

\paragraph{裸阳}

贝莱奉命到索拉利星侦破机器人杀人案,遇到死者妻子嘉蒂亚,……

\paragraph{曙光中的机器人}

贝莱奉命到奥罗拉星侦办人形机器人詹姆斯停摆案件,被告为其设计者法斯陀夫博士,因为只有他才有足够的知识让人形机器人停摆。嘉蒂亚来来到奥罗拉后和法斯陀夫博士是邻居,詹姆斯是服侍她的,她把詹姆斯当成“性伴侣”。同时,法斯陀夫博士的对头——阿玛狄洛的指使人使詹姆斯停摆,试图借此攻击法斯陀夫(只有法斯陀夫有能力去做这种事情)。阿玛狄洛试图消灭贝莱,但后者借助机器人丹尼尔和吉斯卡的力量挫败了前者的阴谋。

吉斯卡具有重要的能力:看透人的心灵并在一定程度上改变人的想法。

\paragraph{机器人与帝国}

两百多年后,地球在贝莱和法斯陀夫的合力下开始向外太空移民,这些人称为“银河殖民者”。银河殖民者控制了很多星球,这给了太空族很大压力,奥罗拉上的阿玛狄洛试图在地球上建立基地,改变地球内部的放射性,从而逐渐使地球成为一个“死星”,借此消灭地球人和太空殖民者。但丹尼尔和吉斯卡通过控制形势和人的心理,挫败了他们的阴谋。

在这个过程里,丹尼尔悟到了“第零法则”,认为机器人应当对人类整体负责,最后吉斯卡停摆前将自己的能力给了丹尼尔。

\subsubsection{一些标注:}
我绝不要出卖我的独立性,以换取某种短暂的快感。”

“如果没有一群异于凡夫俗子的人,就不可能出现天才和圣者,而我不信异于常人的人都集中在好的一端,我认为一定有某种对称存在。

一个孤立体——孤立的个体——可能会说谎,因为他是有限的,所以他会感到恐惧。然而,盖娅是个具有强大心灵力量的行星级生命体,根本就没什么好怕的,因此盖娅完全不需要说谎,或是杜撰一些与事实不符的陈述。”

关于银河帝国三种走向的表述:
\begin{itemize*}
	\item “以川陀为蓝本所建立的第二银河帝国,将是一个父权式帝国,依靠算计建立,依靠算计维持,在无尽的算计中,它永远是行尸走肉。那会是个死胡同
	\item 以端点星为蓝本所建立的第二银河帝国,将是一个军事帝国,依靠武力建立,依靠武力维持,最后终将被武力摧毁。它会是第一银河帝国不折不扣的翻版,
\end{itemize*}
银河中搞不好有百万种智慧生物,却只有一种是扩张主义者,那就是我们。其他的都安分守己地待在母星,隐藏起来……”

理论上,基地公民人人平等,可是出身于联邦原始成员的公民,却比其他世界的人更平等些。至于那些跟联邦之外的世界有渊源的人,则是所有的公民中最不平等的。

双手才是真正的工作界面,

假如第二基地确实存在,却又希望保住这个秘密,那么有一点是绝对肯定的。如果有谁认为它仍旧存在,并且和他人讨论这个可能,甚至在公开场合高谈阔论,闹到整个银河人尽皆知,那么他们一定会立刻用巧妙的手法,将这个人解决掉、铲除掉、消灭掉。

(第二基地)他们的干预尽可能做得精巧、间接和分散。

“你所谓的‘超自然’是什么意思?” “最明显的意思──相信某些实体独立于自然律之外,比如说不受能量守恒或作用量常数的限制。”

历史在在显示,我们无法从历史中学到任何教训。

(地球人和从地球出发的太空殖民者)我们共有三项优势:因为没有机器人,我们用自己的双手打造新世界;因为世代交替迅速,我们一直在求新求变;而最重要的是,地球这颗母星是我们的中心信仰。”

多半要归咎机器人!它们降低了人类的互赖性,填充了人与人之间的空隙。人类彼此间原本存在着自然的吸引力,机器人却将它阻绝,于是整个社会崩解成了一片散沙。

有可能群众人数越多,就越容易受到情感而非理智的影响。

群众显然要比个人容易操纵。

每一个被我强化的心灵,都会再强化附近另一个同质的心灵,接着周遭又会有更多的心灵受到它们的强化。

(奥罗拉的科学)在一个长寿的社会中,压力相对小得多。我们的科学家能用三到三个半世纪的时间,专心研究一个问题,因此逐渐有人认为,自己即使单打独斗,也有机会得到重大的进展。久而久之,就滋生出一种学术性的贪婪——想要自己独力完成某项研究,将科学进展的某个方面视为私产,宁愿眼睁睁看着整体发展慢下来,也不愿舍弃自己心目中的禁脔。结果,太空族世界的整体科学发展就真的变慢了,甚至到了难以超越地球的地步,虽说我们掌握了极大的优势。”

厄俄斯是古希腊的曙光女神,正如奥罗拉是古罗马的曙光女神。”

别人的哀痛总是容易被讲成人生哲理。”

退化”或许就该这么定义:让人很容易适应的事物。

机器人的尸体竟然比人类尸体更像人类,

而我们脚下的这座厄俄斯城,正是这个世界的行政中心。总共有两万人住在这里,因此不只奥罗拉,就算在整个太空族世界,它也是最大的城市。要知道,整个索拉利的人口加起来也只有那么多。”

(奥罗拉)它一天只有22个小时。” “应该说是22.3个传统小时。每个奥罗拉日有10个奥罗拉时,每个奥罗拉时有100个奥罗拉分,每个奥罗拉分又有100个奥罗拉秒。因此,一个奥罗拉秒大约等于0.8个地球秒。”

在一个接受机械劳工的经济体系中,机器人对人类的比例一律会直线上升,任何试图避免这个趋势的法规命令都是徒劳的。上升的速度虽然缓慢,可是永远不会停止。起初人口会显著增加,但机器人增加的速度却快得多。

对机器人下一个简单的命令,看着他乖乖服从,等于在强调自己是人类,而他只是机器人。

有些事物甚至凌驾于你脑中的那组正义之上。人类内心有一种冲动叫做慈悲,化为外在的行动则称为宽恕。”

有待测试的功能越重要、越基本,所需要的设备就越简单。这个道理同样适用于机器人,

(机器人的理解)“所谓的正义,以利亚,就是让所有的法律都发挥应有的效力。”

凡是在公家机关讨生活,个人能力永远比不上交际手腕来得重要,

我无法接受“如果知识代表危险,无知就是解决之道”这样的观点。在我看来,解决之道似乎是善用人类的智慧才对。人类不该拒绝面对危险,而应当学习如何化险为夷。

\subsubsection{一些感想:}
《银河帝国》里的计谋和政治实在是幼稚到家了,简直就是小孩子过家家的水平!另外叙事太直线化,人物太扁平化。阿西莫夫的文采太差了,比刘慈欣还要差,大刘最起码景物描写一笔带过却常常有传神之笔,阿西莫夫却只是干巴巴的叙事。

基地系列写作时间太早,到了机器人系列,阿西莫夫的文笔有了长足进步,以利亚·贝莱的形象就比哈里·谢顿丰满得多,也自洽得多。情节上,机器人五部曲更像是悬疑推理小说,读者的沉浸感会更强烈。

阿西莫夫在漫长的写作生涯中,应该是不断尝试新的技巧和手法的。阿西莫夫的特点在于喜爱通过人物对话来推动情节,优点是显得比较自然(只是相对显得),缺点是人物对话非常多,对人物和社会的情境反映太少,而后者正是科幻小说的魅力所在。科幻小说是生生造出一个世界,一个和现在不同的世界,仅仅通过人物的对话,对社会整体的反映实在太少。 

阿西莫夫在前期作品中对女人以及成人内容描述很少,似乎有读者对他反映,他也承认了这一点,于是在《裸阳》中出现了嘉蒂亚的裸体和男女接触的心理,出现了男女SEX造人的描述,到了《曙光中的机器人》更是把嘉蒂亚的性心理赤裸裸地描述出来,在《机器人与帝国》中她和以利亚甚至太空约炮。想想一个老头子对自己的不断“超越”,还是挺有意思的。

在太空族和地球人的争霸中,作者提到,太空族不受病菌感染,同时有机器人服务(灭菌并不必然导致长寿命,阿西莫夫明显并没有考虑人自身的疾病,比如免疫系统、细胞衰老等),寿命很长,因此社会发展滞后,缺少创新和冲劲。比如太空族的科研工作者都是各自为站,缺少交流,一个人研究出了人形机器人或看透心灵的机器人,别的研究院合多个之力仍然不能做出来。相反,地球人人口数量多,寿命短,因此有冲劲,创新能力强。我对这种社会学的判断持怀疑态度。