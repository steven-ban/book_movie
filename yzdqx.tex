\subsection{《宇宙的琴弦》}
\subsubsection{感想}

这是我11年之后重读这本书。

十一年前,这本书属于老赵,我们刚认识,但当我饥渴地在自习课上并在后排读完这本书后,我们成了好朋友。

那时候我还不懂物理学,只看过粗浅的相对论,这本书为我打开了一扇窗户,让我看到物理学自从广义相对论和量子力学以后的瑰丽画卷。书里的概念我只是一知半解,很多作者费尽心机试图用通俗语言来解释的物理原理和物理概念也丝毫没有弄清楚。我想对普通大众甚至物理学爱好者而言,这本书的内容都显得太艰涩。

幸亏我大学学习了物理学,回头再看这本书,简并自旋玻色子,统计平均涨落本征态,我都懂,可是对于弦理论最核心的原理和结论,依然无法画出轮廓。

我得出结论:弦理论太抽象太难,拿通俗语言解释,简直是白费劲。

或许我可以找篇弦理论的综述看看,可能还明白些。

\subsubsection{标注}

宇宙间的一切事物总是以一个固定的速度——光速,在时空里运动。

没有科学的“教育”,只是培养信仰,而不是教育。没有受过科学教育的人,只能称为受过训练,而非受过教育。