\subsection{《吕著三国史话》}

\subsubsection{标注}
1. 夫惟无意于功名者,其功名乃真。

2. 封建时代,是有其黑暗面,也有其光明面的。其光明面安在呢?公忠体国的文臣,舍死忘生的武士,就是其代表。这两种美德,魏武帝和诸葛武侯,都是全备了的。他们都是文武全才。

3. 所以心计过工,有时也会成为失败的原因的,真个阅历多的人,倒觉得凡事还是少用机谋,依着正义而行的好了。

4. 军队是有系统的,尤其是封建时代的武人,全是效忠于主将的,是个对人关系。

5. 江东就是江苏省里长江以南的地方,现在称为江南,古人却称为江东,而把对江之地,称为江西。古人所说的江南,是现在湖南地方。

6. 军事的胜败,固然决于最后五分钟,也要能够支持到最后五分钟,才有决胜的资格哩。

8. 当时的斗争,遂成为冀州的袁绍、兖州的曹操、荆州的刘表站在一条线上,幽州有实权的公孙瓒、寄居荆州境内的袁术和豫州的孙坚、徐州的陶谦站在一条线上的形势。

9. 政治上最怕的是纲纪废坠。纲纪一废坠,那就中央政府的命令不能行于地方,野心家纷纷乘机割据,天下就非大乱不可了。

10. 朝代的更换,便是这一个天帝的子孙,让位给那一个天帝的子孙。这就是所谓“五德终始”。

11. 至于土,则古人每以自己住居的地方为中心,自然只好位置之于中央;其次序,自然在木火和金水之间了。

12. 把五行来配五方和四时,则木在东方,属春;火在南方,属夏;金在西方,属秋;水在北方,属冬。

13. 古人所祭的地,只是自己所居住、所耕种的一片土地。这便是现在的社祭。所祭的天,也只是代表一种生物的功用。农作物是靠着四时气候的变化,才能够生长成熟的。古人看了这种变化,以为都有一个天神在暗中主持着,所以有青、赤、白、黑四个天帝,青帝主春生,赤帝主夏长,白帝主秋收,黑帝主冬藏。春生、夏长、秋收、冬藏,都是要靠土地的,所以又有一个黄帝,以主土地的随时变化。

15. 亲戚分为两种:一种是父系时代自己家里的人,后世谓之宗室。一种是母亲家里或者妻子家里的人,后世谓之外戚。 伦理上的训条只是一句空话。

17. 我们现在,亲戚二字是指异姓而言,古代却不然。戚字只是亲字的意思。凡是和我们有血统上的关系的,都谓之戚。

18. 宫刑,当隋文帝时业已废除。自此以后,做内监的人,都是自行阉割的。

19. 在古代,必须用兵器伤害人的身体,使之成为不能恢复的创伤,然后可以谓之刑。

20. 此等门客,皇帝名下自然也是有的,这便是所谓宦官。

21. 古人解释宦字,有的说是学,有的说是仕,的确,这二者就是一事。

22. 至于大学,其中颇有些高深的哲学,然而宗教的意味是很浓厚的。

23. 古代的学校亦分为大学小学,所谓小学,只是教授一些传统的做人道理以及日常生活间的礼节,如洒扫应对进退之类。又或极粗浅的常识,如数目字和东西南北等名称之类。根本说不上知识,更无实际应用的技能。

\subsubsection{书评}
这是吕思勉大师关于三国问题的论文合集,文白夹杂,但除了附录里基本文言的几篇,前面主要介绍人物及事件的论文很容易读懂。

读史就要读大师著作,特别是在初学时,大师的观点、史料以及求索精神能够打下坚实的底子,这是鼠辈作书人所远远不及的。我在读钱穆前,也是杂学旁收,但自从看了钱穆的《中国历史政治得失》,很多问题被廓清,迷雾烟消云散,启发性很大。这本也是,想看三国历史,想去除《三国演义》里的一些错误观念和事实,这本书也是很值得一看的。

读史的意义在两端,一是接触更多组织起来的史料,加深对过往人间的认识,知道除我及现实之外,人类的其他故事;二是加强思辨能力,知道但凡一件事,无可也无不可,人类作为个体,可以做出自己的选择,同时亦知人类以往的积淀对今天诸事物的影响,因此不但能看清表面,也能看清背后的演变。刚日读经,柔日读史,其实没有那么绝对,无论是是处在逆境还是顺境,都能从历史中看到成功失败,都能对自己产生启迪与指导。

具体到本书,吕思勉对三国主要人物事件都做了点评,就史料来说,三国的史料只有那么多,治史之人恐怕都烂熟于心,想解读出不同的角度,恐怕还要从“事理”和古代基本常识的角度来看了。在此方面,很多没有治史科学精神的人,只能流于猜想和过度解读,而浸淫国学多年的人,往往可以从多种角度翻出新花样来,给人一种耳目一新的感觉。这本书就多这种普通古代生活常识,读之也能增长见闻。

因此如果有人在读了《三国演义》后想进一步了解三国历史,一定要读这本书不可。满分五分的话,这本书值得五分。