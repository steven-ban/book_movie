\subsection{《霍乱时期的爱情》}

标签: 文学 \  小说\  马尔克斯

作者:加西亚 马尔克斯

\subsubsection{人物}

\begin{longtable}{p{0.2\textwidth} | p{0.35\textwidth} | p{0.4\textwidth}}

    \caption{《霍乱时期的爱情》人物表} \\
    \hline
姓名 & 特点 & 事件 \\
\hline
\endfirsthead

(接上表) \\
姓名 & 特点 & 事件 \\
\hline
\endhead

\hline
\endfoot

赫雷米亚·德圣西莫尔 & 摄影师,50岁 & 氰化物死亡 \\
胡维纳尔·乌尔比诺·德拉卡列医生 & 80岁,备受爱戴的医生,赫雷米亚的好友 &从法国学成归来后制止本省的霍乱,因象棋与赫雷米亚结识,资助他开照相馆 ,因抓鹦鹉而摔下梯子死亡 \\
  & 赫雷米亚的情人,四十多岁,住在奴隶街区中 & 预先知道赫雷米亚的死 \\
费尔明娜·达萨 & 乌尔比诺医生的妻子,72岁 & \\
拉希德斯·奥利维利亚医生 & 乌尔比诺医生的学生,年过半非,略带女人气,医术精湛 & \\
阿敏塔·德昌普斯 & 奥利维利亚医生的妻子 & \\
乌尔比诺·达萨医生 & \\
马可·奥雷里奥 & 医生,乌尔比诺医生的儿子,平庸,50多,无儿无女 & \\
奥莫利娅 & 乌尔比诺医生的女儿,到了更年期 & \\
蒂戈娜·帕尔多 & 乌尔比诺医生家的老女仆 & \\
弗洛伦蒂诺·阿里萨医生 & 乐于助人,举止稳重,75岁,加勒比河运公司董事长 & 年轻时喜欢费尔明娜 \\
特兰西多·阿里萨 & 弗洛伦蒂诺的母亲 & \\
皮奥第五·罗阿依萨先生 & 弗洛伦蒂诺的生父,船运公司老板 & \\
洛达里奥·图古特 & 电报员,弗洛伦蒂诺的老师,德国移民 & \\
埃斯科拉斯蒂卡 & 费尔明娜的姑妈,抚养她长大 &费尔明娜的恋情被父亲发现后,她被哥哥送走,失联三十年后死于麻风病 \\
伊尔德布兰达 & 费尔明娜的表姐,埃斯科拉斯蒂卡姑妈的女儿,两人要好 & 爱情被家人拆散,独自去见阿里萨探知情况,活到近百岁 \\
洛伦索·达萨 & 费尔明娜的父亲 &之前卖骡子为生,向上爬的希望都放在女儿身上,希望她嫁得好 \\
费尔明娜·桑切斯 & 费尔明娜的母亲 &已去世14年 \\
布兰卡夫人 & 乌尔比诺医生的母亲 & \\
马可·奥雷里奥·乌尔比诺医生 & 乌尔比诺医生的父亲,当地医生 & 死于霍乱 \\
弗兰卡·德拉路斯嬷嬷 & 德国人,费尔明娜的教会学校的老师,严厉且假慈悲 & 受乌尔比诺医生请求撮合他和费尔明娜 \\
罗萨尔芭 & 阿里萨去远方城市的船上的游客,一个母亲 & 阿里萨怀疑她是夺去他童贞的人,中途下船 \\
拿撒勒寡妇 & 28岁,生育过三个孩子 & 家被叛军炸掉,跑到阿里萨家里,与阿里萨行房,后开门接客 \\
奥森西娅·桑坦德尔 & 年近五十,性欲旺盛 & 阿里萨的情妇,“制伏”阿里萨,让阿里萨欲罢不能,两人约会时家里被偷了个干净 \\
莱昂十二叔叔 & 阿里萨的叔叔,航运公司老总  & 给了阿里萨工作 \\
莱昂娜·卡西亚尼 & 妓女,黑人 & 阿里萨帮她在莱昂十二叔叔的公司里找到工作,她一路帮他在公司权力斗争中获胜,是为了报恩,两人没有上床 \\
萨拉·诺列加 & 喜爱诗歌,大阿里萨十岁 & 与阿里里萨因诗会结缘,年轻时婚礼被放鸽子,因为诗歌落选而咒骂费尔明娜,但事实上与费尔明娜无关,她只是宣读获奖名单的人,因此与阿里萨分开,结束五年的肉体关系 \\
\end{longtable}

\begin{quotation}
(乌尔比诺医生)他心里明白,自己并不爱她。同她结婚是因为喜欢她的高傲,她的严肃,她的力量,也因为自己的一点儿虚荣心,但当她第一次吻他时,他确定,没有什么障碍能阻止他们建立一份完美的爱情。在那第一个晚上,他们什么都聊了,一直聊到天亮,就是没有谈到爱情,以后也永远不会谈到它。但从最后的结果来看,两个人谁都没有做错。
(阿里萨)他一直都表现得就像是费尔明娜·达萨彻头彻尾的丈夫:肉体上不忠,心灵上却死心塌地;不停地努力摆脱自己所受的奴役,却又从不让自己的背叛给她带去痛苦。
\end{quotation}


\subsubsection{情节}
\begin{enumerate}
    \item 费尔明娜一家来到这个城市,遇见阿里萨,两人通信互诉衷肠。洛伦索 达萨知道后为拆散他们,送走了费尔明娜的姑妈,带费尔明娜去外国旅行。他们没有中断通信。回到城市后,费尔明娜见到阿里萨,突然和他分手。
    \item 父亲死于霍乱, 28岁的乌尔比诺医生从巴黎回到这个城市,带着一些不甘心。他运用自己的医学知识,隔离病源,治疗了很多病例,并累积了声望
    \item 乌尔比诺医生为费尔明娜看病,爱上了她,向她写信求爱 ,犹豫过一段时间后费尔明娜同意两个成婚。
    \item 阿里萨得知费尔明娜要嫁给乌尔比诺医生时意志消沉,母亲托叔叔给他找了远方城市的电报员职位。临行前他在费尔明娜的窗外拉了一次小提琴。在去往远方城市的船上他被一个女人夺去童贞。在船上阿里萨承受的相思带来的沮丧和痛苦,到达港口后决定乘船返航。
    \item 费尔明娜与乌尔比诺医生结婚,去巴黎度蜜月,怀上了孩子。两个人内心并不是真心相爱,但觉得对方“合适”。阿里萨看到怀孕的费尔明娜,内心想要夺回她,于是去努力工作试图功成名就;同时她又无法忘记失恋带来的痛苦,经常找女人一夜情。阿里萨心里充满着爱,写商业文件都是浪漫的语气,于是给人代写情书。
    \item 在五十多年的时间里,阿里萨一直在“重新获得”费尔明娜的幻想中度过。乌尔比诺医生死后,他再次与费尔明娜通信,慢慢地与她接近,安排两人在一条内河航运的船上旅行。两人最终在一起,一起漂流在船上。
\end{enumerate}

\subsubsection{书评}

这本书的阅读体验相当之好。马尔克斯一如继往的娓娓道来,平和细腻,文字里闪动着生活的激情。故事很简单,就是一段五十多年、横跨19和20世纪的苦恋(单相思)。苦恋的主角,是阿里萨,他在十多岁时遇见到商人之女费尔明娜,为她拉小提琴,与她通信。而费尔明娜,那时对爱情仅是一种朦胧的、懵懂的感觉,他喜欢阿里萨,但是在生活的画卷徐徐展开(特别是长期旅行后)时,再次看到阿里萨,看到那个他没有见过几面、却又十分熟悉的瘦小男孩时,却发现他和“想象”中不同,于是拒绝了他:

\begin{quotation}
她回过头,在距离自己的双眼两拃远的地方,她看见了他那冰冷的眼睛、青紫色的面庞和因爱情的恐惧而变得僵硬的双唇。他离她那么近,就像在子时弥撒躁动的人群中看到他的那次一样。但与那时不同,此刻她没有感动爱情的震撼,而是坠入了失望的深渊。在那一瞬间,她恍然大悟,原来自己对自己撒了一个弥天大谎。她惊慌地自问,怎么会如此残酷地让那样一个幻影在自己的心间占据了那么长时间。她只想出了一句话:“我的上帝啊!这个可怜的人!”弗洛伦蒂诺·阿里萨冲她笑了笑,试图对她说点什么,想跟她一起走,但她挥了挥手,把他从自己的生活中抹掉了——

“不,请别这样。”她对他说,“忘了吧。”
\end{quotation}

这种失望的感觉,正是她编织的未来生活与眼前的“可怜”的男孩如此不同。少女的她,对未来生活的设想,并不是仅仅是一些魂牵梦萦的爱情,还应该是生活的满足感和掌控感。她对他的认识,幻想多过实际,而且也没有与他共同生活的欲望。费尔明娜随后选择了一个完美的丈夫,留学巴黎回国的乌尔比诺医生。他生活富足,有良好的教养,风度翩翩,对生活有睿智的洞察力。两人的生活,就是一个标准的幸福婚姻,当然,也有着幸福婚姻标准的走向。她对自己的丈夫,并没有对阿里萨那种激情和幻想,但她清楚地知道,丈夫精准地能给她想要的生活。他们的婚姻也经历着平淡、乏味,甚至丈夫差一点就出轨(但没有真的出轨,这正说明他是完美的丈夫),费尔明娜随后陷入漫长的迷茫和无所事事中。她是圆满婚姻浇灌出来的温室花朵,但她不知道,自己的内心仍然有一块处女地为那段无疾而终的初恋留下可能。在漫长的半个世纪的时间里,她偶尔可以见到阿里萨,但极力阻止自己回想过去,以及想起他。她与他见面,总是平平淡淡,似乎两人仅仅是认识,但没有任何故事。她在逃避,逃避青春期时的那个选择的错误(并非婚姻意义上的)。

阿里萨是一个热情的、包含有爱情的热心的人,写公文都是谈情说爱的色彩。费尔明娜拒绝了他,给他留下了沮丧、伤心、不甘和挥之不去的痛苦。他试图忘记费尔明娜,但只是徒劳。他通过和不同的女人一夜情(但没有生孩子)来浇灭自己得不到费尔明娜的痛苦,但那段爱情却越发生长。他幻想乌尔比诺医生死掉,自己可以“上位”。他遇见到自己喜欢、动心、懂自己的女人,但终于没有把自己的心事说出口。他得到喜爱他的女人的帮助,成为叔叔公司的副董事长,并偶尔可以与费尔明娜在社交场合见面。乌尔比诺医生死去,他终于站在寡妇面前,说:“费尔明娜……这个机会我已经等了半个多世纪,就是为了能再一次向您重申我对您永恒的忠诚和不渝的爱情。”自然,费尔明娜严辞请他滚开,但是,晚上的费尔明娜想念的,不是自己的丈夫,而是这个被她狠狠压抑了半个多世纪的情人。

两个人在漫长岁月里的这种纠结、坚持是本书最为出彩的部分。作者试图向读者剖析,爱情与婚姻并不能划等号,以及如何不能划等号;爱情是一种生命的必然,即使垂垂老矣,也可以愈发醇厚闪亮;爱情并不是在一起,不是性,不是任何其他一种东西,它并不是必然伟大的,爱情只是爱情,就是空气和水只是空气和水,没有什么可以代替它;爱情,至少是阿里萨和费尔明娜的爱情,就是一粒种子,在贫瘠的土地上依然顽强(并不是当事人让它顽强,而是它本身就有了生命,本身就十分顽强)开花结果。作者在这里展现了一种激情,生命的、生活的激情,也是一种热爱,他认真地“记录”着爱情如何在两个人身上生长,如何嬗变,这种细致入微的摹画若无前面那种热爱和激情是无论如何也不能完成的。大多数人的婚姻可能是相似的,但这两个人的爱情则是独一无二的。读者能从这部小说里,得到一种信心,那就是爱情是存在的,是坚强的,并且并不是庸俗的风气能够吹熄的。社会对爱情,有着不同的规训,比如婚姻的要求,比如善恶的、道德的要求,甚至社会对爱情,也有不同的装饰和诱惑,比如很多人错误地把爱情与婚姻等同,分析出“不败爱情的N个秘密”,然而却是买椟还珠,忘记了爱情本身的魅力和力量。很多人把爱情打扮地庸俗、无聊、残忍。费尔明娜即被这种装饰和诱惑说服;自然,她把婚姻经营地很好,而且最终也收获了阿里萨的爱,但在这个故事里,真正的幸福者是阿里萨,读者最为倾心的也是阿里萨。费尔明娜的选择,不过是告诉我们,爱情不会被任何其他的东西伪装,它最终会露出真容。

之后的事情就水到渠成了。阿里萨与费尔明娜通信,乃至定期到她家与她见面,她不拒绝,甚至成了两个人的习惯。子女反对,她则做出剑拔弩张的架势捍卫自己的爱情。最终两人在船上的约会,自然把本书和这件爱情导向了一个必然的终点。

这本书的现实场景,仍然是拉美社会,但这次放在了19世纪末20世纪初,“霍乱”是这个一败涂地的魔幻社会里的流行病,这个社会破败、混乱、贫富分化,人们已经对痛苦麻木。在这样的社会里,生长着一段令人动容的爱情。

评分:5/5