\subsection{《寻龙诀》}
《寻龙诀》影评

这个改编把时间定位到1988年的美国和中国,距原著里三人最后一次盗墓(1983年?)已经有了四年之久。三个本想金盆洗手不再干盗墓勾当,但当天“庆祝”时过火,王凯旋喝多了抱着马桶痛,胡八一和杨雪莉也喝多了于是啪啪啪。事后胡八一和杨雪莉关系开始变得僵硬,胡八一想忘掉这事,杨雪莉痛骂胡八一“大混蛋”但想与生与死。大金牙得到日本人的赞助,想挖辽国公主暮找到“彼岸花”,王凯旋加盟,但出师不利,日本人是个狠角色,BOSS是个四川人被日本人收养,大概是继承了家产和公司,脑子里长瘤子活不了几年,于是想方设法找那个“彼岸花”来“连通生死”。胡八一和杨雪莉闻讯赶来试图劝说王凯旋金盆洗手(又一次?),但不幸也卷入其中,日本人要挟他们一起找到“彼岸花",三个人不得不再次结盟开始盗暮。

王凯旋之所以执著于寻找彼岸花,在于二十年前的一段深刻经历。1969年胡八一和王凯旋到内蒙古插队,都喜欢上了队里的知青丁思甜,但丁只青睐于帅气有才华的胡八一。知青队破四旧试图毁掉辽国公主暮,在草原上挖了石像被不明生物(类似蜂群或吸血蝙蝠)追赶,躲入地下洞穴,发现是日本人挖的地下工事,但里面日本人离奇全部死去,原来是当年日本关东军不知不觉挖到了辽国公主暮,触发了彼岸花(陨石)发出光芒,死亡后变成僵尸。当晚恰好彼岸花再次启动,日本僵尸复活,众人死伤无数,在逃亡中丁思甜、胡八一、王凯旋三人在电梯中丁思甜突然掉落,虽然胡八一拉着她但看到电梯被木块挡住,于是丁思甜不顾安危将木块踢开,自己也掉入火海。事后王凯旋悔恨不已,发誓一定要为丁思甜找到彼岸花。

盗墓过程中状况百出:众人被困循环洞穴、遇到连环”奈何“桥、日本人绑架杨雪莉……但都被胡八一的罗盘八卦一一破解。找到公主墓后彼岸花再次触发,洞穴里死亡的僵尸再次复活。胡八一在千钧一发毁掉了彼岸花,打碎湖底,几人爬入棺材内逃生。

这是一部标准的商业片,3D效果出众但不惊艳,演员演技尚可但并不出彩(脸谱化的设定也没有给演员多大施展空间)。

如果你是冲着看《鬼吹灯》原作的,那一定会失望了。首先人物和原著只是名字和大体性格相同,细致处差别很大,比如原著里沉着冷静的胡八一在电影里流里流气,原著里胸有城府心思缜密的杨雪莉在电影里成了感情受伤害的娇滴滴的小女生,原著里见财起意但生猛非常的王凯旋在这里成了拖油瓶,原著里江湖气十足眼界长远的大金牙在这里成了小丑……而这个所谓辽国公主墓的故事,在原著里当然没有,只是混合了《精绝古城》里辽国墓和《????》里的日本关东军军事基地而已。一众神神秘秘的日本人也是傻得不行,只会在镜头后立正站正。

更让人不解的是丁思甜的设定。三角恋的故事太狗血,以王凯旋之“刚烈“,很难相像还能和胡八一好好相处。

既然都是《鬼吹灯》的同人电影,那么和《九层妖塔》相比较是难免的。整体而言,《九层妖塔》偏离原作更多一些,但年代感更逼真(请无视报纸字体),姚晨的杨雪莉味道更正;《寻龙诀》商业气息更浓厚,画面人设更大气,但剧情还是偏弱。因此如果你是原著的铁杆粉,就不要去看了,看了会心痛。