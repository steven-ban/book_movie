\subsection{《上瘾五百年:烟、酒、咖啡与大麻的历史》}

\subsubsection{一些标注:}
新加坡等于是靠嗜抽鸦片的华人苦力在赡养,殖民政府19世纪的总收入有一半来自鸦片。
严重低估了三项事实:咖啡因类瘾品使用与上瘾之广泛,医疗以外使用烟类早期遭反对之激烈,瘾品用于安抚、控制、剥削劳力(不分是牲畜或人类的劳力)的方法种类之多。

如果没有火,根本不可能有古罗马的长颈盛酒瓶、烟斗、茶、泡水加热的碎麦芽(酿酒用)、提炼的鸦片,以及其他。控制自如的火乃是精神刺激革命所仰赖的原始科技。

我们生活的这个社会,其运作的原则就是要挑动人体内的每根神经,并且让它们维持在最高度的人为紧张状态,要把人类的每个欲望逼到极限,并且尽量制造更多新的欲望与人造的渴求,为的是要用我们的工厂、出版社、电影公司以及所有其他从业者制造的产品来满足这些欲望和渴求。”

刺激精神的物质可以帮助农民和劳工在不堪忍受的日子中苟活下去。
吸烟者代谢咖啡因的速度比不吸烟者快上50\% ,所以要频频续杯才能维持同样的提神效果。

\subsubsection{读后感}
1.人类的某些习惯,看似只是随意和肤浅的个体行为,实质上往往和巨大的社会变革相联系,抽烟喝酒、品茶喝咖啡,甚至吸毒,都和几百年的现代化进程息息相关。如果没有世界贸易的蓬勃发展,没有西方主导下的世界性的生产-贸易-消费市场的建立,没有资本家为了利润最大化的或明目张胆或潜移默化的诱导,英国人肯定不会每天来上一杯下午茶,中国人也不会常常跑去星巴克,落魄的人不会拿威士忌一杯杯灌醉自己,嬉皮士也不会在大麻的刺激下飘飘欲仙。瘾品自古就有,不是什么新鲜东西,但只有到了近代和现代社会,它才会如此堂而皇之地闯入普通人的生活,堂而皇之到我们根本就对此熟视无睹。

2.人生苦短,并且常常充满苦难,如果没有瘾品的刺激,人生便如不加调料的食物那样食之无味。暂时的咖啡因、可卡因、吗啡、酒精、海洛因能让大脑忘却一切苦难并陷入虚无缥缈的快感中。但是,它们提供的快感让大脑难以忘怀,最终上瘾,产生依赖,就像人的舌头对盐、辣椒和味精的依赖一样。有些瘾品是危险的,有些则不那么有害,各国政府在税收和道德间选择了不同的措施,这是利益的妥协。