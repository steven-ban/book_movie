\subsection{《天才、狂人与梅毒》}

标签: 历史 \  艺术史 \  梅毒 \  疾病史 \ 医疗史

作者:【美】德博拉·海登

\subsubsection{何为梅毒?}
梅毒是一个曾经让人闻之丧胆的疾病,由于它是性病,往往与嫖娼有关,因此也让人们保持着奇怪的好奇心。传言咸丰皇帝就喜欢个出嫖娼因而得了梅毒。

梅毒起源于哥伦布“发现”新大陆。他和船员们(以及后来者)把欧亚大陆的一些传染病带到了美洲,杀死了美洲90\%以上的人口;作为美洲的“报复”和上帝的“惩罚”,与当地人滥交的船员们把梅毒带回了欧洲。此后,这种病随着战争和嫖妓而广为传播,往往身居高位的社会名流也得了这种让人羞耻的病。这种病的病因是螺旋体,可以依靠性交和生殖而传播,发病分为三个阶段:
\begin{itemize*}
	\item 初期:下疳
	\item 第二期:皮肤与黏膜出现创伤,全身出现多种炎症
	\item 第三期:螺旋体侵入身体内部结构:内脏、骨头与关节,出现痴呆、脊髓痨等,性格上偏执怪异,但创作能力上升,随后死亡。
\end{itemize*}

为了治疗梅毒,医生了想出了各种办法:用水银、砷、钾盐……这造成了患者的痛苦,而且根本无法根治。只有出现了青霉素之后,这种病才得到了很好的控制。而正是由于青霉素,使得出现第二期和第三期的病人基本不再出现,这降低了现在再去研究后期病情的“样本”。

梅毒在欧洲传播了五百年。19世纪后期,梅毒专家估计\emph{巴黎大约有15\%的人感染梅毒},1907年,茨威格指出\emph{20世纪初维也纳每10个年轻男子就有1-2位诊断出梅毒(通常是因为嫖妓)}。然而,由于有大量的“大人物”也得了梅毒,医生在写诊断书时往往只写症状而不写具体的病名,这也是为了“为尊者讳”。

从中可以看出即使是在20世纪,英法美均已经进入了工业社会,但这些位居人上的名人们也均是疾病缠身(脑膜炎、脊髓痨、中风等),大部分时间都是生活在痛苦之中。

从世界历史特别是印第安人的角度上来看,这也算印第安人对白人的一种报复。大航海触发了启蒙和科技革命,从而造成了西方的近代化,但同时也给他们带来了恶疾。这大概真的是一种报应吧。

\subsubsection{梅毒名人榜}

下面这些人很多其实并没有十足的证据表明得的病就是梅毒,他们往往表现出类似梅毒并发症的症状,因此并不是不易之论。正如作者在《结语》里写的那样,\emph{我们的工作是搜集线索,找出可资辨认而一再重复的模式,把问题留给大家讨论}。

\begin{table}[htpb]
\centering
\caption{梅毒明人榜}
\begin{tabular}{p{0.2\textwidth}|p{0.75\textwidth}}

人物 & 病史 \\
\hline
贝多芬 & 在年轻到老年的时间里展现出梅毒的各种症状:炎症、耳聋失聪、情绪不稳定,但未能完全确诊,各方观点不一 \\
舒伯特 & 25岁患梅毒,31岁死亡 \\
舒曼 & 自称“1831年,我是梅毒病患,以砷治疗”,1856年死亡 \\
夏尔·波德莱尔 & 诗人,可能在1839年11月或12月得了梅毒,当时他18岁,1867年死亡 \\
林肯 & 可能在1835或1836年在小镇上得了梅毒,之后有过服食“蓝色药丸”(包含水银)的经历,但语气不明显,爱戴他的人也反对这样的推论 \\
玛丽·塔德(林肯夫人) & 晚年痴呆并,可能得了梅毒,但证据不全面 \\
福楼拜 & 在巴黎上学期间得了梅毒(1840年左右),后接受治疗,有下疳,1880年死亡 \\
莫泊桑 & 20岁左右(1869年或1870年)得了梅毒,故意传染给女人,1893年死亡,生命中的最后三年病情恶化,出现幻觉,精神失常 \\
梵高 & 疑似梅毒患者,他弟弟西奥和高更都患有梅毒\\
尼采 & 1844年出生,可能在大学时代(1866年)得了梅毒,1889年神经失常,1900年死亡 \\
王尔德 & 同性恋,疑似梅毒患者,死于1900的9月 \\
阿尔·卡彭 & 1899-1947,年轻时就得了梅毒,确诊 \\
雨果·沃尔夫 & 1860-1903,音乐家,确诊 \\
阿尔蒂尔·兰波 & 1854-1891,可能在1887的得了梅毒,确诊 \\
都德 & 1840-1897,44岁时第一次出现脊髓痨,确诊 \\
伦道夫·丘吉尔 & 1849-1895,英国首相丘吉尔的父亲,英国议员,确诊 \\
奥古斯都·马奈 & 1832-1883,确诊 \\
儒勒·龚古尔 & 1830-1870,确诊 \\
海涅 & 1797-1856,德国诗人,1837年出现脊髓痨,确诊 \\
戈雅 & 1746-1828,46岁时疑似患病,未确诊 \\
恐怖伊凡 & 1530-1584,23岁可能首次发病,确诊 \\
\emph{希特勒} & 可能在1900-1920年间得了梅毒 \\
凯伦·布里森 & 女,可能被表哥兼丈夫的表哥感染,1914年得了梅毒,确诊,1926年就有了征兆,1962年去世,享年77岁 \\
詹姆斯·乔伊斯 & 1904年可能就因嫖妓而得病,1941年病发去世 \\
\end{tabular}
\end{table}

其他可能或已经得了梅毒的名人:
\begin{itemize*}
	\item 前乌干达总统阿明
	\item 达尔文
	\item 多尼采蒂
	\item 陀斯妥耶夫斯基
	\item 丢勒
	\item 列宁
	\item 梅里维瑟·刘易斯
	\item 莫扎特
	\item 拿破仑
	\item 帕格尼尼
	\item 爱伦·坡
	\item 拉伯雷
	\item 斯大林
	\item 托尔斯泰
	\item 伍德罗·威尔逊总统
\end{itemize*}

\subsubsection{书评}
本书对梅毒的历史进行了梳理,并且重点讲述了得了或可能得了梅毒的各色名流的患病历史。很多人史学界并没有一致观点认为就是得了梅毒,但有多方证据与得了梅毒有关。虽然没有统一的结论,但梅毒作为欧洲近现代史上的一个阴影,现在还没有散去。

本书翻译十分拗口,语句不通顺,感觉很不专业,也很不用心。

评分:3/5。