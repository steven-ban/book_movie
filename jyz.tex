\subsection{《基因传·众生之源》}

标签: 基因 \  医学 \  生物学 \  遗传学

作者:【美国】悉达多·穆克吉

\subsubsection{主要内容}
作者在前言里说,\emph{本书讲述了基因这个科学史上最具挑战与危险的概念的起源、发展和未来}。

美国国内提倡“优生学”,借用基因缺陷的理由,把两代“巴克”送入收容所,并把20岁的\emph{卡丽·巴克}绝育以防止她传播“缺陷基因”,这真是美国人权的黑历史啊!当然,纳粹德国和苏联所做的更加骇人听闻。纳粹立法规定遗传病患者、犹太人、吉卜赛人、同性恋者甚至政治异议者均需要进行绝育甚至肉体消灭。截至1941年,旨在清除“遗传病”的杀人计划“T4行动”屠杀了将近25万的成人与儿童;在1933年-1943年间,大约有40万人根据绝育法接受了强制绝育手术。有人说,\emph{纳粹主义不过是某种“应用生物学”}。相对地,苏联农学前李森科复活了拉马克的理论,认为后天刺激可以改变基因,据此认为政治异议分子也可以经过后天改造,因此苏联进行了大清洗和集中营。作者归纳说:
\begin{quotation}
纳粹政府相信遗传物质绝对不会改变(“犹太人就是犹太人”),并且使用优生学来改变他们国家的人口结构。苏联政府则相信遗传物质绝对可以重置(“任何人可以成为其他人”),并且希望通过清除所有差异来实现激进的集体利益。
\end{quotation}

作者又说:
\begin{quotation}
纳粹主义与李森科主义的理论基础源自两种截然相反的遗传概念,但是这两种理论之间也具有惊人的相似性。尽管纳粹理论的残暴性无人企及,但是纳粹主义与李森科主义实质是一丘之貉:\emph{它们都采用了某种遗传学理论来构建人类身份的概念,而这些歪理邪说最后都沦为满足政治意图的工具。}
\end{quotation}

\subsubsection{重要知识}
\begin{itemize*}
	\item 人类基因组包括大约21000-23000个基因。
	\item 纳粹对于遗传学研究的两个贡献:促进了双胞胎研究的水平,促进了遗传物质分子结构的研究。
	\item 中心法则:DNA-RNA-蛋白质
	\item 科恩伯格说:\emph{遗传学家仰仗统计,生化学家仰仗提纯}。
	\item 在人体中,\emph{BCL2基因}被激活后将会阻断细胞的死亡级联反应,从而导致细胞出现病理性永生化,并且导致癌症。
	\item 影响同性恋形成的基因位于Xq28区域,其位于X染色体上
	\item 决定性别的基因位于Y染色体上,为SRY基因。当该基因发生突变而失活后,便会影响男性性征的相关发育,导致虽然携带XY染色体但表现出女性的性器官(斯威尔综合征,拥有女性意识)
\end{itemize*}

从作用上来看,基因包含了人类的所有遗传信息,这些信息不仅包括了相关蛋白质合成的信息,还包括重要的调控信息(位于内含子甚至外显子中)。基因的表达与环境息息相关,同时基因上的突变也会影响相应功能的表达。

\subsubsection{基因发现的主要人物和历史}
\begin{longtable}{p{0.1\textwidth}|p{0.3\textwidth}|p{0.15\textwidth}|p{0.4\textwidth}}
\caption{基因的历史}\\
\hline
人物 & 身份 & 发现年代 & 主要观点或发现 \\
\hline
\endhead
毕达哥拉斯 & 古希腊哲学家,数学家 & 公元前530年 & “精源论”,男性精液携带遗传信息,其在身体各处流动吸收遗传物质 \\
柏拉图 & 古希腊哲学家 & 公元前380年 & 需要精心挑选父母才能有完美的后代 \\
亚里士多德 & 古希腊哲学家 & ? & 否定“精源论”,认为男性精子携带信息,女性经血携带材料,共同形成后代 \\
帕拉塞尔苏斯 & 瑞士裔德国炼金术士 & 16世纪20年代 & 接受男性精子里有一个小人的观念(预成论),女性仅提供发育环境 \\
拉马克 & 法国生物学家 & 18世纪 & 提出“用进废退”的进化思想 \\
达尔文 & 英国博物学家 & 1831年开始环游世界,1839年基本完成进化论 & 先于华莱士提出进化论 \\
孟德尔 & 圣托马斯修道院的修道士,多次考教师不成功 & 19世纪50年代开始做豌豆杂交实现 & 发现显性性状与隐性性状,发现隔代遗传 \\
雨果·德·弗里斯 & 荷兰植物学家 & 1900年 & 看到孟德尔的论文,提出与孟德尔类似的观点,提出“突变”概念 \\
威廉·贝森特 & 英国生物学家 & 1905年 & 接受孟德尔的思想,并大力宣传,提出“遗传学(genetics)”的概念 \\
高尔顿 & 达尔文的表弟 & 19世纪80年代 & 提倡“消极优生学”,发现统计上的“均值回归” \\
托马斯·亨特·摩尔根 & 美国细胞生物学家 & 1905年后 & 使用果蝇进行实现,发现了基因连锁定律和基因互换,认识到基因存在于染色体上 \\
斯特提万特 & 摩尔根的学生 & 1911年 & 画出果蝇染色体上6个基因的分布 \\
罗纳德·费希尔 & 剑桥数学家 & 1909年 & 某一个性状可能由多个基因控制,多个基因间的组合可以导致性状在种群中的正态连续分布 \\
狄奥多西·多布斯 & 乌克兰裔美国生物学家 & 20世纪30-40年代 & 采用果蝇进行实验,提出基因型+环境+触发器+概率=表型的思想 \\
弗里德里克·格里菲斯 & 英国细菌学家 & 20世纪20年代 & 细菌之间可以通过化学物质来传递遗传物质\\
赫尔曼·穆勒 & 摩尔根学生 &1926年 & 使用小剂量X射线照射果蝇可以造成大量突变 \\
奥斯瓦尔德·埃弗里 & 纽约洛克菲勒大学研究细菌的教授 & 1940年 & 发现并提取的DNA \\
莫里斯·威尔金斯 & 新西兰物理学家伦敦国王学院生物物理系主任助理 & 20世纪40-50年代 & 研究DNA的三维结构 \\
罗莎琳德·富兰克林 & 英国物理学家 & 20世纪50年代 & 提纯DNA并研究其XRD谱 \\
沃森 + 克里克 &  & 20世纪50年代 & 在威尔金斯和富兰克林的启发下提出DNA的双螺旋结构模型 \\
爱德华·路易斯 & 加州理工的果蝇遗传学家 & 20世纪50年代 & 通过研究果蝇,提出发育时不同的器官和形态由主控“效应”基因编码,工作过程类似于自主部件或子程序 \\
纽斯林-沃尔哈德 & 胚胎学家 & 20世纪80年代 & 提出果蝇发育过程中因子在卵子中不对称沉积,从而造成不同细胞的分化 \\
乔治·比德尔+爱德华·塔特姆 & 斯坦福科学家 & 20世纪30-40年代 & 认为基因可以指导蛋白质分子折叠形成最终构象 \\
悉尼·布伦纳+弗朗索瓦·雅各布 & 分别是巴黎和英国的细菌遗传学家 & 1960年 & 发现细胞内镁离子使核糖体保持完整性,发现RNA的转录 \\
悉尼·布伦纳+罗伯特·霍维茨 & 20世纪70年代 & 研究线虫恒定的细胞数量,发现细胞的程序性死亡 \\
雅克·莫诺 & 法国生物学家 & 20世纪40年代 & 通过研究大肠杆菌,发现乳糖和葡萄糖使得大肠杆菌合成不同的消化酶,即蛋白质调控不同基因的表达 \\
戴维·赫什 + 朱迪思·金布尔 & 科罗拉多大学线虫生物学家 & 20世纪70年代 & 发现线虫发育时同时受到内部的基因调控和细胞间交互的外部影响。 \\
维克多·马克库斯克 & 约翰·霍普金斯大学的内科医生 & 20世纪40年代-80年代 & 统计遗传病与基因突变间的关系,结论:单个基因突变可以导致不同的临床表现;多个基因可能共同影响一个生理功能;同样的突变基因可能导致不同的“外显率”;突变只有统计上的意义 \\
杜德娜 + 卡彭蒂耶 & 生物学家 & 2012年 & 基因编辑 \\
\hline
\end{longtable}

\subsubsection{书评}
本书内容极其丰富,可以说是将“基因“相关的所有内容进行了生动地介绍:从遗传定律的发现和完善,到分子生物学的推进,再到基因工程,甚至是最新的基因编辑……本书成书于2015年,其包括的信息也更新到了2015年,可以说是目前(也是现在2018年)最权威最全面的一部关于遗传学的科普读物。本书不仅有科学内容的展示,也有科学家进行科学研究活动规律的提示和科学思维的养成,也有科学研究的思考,特别是遗传学和人类社会运动(种族主义、纳粹、医学伦理等)相交叉的思考。如果想系统而全面地了解基因,本书是为数不多的优秀范本,可以当作通识教材来读了。

由于历史上优生学、社会达尔文主义和纳粹的人种论点影响了社会的进步,现在已经被西方人当作遗毒来看待,因此西方人对待改变基因来治疗疾病、胚胎干细胞研究、基因编辑等顾虑重重,多次叫停相关的研究。相比之下,中国没有这么多的伦理禁忌和法律规定,因此中国在相关领域的研究已经和他们相近甚至持平。长此以往,恐怕中国在遗传学研究的前沿领域领先西方同行。

如果仅仅是内容多,体系性强,那么本书仅仅是一部普通的科普读物。难得可贵的是,本书的作用深谙写作之道,能将枯燥的生物学知识用娓娓道来的语气讲述出来,其中又穿插了作者家庭、作者遇见的科学家和患者的故事,不仅能用故事来引出知识点,又能在知识点讲解完之后回到这些人物身上,使读者兴致盎然,长时间阅读也能不失趣味。作者不仅是一个科学家,也是一个有着深厚人文素养的现代人,这两者往往很难在一个人身上同时出现,然而这个印度人做到了。值得赞叹的是,本书的翻译也相当地良心,没有很多科普读物中的翻译腔,译文流畅,达到了”信达雅“的要求,说实话,我读了那么多社科和自科的翻译,这本书肯定能排到前三。

评分:10/10。