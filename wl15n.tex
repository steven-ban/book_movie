\subsection{《万历十五年》}

\subsubsection{一些标注}
(明朝军队)本朝的军制规定,常备军由两百万“军户”提供,每户出丁男一人,代代相因不变。

万历皇帝缺乏坚强的意志和决心,但并不缺乏清醒和机灵的头脑,

万历一朝的冲突,文臣占优势,与天启朝厂卫跋扈、缇骑气焰冲天的情形恰为尖刻的对照。

皇帝也是人而并非神,即使他的意志被称为“圣旨”,也并不是他的判断真正高于常人。他的高于一切的、神秘的力量是传统所赋予,超过理智的范围,带有宗教性的色彩,这才使他成为决断人间的最大的权威。

苏州以赖粮著名,“其乡人最无赖”,此地可称“鬼国”[44]。

张居正的根本错误在自信过度,不能谦虚谨慎,不肯对事实作必要的让步。

朝会集合时,就出现一片令人眼目昏眩的现象。他们的朝服为纻丝罗绢所制,四品以上为红色,五品以下为蓝色。朝冠系纱制,侧带两翅;朝靴黑色,靴底边上涂以白色的胶漆。腰带并不紧束,而是轻松地悬在腰间,上镶玉、犀角以及金银等方块,

在申时行充当首辅的年代,全国文官的总数约为两万人,其中京官约占十分之一[26]。
文官位东面西,武官位西面东。

\subsubsection{感想}
不是历史学家,也非历史专业,对大牛的历史著作只能说是抱着学习的态度,完全说不上批判了。

因此,作为读者,我无力去评判作者反复强调的“大历史观”如何如何。看过黄仁宇的《中国大历史》,作者用自己的“大历史观”完整解释了中国历史三千年,归纳总结了一些想法。大体说来,黄仁宇放弃了严谨但破碎的分列一个朝代各制度(经济、税收、官制、军事等)的呆板的方式,而代之以一种融合的整体的讲法。对于明朝在1587以及之前和以后的一些社会问题,作者并没有仅仅归因于表面上的党争、经济凋蔽、军事落后、制度停滞等原因,而是深挖从明王朝建立以来的制度演变,提示出明朝皇权集中、通货紧缩、税制呆板、兵制落后等一系列问题,而仅仅是以六个人的感想和困境写出:万历皇帝、张居正、申时行、海瑞、戚继光、李贽。在这种提示的过程中,有普通人熟悉的故事,但也有作者带来的新知,一路看下去还是挺有趣味的。

黄仁宇在《万历十五年》中所称的“大历史”,据他自己说,是
\begin{quotation}
大历史的观点,亦即是从“技术上的角度看历史”(technical interpretation of history)。
\end{quotation}
并且,作者认为,
\begin{quotation}
今日很多国家外间称之为独裁或极权,其实其内部都还有很多不能在数目字上管理的原因。
\end{quotation}
我觉得,拿“技术角度”来解释十九世纪以前的历史,不算错,但可能无法解释二十世纪的历史。希特勒是独裁,斯大林也是独裁,但德国和苏联两个国家都是工业化的国家,都可以用数字管理。事实上,科学技术的发展,可以给独裁者更多的手段来管理社会。不看苏联的历史,仅仅从乌托邦小说里,也能看出个大概。

作者反复强调,明朝(其实清朝也是)是一个“以自耕农为主”的社会,
\begin{quotation}
洪武皇帝大规模地打击各省的大地主和大家族,整个帝国形成了一个以中小地主及自耕农为主的社会[17]。
\end{quotation}
比如,
\begin{quotation}
南方的农村大多种植水稻。整片田地由于地形和灌溉的原因划为无数小块,以便适应当日的劳动条件。这样,因为各小块间肥瘠不同,买卖典当又经常不断,是以极少出现一个地主拥有连绵不断的耕地。王世贞和何良俊都记载过当时的实况是,豪绅富户和小户的自耕农的土地互相错杂,“莫知所辨析”。海瑞自己在海南岛的田产,据估计不到四十亩,却分成了九十三块,相去几里[23]。这些复杂的情况,使解决农田所有权的问题变得更加困难。
\end{quotation}
对于明朝特殊的政治气候,作者也多次提到,明朝以德治天下(很大程度上,中国历代皆有此影响),不重法(这是我归纳的),
\begin{quotation}
在我们形式化的政府中,表面即是实质。 
本朝以诗书作为立政的根本,其程度之深超过了以往的朝代。
\end{quotation}
所有的政治活动,都围绕着道德展开,从对皇帝的劝柬(对皇帝私生活的干预有时到了令人哭笑不得的地步),到各派间的争斗,都拿道德做面子,
\begin{quotation}
(文官争斗)技术上的争端,一经发展,就可以升级扩大而成道德问题,胜利者及失败者也就相应地被认为至善或极恶。
\end{quotation}
明朝由朱元璋农民起家,但朱有能力有魄力,却缺乏远见,用落后的制度来约束社会发展,子孙后代又不能改革,到后来社会很稳定,但也很落后,
\begin{quotation}
这一帝国既无崇尚武功的趋向,也没有改造社会、提高生活程度的宏愿,它的宗旨,只是在于使大批人民不为饥荒所窘迫,即在“四书”所谓“黎民不饥不寒”的低标准下以维持长治久安。这种宗旨如何推行?直接与农民合作是不可能的,他们是被统治者,不读书,不明理,缺乏共同的语言。和各地绅士合作,也不会收到很大的效果,因为他们的分布地区过广,局部利害不同,即使用文字为联系的工具,其接触也极为有限。剩下唯一可行的就是与全体文官的合作,如果没有取得他们的同意,办任何事情都将此路不通。
\end{quotation}
全书的六个人物,在作者笔下写得有血有肉。

万历这个大胖子,从小受张居正教导,生活在张居正和太监??的阴影下,张居正死后便不断去除张的影响。万历辍朝三十年,又是立储又是争国本,简直是在和百官过家家。

张居正有能力有成绩,但重名重利,落了个身败名裂。

申时行谨小慎微,中人之才。

李贽嘛,我觉得有点像是妄人。他的思想,据作者说,是“分裂”的。但他的一些论点倒是挺有意思,
\begin{quotation}
李贽更为大胆的结论是一个贪官可以为害至小,一个清官却可以危害至大[76]。他尊重海瑞,但是也指出海瑞过于拘泥于传统的道德,只是“万年青草”,“可以傲霜雪而不可以任栋梁者”[77]。对于俞大猷和戚继光,李贽极为倾倒,赞扬说:“此二老者,固嘉、隆间赫赫著闻,而为千百世之人物者也。”[78]在同时代的人物中,他最崇拜张居正,称之为“宰相之杰”,“胆如天大”[79]。张居正死后遭到清算,李贽感到愤愤不平,写信给周思敬责备他不能主持公道,仗义执言,但求保全声名而有负于张居正对他的知遇[80]。
\end{quotation}
李贽对清官的评价可谓精辟。清官往往碌碌无为,清谈天下事,但就是不会办事。清官可能有极强的“理想主义”,但受理想和首先的禁锢,往往放不开手脚,难以成事。这种人,叫清流,他们的言论,叫清议。李鸿章就在给人的书信里,批评过朝廷里这种官员只知高谈阔论却不能成事。而这种人,可能还热衷于党争,党同伐异,并不是特殊材料制成的。因此,海瑞的境遇是
\begin{quotation}
他虽然被人仰慕,但没有人按照他的榜样办事。
\end{quotation}
与之相反的是戚继光,他给张居正送礼,但能打仗,能打胜仗。他不是理想主义者,但在环境和制度允许的范围内干出了成绩。我觉得,这种人才是中国的脊梁。
\begin{quotation}
和戚继光同时代的武人,没有人能够建立如此辉煌的功业。他从来不做不可能做到的事,但是在可能的范围内,他已经做到至矣尽矣。
\end{quotation}