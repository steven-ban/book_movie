\subsection{《民国纪事本末》}

\subsubsection{一些标注:}
婚姻定义即为色衰而有夫如故,法统定义即为身败而有民如故。

民族革命党之民族主义自尊心受侮已甚,练兵、建设稍有起色,即有针锋相对、逐条回敬、不惜以同归于尽求快之心,其下意识期待蒋中正所谓“最后关头”一洗前耻久矣,不复计黄雀在后。

(陈烱明)竞存吊孙文联(一九二五):“惟英雄能活人杀人,功罪是非,自有千秋青史在;与故交曾一战再战,公仇私谊,全凭一寸赤心知。”

日人政令分歧,自乱阵脚者颇多,推本溯源,皆明治宪法之私智所赐。

正常国家化犹如传薪接棒,实为跨代跨派之长时段历史运动,劳作者不居,居住者不劳。

五四以来,华民族主义所仇者,一向为友华势力,于仇华者无寸伎可施。“

成功者必然革命,失败者必然反革命。

三月二十四日,南京党军劫掠英、美、日领事馆,伤英领事、杀美金陵大学校长等外侨,外人集体撤离。英美军舰入宁,与党军交火。 其曲在华,责无可卸。反帝排外自始即有自毀性格,遇鸽派则出以横暴,遇鹰派无术御之,破国亡家,率皆出此。

条约体系瓦解,真能从中牟利者,绝非高唱反帝之国府。列强势力均衡破坏,日俄争霸取而代之,中国处境其实更为恶化。北洋老奸巨猾,深知弱者依赖法律保护远过于强者,不拟大动干戈;而愤青革命家只知意图伦理,拒绝承认身为弱者、不公正法律犹稍胜于赤裸暴力,诚所谓“本不植高原,今日复何悔”。

(1922年)八月二十六日,国会宪法审议会决议:中华民国为联省共和国,各省得自制省宪。

名实悖误为民国史一大特色,民族主义最擅引寇入室、民主英雄一味扩张公权、群众革命无不增贡于民、新学健将恭行以吏为师,反不及家长君主及保守核心相对而言较少践踏群己权界。

每易岁必易法统,何怪乎民视法如草纸。

日式外交,擅长受横暴之名实,享微薄之实惠,恰与俄人相反,正合“俄狡倭暴”论。
五四学潮即张本于袁氏五九国耻日纪念,华民族主义洪流溃决,革命吞噬立宪,救亡压倒启蒙,皆自此出。

“联美制日策”与民国相始终,自善后大借款、“二十一条”,经凡尔赛-华盛顿会议,直达珍珠港。

败局之成,诸将不受驭为内因,民党-日本联盟抵制为辅因,大战将兴,列强(以英为首)无力维系远东国际协调(即民国统一和平)为主因。

北洋最可取者,即在“司法不党”确能践行,明知其效多有碍于秉政者。

霸术以削藩为首,袁氏能覆国会而不能废联省制,无怪其亡。

民国实为省联而非联省,袁氏亦不以大总统自任。

专制中央集权历史合法性原在慑服豪强,令顺民得以苏息。袁氏之败,实不在其专制,而在其削藩功败。

论政者之不宜从政也明。

能代袁者,有其权诈而欠其大度;能代孙者,受其虚浮而乏其精诚。

九月二日,孙文于京讲演:振兴实业,十年筑路二十万里。 文人论事,不切实际多类此。孙文之铁路玩具热,给予袁氏一大机会解脱张案压力,铁路之寸基未成,自不足道。

民初诸法皆不甚重司法独立,法司亦有署无人,更易起官吏侵越之心。

吾民族之团体尽忠观念过于僵硬,必以议会政治必有之妥协调和为失德,亦宪政破产一要因。可注意者,此类节烈论暗合宋儒,而于革命两党之手发扬光大,北洋诸公尚不过小巫耳。

党国虽有法币改革、关税改革之利,终以间接税(关盐统三税)为重,有土而无民,上重而下轻,沪税居天下之半,津门次之,抗战军兴,二埠先亡,后之国府,乃如晚期食道癌,纯以耗蓄积打点滴(华府美援)延寸阴,其亡也宜。可异者不在其亡,而在其垂绝残焰竟能拖垮强邻,复我深仇。

就史论实,行政统一、地方自治本为宪政平衡木,民党之实迹纯系机会主义,在野则主省权,在朝则主集权,一如孙、蒋二公居元首则主总统制,居下位则主内阁制,喜剧色彩浓厚。

欲左右逢源者,多左右不是人,黎公之谓也。

黎公好名,不免于好名者天然陷阱——妄图人人满意。

二十世纪之多难,中枢即在十九世纪自由主义瓦解,各国愈居下位,受害愈深。

民初尚属天真时代,出言必践,信义尚重。后之非常大总统以复国会为词,国会已复,自食其言,而激发陈炯明兵变,判若两人矣。有此表现者非独孙文一人,民初之贤人大抵即民末之权奸。政风日下,廉介者段吴反有顽钝之名,自趋毁灭。此事若有责任问题,吾民族崇暴轻法之恶根性亦当分责。

存尊号,待以外国君主之礼,暂居宫禁,岁费四百万,日后移居颐和园,侍卫留用。满蒙亲藩黄教僧团旧有封地津贴一准前清旧例。各族权利平等宗教自由,风俗礼仪依旧。
国之将亡,大老宦囊厚积者必求仕新朝,勤王赴难者必出“前异见者”,如明之黄道周辈,俄之梅德韦杰夫辈,清人亦非例外。

民党以北美南洋华人社区为基本选民(尚需与保皇党中分义工捐款),其次为留日学生,再次为新军中级军官,而后帮会,士绅极少。光复会海外组织远逊孙梁,几纯以江浙士绅卫系,善战敢死过于民党,而活动范围狭窄更甚于民党,失江浙绝其命脉,章炳麟“革命军起革命党消”不幸中于自身。光复会员绅士气重,好逞意气,易散难合,耻于联络下层,亦难幸免于政风日趋下流之际。

封建之国如守万山,败而据险甚易,不可征服——马基雅维利

临时政府日坐愁城,促袁接手烂摊之急迫,尚速于袁氏夺权步骤,正所谓“看人挑担不觉沉”。

无洋款即无左李功业,清室必如唐明,葬身流寇之手,汉民必受闯献之毒,乌得有民国?清亡杀戮极少,打破华史成例,末君亦得善终,无往而非帝国主义遗德),

终民国之世,宁汉隐为敌国,虽昔日之信友,分据二地必成敌手。直至李、白中南溃灭,蒋氏亡宁就台,余烬稍息。

军权保护财权、财权支持政权实为民初政争铁例。

就宪政而论,民初省内集权制与民国联省制极不协调,违背联邦以县乡自治为节制州权根本之义,已成民元约法致命之伤。然主要责任者不在迁就现实之宋教仁,而在洪、杨、曾、左制造之既成事实。

凡真行民主之处,庸民之道德大多数必不容革命家久居其位,而“谨愿者亦为之”绝难长于蜜月。故革命、立宪就思想人物阶级皆分两途,同舟异梦,本属必然。革命者迟早须面对选择:弃血战经营之壮志重权于鼠目短见之庸民以全立宪;抑或以革命大义驱庸民于夹道险途,“强迫你进步”,不复以立宪为意,视之如视言论自由,仅为夺权手段耳。
袁公未出成算已定,以立宪议和为纲。虽不免为己谋,不谓之此时华人所能选择之最佳出路,不可得矣,取大位亦非幸致。

辛亥军务,与其视为南北兵争,毋宁视为南北当局各自与财政崩溃社会解体进程争时。
以彼立场,辛亥之宪法意义,即在系于帝室之宗藩关系为汉臣单方面撕毁,蒙藩效忠者本即皇室而非国家,自此即可自由行动。

清室之亡,浅而言之,可谓亡于其自伤技术高于革命党劣质炸弹;统而言之,实系秦政防猜之术再度(既非首度亦非末次)完成其设定功能。

文学社加盟革命,极具业余性质,反侦察意识薄弱,势将为布尔什维克所笑。可见中产良家子极不适于密谋造反,开国帝王必出流氓黑帮,有其“存在合理性”。

不取南人为北洋心法,以北人朴厚敢死,易于驯顺,南人巧诈多智,自喜好争故。非仅便于求近。虽吞江汉,家法不改)。

诸军无论出身、新旧,皆受吸毒式招收游民—军饷膨胀—财政崩溃—任用私人—把持饷源—勒逼地方之经典进程驱使。

多方角力,清室收完败、民党取虚名,唯地方实力派坐收全功。

长曾李传人私军之弊,举社稷殉之,余毒流于异代。

\subsubsection{一些感想}
作者抛弃革命史观(不仅共产党有,国民党也是很左很革命的,两者一脉相承,革命理论都和孙中山和苏俄有很大关系,只不过前者主要学习苏俄的共产主义,后者主要师承孙中山三民主义),以编年史的形式从宪政的演变来重新叙述民国史。只这一样,此书便值得一读。

宪政是晚清以来大多有识之士的追求,但在实现路径上多有分歧,加上太平天国以来南方汉族实力派崛起以及越来越拥兵自重,民国始终难以在有足够中央权威的条件下实施有计划有步骤的宪政过程,加上各派私心颇重,互相疑忌,程序正义败给狡诈权谋,正人君子不敌厚黑大师,终于一步步自毁法统,沦入武力革命的深渊。感想:宪政需要有足够的中央权威,需要中央有足够的诚意,也需要民众有足够的觉悟。百年开外,不知今天的中国离宪政还有多远。

作者喜用春秋笔法,一字褒贬。袁世凯乃国会选举产生,南北双方都接受,故以袁大总统称之;自乱法统,压制国会权力,则以其名呼之。日本大败在即,蒋介石促日投降,朱德也促日投降,可见共产党虽名义依顺政府,但早已把自己当成独立政府,内战是早晚的事。

以1922年废除新旧约法、大总统权力凌驾国会并另组国会为分界线,民国历史截然分为法统和革命时代。前者不过十年,但一路坎坷伤痕累累跌宕起伏;后者27年,却不过是派系混战一马平川。历史良机一旦错过,再回头便不可能。一百年的弯路,其实在那宪政黄金十年里就已经注定。