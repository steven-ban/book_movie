\subsection{《包法利夫人》}

作者:【法】福楼拜

小说情节比较简单,主要讲了主角包法利夫人(艾玛)和丈夫夏尔·包法利之间的成长、相识、结婚的故事,主角是艾玛,笔墨重点放在她在婚后厌恶丈夫、渴望“浪漫爱情”和上流生活以及因此与两个人的偷情故事。艾玛为了维持自己的偷情生活,欠了商人和高利贷者一大笔钱,支出过高,无法还债,最后抵押掉自己家的房子也无法还清,于是走上了自杀之路。

艾玛只是一个农村农场主出身,少女时期去上了都会学校,读了不少浪漫爱情故事,于是对这样的生活充满了憧憬与羡慕。
\begin{quotation}
在她奔放的热情中,却有讲究实际的精神,她爱教堂是为了教堂的鲜花,爱音乐是为了浪漫的歌词,爱文学是为了文学热情的刺激,这种精神和宗教信仰的神秘性是格格不入的,正如她的性格对修道院的清规戒律越来越反感一样。
\end{quotation}

但是自己出身比较低,无法攀上贵族名流的生活,只能嫁给老实本分的乡村医生夏尔·包法利。夏尔长相普通,没有远大理想,医术平庸,生活不算好,而且缺少情趣,没有丝毫浪漫之情。他对艾玛百依百顺,宠爱(从他自己的角度来看)有加,物质上达到了自己的极限,但这仍然无法满足艾玛那一颗浪漫之心。一开始,艾玛对婚姻仍存在着一些渴望,世俗和宗教道德伦理对她具有一些约束力,她只是想想浪漫爱情,并没有跨出那实质性的一步。但当夏尔在容镇做脚骨手术失败后,他名声迅速下降,再也无法跨越平庸乡村医生的圈子,艾玛对他生出更多的厌恶和绝望之情,
\begin{quotation}
她觉得他是个小人物,没本事,不中用。总而言之,在各方面是个可怜虫。
\end{quotation}

于是就想“摆脱”他,奋不顾身投入到偷情的生活中,把自己的希望寄托于情人身上,带她远走高飞,给她浪漫爱情的满足。

小说情节主要在容镇这个法国普通的乡镇上进行。在容镇之前,艾玛参加了一次巴黎的舞会,这个舞会上她遇见了不少上层贵族名流,特别是子爵,这成为她对浪漫爱情和上流生活的具像,即使在最后走投无路时,艾玛仍然在脑海中看见子爵从自己眼前走过。但舞会只是昙花一现,这以后艾玛再也没有参加过同样的盛大舞会,而是被自己的出身和财富限制在乡村中,限制在平庸和粗俗中。

艾玛的第一个情人是莱昂,这个年轻人只是一个药剂师的实习生,艾玛漂亮有气质,这对莱昂产生了巨大的吸引力,他不断找机会与艾玛接触,艾玛何尝不知。但此时的艾玛仍然没有跨出那一步,而是在本能驱使下不断与他会面,但没有表明心迹。莱昂的自卑也限制了自己的行为,他没有表现得地过于狂热。情事无疾而终,莱昂离开容镇去做事业。

无聊的生活中,艾玛遇见了隔壁镇的小地主罗多夫,他岁数比艾玛大十来岁,人高马大,惯于情场。他一下子就看出艾玛想要出轨,于是借故与艾玛接触并很快让艾玛踏出出轨的一步。出轨的核心的欺骗,而罗多夫惯于如此,于是两个人的感情立刻变得火热。偷情是刺激的,是无聊生活的调味品,艾玛需要这个调味品。两个的感情随后出现了冷却,但夏尔手术的失败让艾玛十分生气,她立刻恢复了与罗多夫的炽热偷情,并希望对方带自己走。她开始偷偷花钱,借债,并且越花越多,一发而不可收,这为她以后的灭亡埋下了伏笔。但罗多夫是理性的,他并不希望带艾玛远走高飞,在两个约定离开的当天就反悔,并不再见艾玛。

第一次出轨失败,艾玛的浪漫梦想破灭。她没有反悔,而是越发堕落。她是矛盾的,
\begin{quotation}
哪里找得到一个刚强的美男子,天生的勇敢,既热情洋溢,又温存体贴,既有诗人的内心,又有天使的外表,能使无情的琴弦奏出多情的琴音,能向青天唱出哀怨动人的乐歌?为什么她就碰不到一个这样的男子?啊!不可能!再说,也不值得追求,到头来一切皆空!一切微笑都掩盖着厌烦的呵欠,一切欢乐下面都隐藏着诅咒,兴高采烈会使人腻味,最甜蜜的吻留在嘴唇上的只是永远不得满足的淫欲。
\end{quotation}

这次的她见到了几年后的莱昂,两人旧情复燃,莱昂变得自信而大胆,艾玛借故学钢琴欺骗夏尔,并定期与莱昂幽会。欠债终于使她不堪重负,无法掩盖,即使卖掉了夏尔的旧房子也无事于补。到了最后一步,艾玛决定自杀,她找到药剂师的砒霜,痛苦而缓慢地死去。生前无限风光,热爱浪漫的她,死得狼狈而肮脏:她的嘴角流出黑色的液体,她的眼神空洞而无神,她的遗体前是神甫和药剂师无聊的辩论……她死后,福楼拜既同情又讽刺地说:
\begin{quotation}
(神甫)用右手大拇指沾沾圣油,开始行涂油礼:先用圣油涂她的眼睛,免得她贪恋人世的浮华虚荣;再涂她的鼻孔,免得她留恋温暖的香风和缠绵的情味;三涂她的嘴唇,免得她开口说话,得意地叫苦,淫荡地得出靡靡之音;四涂她的双手,免得她挑软拣硬;最后涂她的脚掌,免得她幽会时跑得快,现存却走不动了。
\end{quotation}

夏尔对自己的妻子深信不疑,一直是一个顾家的好男人,始终不相信自己的妻子会出轨背叛自己。直到艾玛死后,他看到她写给罗多夫的信,才恍然大悟。但艾到的形象在他脑海里挥之不去,他深情地爱着她,在心底原谅了她,最后他死在去艾玛的回忆里。他死后,他们的女儿被送走,成为童工。

福楼拜文采斐然,对艾玛的心理描写十分细腻生动,对她心态的变化和捕捉十分灵敏,是现实主义的高峰。福楼拜主要是现实主义手法,但在夏尔的死亡这些事上,又受到之前浪漫主义文学的影响,两相结合十分自然好看。

福楼拜不仅讲了一个女人被欲望和想像毁灭的故事,还将眼光放到了十九世纪上半叶法国普通乡村人物的描写上。这些人物平庸而世俗,既有胆小怕事的莱昂之流,也有卑鄙嗜利的高利贷者,还有虚伪势利、沽名钓誉的药剂师。福楼拜下笔冷静,没有对人物的褒贬,但几句话就栩栩如生,干净利落。

本书的翻译是许渊冲先生,翻译十分流畅生动,达到信达雅的水准,读起来十分舒服。

评分:5/5。