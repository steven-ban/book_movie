\subsection{《素女经》}
\subsubsection{作者冯唐}

不停的贫,贫,贫,抓一些冷僻知识(其实不算多冷僻,百度百科上很容易找到,所以只能说是抖机灵,不算是卖弄),反复地讲道理说感想。

故事本身很简单:黄冈学霸清华毕业生斯坦福出身的田小明,在美帝遇见读研的师妹白白露,两人搞在一起。田小明受发小王大力邀请回国创业,创办生物高科技公司,遇见合作的咨询公司的万美玉,出轨也搞在一起,两人身体契合,心灵也契合。后来白白露发觉,也怀孕,万美玉也让田小明二选一,田小明暂时选择婚姻,半年后又出轨找万美玉,因为受不了后者的控制欲,跳楼坠亡,灵魂出窍又还魂,送进精神病院,出来后去西藏成了神迹一样的人物(这段无意有意在恶心西藏)。

这本书是黄书,性描写露骨,虽不算多出彩多优秀,但结合在情节里也算水到渠成,没有多刻意。但人物形象模糊,除了话唠外难以让人记清楚(有这种特点的小说,人物往往沦为作者观点的传声筒)。

作者常有妙语,可供消遣,如:
\begin{quotation}
像坏人的红酒就是好酒,一时让人开心,过后一直伤心。像小坏蛋的就是挺好的红酒,像大坏蛋的就是超好的红酒。
\end{quotation}

\subsubsection{田小明}
清华和斯坦福高材生,文理兼通,喜爱自摸和A片,更喜欢美女。讲话把生殖器和吹牛逼挂在嘴边,有趣,但不深刻;嘴欠,但说不上油滑;坦诚,但遮遮掩掩做不到完全的赤裸相见;还算专情,但责任感让位于生物本能和一时痛快。田小明懂女人,会搞女人,但丝毫不专一,并且毫不愧疚(或者毫不要脸)地把这归咎于生物本能。活的敞亮,但很猥琐,有高富帅的面目,但处处都为满足屌丝意淫的幻想(或许仅仅是作者冯唐的幻想)。

\subsubsection{白白露}
郑州人,清华大学电子系毕业,爱运动,长的漂亮但肯定不惊艳。似乎对男人本性放的很开,和田小明两地分居还叮嘱他可以嫖娼但不能有外遇(女人特殊的洁癖和贞操观)。分居久了田小明不再陪她也陪不了他就不再联系,但偷偷设置监控,监督田小明的一言一行,发现他把女人往家里带,回国生孩子,试图把田小明拉回自己身边,但田小明还是偷偷去找万美玉。

作者似乎很爱定义理工和文科学生的思维差别,也爱用清华和北大学生的梗。白白露作为清华理科女,具有所谓的“理科思维”,会搞硬件也会写程序,还会装监控搞偷窥。回国前白白露很大度,之后成了小家子气的女人,滑向心胸狭窄和控制狂魔的深渊。

\subsubsection{万美玉}
工作狂,条理分明,田小明身体和心理的红颜知己,“不顾一切”地和田小明约炮,试图让田小明脱离婚姻。在控制欲这一点上,和白白露如出一辙。