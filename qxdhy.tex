\subsection{《球形的荒野》}

作者:【日】松本清张

1944年,日本战败在即,国内政治和军队里出现了两派:一派是主和派或海军派,他们认为日本大势已去,战败不可避免,力主和盟国达成停战协议;另一派是死硬派或陆军派,他们坚定地认为日本还可以再战八年,力争战斗到底。在中立国瑞士,日本的外交人员也明显地分成了两派,其中主和派的野上显一郎是一等书记官,他偷偷与英美的谍报和外交人员接触,通过交换日本的政治军事信息而求取日本在战后的某种程度的优待,并脱离日本国籍,获得了法国国籍,逃往盟国,日本方面为了掩饰,对外通报他生病而死,并让当时的下级村尾芳生把骨灰带回了日本。

然而,十多年以来,野上思念自己的妻子孝子和女儿久美子,虽然他已经有了新的妻子(温柔体贴,有点像日本人,对丈夫的这种情感十分包容和支持),但仍然怀着思念和新妻子一起回日本,想暗中见妻女一面。他重拾自己的爱好,在日本的古寺里签下自己的假名留作自己对故国的思念之情,但被同样去古寺游览的节子(她是久美子的表姐)发现,野上学的是米芾的书法,在日本比较少见,辨识度很高,从而引出了本书的故事。野上在日本与自己当时的部下同事见面,并让他们安排自己与女儿的相见,先是通过请妻女看歌舞伎的方式暗中观察她们,又托人让画家为女儿画素描的方式,扮作杂工来近距离观察女儿久美子,然后又匿名写信让久美子一个人去京都相见,只不过由于警察介入跟踪保护,无法相见。久美子对警察暗中跟踪很生气,又留在京都一天,见到了一对“法国夫妇”并与之交谈,后来去的酒店也是同一所,野上给久美子打电话听女儿说话,但没有相认导致久美子十分疑惑。当晚死硬派组织派人从酒店窗外向化名赶来的野上的旧部下开枪,来警告他们当时的“卖国”行为。

另一方面,当时日本在瑞士外交部门里的死硬派回国后依然贼心不死,他们发现了野上回日本,于是赶来东京,但被主和派的当年的手下发现并杀死。死硬派一直跟踪当时的主和派,并复仇杀死了当年手下。故事的最后他们还一直在活动,在做杀掉“卖国贼”的事情。

在这个过程里,久美子的男友添田彰一一直在追踪这些事件,他最后知晓的事件的真相,安排久美子与野上会面,两人在海边依然没有相认,但野上一直在以一个在法国多年的日本人与女儿交谈,还为她唱了小时候的儿歌。节子的丈夫芦村亮一是大家教授,他之前已经见到了野上并知道了事情的全貌,但没有告诉其他人。

整体来说,这本小说的推理性并不强,30\% 的篇幅后读者就能猜到野田没有死并且一直在活动,只不过事情的真相特别是战时战后日本的政治态势对具体情节的影响没有和盘托出。这里涉及两个命案:一件是当时的武官、死硬派伊东忠介的死(凶手就是使馆里主和派的门田源一郎,他之后就被死硬派灭口),一件是画家的死。还是就是京都酒店的枪击案。不过这几件事的关联性不强,都是和野上的行踪有关联但没有绝对的因果关系。

本书的写法也比较松散,是全知型的写法,作者是上帝来叙述事件,但是总是在不同的人物之间进行切换(例如最多的添田彰一、久美子、芦村亮一,以及叙述者本身的对事件的补充),但并不是《冰与火之歌》里那种严格的第一人称视角,而是类似于游戏中的“越肩”视角,书写角度实际上还是全知全能的作者。我个人感觉在推理小说或者侦探小说里,这种写法并不好,不容易造成紧张刺激的气氛,更不要说让读者自己代入到侦探视角参与推理了。本格派的推理小说一定要有一个“侦探”(其具体身份也可以是警察,也可以是马普尔小姐那样的第三者)来将事件串起来,把证据全摆出来,本书的这个人就是作为记者的添田彰一,他追问事件到了让人讨厌地地步,一幅“很抱歉,给你造成麻烦了,但是你必须得说出来事件的真相”的样子,不知道当时日本的记者是否有这么大的权力,以及世人特别是政府官员、社会名流会不会都给记者这么大的面子(我觉得不会),而且里面的人说话也不是直来直去的(特别是芦村亮一和添田彰一都知道了蝍上没死,两人都想告诉对方,但都支支吾吾不说)。

野上对久美子的情感在最后一刻表现得十分感人,除此之外本书的叙述都是很冷静客观的,作者也不会随意发表评论。

评分:2/5。