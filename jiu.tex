\subsection{《咎》}

作者:凯特·肖邦

\subsubsection{人物}

\begin{longtable}{p{0.1\textwidth} | p{0.4\textwidth} | p{0.4\textwidth}}
    \caption{《咎》人物表。} \\
\hline
人物 & 特征 & 事件 \\
\hline
\endhead

\hline
\endfoot

杰罗姆-拉弗姆 &天主教徒 & 猝然去世,有1400亩的种植园 \\
特蕾莎 & 杰罗姆的遗孀,刚刚30岁,年轻漂亮,没有孩子,生性脆弱 & \\
海勒姆大叔 & 杰罗姆的仆人 & \\
格雷戈尔 & 特蕾莎的侄子,舞步优雅,雄辩,长相帅气,眼睛有神,棱角分明,黑眼睛炯炯有神,声音低沉轻柔,女声女气,带一点哀怨惆怅 & \\
大卫-霍斯默 & 39岁,身材高瘦,面色蜡黄,黑发不少已经灰白,脸上有皱纹,热切追求功名机遇,不注重保养 & 想要买下特蕾莎土地上的伐木权,拥有一个锯木厂,儿子曾经死掉,两年前离婚,父母早死,独自照顾妹妹长大 \\		
贝琪 & 特蕾莎的女仆	& \\
梅莉森特 & 大卫-霍斯默的妹妹,修长的身形,橄榄色皮肤,深棕色眼睛和头发,好奇心重 & \\	
乔森特	& & \\	
范尼·拉莫瑞 & 霍斯默的前妻,交往下等人,酗酒 & 父母早死 \\
洛伦佐-沃辛顿的太太 & 朋友叫她“贝拉”(美女),住在范尼对面 & 只生了一个12岁的女儿 \\
杰克-道森的太太 & 名字“卢”	& \\
伯特-罗德尼先生 & 贝拉和卢的朋友,喜欢和她俩约会 & 妻子和女儿不在身边 \\

\end{longtable}

\subsubsection{主要情节}
霍斯默爱上了拉弗姆夫人,向她示爱,拉弗姆夫人对霍斯默也有情意

拉弗姆夫人是个天主教徒,认为男人应当有男子气概。霍斯默听从她的意见,接回了范尼,与她重新结婚,回到锯木厂

霍斯默虽然和前妻复合,但仍对拉弗姆夫人心存爱意

梅利森特经常和格雷戈尔一起出去

她刚开始对他没感觉,不喜欢他说话的方式,但慢慢喜欢上了他

梅莉森特拒绝了格雷戈尔的爱,格雷戈尔很爱她

乔森特命丧锯木厂,起火,格雷戈尔开枪打死他

梅莉森特因此不再理睬格雷戈尔,她第二天就离开了种植园

格雷戈尔与人发生口角,被打死,梅莉森特为他穿起了丧服,她内心觉得她可能爱上了他

范尼知道了霍斯默和拉弗姆夫人的事情,生气出走。在河边的小屋里,落水,被霍斯默救起。范尼死去。

霍斯默和拉弗姆夫人最终结合。梅莉森特外出旅行。

\subsubsection{书评}
特蕾莎应该是作者意志的投射:她是个完美的女人,为人公正、坚强,没有美国佬的做派,眉眼间大方地流露着亲切的同情。但她对自己也存在一定的疑惑:她其实喜欢霍斯默,她一直以来都是尽力正直地帮助别人,是否一定正确?

小说本身并不吸引人,情节十分简单,只靠作者的意志力来推进,人物形象单薄,性格只能靠大段的心理描写和作者的上帝视角来说明,人物行为缺少动机,作为小说而言,是平庸甚至不合格的。至于婚姻与人性冲突这一人类永恒的主题,在作者这里成为一个机械和肤浅的矛盾,为了解决这个矛盾,作者杀死了范尼。如果想成为伟大的小说,作者必须让范尼活着,甚至让范尼和霍斯默冰释前嫌重新相爱,让霍斯默和拉弗姆夫人挣扎、冲突、决裂,让这一矛盾真的在两个人身上燃烧。因此,作者显然取了巧,我认为造成这种现象的原因在于作者根本就没有好的文学审美,没有人文积淀,因此她笔下的小说只可能带入自己的浅见和认知,而没有发人深省的艺术力量。

小说里有大量的心理描写,人物转换随意,视角混乱,使人难以沉浸到情节里。

这是作者的处女作。一般而言,处女作的写作技巧比较稚嫩,但作者的创作理论会更容易显现。比如金庸的小说,后期虽然技巧更好,但更体现他本真的创造旨趣的反而是前期小说。

小说涉及19世纪下半叶美国南方的种植园生活,主要是法国文化区(如天主教的节日)。作者的南方人痕迹很重,甚至对黑人与白人之间的平等理念有抵触。

归入“女权主义小说”似乎有点牵强,难道仅仅是因为作者本身是女性并且去反思婚姻制度与爱情之间的取舍吗?作者在小说里表达的,虽然确实对婚姻有一种反思和反叛,但并没有多么深刻,陷入一种非黑即白的思辨中。仅仅因为一种解读出来的政治意义来拔高小说,本身并不可取。作者的观点,用一篇短文就能说清楚,非要编一个蹩脚的故事。总之,这是一篇不值得读的小说,读了简直是浪费时间。

评分:2/10。