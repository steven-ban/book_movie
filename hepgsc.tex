\subsection{《黑色皮革手册》}

标签: 推理小说 \ 社会派 \  松本清张

作者:  松本清张

野心成就了我,也毁灭了我。

\subsubsection{人物}

\begin{table}[htp]
    \centering
    \caption{《黑色皮革手册》人物表}
    \begin{tabular}{p{0.1\textwidth} | p{0.3\textwidth} | p{0.5\textwidth}}
人物 & 关系 & 事件 \\
\hline
岩井叡子 & 妈妈桑,34-35岁 & 经营酒吧10年,元子在她店里实习一年 \\
原口元子(化名春江) & 至多33岁,叡子的高中同学 & 白天在银行上班,晚上去陪酒,刚来半个月,毕业后就工作了15年 \\
千鹤子 &  酒吧小姐 &   \\
楢林谦治  & 妇产科医院院长 &  妻子多病,多年包养中冈市子,之前光顾叡子的酒吧,后来转向原口元子的酒吧 \\
A画家 &    &   线索人物,视角转到元子后消失 \\
藤冈彰 & 东林银行千叶分行经理 & 元子事发后被调地方,后死去 \\
村井亨 & 东林银行千叶分行副经理 & 元子事发后到董事手下工作,报复元子 \\
波子 & 原口元子新酒吧的工作人员,漂亮虚荣轻浮 & 楢林包养她,给她钱开酒吧,后被抛弃,转到董事手下,后买下更大酒吧,报复元子 \\
美津子 & 元子酒吧的女侍 &  \\
里子 & 元子酒吧的女侍 &  \\
中冈市子 & 楢林妇产医院的护士长 & 工作二十多年,与楢林一同逃税卖死胎,被包养多年,已经离不开楢林谦治 \\
岛崎澄江 & 梅林料亭的女侍,30多岁,举止优雅 & 托言想跳槽到元子的酒吧挣钱,欺骗元子,其实是桥田常雄的情妇 \\
桥田常雄 & 医学院考试补习班校长,油腻好色 &   \\
江口虎雄 & 江口大辅的叔父,70岁,好色 & 有桥田常雄补习班内学员向医学院行贿的“记录”,曾因江口大辅的关系在桥田常雄的补习班当校长 \\
江口大辅 & 参议员 & 刚刚去世 \\
安岛富夫 & 江口大辅生前的秘书 & 想竞选江口大辅选区,诱惑了元子开房,后离开 \\

    \end{tabular}

\end{table}

\subsubsection{事件}

\begin{itemize*}
    \item 楢林谦治与中冈市子在经营妇产医院时,为了逃税而少报营业额(去他们妇产医院打掉孩子的都是不光彩的,不走医保),并将打掉的胎儿卖掉。两人将钱以不同的名义存在多个银行。这种行为被原口元子注意到。
    \item 原口元子从东林银行储户账户中盗领大量资金,并把具体情况写在黑色皮革手册上。
    \item 原口元子从银行辞职去开酒吧。
    \item 原口元子以交出黑色皮革手册中的数据作为条件,威胁藤冈彰、村井亨和律师三人不要把她挪用的钱还出去(她连三分之一都不想还给银行),并让他们定了保证书来保证不用还钱
    \item 几个月后原口元子用钱在银座豪华地段开了一家酒吧“卡露内”。酒吧的波子被楢林谦治看上,后者为她花钱盘下店面准备开店。
    \item 原口元子让自己公司的员工的妹妹做实习护士,潜入楢林的医院的内部刺探情报,发现医院把堕胎下来的胎儿和胎盘冷冻后送往别处。
    \item 原口元子从中冈市子口中得知楢林妇产医院的内幕,利用波子和她的矛盾激怒她出去开咖啡店,然后诱惑楢林谦治和自己开房,利用自己复印的妇产医院的财产内幕威胁他给自己五千万。楢林谦治抛弃波子,与中冈市子复合。
    \item 桥田常雄想买下梅村。元子以可以攒钱为由劝说岛崎澄江代替自己去和桥田常雄幽会。
    \item 元子看上了更大的酒吧鲁丹俱乐部,从游手好闲的兽医那里打听到消息要出售。她搭上安岛富夫,后者“答应”帮她扳倒桥田常雄。元子坠入情网,爱安岛,但后者在跟她接吻、开房后很快不再与她联系,元子也慢慢忘掉。
    \item 元子探知鲁丹情况,利用从江口虎雄处拿到的资料去威胁桥田常雄,逼他将梅村料亭转让给自己,并给自己大量钱,桥田常雄似乎答应了她。
    \item 元子与鲁丹老板谈判,约定先付四千万,如果买不下则再付给四千万。
    \item 桥田常雄戳穿元子,说自己从没有买下梅村,元子得知梅村转让是注销转让房产,属于逃税手段。她去找鲁丹老板理论,与波子发生冲突,流产,被送到楢林妇产医院里,抬进了中冈市子的手术室。
\end{itemize*}

\subsubsection{书评}

\begin{quotation}
她没有情人,也没有亲近的好朋友,总是独自旅行。她在旅游地时常碰到出手阔绰的团体或情侣,可是她只能俭约地旅行。\\
其实她早已习惯这样的生活,并不觉得落寞孤寂。她习惯把自己关在银行界的白色围墙里。\\
不过,她后来发现了自由而缤纷多彩的世界,极想早日冲出白色的围墙。因为只要你有才干,就能尽其所能发挥。这社会是多么生动有趣,充满无限的可能性,就像夏日阳光般绚丽多彩。然而,那绚烂的阳光突然蒙上了阴影。
\end{quotation}

这只是一个普通得再不普通的银行女职员,没有亲密的爱人,似乎也没有家人,更没有要好的朋友。她相貌平平,没有女人味,在银行一干就是十几年,成了大家眼中的不受欢迎的老员工。她一个人形单影只,孤独地让年华老去。她“不觉得落寞孤寂”吗?我觉得未必,不然也不会有后来长期压抑带来的反抗了。一个真的内心丰富的人,即使看到了绚丽多彩的世界,也会安之若素。然而,原口元子不是这样的人,她其实是积累了几十年的物质欲望,被开了一个口子之后喷薄而出了。

\begin{quotation}
她享受着秘而不宣的喜悦和窃占公款的快感。这是她对自己在银行上班以来长期受到同事排挤冷落的心理报复。……黑色皮革手册果真发挥了强大的作用。当她看到敌人露出惊慌的表情时更是快意畅然。
\end{quotation}

她利用熟悉银行和财务操作流程的便利,偷偷记录了支行里的违规记录,记在黑色皮革手册上,并拿来要挟银行经理和副经理,他们无计可施,只能给了江口原子七千多万日元。得手的她利用在酒吧实习工作的经验,开了一家新酒吧,成了妈妈桑。然而,她的心气并不是开一家小酒吧,而是开一家更大更豪华的新店,于是她把目光锁在了楢林谦治身上。他是一家妇产医院的院长,利用法律漏洞逃了大量的税,并以他人名义存在了多家银行上。原口元子利用他想要抛弃的情妇、情妇本人却无法离开他(她在这家医院干了十几年,把青春都放到了楢林院长身上,因此对他另结新欢波子心生嫉妒怨恨),把这些违法账目记到了第二本黑色皮革手册上,并用开房当诱饵,威胁楢林谦治给她五千万日元。她成功逼走了波子。"用不正当的手段搞钱也是理所当然吧?这是她对长期以来苦闷生活的报复!"她的道德水平就不高,居小礼而无大义,堕落也是理所当然。

这本书主要是以原口元子的视线展开。到了这一步,原口元子自以为得计,于是把目光瞄准了更大的目标——开医学院培训班的桥田常雄口中的高级酒吧鲁丹俱乐部。她故伎重施,利用梅村女侍岛崎澄江色诱桥田常雄,她自己则找到常去自己酒吧的议员秘书安岛富夫合计找证据扳倒桥田常雄,安岛介绍她去找桥田常雄培训班的前校长江口虎雄,说他有培训班与医学院受贿的证据。他们不费吹灰之力就拿到了证据,元子将它抄在第三本黑色皮革手册上。岛崎澄江与桥田常雄开房的红色灯光刺激了原子内心的对男人的生物欲望,她短暂地爱上了安岛富夫,与她在晚上接吻,以至于开房。然而,安岛富夫很快与她脱离了联系,她也很快放下了这段短暂的爱情,转而筹钱去买下鲁丹。她向长谷川庄治签谭合同,以四千万为首付,如果之后不买则再付四千万。她向桥田常雄展示了自己的第三本皮革手册,以此威胁他为她转让梅村(价值一亿六千万日元多)并给岛崎澄江一千五百万日元(她自己私吞一千万日元)。自以为得计的她信心满满地准备买下鲁丹,然而桥田常雄很快向她摊牌:她被耍了。

桥田常雄、岛崎澄江、安岛富夫、江口虎雄、长谷川庄治甚至于给元子提供小道消息的兽医(这一点是元子推断的)就是一伙的,他们联手在元子面前唱了一出戏,目的就是打垮元子。江口虎雄的所谓证据是假的,岛崎澄江是桥田常雄的情妇,江口虎雄是在玩弄自己的身体和感情(甚至嘲笑她的僵硬的体位、笨拙的动作和让男人提不起兴趣的颜值),桥田长雄没有买下梅村仅仅是在房产登记上做了手脚让元子误以为是他已经买下,长谷川庄治明知道元子会违约拿不出四千万却引诱她签下合同。元子从楢林谦治得手之后的一切,都是这些人的计谋。

最后元子失败了,她无法买下鲁丹,甚至要将自己辛辛苦苦经营的卡露内卖给波子(没错,波子投靠新的金主,买下鲁丹,并且与桥田常雄他们串通)。气急败坏的元子与波子撕打,却动了胎气流产,为她做手术的,竟然是楢林谦治和中冈市子。

元子失败了,败给了这些人的阴谋,也败给了日本这个社会。在这个社会里,以桥田常雄为代表的所谓精英分子,无视国家法律,疯狂地为自己牟取私利,偷税漏税、贪赃枉法是家常便饭,而且他们结成了攻守同盟,对敢于反对他们的人(比如元子)进行打击报复。而元子呢?她既让人可恨,又让人可怜。说她可恨,是因为她也想贪图金钱享乐,视国法于不顾,也视中冈市子、岛崎澄江这样同为女性的人的幸福于不顾,试图通过欺骗的办法来获利。元子对待他人,对待社会,十分冷漠。她可以说是有着野心,有着物欲,而对人情对他人则缺少打从内心深处的关心。她她她托人办事会去买贵重的礼品,但待人并非真心,而是利用。说她可怜,是因为她的野心,无非是开一家更大的酒吧,她所有的梦想、欲望、精力都放在了这上面,她除了金钱上的欺骗,其实并没有像楢林谦治和桥田常雄那样罪大恶极,更大的罪恶俱乐部里,她还不够格。她的心还没有那么狠,她在这些人面前,还太稚嫩,以至于被人看穿心思,加以利用。

元子身上,缺少作为女人的幸福的能力,中冈市子说她不懂得女人,她自己也是毫无情趣,缺少女人味。她成为物质欲望的陪葬品。当然,这本书里,女人完全是有钱有权男人的玩物,像波子就是从一个男人换到另外一个男人去依附,中冈市子则是完全离不开包养自己十多年的楢林院长,即使后者另结新欢,她也可以“原谅”他,再次回到他身边。这些女人,也不是没有经济能力,而是感情上、人格上离不开男人的包养。像中冈市子,就是有着像样工作的人,但她似乎被社会排斥,中了毒一样留在楢林谦治身边。

这种悲凉绝望的感觉,是不少日本文艺作品共同展现出来的,从太宰治到《小偷家族》,这种社会的冰凉、冷漠,应该是日本社会的底色。中国社会也有罪恶绝望,比如鲁迅笔下的中国,但中国的温情整体上比日本好太多。

这本书的节奏感不错,先是借画家A的视角来引出主角元子,然后就一直以元子的第一视角展开,对元子如何从一无所有到一步步地开酒吧的心境描写得很细密,特别是她从一家酒吧到看上更大酒吧、坠入得手的那种狂喜和催生出更大欲望、坠入情网的怦怦跳的心情、马上要买进酒吧的激动和迫不及待的心情,都很抓人。她眼里的世界,是如何冷漠,她对波子的嫉妒、不服输和好胜心,一直强迫着她往前走,再也难以回头。然而,她很快地失败了,更糟的是她竟然怀上了欺骗她的安岛富夫的孩子,并最终流产,走向了悲剧的命运。这本书的基调是绝望的,元子周遭的世界是黑暗的,她对这个欲望的世界有着美梦,想融入它,却做不到。

松本清张讲故事的能力在这本书里发挥得淋漓尽致。这本书并不是推理小说,虽然是“社会派”的名头,但推理很少,更多是运用铺垫、暗示、怀疑对读者进行“欺骗”,但读这类小说多的人,应该在阅读到一半时开始怀疑元子是不是进入了阴谋。

评分:4/5。