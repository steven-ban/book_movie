\subsection{《他在她的国》}

作者:夏洛特·铂金斯·吉尔曼

\subsubsection{事件}
艾拉多跟随我们去了真实世界,欧洲正在打一战。我和艾尔玛一起游历世界。艾尔玛反对战争,认为基督教没有制止战争。我们讨论两个社会的差别。

我们游历了欧洲、埃及、印度、中国、日本,经夏威夷到了美国西海岸。

经过对社会的考察,艾拉多为美国开了药方。

艾拉多和我回到她国,我们的儿子降临。

\subsubsection{书评}
如果说前作《她的国》是传统社会中男人在她国时观念被冲击和颠覆的话,本作就完全倒了过来,是她国眼中的现实世界。

如果说前作是通过反差来反思现实社会中的性别差异的话,本作就是直接对现实世界做出反思和批评了。

关于战争。作者似乎是个和平主义者,她国里就没有战争,也没有产生战争的因素。但在现实世界,当时正在打一战,这是人类历史上规模最大也是最惨烈的战争之一,同盟国和协约国把自己的士兵送上战场去送死,而打的目的却似乎没有那么深思熟虑。作者借艾拉多之口,认为人类历史上的战争都是男人的责任,与女人无关,这当然是无稽之谈。男人掠夺、杀戮难道不是为了整体社群的发展和壮大吗?战争就没有女性的支持吗?人类并非食草动物,面临的生存环境十分逼仄,从古猿时代的天敌的屠杀,到渔猎时代的环境压力,再到农业时代的天灾和人口压力,人类社会一直是在匮乏和不安定中度过的,并非田园牧歌的悠闲。我恶意揣测一下,作者作为女性,虽然经过良好的教育,但显然生活在过优渥,没有尝过生存的艰难,没有被别的民族欺压,于是认为人类的和平、友爱、善良、容忍都是理所当然的。这是十分片面和短视的行为。19世纪正是世界弱肉强食的社会,欧美的发达是建立在开拓大量殖民地和对其他国家的经济压迫上的,作者口中欧美社会的”文明“恰恰需要那些不文明社会的付出才能实现。

艾拉多他们游历了中国,可以看出作者对当时中国的态度。在作者笔下,中国地大物博,历史悠久,可以看作一个伟大的民族,但中国当时十分落后,还保留着女人裹脚的恶习,生态环境恶劣(缺少植被)。这大约是当时中国的实情。但艾拉多(其实也是作者)对中国的那种厌恶让100年后的我感到十分反感。任何一个民族,都有落后的时候,身为”先进“民族的一员,对于落后民族抱着哪种感情,可以看出他心胸是否博大。欧美人对亚洲,总是有一种居高临下的心态,同时也不愿意真的去了解这些民族的真实情况就妄下诊断。这是西方中心主义的流毒,是当时时代的局限。

对于日本,作者表现出了更多的喜爱。当时的日本经济比中国好太多,环境整洁,但作者随即指出,日本普通民众十分贫困,女人承受了大量的劳动,其地位十分低下。这大概也是实情。

美国是他们重点探讨的国家。艾拉多认为美国民主强大,是人类的希望。同时,艾拉多对美国起源时的奴隶贸易、亚非劳工的不平等进行了谴责。最重要的,艾拉多认为美国女性没有选举权,没有参与到理应是每个人的游戏的民主中,是这个国家最大的不平等。总的来说,这反应了作者和那个时代的美国人对自己国家的反思,很多问题还是很尖锐的,可以看出美国人善于反思,一直在进步。不过,像种族主义这样的顽疾,解决问题比看到问题复杂多了。对于美国的经济制度,作者实际上对美国的自由放任主义和私有制进行了批判,认为让家庭或私人占有公共的资源造成了美国的不平等,提倡国有化(虽然当时没有明确的凯恩斯主义)和社会化的经营方式。

她国是一个停滞的社会,是一个自足的社会,作者把它当成理想国,是荒唐的行为。而作者参照这样一个乌托邦来给世界和美国开药方,也是愚蠢和短视的行为。艾拉多走马观花的考察,肤浅的理解,竟然自称“学术研究”,敢在学术会议上和专家对喷,真是民科治国的典范。作者认为美国应当让更多人上高中,提倡公民教育,提倡报纸共有而非属于资本家,但提不出可行的举措(因为根本就不可行)。作者用家庭主妇式和菜市场买菜的方式核算火车卧铺的成本认为太贵而应当降价,无视经济学原理,显得浅薄和荒谬。

作者对社会主义(民主社会主义)表现出兴趣,反对放任自由的经济。

对于人性,作者明显地把人类的天性分为“父性”和“母性”,认为前者造成了这个社会的冲突和战争,提倡社会增加更多的母性来调和。作者提倡女人应当具备”母性“,好好养育孩子,这本身就是对女性的禁锢。作者显然认识到了职业化对女性的意义,但显然没有认识到这件事情的重要性。女性只有直接参与到社会化的职业中,才能真正和男人一样改造社会,从而摆脱家庭的禁锢,成为真正的第一性。

总的来说,艾拉多的世界游记只能说是浮光掠影,见到的问题十分表面,表现出作者对于世界和历史的浅薄与无知。另外,作者把艾拉多塑造成仁慈而敏感的圣母,世界是她的婴儿(多么女性化的视角!),见到不平等和恶行就动辄流泪心痛,而“我”则成了她的闺蜜而非丈夫,两人更像是旅伴而非夫妻,说话方式如出一辙,显露出作者对人物性格和互动的力不从心。从文学上来讲,这本小说根本没有个小说的样子,几乎没有情节,全是大段的对白,以此借艾拉多之口表达对社会的想法,人物形象单薄不可信,没有性格。因此,作者的想法,用来写论文更好,非要写成小说就勉为其难了。从小说创作的角度来看,本书是失败的,其小说性还不如前作《她的国》。

评分:2/5。