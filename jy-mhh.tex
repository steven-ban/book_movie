\subsection{《解忧杂货店》}

封面就说这不是一本推理小说,读下来才知道,虽然不是推理小说,但耗费的脑细胞并不少,交叉变换人物视点叙事本来就需要记住前面的情节和暗示,其阅读“难度”其实是比作为推理小说的《梦幻花》要高的。 
其实整个读下来,这本小说的核心就在于“时间机器”的设置,几个人的命运在浪矢杂货店和丸光园中交汇重合,又远离,然后再次汇集。 
其实这种小说没有太大的深度,读完也觉得是推理小说的皮,但核心是一碗鸡汤。

\subsection{《梦幻花》}

说实话读完这本小说有点失望,感觉作者和能写出《白夜行》和《嫌疑人X的献身》不是一个人。其实想想,这本小说的叙事技巧并不比前两者差,只是立意不高,强行为国民上课(关于核能和”家庭责任“)。为什么读下来没有前两者好?因此这本小说灌注的感情远远差于前两者。你读不出作者强行编故事渲染出的那份感动。 
小说只有三个视角:一心破案以与儿子交流的警察下濑,被父亲强行棒打鸳鸯而耿耿于怀的下治(?),以及放弃奥运梦想表哥爷爷相继被杀的秋山??。整个故事就是个一本道,前面的铺垫作者马上就迫不及待地揭开,不需要读者费脑筋。 
真相让人大跌眼镜。堂堂大日本帝国,对于疑似毒品的东西不是派个研究机构秘密保护起来,而是交给警察和医生家族,美其名曰”家族责任“。家族承担得起一国毒品滥觞之责任吗?

不过大夏天的读完两本小说确实还算满足。东野圭吾还是好好加油写出个经典出来吧!如果只是整天写一些水平低的小说,虽然赚钱多,但后世留不了名啊! 
东野圭吾的小说时间跨度大,而且有很强烈的年代感,用特定年代的一事一物就把年代传达出来,笔墨精炼,但类似猜谜语的快感还是让人兴趣盎然。