\subsection{《中华帝国的衰落》}

作者:【美】魏斐德(Frederic Wakeman, Jr.)

本书作者是被称为美国“汉学三杰”之一的魏斐德的著作,是美国多所大学的”汉学“专业的教科书。本书主要聚集中华帝国晚期的历史演进,涉及从晚明到民初大约300年的历史。本书囊括了这三百年历史的主要历史事件和社会变动,是了解明清史的普及性读物。

读书之前,可试问自己几个问题:
\begin{itemize*}
	\item 明清时的主要社会变动是什么?中国各阶级是如何应对的?
	\item 明清时的主要历史事件是什么?它们之间有何关联性?
	\item 外国人在中国的近代化中起到了什么作用?
\end{itemize*}

这几个问题比较抽象,但从中可看出读者对这段历史的大致认识。

那么本书是如何回答这个问题的呢?这本书的脉络或曰主线,是\emph{士绅这一阶级在明清社会中的作用以及不断的变化}。本书作者将士绅分为上层士绅和下层士绅,他们皆从科举制度中产生。作为直接统治阶层的主要是上层士绅,他们与皇家相合作,维持着帝国宏观层面的运转。但是,由于清朝统治者的满汉有别的思想,上层汉族士绅并不能取得独立的地位,他们的权力被牢牢地控制在满族人手里。下层士绅被拒绝在帝国的官员队伍之外,只能依附在中央和地方体制下充当塾师、讼师和小吏的角色。然而,清朝人口的快速增长和土地的紧缺,导致基层治理能力的相对不足,他们渐渐开始在基层活跃。太平天国\footnote{至少前期主要是客家人的联结,这一点让人很意外。}的爆发和随之而来的捻军、回乱、新疆叛乱等一系列社会危机,八旗绿营的逐渐腐朽,使得清王朝不得不依赖于汉族上层士绅组织的地方团练。这些士绅与下级士绅通过乡谊的连接,逐渐控制了地方的事权、税权和兵权,成为汉族士绅崛起的依靠。太平天国运动,成为中国传统帝国体制和满汉体制发生变化的转折点,左右了中国后续的政治选择。作者抓住士绅以及背后的儒家文化这一主要脉络,将中国近现代史的这种演变贯穿起来,对于长久以来浸淫在单一的革命叙事话语下的中国人,应该是比较新鲜的观点(当然,对于真正了解这一部分历史的人来说,可能流于简单化和程式化)。

两个重要的事实:
\begin{itemize*}
	\item 地方自治运动还把城市士绅与以商会为代表的新兴资产阶级联系了起来。联合后的两个阶级构成了各省主张变法的精英分子,加速了辛亥革命的爆发。
	\item 客家人不仅是太平天国运动的核心力量,成为19世纪中国历史转折点的那场叛乱,也是被客家空想家——洪秀全的神秘启示激发起来的。
\end{itemize*}


由于以上原因,本书还是值得一看的。当然,受限于篇幅,本书对于历史事件的讲述还是不够详细,因此正如本文开头所言,这是一种普及性的读物以及一个西方视角下的历史认知。本书虽然为美国人所写,但大的历史错误基本没有,其可靠性还是值得保证的。不过,联系到西方普通知识分子对中国的狭隘甚至无知,本书的内容可能并不为西方人所熟知,可见西方汉学在西方并非显学。

评分:7/10。
