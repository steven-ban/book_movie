\subsection{《岳村政治》}
这本书2012年初买到手,当时只看了第一章前两节和附录《衡山调查记事》的三分之二,之后研究生做实验做课题,来郑州工作,忽忽四年都没有再翻开看看。2014年租房住,书放在集体宿舍,2016年七月才搬到新家。后来收拾好次卧放进书桌,这才回到灯下看书的日子。

佩服作者于建嵘为了获取第一手资料,不是坐在书斋空想,而是前后历经两年(1999-2001年)深入乡村、镇和县里实地考察,与农民、村干部、镇干部和县委交流询问,吃住在农民家里,往往需要记录写作到凌晨、坐农用车冒雨赶路,多次感冒。这种精神,恐怕不是随便一个博士生能做到的。

从行文可以看出,于建嵘是个很有情怀的人,虽为学者,但面对农民丝毫不摆架子,也有能力和农民深入交谈以获取真实想法(这在田野调查里很重要),同时又能坚持原则敢于当面拒绝不合理的请求,这一点恐怕大多数学者不会愿意做或做不了这么好。于建嵘出身农民,小时候是黑户,对农民的感情是真挚饱满的。后来于建嵘在北京画家村,进京上访的人喜欢找他,他也乐于帮助他们,又是做饭又是给钱。他写书前干过律师,还会作画,《母亲》这幅画和罗文中的《父亲》也有异曲同工之处。他给官员讲学,也乐于引导他们关注底层和弱势群体。总之,他的人格,我佩服,见贤思齐,我也要学习!

正因为深入基层,使得作者有了大量的访谈笔录、文字资料(如乡约、政府文件等)和调查问卷,对一些最基本问题有了实证性的记录(比如农民的负担、村干部的工作难处、镇政府的财政状态等),这恐怕是本书最有价值的部分。附录里的调查记录,虽然像写日记,但文字中间有推论和思考,读起来比正文的要严谨。

本书为农村社会的几千年变迁(主要是近一百年)给出了清晰的理论框架和模型,着重阐述了乡村社会在国家主导的社会制度变革中的合作及抵触,并分析了包括各级基层政府、传统宗族势力、家庭、各级干部如何共同作用于这一变革。

作者给出了农村社会的民主化方向的畅想,但结论显得仓促,缺少足够的论据。事实上,从作者完成本书(博士论文)到现在已经16年了,农村社会除了农民负担更轻、收入提高显著、更多更快的人才及劳动力流失、基层更失控、群体性事件更多以外,政治结构并没有什么变化。看来,农村的民主化社区的努力基本是失败的,城镇化才是方向。

湖南农村的发展水平,基本上比华北要快上十年。