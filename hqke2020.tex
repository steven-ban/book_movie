\subsection{《环球科学(2020年合订本)》}

标签:自然科学 \ 科普

\subsubsection{标注}
果蝇在清晨和傍晚更偏爱绿光,在中午却喜欢红光。这样的变化主要受到生物钟的调控。果蝇具备二色视觉,能分辨不同色调的绿色和蓝色,红光对它们来说相当于很浅的绿色,在中午时分选择红光有助于让它们找到凉爽的区域。携带特定基因突变而导致生物钟失灵的果蝇会始终待在有绿光的区域,而生物钟错乱的果蝇会徘徊在绿光和红光之间,表现出异常的行为模式。不过在任何时候,果蝇都会回避蓝色的光线,因为蓝光接近紫外线区域,具有较高的能量。

打哈欠可能起到了促进大脑降温的作用,原理就是:当你深吸一口气时,吸入的空气会使大脑稍稍降温;同时,伸展下巴增加了流向大脑的血液,这是另一个降温因素。

表面自由能是恒温恒压时材料表面相对于材料内部所多出的能量(因为材料内部的分子和表面的分子受到的力不同)。固体表面自由能越高,越容易被润湿。

维尔切克认为,超越标准模型的新理论不会带来新的技术革命,但是,通过全新或是更加深入的方式理解现有的理论,依然能带来新的技术,这同样是令人激动的探索。

梦的形成不需要全脑的参与,而仅需要激活后皮质热区这一特定区域即可。

\subsubsection{书评}
很久没有看过这种杂志类的科普了。高中时候在学校外面地推上看到《Newton》这样的彩页印刷的科普杂志,把太空、物理图景画面栩栩如生,我极为震撼,生动趣味性十足,让我彻底喜欢上了物理学,至今我也觉得这是我看过的现代物理学、宇宙学最好的诠释。后来也看了几期《环球科学》,知道这是《科学美国人》的中文版。但是上了大学之后,就基本上再也没有看过这类杂志,而是看一些专门的书籍。相比书籍,这类杂志的时效性更好(虽然中文翻译有一定的滞后),并且品类更多,短小易读性好。2010年之后看的一些果壳网、科学松鼠会、博物杂志的科普文章,也都挺不错。

这类科普,让我发现中国和美国之间对待科学上的不同。中国人讲究实用,讲究集体和服从,因此对专家和权威的话更为信服,但大多数人不会主动去探究何为科学精神(特别是怀疑精神、逻辑、科学实验方法等)。美国的精英层讲究科学,毕竟近代科学是西方发源的,并且美国的平均收入更高,工业化现代化时间长,已经形成了在长期的科学氛围。不过美国底层民众相信科学的有,相信宗教的也多,这两者很多时候是互斥的,因此美国人科学精神的上限很高,但下限极低,明显低于中国接受了九年义务教育的人,比如他们借宗教之名不相信地球是圆的,相信人是上帝造出来的,等等。美国的科研机构想要让国会拨款,就要讨好民众,教育民众,因此他们开发科普的书刊杂志、广播电视节目的动力更高,而且也有不少精品,比如他们制作的各种讲相对论、宇宙学、黑洞、量子力学的纪录片就很多,这一点国内还没有形成风气。

这次再看2020年全年的杂志,让我对美国的科研和科普有了新的认识。物理学前沿已经半个多世界没有足以撼动人类认知的重大突破了,因此可以看到这些内容都是关于如何发现希格斯玻色子、如何建议国会拨款建造加速器、如何观察宇宙中的各类光谱等内容。另外,美国向来喜欢研究生物医学(包括精神疾病类),这些内容相当多。还有环境保护类(如气候变化、美国森林大火)、人类学等内容,这都是国内科普界很少涉及的。但是,这么多年来,中国的科研事实上已经很强了,但这类杂志似乎在有意避开中国科学家的研究,所采用的研究论文和成果(特别是欧洲一些国家)并没有比中国人做的更好,不知道这是不是美国科普圈子里一向的偏见还是歧视。这些内容,平时读一下是有益的,但是并没有给人以震撼和革新之感。当然,电子版的看图片比较费劲,可能看纸质版的会好很多吧。

评分:4/5。