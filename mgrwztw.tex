\subsection{《民国人物在台湾》}

标签: 国民党 \ 民国  台湾

作者:张林 \ 丁雯静

本书由凤凰卫视出品,有同名纪录片,本书应该是记录片的文字版。本书的主要内容,是讲述了国民党主要高层在败退台湾之后的生活,但其实还有在大陆时期的介绍,可以看成是这几个人物的小传记。

本书涉及的人物:

\begin{description*}
    \item[阎锡山] 1883年生于山西五台,小地主家庭,9岁入私塾,1902年考入山西武备学堂,1904年留日进入振武学堂和日本陆军士官学校,1905年加入同盟会以及核心组织“铁血丈夫团”,1909年回国,1911年在山西起义,成为山西都督。盘踞山西,以自己利益为重,置身全国革命事外,只求保存实力,对蒋、日、共都是反复无常,残酷冷血,暴戾乖张,故弄玄虚,杂糅了传统儒家、马克思主义、三民主义等思想,在山西兴办实业。抗战时与日本谈判,希望日本不进攻山西。解放战争中太原战役被解放军消灭了主力,差一点抓住胡耀邦,部队中还有对他感恩的日本人,城破后五妹殉“国”。1949年坐飞机逃跑,因金条太重而飞机飞行不便,于是少带几个卫士。原配夫人竹青,未生育,在外娶姨太生儿育女。1960年去世。
    \item[陈诚] 1898年出生于浙江青田县,世代务农。冒名顶替考进保定陆军军官学校,学炮兵。1923年随孙中山出征西江,负伤见到蒋介石,1924年调入黄埔军校任教育副官,受蒋赏识,从此追随蒋一生。1930年任上将,个子低,文弱,经常在前线,但军事能力一般。参与蒋嫡系的很多战役,如围剿、武汉会战、解放战争中的东北战事等,多有失败。对蒋敢于犯颜直谏。1948年受蒋委命经营台湾,稳定住了局势。1950年任行政院长,后成为蒋的副手,权势日隆,压过蒋经国,与蒋家父子的关系变得微妙。与谭延闿三女谭详结婚,感情好,儿女均成材。1965年因肝癌去世,墓园改为“辞修公园”。
    \item[白崇禧] 1893年生于广西临桂县,回族,父亲弃文从商,死后家中败落,1911年加入广西学生军,1916年毕业于保定陆军军官学校。1921年面见孙中山要求参加革命,任黄绍竑的参谋长。1925年与桂林名媛马佩璋结婚,性格坚强。1926年认识李宗仁,成为新桂系主要人物。1938年提出“游击战与正规战相配合,积小胜为大胜,以空间换时间”,台儿庄战役协助李宗仁。1949年败退台湾,之后被蒋冷处理(三次下野都与李白二人有关),任闲职,甚至被特务弄乱家。1954年被弹劾侵吞国库资产。在台湾任回族协会会长,兴建清真寺。曾被蒋派人谋杀,未遂。1964年李宗仁向大陆示好,白崇禧在公开场合劝他。1966年死于家中地板上。其子白先勇。
    \item[何应钦] 1890年4月生于贵州兴义,父亲为商人,排行第三,小时成绩优异,1908年留日加入同盟会,1915年毕业于日本陆军士官学校。黄埔军校时相当于是蒋介石的学生,东征时救了蒋介石,劝蒋发动“中山舰事件”,中原大战时任蒋手下。北伐末期蒋第一次下野时不支持蒋,受蒋记恨,怀疑他有异心。1935年签订《何梅协定》,为卖国条约。西安事变后建议用武力解决,被蒋记恨在心。抗战胜利后受降,对日本人太客气而被指责。抗战后期豫湘桂战役时去贵州救急,收复一些失地,后蒋倚重陈诚,解放战争时没有担负重要职务。因与日本人交好退居台湾后日本人帮助训练军队。退居台湾后1952年国民党七大不是中央常委,被驱逐出权力中心,只任一些虚职。1956年任台湾旅游观光名誉会长。1961年排练反共歌舞剧《龙》,后在“中国童子军副会长”。晚年在家中种兰花,生活规律。妻子王文湘,是他胞妹,1916年认识,后者终身未育,过继了弟弟的女儿,怕老婆。1987年死于心脏病。
    \item[胡宗南] 生于1896年,浙江镇海县人,黄埔一期,与蒋是同乡,受蒋信任一路高升,1945年手下兵力60万,称“西北王”。1947年攻下延安,很快失败。1950年去浙江沿海的大陈岛任“浙江反共救国军”总指挥。1937年认识杭州大三学生叶霞翟,1947年两人才结婚。1953年在日本人帮助下秘密训练部队。低调清贫,1962年心脏病发死亡。
    \item[吴稚晖] 江苏武进(常州)人,天性顽皮,24岁以第一名入学成绩就读于江阴南莆书院,1891年用篆书乡试中举,与康有为约定不再科举,曾闹事县太爷、江苏学政,原因是他们不尊重知识分子。1897年任教于北洋大学、上海南洋公学,1901年留日读日本东京高等师范,1905年参加同盟会,1911年回国不就教育总长,1913年任教育部读音统一会议长,1918年办中法大学,勤工俭学政策带出来不少中共高层,1924年出任国民党一大并当选为中央监察委员,此后追随蒋充当反共先锋,鼓动蒋发动四一二反革命政变。喜欢骂人,出语粗鄙,毫无斯文,周恩来斥为“小丑”。与蒋经国交好,出卖过陈独秀儿子陈延年。1949年随蒋入台,1953年死,海葬在大小金门海域。
    \item[于右任] 1879年出生于陕西三原,母亲改嫁,后被婶娘抚养,17岁中秀才,20岁时受新任陕西学政叶尔恺赏识,1900年因讽刺慈禧被通缉,逃往上海租界。成为马相伯学生(两人差39岁),写报刊论政,办《神州日报》,民国成立任监察院长。1931年再任监察院长,1949年随蒋撤回台湾,平民化的他拒绝蒋安排的官邸,住在闹市,弹劾李宗仁(当时出走美国)和俞鸿钧。喜好诗词书法,经常写字送给政客来结交。结四次婚,幼子和母亲留在大陆,国共和谈时经常牵线搭桥。1964年死于拔牙并发症。立铜像于玉山,后被毁坏。
    \item[孙科] 1894年出生广东香山,父亲孙中山,在檀香山长大,妻子当地华侨陈淑英,1912年回国,平时喜欢读书。对共产党态度一直摇摆,1927年四一二反革命政变以后支持蒋,之后反蒋,1932年任立法院长,主持宪法起草,限制总统权力,但1936年被蒋修改。喜欢英美模式,想用该模式改造中国,1947年选举败给李宗仁,1949年入去香港,1952年定居美国,穷困,需要自己做家务,一直在家读书。与原妻一直没有离婚,后断断续续与严蔼娟同居四年,生下孙穗芳(13岁时见证了孙家女人争斗),后与蓝妮结婚。1965年回台(此时蒋经国已经巩固权力,蒋家对其他国民党元老已经不怕威胁),做闲职,1973年急性心梗去世。
    \item[陈立夫] 1900年出生于浙江湖州,北沣大学读矿科,立志工业建国,从煤铁入手,“志在采矿”。陈其美是其二叔,因此称呼蒋介石为“二叔”。1924年取得美国匹兹堡大学冶矿硕士,1925年成为蒋介石幕僚,劝蒋介石不要去苏联,起草北伐宣言。1927年奉蒋介石之命成立“中央组织部调查科”,1931年担任国民党组织部部长(接了哥哥陈果夫的班,两人及其党羽合称为CC系)。抗战中坚持为教育拨款,仅次于军费,建立助学贷款(后未偿还)。入台后因CC系人数众多,与陈诚关系较近,恐被蒋介石猜忌,因此自觉离开台湾去美国,由于家穷而养鸡,1962年因父亲病重而短暂回台,后立即回美,但鸡场因火灾被毁,整日埋头读书。1966年因蒋介石八十大寿回台定居,生活低调,1988年提出“以中国文化统一中国,建立共信”,获大陆赞赏,90年代计划与邓小平会面未果,2001年去世。妻子孙禄卿为同乡,学习国画,是娃娃亲,两人感情甚笃。
    \item[严家淦] 1905年出生于江苏吴县木渎镇,父亲为木渎首富严国馨。18岁保送上海圣约翰大学理学院,主修化学,辅修数学,成绩优秀,家境优渥(当时有自己的汽车)。1931年到福建实业公司工作,1937年充当福建省主席陈仪秘书,办理发电和公路事业,推行对农民有利的实物地租改革,1945年随陈仪到台湾参与交通工作,新历二二八事件,被林献堂所救,1948年任财政部长,主持军饷发放(亲自调查部队空额)和币制改革工作,发行新台币,影响至今。1954年任台湾省主席,不争功不夺利,受蒋家赏识,后任行政院长,“懂事”地配合蒋介石为蒋经国铺路,1975年蒋介石死后任“总统”,1978年“还政”蒋经国。由于待人温和,性格内敛,得以善终。家庭关系和谐融洽,与妻子刘期纯关系好。
\end{description*}

这些民国高官,本身就顶着失败的名头,退居台湾后由于与蒋关系的不同而命运迥异。如果那个时代已经远去,这些人早已作古,他们的子女也都老去甚至死去,现在的台湾已经抛弃了国民党当年的革命叙事和民族主义,彻底走向“台独”的死胡同和“小确幸”的自我感动,心理上与当年的“民国”可谓形同陌路。大陆的人不会给他们太正面的评价(毕竟很多都是反共急先锋),而台湾的人,也渐渐忘记了他们的功过,这些人,真是可悲可叹。

本书内容十分浅近,没有文献引用,就像是故事集,不知道考证是否严谨。而且对人物的分析浮于表面,简单看一下即可,没有必要深究。另外,本书引用了不少高官后人的评论,明显有为尊者讳的倾向(比如白先勇),这部分内容也不值得看。

评分:1/5。
